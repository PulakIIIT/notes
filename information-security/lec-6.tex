\section{Information Theory - Lecture 5}
\section{Our first example of circumventing an impossibility}

One way function: hard one way, easy the other
Trapdoor one way: One way function with a trapdoor that makes the hard way easy with the key.

\section{Exploring discrete log problem}

Take a seed, that is a member of $Z_p^x$.

Construct the following sequence of bits: 
$\lbrack MSB (x1)~MSB (x2) ~\dots~MSB(x_i) \rbrack$

\texttt{|| = concatenation}
$x_j =  g^x_{j - 1}$ in $Z_p^x$


Given $(g^x mod p, p, g)$ what is the $MSB(x)$? Is this actually as tough
as trying to find $x$?


\subsection{Theorem: $LSB(x)$ is easy to get}
\subsubsection{Proof}


Fermat's little theorem: $\forall x \in Z_p^x, x^{p - 1} = 1$
\subsubsection{Proof of fermat's little theorem (raw number theory)}
Everything happens in $x^{p - 1}$.

$S = {a, 2a, 3a, ... (p - 1)a}$.
We show that this is a permutation of
$S' = {1, 2, 3, ..., (p - 1) }$

$a \neq 0$. So, $a \cdot x = 0 implies x = 0 since Z_p^x is integral domain$

Suppose two elements are not distinct in $S$. This means that $ai - aj = 0$
Hence, $p | a(i - j)$. But, $a < p$, $(i - j) < p$. Hence, their product 
cannot be divisible by $p$ (product of two numbers less than a prime numbers.


Multiplying all numbers in S should be equal to multiplying all numbers in S'


$a^(p - 1) (p - 1)! =(congruent mod p)= (p - 1)!$
Hence $a^(p - 1) =(congruent mod p)= 1$

\subsubsection{Continuing proof of LSB of discrete log is easy}

$(g^x)^(p - 1) = 1$
$(g^x)^{\frac{p - 1}{2}} = +-1$

When $x$ is even, this will be $+1$. When $x$ is odd, this will be $-1$

In some sense, we are computing $(g^x|p)$ (legendre symbol)

Hence, we can find $LSB(x)$. Note that this will fail if $g^(p - 1)/2 = 1$,
but $g$ is a generator of $Z_p^x$ so it can't happen (g has order |p - 1|).


\subsection{Can we not use this to "peel bits" off? We can peel more than just LSB}

If $4 | p - 1$, then we can reapply the same method to get *two* bits.


$g^{x^{\frac{p - 1}{4}}}$ =

$1. if x == 0 mod 4 -> 1$
$2. if x == 1 mod 4 -> ?$
$3. if x == 2 mod 4 -> ?$
$4. if x == 3 mod 4 -> ?$

\subsection{Hardness given ability to get MSB}

Assume there is an algorithm to find $MSB(x)$ given $y = g^x$ (everything in $Z_p$)

We want to find $sqrt(y)$. That is, it finds $z$ such that $z^2 = y$. Suppose 
sqrt(y) \textit{does exist}.

Note: algorithm to find roots of polynomial in a FF efficiently (?) Look this up. If we 
have this, we can nuke this problem.

SQRT-WHEN-SQRT-EXISTS(y):
Compute $a = y^\frac{p + 1}{4}$.
$a^2 = y^\frac{p + 1}{4})^2 = y^(\frac{p + 1}{2})$.

We know that $y$ is a quadratic residue, so $y = g^2k$
So, $a^2 = (g^2k^\frac{p + 1}{2}) = g^(k * \frac{p + 1})$
<Lost, do the arithmetic yourself>.


\texttt{x -> x/2} if x is even
\texttt{x -> x - 1} if x is odd.

If we have the trace of the function fixpoint (0), then we can reconstruct x.

Going to $x/2$ is difficult because we have two square roots in $Z_p^x$.

\subsubsection{Brilliant: }
If we have an MSB algorithm, then this step can be *made unique*.
If the number is between $0..(\frac{p-1}{2})$, then MSB = 0. Otherwise,
if it is in the other portion, $MSB = 1$. So, we can use $MSB$ because
we know that the sqrt will be $g^\frac{p - 1}{2} = -1$ multiplicative
factor away from each other (the roots of x are $c+-k$)

So, given MSB, we can find discrete log.
Therefore, MSB is just as hard as discrete log, because:

discrete log == MSB algorithm + LSB algorithm (WTF)


\subsection{One way function / Permutation}

$F: \{0, 1\}^n \to \{0, 1\}^n$

There exists PPTM such that $P \lbrack M(x) == F(x) \rbrack = 1 - negligible$.


For all PPTM A, forall x chosen at random from domain(f), 
$P[A(f(x)) \in f^{-1}(f(x))] = negligible$

\subsection{Hard-core predicate of a one-way function $f$}

$H:{0, 1}^n -> {0, 1}$ is a hard core predicate of $f$ if 

1. x -> h(x) is easy,


$\forall PPTM A, \forall random x in dom(f), P[A(f(x)) = h(x)] = negligible + \frac{1}{2}$

As in, should be negligible from random guess (since range is ${0, 1}$).



\subsection{Convert one-way function to PNG}

PNG(s) $h(s1) || h(s2) || h(s3) ... || h(s_n)$

$|| = concatenation$

$s_i = f(s_{i - 1})$
$s_0 = s$


\subsection{General construction of hard-core predicates}

For a one-way function f, the XOR of a random subset of bits will be a hardcore
prediate.

Let $I$ be the index set, $I \subset \lbrack 1 \dots n \rbrack$.
$H(x) = \texttt{XOR} x_i, i \in I$ will be a hardcore predicate.


\subsection{Exercise}
if $p = s.2^r, maximum r$, LSB is $0$th bit, $r$th bit is a hardcore predicate.
