\section{Information Theory - Lecture 4}

\begin{itemize}
    \item Two famous relaxations
    \item Modern approximations
    \item Modern definition of Security
\end{itemize}

\subsection{Shannon's perfect secrecy}
Perfect secrecy assumes secrecy required for infinite time. We can relax 
this to be some large number.

However, if we accept this, then the scheme \textit{will be breakable} by brute force
since our key space is finite. This naturally forces another relaxation:

We allow a small error term of probability in terms of failure of secrecy.



\subsection{Representation Change}

Pick two representations of the same information. Converting from one (say A)
to the other (say B) is easy, but converting back from B to A is hard.

\subsubsection{$P \neq NP \implies$ existence of trapdoor function}

Verifying an NP complete problem will be in $P$. Computing it will be $NP$.


If we have a certificate, then a non-deterministic turing machine can solve in
polynomial time by guessing the certificate.

Other direction, if there is a non-deterministic turing machine that can solve
in polynomial time implies a certificate, because the "path" in the
non-deterministic TM will be the polynomial time certificate vertification.

\subsubsection{Using the relaxed definition}

We have a secure channel and an insecure channel, how do we improve 
bandwidth?

the field divided into two:
- Assume we have a slow secure channel and a fast insecure channel, how do I create
a fast secure channel? (Slow secure + fast insecure =? Fast Secure) / Private key crypto.

- (No secure + Slow insecure =? Slow secure) / public key crypto.

\subsection{Formalization}

If the probability that any adversary can win the game is $\frac{1}{2}$.

$$\forall adversary, P \lbrack b' = b \rbrack = \frac{1}{2}$$


For all probabilistic polynomial time turing machines A, if A interacts with a protocol in the game,
the prob. that the output of the game will be b, is bounded by $1 / 2 + \mu$, where $\mu$
is negligible, then the game is secure.


\subsubsection{negligible}
A function $\mu$ is said to be negligible if:


$\forall p \in polynomials, \exists n_0, \forall n \geq n_0, \mu(n) \leq \frac{1}{p(n)}$.

This is equivalent to saying that $\mu \leq 2^k$ because $2^k$ will always
outgrow any polynomial $p$.

If the adversary has some non-negligible chance of doing better than
$1 / 2$, then he can repeatedly reapply the strategy some polynomial number
of times to "blow up" the advantage (see: randomized algorithms).

How close to $1$ we can get by re-running is how away from $\frac{1}{2}$ we are.

Roughly, by repeating $M$ times, we can push it to $\frac{1}{2} + M \mu(n)$.


However, if a function is negligible, polynomial times multiplication with
negligible will continue to be negligible. It's some weird ideal in $R[X]$? 


\subsection{The game to be played}

1. Fix message space with all messages of equal length.
2. The adversary chooses two message of his choice, M0 and M1.
3. We encrypt one of the messages, call it C
4. The adversary has to find out which one (C = encrypt (M0) or C = encrypt(M1))

The adversary wins if  the adversary does not satisfy the security criteria.
