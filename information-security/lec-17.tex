\chapter{Quantum Secret Key Establishment}

\section{Three Polarizers Experiment}

Photon is a single Qubit system: Vector in $\mathbb{C}^2$

$|\psi> = a |0> + b|1>, a, b \in C$

Normalized, such that $|a|^2 + |b|^2 = 1$.


Postulates of QM:
\begin{itemize}
    \item For any system, there is a state vector in Hilbert space.
    \item Measurement postulate: If we measure a qubit, then with probablibity of $|a|^2$, we get $|0>$, with probablibity of $|b|^2$, we get $|1>$.
        After measurement, things collapse to one of the measured values
    \item  Evolution postulate: Will evolve according to schrodinger / unitary transformation
\end{itemize}


\section Quantum Key exchange

$A$, $B$. Have two channels, classical and quantum.
$E_{(ve)}$ is eavesdropping on both channels.


\begin{align*}
\ket{+} = \frac{\ket{0} + \ket{1}} {\sqrt(2)}
\ket{-} = \frac{\ket{0} - \ket{1}} {\sqrt(2)}
\end{align*}

Step 1: $A$ chooses to encode $0$ as either $\ket{0}$ or as $\ket{+}$
$A$ chooses to encode $1$ as either $\ket{0}$ or as $\ket{-}$


Consider a stream of random bits: $r_0, r_1, \dots, r_n$, $s_0, s_1, \dots s_n$

$r_i$ is the value we encode. $s_i$ dictates how we encode $r_i$.

$r_i = 0, s_i = 0$, then we sent $\ket{0}$.
$r_i = 0, s_i = 1$, then we sent $\ket{+}$.
Et cetra.


$A$ encodes $r_i$ with respect to $s_i$ and sends the qubits to bob.
So, $n$ qubits have been sent. All the qubits can be eavesdropped by $E$.



Step 2: $B$ receives these qubits and measures them in one of the two bases
\textit{at random}. (either 01 or plus-minus). $B$ records the answers.

$B$'s choice of basis will be governed by random bits $s_1', s_2', \dots s_n'$
Let the answers on measuring in the $s_i$ basis be $r_1', r_2', \dots r_n'$. 


Note that if $s_i = s_i'$, then $r_i = r_i'$. If $s_i \neq s_i'$, then
$r_i = r_i'$ with $0.5$ probability.



Step 3: $A$ and $B$ publish the $s_i$, $s_i'$. (publish measurement basis info)
Ignore all indeces where $s_i \neq s_i'$. So, from now on, we will consider
the index set $I_{eq} = { i \in [1\dots n] \vert  s_i = s_i' }$

Analyzing Eve: Eve has a $s_i''$ that is used to measure the channel.
When $s_i = s_i'$, but $s_i'' \neq s_i$ (that is, Eve is measuring using a 
different basis from $A$ and $B$).

Assume WLOG that $A$ and $B$ are using the 01 basis. Now, if Eve measures
using +- basis, then the original $\ket{0}$ or $\ket{1}$ will now become
$\ket{+}$ or $\ket{-}$.

So now, when Bob measures, he will only \textit{get the original with 0.5 probability!}

We can detect the presence of someone who is intercepting the channel
with this principle.


So, there will be ~25 percent chance that $s_i \neq s_i'$ when $r_i = r_i'$, if
Eve is eavesdropping. If not, then $s_i \neq s_i' \implies r_i \neq r_i'$.

Step 4: Randomly sample some subset of the $r_j$ for those indeces with $s_j = s_j'$.
If eavesdropper did not exist, then all the $r_j$  will match. If eavesdropper
exists, then some values will be mismatched.

This will let us \textit{detect} the presence of an eavesdropper.

So now, if the eavesdropper is not there, the rest of the
\textit{unpublished $r_k$'s} where $s_k = s_k'$ become
our secret key. This is shared, since we know that this will not
be corrupted, due to the lack of an eavesdropped. We also know that the
values will be equal since we use the measurement basis ($s_k = s_k'$)


\subsection{Why can't eve keep a copy of the qubits?}
In theory, Eve could have tried to keep a copy of the qubits, and then measured
once the $s_i$ have been published. However, no-cloning prevents this from happening.

\section{Dealing with noise}

On a noisy quantum channel, we will need to deal with the case that possibly
$s_i = s_i'$, but $r_i \neq r_i'$.


\section{Dealing with noise}

On a noisy quantum channel, we will need to deal with the case that possibly
$s_i = s_i'$, but $r_i \neq r_i'$. We can use classical error correction on
the $r_i$'s to deal with this stuff. We do not go this analysis in the lecture.


\section{Two qubit systems}

four classical possible outcomes: $\ket{00}$, $\ket{01}$, $\ket{10}$, $\ket{11}$.
Superposition of all four is allowed.

$\ket{\psi} = a_{00} \ket{00} + a_{01} \ket{01} + a_{10} \ket{10} + a_{11} \ket{11}$.

\section{$n$ qubit system}
$n$ qubit system will have $2^n$ classical outcomes - we will need $2^n$ complex amplitudes.
On the other hand, for am $n$ bit system, we will need, well, $n$ bits.

\section{Why quantum computer?}

Note that a quantum computer runs $2^n$ computations in parallel, samples one of
them (by nature's choice), and deletes all $2^n - 1$ values.

the argument is that, for a classical computer, even \textit{deleting}
$2^n - 1$ values will take exponential time. However, we can at least
exploit this "deleting" property.


\end{document}
