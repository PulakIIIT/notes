\documentclass[11pt]{book}
\usepackage[sc,osf]{mathpazo}   % With old-style figures and real smallcaps.
\linespread{1.025}              % Palatino leads a little more leading
\usepackage[euler-digits,small]{eulervm}
%\documentclass[10pt]{llncs}
%\usepackage{llncsdoc}
\usepackage{amsmath,amssymb}
\usepackage{graphicx}
\usepackage{physics}
\usepackage{makeidx}
\usepackage{algpseudocode}
\usepackage{algorithm}
\usepackage{listing}
\evensidemargin=0.20in
\oddsidemargin=0.20in
\topmargin=0.2in
%\headheight=0.0in
%\headsep=0.0in
%\setlength{\parskip}{0mm}
%\setlength{\parindent}{4mm}
\setlength{\textwidth}{6.4in}
\setlength{\textheight}{8.5in}
%\leftmargin -2in
%\setlength{\rightmargin}{-2in}
%\usepackage{epsf}
%\usepackage{url}

\usepackage{booktabs}   %% For formal tables:
                        %% http://ctan.org/pkg/booktabs
\usepackage{subcaption} %% For complex figures with subfigures/subcaptions
                        %% http://ctan.org/pkg/subcaption
\usepackage{enumitem}
\usepackage{minted}
\newminted{fortran}{fontsize=\footnotesize}

\usepackage{xargs}
\usepackage[colorinlistoftodos,prependcaption,textsize=tiny]{todonotes}

\usepackage{hyperref}
\hypersetup{
    colorlinks,
    citecolor=black,
    filecolor=black,
    linkcolor=black,
    urlcolor=black
}

\usepackage{epsfig}
\usepackage{tabularx}
\usepackage{latexsym}
\newtheorem{lemma}{Lemma}
\newtheorem{observation}{Observation}
\newtheorem{proof}{Proof}
\newcommand\ddfrac[2]{\frac{\displaystyle #1}{\displaystyle #2}}

\def\qed{$\Box$}
\def\proof{\textit{Proof. }}
\newtheorem{corollary}{Corollary}
\newtheorem{theorem}{Theorem}
\newtheorem{definition}{Definition}


\newcommand{\textbb}[1]{$\mathbb{#1}$}
\DeclareMathOperator{\negl}{negl}
\DeclareMathOperator{\rand}{rand}
\newcommand{\binary}{\{0, 1\}}
\newcommand{\zeroone}{\binary}
\newcommand{\gen}{\texttt{Gen}}
\newcommand{\enc}{\texttt{Enc}}
\newcommand{\dec}{\texttt{Dec}}
\newcommand{\N}{\ensuremath{\mathbb{N}}}
\newcommand{\Z}{\ensuremath{\mathbb{Z}}}
\newcommand{\xor}{\oplus}


\newcommand{\pr}[1]{Pr\left[ #1 \right]}

\newcommand{\privkeysecurity}[1]{\ensuremath{PrivK_{A, \Pi}^{\texttt{#1}}}}

\title{Principle of Information \& Security}
\author{Siddharth Bhat}
\date{}

\begin{document}

\maketitle
\tableofcontents

% http://mirrors.ibiblio.org/CTAN/macros/latex/contrib/physics/physics.pdf
\newcommand{\qdot}{{\dot q}}

\chapter{Lagrangian, Hamiltonian mechanics}

Mechanics in terms of generalized coords.
\section{Lagrangian}
Define a functional. $L$ over the config. space of partibles $q^i$, $qdot^i$.
$L = L(q^i, qdot^i)$.  We have an explicit dependence on $t$.



$L = KE - PE$

Assuming a 1-particle system of unit mass,
$$L = \frac{1}{2} \qdot^2 - V(q)$$

Assuming an n-particle system of unit mass,
$$L = \sum_i \frac{1}{2} {qdot^i}^2 - V(q^i)$$ 

\section{Variational principle}

Take a minimum path from $A$ to $B$. Now notice that the path that is
slightly different from this path will have some delta from the minimum.

Action
$$S(t0, t1) = \int L \dd t = \int_{t0}^{t1} L(q^i, qdot^i) \dd t$$.
Least action: $\delta S = 0$

% \begin{align*}
%     \delta S &= \delta \int L(q^i, qdot^i) \dd{t} \\
%              &= \int \delta L(q^i, qdot^i) \dd{t} \\
%              &= \int \pdv{L}{q^i} \delta q^i + \pdv{L}{qdot^i} \delta qdot^i \dd{t} \\
% \begin{align*}




\section{Operators}

\subsection{Projectors --- $P$}

Suppose $W$ is a $k$-dimensional vector subspace of the $d$-dimensional 
vector space $V$. 

Using Gram-Schmidt, it is possible to construct an orthonormal basis
$\ket{1}, \ket{2}, \dots \ket{d}$ for $V$ such that $\ket{1} \dots \ket{k}$
is an orthonormal basis for $W$. Then the projector $P$ is defined as:
\begin{align*}
    P_W \equiv \sum_{i=1}^k \ketbra{i}
\end{align*}

\begin{itemize}
\item $P^\dagger = P$ (Immediate from writing in $\ket{i}$ basis)
\item $P^2 = P$ (Immediate from writing in $\ket{i}$ basis)
\end{itemize}

$Q = I - P$ is the projector onto orthogonal complement of the subspace that $P$.
projects into. This projects onto the $\ket{k+1} \dots \ket{d}$ basis.

\subsection{Normal operator}
\begin{align*} A A^\dagger = A^\dagger A \end{align*}


\begin{theorem}
Spectral theorem for normal operators:
Any normal operator $M$ on a vector space $V$ is diagonal with respect to some
orthonormal basis for $V$.
\end{theorem}
\begin{proof}
Let $\lambda$ be an eigenvalue of $M$. $P_\lambda$ is the projector onto
$\lambda$'s eigenvector. $Q_\lambda = P_\lambda^\bot$ is the orthogonal complement projector
of $P$.

We first establish a fact about $P M Q$:
\begin{align*}
&M M^\dagger \ket \lambda = M^\dagger (M \ket \lambda) = \lambda M^\dagger \lambda \\
&\text{Hence, $M^\dagger v \in P$.} \\
&Q (M^\dagger P) = 0 \implies (P M Q)^\dagger = 0 \implies P M Q = 0
\end{align*}

Next, we prove some properties of $QM$ and $QM^\dagger$
\begin{align*}
QM = QM(P + Q) = QMP + QMQ = QMQ \\
QM^\dagger = QM^\dagger(P + Q) = QM^\dagger P + QM^\dagger Q = (PMQ)^\dagger + QM^\dagger Q
\end{align*}

\begin{align*}
&\text{QMQ is normal:} \\
&(QMQ)^\dagger(QMQ) = Q^\dagger M^\dagger Q^\dagger Q M Q = Q M^\dagger Q M Q = Q M^\dagger M Q \\
&(QMQ)(QMQ)^\dagger = (Q M Q) (Q^\dagger M^\dagger Q^\dagger) = Q M Q M^\dagger Q = 
Q M M^\dagger Q = Q M^\dagger M Q = (QMQ)^\dagger QMQ
\end{align*}

\begin{align*}
&M = (P + Q) M (P + Q) \\
&M = P M P + P M Q + Q M P + Q M Q \\
&M = P M P + Q M Q \\
&M = \lambda_i \ketbra{i} + Q M Q \\
&\text{Since $Q M Q$ is normal, and we are performing induction on dimension, and $P \bot Q$,} \\
&M = \lambda_i \ketbra{i} + \sum_k \lambda_k \ketbra{k} \\
&\text{Hence M is normal}
\end{align*}
\end{proof}

\begin{theorem}
Any diagonalizable operator is normal
\end{theorem}
\begin{proof}
Let $M$ be diagonal with respect to basis $\ket{i}$.
Then, $M \equiv \sum_i \lambda_i \ketbra{i}$.
Now, $M^\dagger= \sum_i \lambda_i^* \ketbra{i}$. 
\begin{align*}
&M M^\dagger = \bigg(\sum_i \lambda_i \ketbra{i}\bigg)
    \bigg(\sum_j \lambda_j^* \ketbra{j}\bigg) \\
&M M^\dagger = \sum_i \lambda_i^* \lambda_i \ketbra{i} \\
&\text{Similarly,}  \quad M^\dagger M = (\sum_i \lambda_i^* \lambda_i \ketbra{i}) 
\end{align*}
\end{proof}

\subsection{Unitary operator}
\[ U U^\dagger = U^\dagger U = I \]
\begin{itemize}
\item unitary operator is normal.
\item unitary operator preserves inner products.
\begin{align*}
\bra{b'} \ket{a'} = \bra{b} U^\dagger U \ket{a} = \bra{b} I \ket{a}
\end{align*}
\end{itemize}

\subsection{Positive operator}
Special class of Hermitian operator.

\begin{align*}
 \forall v \in V, \bra v A \ket v \geq 0
\end{align*}

If the inner product is strictly greater than zero, then such an operator
is called as \emph{positive definite}. If it is greater than or equal
to zero, it is called \emph{positive semidefinite}.

\begin{theorem}
A positive operator is Hermitian
\end{theorem}
\begin{proof}
\textbf{TODO}. Proof most likely follows real case, where we use
cholesky to write it as $A^T A$ and then show that it is normal. We then
use the fact that its eigenvalues are greater than or equal to zero
to establish that it is Hermitian.
\end{proof}


\chapter{Maxwell's equations in Minkowski space}
% http://www.physics.ucc.ie/apeer/PY4112/Tensors.pdf

Let us first review Maxwell's equations:

\begin{align*}
&\div E = \frac{\rho}{\epsilon_0}~\text{(Electric charges produce fields)}\\
&\div B = 0~\text{(Only magnetic dipoles exist)}\\
&\curl E = - \pdv{B}{t}~\text{(Lenz Law - time varying magnetic field induces current that opposes it)} \\
&\curl B =  \mu_0 \bigg(J + \epsilon_0 \pdv{E}{t} \bigg)~\text{(Ampere's law + fudge factor)}
\end{align*}

\section{Constructing $F$, or Tensorifying Maxwell's equations}

Begin with the equation that $\div B = 0$. This tells that $B$ can be written
as the curl of some other field:

\begin{equation}
    \boxed{B \equiv \curl A}
\end{equation}

Expanding this equation of $B$ in tensorial form:
\begin{equation}
    \boxed{ B^i = \levicevita^{ijk}  \partial_j A^k }
\end{equation}

Next, take $\curl E = - \pdv{B}{t}$.


\begin{align*}
&\curl E = - \pdv{B}{t} = \pdv{(\curl A)}{t} = \curl{\pdv{A}{t}} \\
&\curl (E + \pdv{A}{t}) = 0 \\
&\text{writing this as the divergence of some field $\phi$ scaled by $\alpha : \reals$} \\
&E + \pdv{A}{t} = \alpha \big(\div \phi\big) \\
&E = \alpha \div \phi - \pdv{A}{t}
\end{align*}

Since electrostatics is time-independent, we choose to think of $\alpha = -1$, 
so we can interpret $\phi$ as the potential.

\begin{equation}
     E^i = - \pdv{\phi}{x^k}  g^{ik} - \pdv{A}{t}^i
\end{equation}

A slight reformulation (since we know that in Minkowski space, $\partial_t = \partial_0$)
we get the equation:


\begin{equation}
    \boxed{ E^i = - g^{ik} \partial_k \phi - \partial_0 A^i}
\end{equation}

We get the metric $g^ik$ involved to raise the covariant $\pdv{\phi}{x^k}$
into the contravariant $E^i$.

(\textbf{Sid question:} how does one justify switching $\curl$ and $\partial$? It feels like some algebra)

\textbf{Here be magic!} We define A new rank-$2$ tensor in Minkowski space-time,
called $F$ (for Faraday),

\begin{equation}
    \boxed{F_{\mu \nu} \equiv \partial_\mu A_\nu - \partial_\nu A_\mu}
\end{equation}

(\textbf{Sid question:} why is this object $F_{\mu \nu}$ covariant? What does this \textit{mean}?)

\begin{lemma}
$F_{\mu \nu}$ is antisymmetric.
\end{lemma}

\begin{lemma}
$F_{\mu \nu}$ has 6 degrees of freedom
\end{lemma}
\begin{proof}
Number of degrees of freedom of $F$: 
\begin{align*}
\frac{4^2~\text{(total)} - 4~\text{(diagonal)}}{2~\text{(anti-symmetry)}} = 6
\end{align*}
\end{proof}

Notice that $F$ is a 1-form!

\section{Expressing $B$, $E$ in terms of $F$}
We now wish to re-expresss $B^{ij}$ and $E^{ij}$ in terms of $F$, so that
this $F$ captures all of maxwell's equations.

\begin{align*}
    B^i &= \levicevita^{ijk}  \partial_j A^k = \levicevita^{ikj} \partial_k A^j \tag*{by $k$, $j$ being free variables} \\
    B^i &= \frac{1}{2} \bigg( \levicevita^{ijk} \partial_j A^k + \levicevita^{ikj} \partial_k A^j \bigg) \\
        &\text{Substituting $\partial_j A_k - \partial_k A_j = F_{jk}$, } \\
    B^i &= \frac{1}{2} \levicevita^{ijk} F_{jk}
\end{align*}


So, $B$ in terms of $F$ is:
\begin{equation}
    \boxed{B^i = \frac{1}{2} \levicevita^{ijk} F_{jk}}
\end{equation}

Similarly, we wish to write $E$ in terms of $F$. The algebra is as follows:
\begin{align*}
    E^i &= -g^{ik} \partial_k \phi - \partial_0 A^i \\
    E^i &= -g^{ik} \partial_k \phi - \partial_0 g^{ik} A_k  \tag*{Is this allowed? Am I always allowed to insert the $g_{ik}$?} \\
    E^i &= -g^{ik} (\partial_k \phi + \partial_0 A_k) \\
\end{align*}

Since $k = \{1, 2, 3\}$ ($k$ is spacelike coordinates), and we would like to
relate $\phi$ with $A$ (to unify $E$), we \textbf{set}:

\begin{equation}
    \boxed{A_0 \equiv - \phi}
\end{equation}

Continuing the derivation,



\begin{align*}
    E^i &= -g^{ik} (\partial_k (- A_0) + \partial_0 A_k) \\
    E^i &= -g^{ik} (\partial_0 A_k - \partial_k A_0 ) \\
    E^i &= -g^{ik} F_{0k}
\end{align*}


So, finally, the relation is:

\begin{equation}
    \boxed{E^i = -g^{ik} F_{0k}}
\end{equation}

\textbf{TODO: Find out how $E^i = c F^{i0}$}


\begin{equation}
    \boxed{E^i = c F^{i0}}
\end{equation}

\section{Other ramifications of Maxwell's equations on $F$}

\subsection{Ramification 1}
\subsubsection{First part, using 4th equation}

We next consider the 4th Maxwell equation:

\begin{align*}
    \curl B &= \mu_0 J + \epsilon_0 \mu_0 \pdv{E}{t} \\
    \curl B &= \mu_0 J + \frac{1}{c^2} \pdv{E}{t} \\
            &\text{Converting to indices,}\\
    (\curl B)^i &= \mu_0 J^i + \frac{1}{c} \pdv{E^i}{ct} \tag{From $\partial_{ct} = \frac{1}{c} \partial_t$} \\
                &= \mu_0 J^i + \frac{1}{c} \pdv{E^i}{X^0} \\
                &= \mu_0 J^i + \pdv{F^{i0}}{X^0} \tag{From $E^i = c F^{i0}$} \\
                &= \mu_0 J^i + \partial_0 F^{i0}
\end{align*}

Now, we start to simplify the LHS, $\curl B$:

\begin{align*}
    &(\curl B)^i = \levicevita^{ijk} \partial_j B_k \\
    %
    &\text{Since $B^k = \frac{1}{2} \levicevita^{klm} F_{lm}$,} \\
    %
    &\text{$B_k = \frac{1}{2} \levicevita_{klm} F^{lm}$,} \tag{\textbf{TODO:} this is scam} \\
    %
    &(\curl B)^i = \levicevita^{ijk} \partial_j (\frac{1}{2} \levicevita_{klm} F^{lm}) =
    \frac{1}{2} \levicevita^{ijk} \levicevita_{klm} \partial_j F^{lm}\\
\end{align*}

\textbf{Aside: We need to know how to evaluate $\levicevita^{ijk} \levicevita_{klm}:$}
\begin{align*}
    \levicevita_{i_1, i_2, \dots, i_n} \levicevita_{j_1, j_2, \dots j_n} =  
    \det{
    \begin{vmatrix}
        \delta_{i_1 j_1} & \delta_{i_1 j_2} &\dots &\delta_{i_1 j_n} \\
        \delta_{i_2 j_1} &\delta_{i_2 j_2} &\dots &\delta_{i_2 j_n} \\
        \vdots           &\vdots  & \ddots & \vdots \\
        \delta_{i_n j_1} & \delta_{i_n j_2} & \dots & \delta_{i_n j_n}
\end{vmatrix}}
\end{align*}

Hence, \textbf{TODO: HOW?}
$\levicevita^{ijk} \levicevita^{ilm} = \frac{-1}{2} (\delta^j_i \delta^k_m - \delta^j_m \delta^k_l)$


Plugging both equations together,

\begin{align*}
    \frac{1}{2} \levicevita^{ijk} \levicevita_{klm} \partial_j F^{lm} &=  \mu_0 J^i + \partial_0 F^{i0}  \\
    %
    \frac{1}{2} \big[ 
   \frac{-1}{2} \big(\delta^i_l \delta^j_m - \delta^i_m \delta^j_l\big) \big]
   \partial_j F^{lm} &=  \mu_0 J^i + \partial_0 F^{i0} \\
    %
    \textbf{Something is fucked here with respect to $\partial_m F^{mi}$} \\
    %
    \frac{1}{2} \big[ \frac{-1}{2} \big(
    \partial_m F^{im} - \partial_m F^{mi} \big) \big] &= \mu_0 J^i + \partial_0 F^{i0}   \\
       %
        \textbf{$F$ is anti-symmetric, so rewriting $-\partial_m F^{mi} = \partial_m F^{im}$} \\
       %
    -\frac{1}{2} \big[ \partial_m F^{im} \big] &= \mu_0 J^i + \partial_0 F^{i0}   \\
       %
       \textbf{In the notes, the $\frac{1}{2}$ does not exist} \\
       %
    - \big[ \partial_m F^{im} \big] &= \mu_0 J^i + \partial_0 F^{i0}   \\
       % 
    \mu_0 J^i + \partial_0 F^{i0}  + \partial_m F^{im}  &= 0 \\
        % 
    \mu_0 J^i + \partial_\mu F^{i\mu} &= 0 \tag{$\mu = \{0, 1, 2, 3 \}$}
\end{align*}

This gives us a continuity-style equation, linking the current density $J$ to
the rate of change of $F$.
\begin{equation}
    \boxed{ \mu_0 J^i + \partial_\mu F^{i\mu} = 0 \tag{$\mu = \{0, 1, 2, 3 \}$} }
\end{equation}


\subsubsection{Second part, using 1st equation}

\begin{align*}
    &\grad E = \frac{\rho}{\epsilon_0} \\
    &\partial_i E^i = \frac{\phi}{\epsilon_0} \\
    &\text{Substituting $E^i = c F^{i0}$, } \\
    &c \partial_i F^{i0} = \frac{\rho}{\epsilon_0}  = \frac{\rho \mu_0}{\mu_0 \epsilon_0} = \rho c^2 \\
    &\partial_i F^{i0} = \mu_0 c \rho \\
    \text{Since $F$ is anti-symmetric, $F^{00} = 0$, Hence:}
    &\partial_0 F^{00} + \partial_i F^{i0} = \mu_0 c \rho \\
    &\partial_\mu F^{\mu 0} = \mu_0 c \rho
\end{align*}

\begin{equation}
    \boxed{ \partial_\mu F^{\mu0} = \mu_0 c \rho}
\end{equation}

\subsubsection{Combining part 1 and part 2:}


\begin{align*}
    \mu_0 J^i + \partial_\mu F^{i\mu} = 0 \tag{From $B$}  \\
    \partial_\mu F^{i\mu} = -\mu_0 J^i 
    \partial_\mu F^{\mu 0} = \mu_0 c \rho \\
    \partial_\mu F^{0 \mu} = - \mu_0 c \rho \\
\end{align*}

To combine these equations, \textbf{we set:}
\begin{equation}
    \boxed{J^0 \equiv c \rho}
\end{equation}
We arrive at the unified equation:

\begin{align*}
    \partial_\mu F^{\nu \mu} = - \mu_0 J^{\nu}
\end{align*}

Choose units such that $c = \frac{h}{2 \pi} = G_n = 1$, which gives us:


\begin{align*}
    &\partial_\mu F^{\nu \mu} = -  J^{\nu} \\
    &\textbf{$F$ is antisymmetric, so flipping indices} \\
    &\partial_\mu F^{\mu \nu} =  J^{\nu} \\
\end{align*}

\begin{equation}
    \boxed{ \partial_\mu F^{\mu \nu} =  J^{\nu} }
\end{equation}

Note that this is \textbf{Ampere's law!}

\chapter{Hierarchy Theorems}

$\exists L$, such that $\forall f: \mathbb{N} \to \mathbb{N}$, where $f$ is
space/time constructible,
\begin{align*}
    Space(f) &\supsetneq Space(o(f)) \\
    Time(f) &\supsetneq Time \bigg (\frac{o(f)}{log f} \bigg )
\end{align*}

So, there is a Hierarchy of complexity classes in time and space.

\subsection{Proof sketch}
We exhibit a languag $A$, such that $A \in Space(f(n))$, and $A \notin Space(o(f(n))$.

Let $D$ decide $A$. D's definition:
\begin{itemize}
    \item compute $f(n)$ and mark the end of $f(n)$ cells. If the read-write head
        ever crosses it, \texttt{REJECT}, \texttt{HALT}. We first need
        $f(n)$ to use $f(n)$ cells or less to compute. This is called as 
        \textbf{space-constructribility}. ($f: \mathbb{N} \to \mathbb{N}$ is
        space-constructible iff given $n$, $\exists$ TM which computes $f(n)$ 
        using at most $f(n)$ cells). Also, we want $f(n)$ to be at least
        $log(n)$. Clearly, this process is in space $f(n)$.

    \item We now need to "seaparate" A from the smaller classes. If A can be
        solved in a smaller space (ie, we cannot separate $A$), then
        there must be a TM (say, $D'$) which decides A in space less than $f(n)$.
        So now, we need to choose some input such that $D'$ is different from $D$.
        We can use diagonalization to construct such a function.

    \item let the input be $x$. Let $x = M10*$ for some TM $M$. if not,
        \texttt{REJECT}, \texttt{HALT}.

    \item Let $D$ simulate $M$ on input $M$. If $M$ takes less that $f(n)$
        time to run on $M$, then $M$ can decide A in time less than $f(n)$.
        So now, $D$ knows how much space $M(\langle M \rangle)$ requires. if $M(\langle M \rangle)$
        accepts, we reject.  If $M(\langle M \rangle)$ rejects, we accept (diagonalization).

    \item To find out whether $M(\langle M \rangle)$ rejects, note that it is space-bounded, so
        we can just check how many states of the configuration space it visits.
        If it has not halted after visiting all states in the configuration
        space, we can conclude that $M(\langle M \rangle)$ does not halt. The configuration space
        is $O(2^{f(n)})$. So we need to run $D$ for time $O(2^f(n))$, and then
        \texttt{REJECT} if it continues running.
\end{itemize}

\textbf{Arjun Q:} Are there examples of non-space constructible functions, which are non-trivial?
Other than ones that are too-small?


Proofs of time are similar to the space separation theorem.
\begin{itemize}
    \item compute $t(n)$ ($t(n)$ should be time-constructible).
        decrement a counter initialized to $t(n)$. if this hits $0$, 
        \texttt{REJECT, HALT}. We get a $log$ factor due to the slowdown
        of keeping time. (People are trying to speed this up).

    \item once again, repeat the same construction used for $SPACE$.
\end{itemize}

\section{Savitch's Theorem:  $\texttt{NSPACE($f(n)$)} \subseteq \texttt{PSPACE($f(n)^2$)}$}
$NSPACE(f(n))$ -- one branch of a NTM $N$ decides $L$ in space $O(f(n)$.

Configuration space is $O(\texttt{alphabet}^{f(n)}) = O(2^f(n))$ -- otherwise,
configurations are repeated.

Our branch depth is exponential in $f(n)$. So, we need to keep track of
$O(2^{f(n)})$ data.

Given $\langle C_1 \in Config(N), C_2 \in Config(N), t \in \mathbb{N} \rangle$
if we can find whether $C_1$ goes to $C_2$ in $t$ space, then we can solve
our original problem.

This can be solved by recursion by asking if there exists a $C_{mid}$, such that
$C_1 \to C_{mid}$ in $t/2$ steps, similarly $C_{mid} to C_2$ in $t/2$ steps.

% F(C1, C2, t)
% for all config C{mid}:
% a1 = F(C1, Cmid, t/2)
% a2 = F(Cmid, C2, t/2)
% accept if both accept

\section{Cook Levin theorem}
$L$ is \texttt{NP-complete}, if 

\begin{itemize}
    \item $L \in NP$
    \item $\forall L' \in NP$, there exists a Karp reduction from $L'$ to $L$: $L' \leq_p L$ (\texttt{NP-hard})
\end{itemize}

$A \leq_p B$ if there exists a poly time computable function $f: \Sigma^* \to
\Sigma^*$ such that 

$$w \in A \Leftrightarrow f(a) \in B$$
Karp reduction = poly time mapping reduction.

Define \texttt{SAT}, and show that \texttt{SAT} is \texttt{NP-complete}.


We have boolean formulas $\phi$, which is given in CNF.

$$\texttt{SAT} = \{ \phi~\vert~\phi~\text{is in CNF (product of sums),}~\phi~\text{is satisfiable} \}$$.

This is clearly decidable since we can try all possible assignments.

It is in NP since a NDTM can try to guess assignments.

To show that this is NP-complete, take any language $L'$ in \texttt{NP}. We provide
a karp reduction to \texttt{SAT}. We take the poly-time checker for $L'$ into a
SAT problem $\psi$, such that \textbf{iff} a solution for $\psi$ exists, then the poly
time checker will accept the string, and vice versa (for reject).



\section{\texttt{EXPSPACE} completeness - $EQ_{REG\uparrow}$}
$r \uparrow \equiv \exists k \in \mathbb{N}. r \uparrow$ is regular if $r$ is
regular.  We need this operator to control input size.

Question: Check if two regular expressions are the same -- We show that
this $\notin \texttt{PSPACE}$, and hence $\notin \texttt{PTIME}$. We show that
this problem is $\texttt{EXPSPACE}$ complete.

\section{Information Theory, CPA security - Lecture 5}


Most theorems will read as: if X is true, then the protocol $\Pi$ is secure.

\section{Our first example of circumventing an impossibility}

\section{PRNGs - Pseudo random number generators}
This allows us to break $|K| \geq |M|$.
This is still a one-time pad, but it allows us to create $|K| << |M|$.


Deterministic program $G$.  Takes as input $n$-bit string, returns $l(n)$ bit string. We have two
assumptions.
\begin{itemize}
\item 1. $l(n) > n$. Expansion.
\item 2. Pseudorandomess.
\end{itemize}

\subsubsection{Pseudorandomness}
for all PPTM(probabilistic polynomial turing machine) $D$,
$$ |P[D(r) = 1] - P[D(G(s)) = 1]| \leq \negl(|s|) \quad  r \leftarrow \binary^{l(n)} \quad  s \leftarrow \binary^n$$
$r, s$ are chosen uniformly at random. The probability distribution is over the
random coins used by $D$, along with the uniform distributions of $r$ and $s$.


\begin{itemize}
\item Strings of length $l(n)$. pick one at random. probability of picking one of them is
\item Strings of length $n$, and then we inject into $l(n)$ with $G$. Clearly, $|Im(G)| < 2^{l(n)}|$.
So, we can sample all $|Im(G)|$. If we are in a pseudo-random world, it will repeat for sure (with $P = 1$).
If we are in the non-PRNG world (true randomness), the chance that something repeats will be negligibly small.
\item We cannot distinguish with polynomial samples, however. So, PPTM is a good choice for a distinguisher.
\end{itemize}


Given that we have to assume PRNGs exist, there are different ways to proceed:
\begin{itemize}
\item Heuristics - Assume that the PRNG we write is a true PRNG, and then get to work.
\item Specific mathematical assumptions - Assume that certain problems are hard. Build PRNGs from this mathematical assumption.
\item Provable Security - If there exists even one hard problem $P$, then we can use that to build a PRNG.
\item Proven security - prove PRNGs exist.
\end{itemize}


\subsubsection{Assume PRNGs exist. We will build a secure encryption scheme}

% \begin{align*}
%     &M = \binary^{l(m)} \quad K = \binary^m \\
%     &Gen: k = {0, 1}^n$ \\
%     &Enc_k(m) = m \xor G(k) \quad Dec_k(c) = c \xor G(k)
% \end{align*}


Note that this is just one time. If they attacker can see two ciphertexts, they can XOR the ciphertexts to get the XOR of the cleartexts.

\paragraph{Proof that this is sane}.
If the adversary can differentiate between $M_0$ $M_1$, we will use it to break the PRNG (as in, distinguish between PRNG and RNG).
Call the adversary A. It can generate 2 messages $M_0$ and $M_1$. When given $encryption(M_b) = G(k) xor M_b$, he can guess $b = 0 \/ b = 1$ with non-negligible probability.
Call the distinguisher $D$. $D$ has to distinguish between truly random and pseudo random world for our proof.
Given a string $w$ and ask if $w$ can be distinguished by $A$. We can pick $w$ from the PRNG world or the RNG world.

If $A(w xor M0, w xor M1)$ gives us the  correct value(can distinguish), then we are using the PRNG. Otherwise, it is the RNG.


\subsubsection{CPA secure}

Adversary gets to pick $M_0$ and $M_1$, we choose a bit $b$ at random and give encryption of $M_b$. He has an oracle that has oracle access to the encryption algorithm. Even with this, he should not be able to guess $b$.

\paragraph{No determistic algorithm can be CPA secure}

The adversary will ask for encryption of $M_0$ and encryption of $M_1$. He gets back $C_0$ and $C_1$. Then, we can compare that to our result, and find the random bit $b$.


\paragraph{How to create CPA secure}
$C = <R, c xor enc(R)>$ where R is a random string.
Decryption will never fail. if we know R, we can xor twice.
Encryption will not fail because encryption of random data is still random.

We have a problem of length doubling: For one length of data, we need R as well.

\subsubsection{Indexable PRNGS}
A PRNG that we can index at a point, and it will start generating from that index. They are called ``pseudorandom functions''.

Consider $Z/pZ^x$. All numbers except $1$ in $Z/pZ$ are generators.

Discrete log: Given $g^x mod p$, given $g$, given $p$, find $x$. (log in a group). We know that Discrete log is hard. Let us try and build a PRNG.


Step 1. Given a PRNG that expands 1 bit, we can use it to create a PRNG that expands any number of bits $n$
$s = seed$. 
$G(s), G(G(s)), G(G(G(s))), G^n(s)$, take the extra bits from each $G^i(s)$. This is a PRNG.

This is a PRNG.

Assume we can break this PRNG. $s_1 s_2 ... s_n$ = stuff from PRNG is distinguishable from $r_1 r_2 r_3 ... r_n$ = Random info.

Construct $s_0 s_1 s_2 .. s_n$, $r_0 s_1 s_2.. s_n$, $r_0 r_1 s_2 s_3 ... s_n$. $r_0 r_1 r_2 r_3 .. r_n$. We know that we can
distinguish first from last. Hence, there must be an adjacent set of strings that can be distinguished, since ``distinguishable'' is transitive (why?)
so, if $r_i  dist r_{i+2}$, we need to have either $r_i dist r_{i + 1}$ or $r_{i + 1} dist r_{i + 2}$. However, between these strings, we have only edited $s_i$.
So, we are able to distinguish one bit extra. This means we can actually distinguish the output of $G$.


Step 2. if we can find $MSB(x)$, we can find x in polynomial time. So, all we need to do is to break $MSB(x)$.


Step 3. Create PRNG that produces one bit output using discrete log.

Take seed s. output $MSB(s_1 = g^s mod p)$. So we now have a PRNG that can create one bit. Second output: $MSB(s_2 = g^{s_1} mod p )$
Third output: $MSB(s_3 = g^{s_3} mod p)$.

Hence, if discrete log is hard, we can get a PRNG. 

\subsection{Multi message Indistinguishability experiment}

This is defined for an encryption scheme $(\gen, \enc, \dec)$.
\begin{itemize}
\item Adversary outputs a pair of vector of messages $(\vec m_0, \vec m_1)$. Each
    vector contains the same number of messages, and the $i$th messages have the
    same length. That is, $|m_0[i]| = |m_1[i]|$.
\item A random key is created: $ k \leftarrow \gen$, and a random bit
    $b \leftarrow \binary$ is chosen. For all $i$, $c[i] \leftarrow \enc_k(m[i])$
    is computed. $\vec c$ is given to the adversary $A$.
\item The adversary $A$ outputs a bit $b'$. The output of the experiment is $1$
    if $b = b'$, and $0$ otherwise.
\end{itemize}

Security definition of the cryptosystem remains unchanged.


\paragraph{Weakness of one time pads under this threat model}
Note that one time pads will fall to this threat model, since repeatedly
ciphering data with a one-time pad will allow us to extract data from the
one-time pad. Indeed, any deterministic scheme can be attacked under
this threat model. So, we now need probabilistic encryption schemes.

\paragraph{Attacking all deterministic cryptosystems under multi message threat model}
Let $m_0 \equiv (0^n, 0^n), m_1 \equiv (0^n, 1^n)$. Run this through the experiment. We will
be given $c \equiv (c_0, c_1)$. If $c_0 = c_1$, then  we know that the message
was $(0^n, 0^n) = m_0$, and is $m_1$ otherwise. We know this since the encryption
function is deterministic, and  hence $\enc_k(m_0) = \enc_k(m_1) \implies m_0 = m_1$.


\chapter{Gaps in space and time}

We wish to study what is not computable given some resource.
If there resource is time, we want to understand what can be solved
in $t(n)$ but not in smaller than $t(n)$ --- in the sense of $o(t(n))$.

We can try to construct a hierarchy of problems that can be solved
given increasing time. 

\begin{align*}
f(n) \in o(g(n)) &\equiv \lim_{n \to \infty} \frac{f(n)}{g(n)} = 0 \\
f(n) \in O(g(n)) &\equiv \lim_{n \to \infty} \frac{f(n)}{g(n)} \in  O(1)
\end{align*}

\section{Space Hierarchy}

A function $f: \mathbb{N} \to \mathbb{N}$ is said to be \textbf{space constructible}
if there exists a turing machine that on input $1^n$, it computes $f(n)$
using space $O(f(n))$. So the output can be $1^{f(n)}$ say, since that uses
space $O(f(n))$.

Most common functions such as polynomials, exponentials, and logarithms
are all space constructible.

\begin{theorem}
Let $f$ be a space-constructible function. There exists a language $L$ which
can be decided in $O(f(n))$ space, but not in $o(f(n))$ space.
\end{theorem}
\begin{proof}
The proof is to \textbf{construct} a language which can be decided on $O(f(n))$
space, but not in $o(f(n))$ space. Such a language tends to be artificial due
to the construction having to work \textit{for all $f$}.

We need two properties for this language $L$ we create:

\begin{itemize}
\item It is \textbf{not decidable} in $o(f(n))$ space.
\item It \textbf{is} decidable in $O(f(n))$ space.
\end{itemize}

We will use diagonalization to show an construct an $L$ that 
\textbf{cannot be decided} in $o(f(n))$ space. List each TM that runs in 
$o(f(n))$ space. This collection of all TMs (viewed as strings) is written as:

$$ALLTM = \cup_{i=0}^\infty \{0, 1\}^i$$


We will define a language $L$ which cannot be decided by \textbf{any} TM
on the above list.

We will create a matrix of the form $DECIDE(i, j) = M_i(\langle M_j \rangle)$.
That is, we feed $M_i$ the string of $M_j$.($\langle M_j \rangle$ interprets
the machine $M_j$ as a string).

Now, create a language $L$:

\begin{align*}
L \equiv \{ M~\vert~M ( \langle M \rangle ) = 0 \}
\end{align*}

Note that $L$ is \textbf{not decidable} in $o(f(n))$ space. Proof by contradtiction:
Assume such a machine $M_c$ ($c$ for contradiction) exists. We now ask if $\langle M_c \rangle \in L$?

\begin{itemize}
\item If $\langle M_c \rangle \in L$, then $M_c (\langle M_c \rangle) = 0$ (by the definiton of $L$).
But since $M_c$ \textbf{decides} $L$,
$M_c (\langle M_c \rangle) = 0 \implies  \langle M_c \rangle \notin L$. \textbf{Contradiction}.

\item On the other hand, say that $\langle M_c \rangle \notin L$, then $M_c (\langle M_c \rangle) = 1$
(by the definition of $L$).
But since $M_c$ \textbf{decides} $L$, $M_c (\langle M_c \rangle) = 1 \implies \langle M_c \rangle \in L$. 
\textbf{Contradiction}.
\end{itemize}


We now move to show that $L$ \textbf{can be decided} in $O(f(n))$ space.
Consider a machine \texttt{INTERPRET} that does this:

\begin{minted}{python}
def INTERPRET(w):
    Mw = convert_to_TM(w)

    # Naive solution: Try to run Mw, see what happens.
    # flag = Mw.run(w)

    # Problem 1: How do we know it runs in o(f(n)) space?
    # flag = Mw.run_with_bounded_space(w, space_bound=f(n))

    # Problem 2: How do we know that Mw halts?
    # Count the size of the config. space, and reject if Mw
    # takes more steps than the configuration space size.
    flag = Mw.run_wth_bounded_space_and_steps(w, space_bound=f(n), 
                                                 steps_bound=Mw.config_space_size())


    return !flag
\end{minted}
\textbf{For more details, read Sipser chapter 9}
\end{proof}



\begin{corollary}
For two functions $f1, f2 : \mathbb{N} \to \mathbb{N}$, if
$f1 \in o(f2)$, then $\dspace~(f1) \subsetneq \dspace~(f2)$. 
(\textbf{Sid note: we do not need the condition that $f1 \neq f2$ thanks to the fact
that in $o(n)$, the limit tends to $0$})
\end{corollary}

\section{Time Hierarchy}

\begin{theorem}
Let $f$ be a time-constructible function. There exists a language $L$ which
can be decided in $O(f(n))$ time, but not in $o \bigg(\frac{f(n)}{\log(f(n)} \bigg)$ time.
\end{theorem}
\begin{proof}
Proof is the same as that of space hierarchy (roughly).

We get the $\log$ factor for us to simulate a $f(n)$ time turing machine.
We do not know how to perform the simulation with constant overhead.
\end{proof}

\begin{corollary} $\texttt{P} \subsetneq \texttt{EXPTIME}$ \end{corollary}


\chapter{Probabilistic encryption}

Determinisim fucks over security. Since now-a-days, servers encrypt pretty much
everything you send them, you can try to mount a chosen plaintext attack.


\section{Truly random functions} 
Look at all functions from r to x. Pick one such function and use that. Number
of such functions: $2^{n^{2^n}}$ 

Number of bits to index this set: $$\log
(2^{n^{2^n}}) = 2^n log(2^n) = n \cdot 2^n$$

\section{Pseudorandom Function (PRF)}
We need distributions on functions. We define this by using keyed functions.

\begin{align*}
F: (k: \binary^n) \to (r : \binary^n) \to (x: \binary^n)
\end{align*}

firstst string is key, second string is what to encode, output is encoded.
In general, we fix a key, and then consider the function $F_k$. We assume
that $F_k$ is efficent. That is, there is a deterministic polynomial time
algorithm that can compute $(F_k(x)~\forall k, x)$.

Intuitively $F$ is called a pseudorandom function if the function $F_k$
for a randomly chosen $k$ is indistinguishable from a random function chosen
from the set of all functions having that domain and range. 

Note that the space of all functions is $(2^n)^{2^n}$, while the space of keys
is just $(2^k)$.


If we have key size as $n \cdot 2^n$, then $F_k$ (the kth function in the set
of all TRFS from r to x) will be truly random.

We formally define them as:
\begin{itemize}
\item Efficiency of computation:  given $x$, computing $f_k(x)$ is easy.
\item Pseudorandomness: for all PPTM A, 

\begin{align*}
|P [ A^{f_k(\cdot)} = 1 ] - P [ A^{f_n(\cdot)} = 1 ]| \leq \negl(n)
\end{align*} 
where $k \leftarrow \binary^k$, is a key that is chosen uniformly at
random from the key space, and $f_n \leftarrow (\binary^n \rightarrow \binary^n)$ 
is chosen uniformly at random from the space of functions.
\end{itemize}

\section{CPA security from pseudorandom functions}
Let $F$ be a pseudorandom function. Define an encryption scheme as follows:
\begin{itemize}
\item $\gen \equiv k \leftarrow \binary^n$. Choose a key at random
\item $\enc(m) \equiv ( r, F_k(r) \xor m)$
\item $\dec((r, c)) \equiv f_k(r) \xor c$.
\end{itemize}

This as seen before is CPA secure, but is problematically length doubling.

\section{Pseudorandom permutations}
A pseudorandom permutation is much like a pseudorandom function, except it
is bijective, and there is a polynomial time algorithm to compute both
$F_k(\cdot)$ and $F_k^{-1}(\cdot)$.

\begin{definition}
    Let $F: \binary^* \times \binary^* \rightarrow \binary^*$ be an efficent
    keyed permutation. We call $F$ a pseudorandom permutation if for all
    PPTM $D$, there exists a negligible function $\negl$ such that:
    \begin{align*}
        \left| \pr{D^{F_k(\cdot), F_k^{-1}(\cdot)} = 1} - 
        \pr{D^{f_n(\cdot), f_n^{-1}(\cdot)} = 1} \right| \leq \negl(n)
    \end{align*}
    Where $k \leftarrow \binary^n$ is chosen uniformly at random, and $f_n$
    is chosen uniformly at random from the set of all permutations of n
    bit strings.
\end{definition}

\section{Modes of operation}
A mode of operation is essentially a way to encrypt arbitrary length
messages using a block cipher. We will see methods that have better
\emph{ciphertext expansion}: The ratio between  the length of the message
and the length of the ciphertext.

\subsection{ECB --- Electronic Code Book} 
Given plaintext $m = \langle m_1 m_2 \dots m_l \rangle$, cyphertext
encrypts each block randomly, usin the PRF: $(c = \langle F_k(m_1), F_k(m_2) \dots F_k(m_l) \rangle)$

This is deterministic, and so cannot be CPA-secure.  It doesn't even have
indistinguishable encryption in the presence of an eavesdropper. This is
because if the same block repeats in the plaintext, it repeats in the
ciphertext.


\subsection{CBC --- Cipher Block Chaining}

Like the name says, chain blocks for messages. we perform $c_k = F_k(m_k \xor c_{k - 1})$. This creates a chain of dependences.

\subsection{Output feedback mode} 
Here, we set $r_n = f_k^n(r_0)$, where $r_0 \equiv IV$. This allows us to
compute $r_n$ without having to have the entire history.

\begin{align*}
    &r_0 = \text{public} \\
    &r_n = f_k(r_{n - 1}) \\
    &c_k = m_k \xor r_k
\end{align*}




\chapter{Probabilistic proofs}
\section{Completeness and Soundness}
\begin{definition}
Completeness: For every true assertion, there is a valid proof.
\end{definition}

\begin{definition}
Soundness: For every false assertion, no valid proof exists.
\end{definition}

A good proof system must also be such that the verifier is efficient
(that is, polynomial time).

If we ask that a proof system must be sound and complete, there is no 
scope for error! Further, it is not clear if the verifier and the
prover can "talk" to each other. If we choose to allow interactions, what
are the implications?


We relax the assumptions this way --- Relaxed compleness states that
for every true assertion, there is a
proof strategy that will convince the verifier with probability 
at least $> \frac{1}{2}$.  
Similarly, relaxed soundness states that for every false assertion,
every proof strategy fails to convinve the verifier with probability
at least $> \frac{2}{3}$. 

The formalization is as follows:
\begin{definition}
Interactive proof systems
\begin{itemize}
\item An interactive proof system for a language $L$ consists of two
entities: a prover $P$ and a verifier $V$.
$P$ and $V$ share common input, and work for $R \in \mathbb{N}$ rounds.

\item In each round, the prover can send the verifier a message that 
is polynomial in the length of the input.

\item The verifier can send a polynomial length reply to the prover.

\item The verifier is a randomized polynomial time turing machine. Time
is measured as a function of the length of the input.

\item \textbf{Completeness}: $\forall x \in L$, there exists a prover strategy
so that the verifier accepts with probability $> \frac{2}{3}$.

\item \textbf{Soundness}: $\forall x \notin L$, any prover strategy will lad
the verifier to accept with probability  $< \frac{1}{3}$.
\end{itemize}
\end{definition}

- Historical ciphers:  (breaking)
+ ceasar and shift
+ Monoalphabetic substitution cipher
+  Vigenere cipher

- 17th century: Kerchoff's Principle: (don't use obscurity)
+ Shannon's pessimistic theorem 
+ One time pad is perfectly secure
+ |M| <= |K| (limitation of perfect security)


- Two relaxations
+ PPTM adversary
+ Negligible p of error
+ $f(n)$ is negligible iff $\forall p \in R[x], \exists N_0, \forall n \geq N_0, f(n) < 1 / p(n)$.
++ examples: $f(n) = \frac{1}{2^n}$. $f(n) = \frac{1}{eps^n}$ where $eps > 0$.


- PRG
- Secure encryption

- CPA
- CPA secure : Adversary has free access to encryption oracle
- So, we need probabilistic encryption to offer CPA security.
- PRF
- CPA secure encryption

- CBC
- IFC
- Random counter mode

- Convert Pseudo random function to pseudo random permutation. If both
  forward and backward are efficient, then it's a block cipher. We did this
  using a "Feistel structure".

$f' :: Z x Z \to Z x Z$
$f' = (x, y) \to (y, (F_k(y)~\texttt{xor}~x))$

This function is invertible.
Each application is a "fiestel round".
Apply this as many times as wanted, at least 4 is recommended.


- 3 DES. 2 keys of 56 bits each.


- CCA secure (chosen ciphertext attack)
  Adversary does not know what the message is. He can actively modify the 
  *ciphertext*.

- Semantic security


- MAC : message authentication code - solves problem of data integrity.

CPA secure + MAC => CCA secure.

c -> cpa secure(c) + mac (c)


What is information?

- Shannon,  Kolmogrov, Lenin
- Randomness, Space, Time.

\chapter{Design models of parallel algorithms}

\section{Partitioning}
This is similar to divide-and-conquer, but we don't need to \textit{combine}
solutions! We can treat problems independently and solve it in parallel.
Examples are parallel merging and searching.

We generate subproblems that are independent of each other.
Example is quicksort. Once we partition the array into two subarrays,
we sort the subarrays recursively.

\subsection{Merging in parallel by partitioning}
Two sorted arrays $A$ and $B$ are to be merged into an array $C$.


We define a function $Rank(x_0, X) = |\{ x < x_0~\vert~x \in X \}|$. Note
that the position of $x_0$ in $sorted(X)$ is equal to $Rank(x_0, X)$.
\textbf{Claim:} $Rank(x, C) = Rank(x, A) + Rank(x, B)$.


For $x \in A$, $Rank(x, A)$ is immediately available (since $A$ is sorted).
We need to find $Rank(x, B)$, but we can find this using binary search through $B$.


Time for each binary search is $O(\log n)$. Total time for merging is
$O(\log n)$, since we are doing each binary search in parallel --- we just need
to read the array $B$, no need to update. The total work is $O(n \log n)$, since
we are performing $O(\log n)$ work for $n$ elements.


Note that this is \textbf{non optimal}. The sequential algorithm has
a time complexity of $O(n)$.


We are going to try and reduce the work to $O(n)$. 

\subsubsection{Merging, take 2: optimal work}
General technique is to solve a smaller problem in parallel, and then
extend the solution to the entire problem!

\begin{itemize}
    \item The problem size to be solved is guided by the factor of non-optimality
        in the current algorithm. We need to reduce the total work to $O(n)$.

        For input size $n$, we do $O(n \log n)$ work. So, for input size $n / \log n$,
        we do $O(n / \log n \times \log (n / \log n) \sim O(n) + O(\log(\log(n)) \sim O(n)$.

    \item We pick every $\log n$th element of $A$. We merge the selected elements
        of $A$ and $B$. However, we still perform binary search on the entireity of $B$.

    \item Pick elements $A[\log n], A[2 \log n], \dots, A[ n - \log n], A[n]$, and
        rank the, in $B$ (ie, find their corresponding positions in $B$.)

    \item Define $[B_{r(i)}, \dots, B_{r(i + 1)}] \equiv \text{portion of $B$ between $A[r \log i]$, $A[(r + 1) \log i]$ in $B$}$.

        \begin{minted}{py}
        A = (5) 6 9 12 (15) 18 19 (21) 23 26
        B = 1 4 (..5..) 7 8 10 11 12 (..15..) 16 17 20 (..21..) 22

        In the output array, we can merge
        the array of B between the (..) elements of A
        \end{minted}

        The problem is that the size of $\log n$ per chunk in $A$ does not mean
        that the size is $\log n$ in $B$.


        \begin{minted}{py}
        A = (5) 6 9 12 (15) ... (...) ...
        B = 6 6 6 6 6 6 6 ... 6

        In this case, the entireity of B is between [5, 15]
        \end{minted}

        So, if we can somehow control the size of $B$, so, we can perform
        binary search in $O(\log n)$, with $n / \log n$ processors.

        We then need to perform the merge with $O(\log n)$, 
        \textbf{under certain conditions}.  There are again $n / \log n$
        such merges.


        The work is $O(n)$.


        So now, the only thing we need to control is the size of partitions
        of $B$.


    \item If $[B_{r(i)}, \dots, B_{r(i + 1)}]$ is too large, then we can
        pick $\log n$ items of this section, and we can rank them in $A$!
        Each piece in $A$ will be smaller than $\log n$, since the partition
        of $A$ was already $\log n$.

    \item we can merge two sorted arrays of size $n$ in time $O(\log n)$
        with work $O(n)$.  This algorithm works in \texttt{CREW}.
        We can improve this  further, we will see this later.
\end{itemize}

\subsection{Searching faster}

Each binary search takes $O(\log n)$ time, and we have $O(n / \log n)$ subproblems,
each of size $O(\log n)$. 

Can we make search faster?

\begin{itemize}
    \item Consider a sorted array $A$ with $n$ elements. We want to search
        for an element $x$.
        Given $p$ processors, we can search at the indeces $1, n / p, 2n/p, \dots, n$.

    \item Record the result of each comparison as $1$ or $0$.
        $cmp[i] = 1 \equiv A[i] < x$, $cmp[i] = 0 \equiv A[i] \geq x$.
        More succinctly, \verb|cmp = map (\a -> a < x) A|.

    \item $cmp$ will either have all 0s, all 1s, or a shift from 1s to 0s.

    \item If we have a shift from 1s to 0s, we know that $x$ is likely
        in the $n/p$ segment corresponding to the shift from 1 to 0.

    \item So now, we can recursively search that small segment.

    \item $T(n) = T(n / p) + O(p)$. ($O(p)$ since $cmp$ has length $p$).
        Hence, $T(n) = T(n / p) + O(1)$. This gives us $O(\log n)$ when $p = 1$
        (make sure this is correct, there is some \textbf{off by one here}.

    \item When $p = O(\sqrt n)$, the time taken will be $O(\log n / \log (\sqrt n)) = O(1)$
        This looks useless from a work point of view, but we want to see what this is
        good for!
\end{itemize}

\subsection{From parallel search to merge}
\begin{itemize}
    \item We have two sorted arrays $A$ and $B$, which we want to merge.
    \item We want to rank some elements of $A$ to create paritions of $B$.
    \item Let us take $\sqrt n$ elements of $A$ in $B$.
    \item We have $n$ processors, so each search can use $ n / \sqrt n = \sqrt n$ processors.
    \item each search now finishes in $O(1)$  time.
    \item the problem is that the partitions of $A$ are much larger now (they are $\sqrt n$ large).
    \item we have a $\sqrt n$ sized piece of $A$, and we have a size of B that is of size $(?)$.
        Note that for each piece of $A$, we now choose to allocate $\sqrt n$ processors.
    \item So, we pick $n^\frac{1}{4}$ elements of $A$ in $B$, each of which
        uses $n^\frac{1}{4}$ processors. Size of each piece is now $n^\frac{1}{4}$.
    \item So, we pick $n^\frac{1}{8}$ elements of $A$ in $B$, each of which
        uses $n^\frac{1}{8}$ processors. Size of each piece is now $n^\frac{1}{8}$.
    \item We reduce the sequence $n \to \sqrt n \to n^{\frac{1}{4}} \to n^\frac{1}{8} \dots \to O(1)$.
        This can be done in $\log \log n$ steps!
\end{itemize}
\end{document}

\section{Hashing - collision resistant hashing schemes}

Key superscript: index
Key subscript: secret.

Probability that we can find collision for $H^s$ by a PPTM must be negligible.


\section{Merkle Damgard Transform}

Given a collision resistant, hash function of the form $(h: \{0, 1\}^{2n} \to {0, 1}^n)$.
(Fixed length)

Then we can construct a collision resistant
hash function of the form $(H: \{0, 1\})^{*} \to \{0, 1\}^n)$.

$m = m1 || m2 || m3 || .. m_n$

$m_i$ is $n$ bits.


$z_1 = h(m_1, IV)$
$z_2 = h(m_2, z_1)$
$z_t = h(m_t, z_{t - 1})$

Finally, $H(m) = h(z_t, |m|) (?)$

If $h$ is collision resistant, then $H$ is collision resistant

\chapter{Tree processing}
\section{Traversal via an Euler tour}
\begin{definition}
an \textbf{Euler tour} is a cycle of a graph that includes every edge of the
graph exactly once.
\end{definition}

\begin{lemma}
A directed graph $G$ \textbf{has an Euler tour} iff for every vertex,its in-degree
equals its out-degree.
\end{lemma}

For a tree $T = (V, E)$, to define an euler tour, we make it a directed graph.
$T_e = (V_e, E_e)$, where $V_e = V$, and $E_e = \cup_{(u,v) \in V} \{ (u, v), (v, u) \}$
That is, each $(u, v)$ in $E$ creates two edges $(u, v)$, and $(v, u)$ in $E_e$.
$T_e$ will have an Euler tour.


We have to define a successor function $s: E_e \to E_e$. Here, the successor for an edge.
For a node $u$ in $T_e$, order its \textbf{neighbours (both incoming and outgoing)}
$v_1, v_2, \dots v_d$. This can be done \textbf{independently at each node}. 
For $e = (v_i, u)$, set $s(e) = (u, v_{i + 1~\text{mod $d$}})$. This choice of $s$ is valid
since we always have both edges $(x, y)$ and $(y, x)$, and we are therefore
assured that $(v_i, u)$ will be an incoming edge, and $(u, v_{i + 1}$ will be
an outgoing edge. Also, compute $i + 1$ modulo $d$, so that we eventually
cycle.

\textbf{TODO: relabel vertices to $[0..(d - 1)]$ so that modulo works properly}
\textbf{TODO: add example}

\begin{theorem}
$s$ actually constructs a tour.
\end{theorem}
\begin{proof}
Induction on number of vertices. If $n = 1$, obviously true. 
If $n = 2$, at most one edge present. We will go along the edge and come back,
which is a valid tour.


\begin{itemize}
\item Let the tour be well defined for $n = k$. We will prove it for $n = k + 1$.
\item Every tree has at least one leaf, call it $l$. Create a tree $T' = T/\{l\}$.
\item Let $u$ be a neighbour of $l$ in $T$.
\item Let $N(u) = \{ v_0, v_1, \dots v_i = l, v_{i + 1}, \dots v_d \}$.
\item Set $s_{new}(u, v) \equiv (v, u)$. Set $s_{new}(v_{i - 1}, u) \equiv (u, v)$.
\item For all other vertices, $s_{new}(e) = s(e)$.
\end{itemize}
\end{proof}


\section{Using euler tours for traversal}
Operations on a tree such a rooting, preorder, and postorder traversal
can be converted to routines on an Euler tour.

\subsection{Rooting a tree}
Designate a node in a tree as the root. All edges in the tree are
directed towards (or away) from the root.

\begin{itemize}
\item let $\{v_1, v_2, \dots v_d\}$ be the neighbours of root node $r$. 
\item we mark the final edge of the tour as \texttt{NIL}, so we get an
Euler path, and not an Euler tour.
\item the edge $(r, v_i)$ appears before $(v_i, r)$.
\item so the edge $parent \to child$ appears before $child \to parent$
\item So, if $uv$ precedes $vu$, then set $u = parent(v)$. Orient the
edge $uv$ as $v \to u$ (that is, $child \to parent$), since we want all edges towards the root.
\end{itemize}

\subsection{Preorder traversal}
We have a rooted tree with $r$ as the root. In a preorder traversal, a node is
listed before any of the nodes in its subtrees.

In an Euler tour, nodes in a subtree are visited by entering subtrees,
and the exiting towards the parent.

If we can track the first occurence of a node in an euler path, this will
tell us the preorder traversal. Note that edges in the euler tour occur
first as $parent \to child$, and later as $child \to parent$. So, we can
look at the sequence of edges in the euler tour, and find the preorder
numbering.

\subsection{Expression tree evauation of binary trees}
Tree may not be balanced.

We use the \texttt{RAKE} technique to evaluate subexpressions. We rake the
leaves from the expression tree --- we remove the leaf node and its parent.

\begin{itemize}
\item $T = (V, E)$ is a tree rooted at root node $r$. $p: E \to E$ is the 
parent function.
\item One step of the rake operation at a leaf $l$ with $p(l) \neq r$ involves:
    \begin{itemize}
    \item Remove node $l$, $p(l)$ from the tree
    \item Make the sublings of $l$ as the child of $p(p(l))$. That is, graft
    the siblings of $l$ to the grandparent of $l$.
    \end{itemize}
\end{itemize}

Why is this a good technique? Can this be applied in parallel to several leaf
nodes? Yes, it can be applied to leaf nodes that don't share the same parent.
In general, there is a richness of leaf nodes in a tree, since there
are only $n - 1$ edges.

Each application of rake at all leaves reduces the  number of leaves by half.
Each application of \texttt{RAKE} is $O(1)$. So, total time is $O(\log n)$.


\begin{minted}{python}
def shrinkTree(R):
    compute labels for leaf nodes, store in array A (exclude leftmost
    and rightmost nodes in this A)

    for _ in range(k):
        apply rake operation to all odd numbered leaves that are
        the *left* children of their parent

        apply rake operation to all odd numbered leaves that are
        the *right* children of their parent

        update A to be the remaining even leaves
\end{minted}

Applying \texttt{Rake} means that we can process more than one leaf node
at the same time.

Fo expression evaluation, this may mean that an internal node with 
only one operand gets raked.

\begin{minted}{python}
          + g(u)
  
    + p(u) 

Y     X (u)


--After raking--

   + g(u)
Y
\end{minted}

\begin{itemize}
    \item Transfer the impact of applying the operaot at p(u) to the sibling of u
    \item $R_u = a_u X_u + b_u$
    \item $X_u$ is the result of the subexpression at node $u$ -- $X_u = f(left, right)$
    \item adjust $a_u$ and $b_u$ during any rake operation appropriately
    \item Initially, at each leaf node, $a_u = 1, b_u = 0$.
\end{itemize}



\begin{minted}{python}
          + g(u)
  
    + p(u) 
    X_w
    a_w
    b_w

v        (u) 5, 1, 0
X_v,
a_v,
b_v

--After raking--

     + g(u)
v
X_v'
a_v'
b_v'
\end{minted}

\begin{itemize}
    \item Before removing $p(u)$, the contribution of $p(u)$ to $g(u)$ will be $X_w a_w + b_w$.
    \item we want what $p(u)$ used to calculate to be what $v$ calculates after.
    \item $X_w = (X_u a_u + b_u) + (X_v a_v + b_v)= (X_v a_v) + (X_u a_u + b_u + b_v)$
    \item What $p(u)$ used to calculate is: $a_w X_w + b_w = a_w (a_v X_v + a_u x_u + b_u + b_v) + b_w = a_w a_v x_v + a_w (a_u X_u + b_u + b_v) + b_w$
    \item what $p(v)$ should be: $a_v' = a_w a_v$, $b_v' = a_w (\dots)$
\end{itemize}

For other operators, proceed in a similar fashion (\textbf{TODO: do this and send to kiko, he seems interested!})


\section{Secure multipart communication}

\subsection{Synchrony}
Existence of rounds. Send messages per round. Messages are
recieved by all per round.

Or, equivalently, there exists a global clock.

\subsection{Problem statement}

A, B, C have $x_a, x_b, x_c$ inputs.

We wish to compute $f(x_a, x_b, x_c)$ without revealing $x_a, x_b, x_c$.

For this, we **do not** need trapdoor one-way permutations!
Somehow, the existence of 3 people allows us to sidestep the
requirement of trapdoor one-way permutations.

We can simulate a trusted virtual server that is not under the control of
A, B, C, that can compute $f(x_1, x_2, x_3)$ without revealing.

Adversary can only eavesdrop on **one of three parties** at any given time.
We do not know which party adversary is spying. Adversary has access to NP-oracle.

\subsubsection{Kannan philosophy}
The adversary is *omnipotent* (can solve NP complete problems in P). BUT, he
is not *omnipresent* (can only eavesdrop on one of A, B, C at a time).


\subsection{Machinery: Key management}

Can the key be stored in a network of $n$ memory spaces (called n ``shares'' of the key)
such that upto $t$ shares reveals nothing about the secret. all $t + 1$ or more shares
reveals the secret.

\subsubsection{Kannan philosophy: Rate of papers being published?}
Supposedly, when he had last seen this area and surveyed it a couple decades back, there were ~10,000 papers.
However, for some reason, we are unable to actually *see* this in our research life.

That is, per year, the course does not change by so much. why is it that the rate of increase of knowledge is
uncorrelated with the rate of papers being published?

He argues that most papers are trash, indirectly. ``We have learnt something about the art of generating
problems and solutions which is useful for training our mind, but not a contribution to the world.''

Maybe by the end of today we can see why that happens? (what? Does the crypto scheme shed some light on this?)

\subsubsection{Proof: Shamir's Secret Sharing scheme}
Consider FF, finite field, characteristic p. He picks $Z/pZ$. (Note: I will continue to use FF for finite field).

Pick a polynomial of degree $t$, with $t + 1$ coefficients.

$P(x) = \sum_{i=0}^t a_i x^i$. $a_0 = S$. $a_i, \text{where} i > 0 = random element from F (Z_p in our case) $ .

The secret is $a_0 = S$.

The $i$th share is the polynomial evaluated at $i$.
$ Share_i = P(i), i = 1, 2, ... n$


Clearly, $t + 1$ or more shares reveals the secret, since we will have $t + 1$ points of a $t$ degree polynomial.

However, given only $t$ polynomials, we do not know anything about the constant term. We can look at this as a
generalization of a one-time-pad for $n$ terms.



\paragraph{Consider an example for t = 2}

\begin{align}
P(0) &= S \\
P(1) &= S + a1 + a2 \\
P(2) &= S + 2a1 + 2a2
\end{align}

Having $P(1)$, $P(2)$ is not enough to reconstruct $P(0)$, the secret.


\paragraph{Vandermonde matrix}
This construction can be seen as a vandermondle matrix:

$Share = A \cdot v$, where:


$A = \begin{bmatrix}
 1 & 1 & 1  & 1\\
 1 &2 &2^2  &2^3 \\
1 &3 &3^2 &3^3 \\
1 &4 &4^2 &4^3
\end{bmatrix}$
$v = [S a_1 a_2 \cdots a_t]^T$.
$A_{ij} = i^j $. A has full rank, A is invertible.

\paragraph{Hardness}
We know that $P[S=s] = \frac{1}{|F|}$ (since we pick each $S$ randomly. We use finiteness of F here).

Compute $P(S=s | S_1=s_1, S_2=s_2) =? \frac{1}{|FF|}$. If it is equal to $\frac{1}{|FF|}$, then it's a one-time-pad.

We take this on faith, and say that this will work. Lookup proof.


\section{Use machinery to solve secure multiparty communication}
In this case, we use the key management scheme to share our memory data.
We split our memory into $n$ shares, and give the $n$ shares to different parties.

We need $|FF| > n$, so that we can have a large field.

Now, we've built *secure memory*. We have yet to show that we can perform operations over this thing.

\subsubsection{Suppose $n = 3, t = 1$ (that is, any one should not get the secret, out of 3 machines)}


In this case, we have a 1-degree polynomial since $t = 1$. So, $P(x) = rx + s$.
(r = random, s = secret).

Hence, $S_a = r + s$, $S_b = 2r + s$, $S_c = 3r + s$.

We will construct an instruction set consisting of $add$, $multiply$ *over our FF* (kannan will be doing it over $Z_p$).

We will need $z <- x + y (in FF)$, $z <- x * y (in FF)$.


\subsubsection{Constructing $+$}

$x$ is shared according to $P_x(t) = r.t + x$,
$y$ is shared according to $P_y(t) = r'.t + y$.

A has $x_a, y_a$. B has $x_b, y_b$. C has $x_c, y_c$.
A should have $z_a$, B has $z_b$, C has $z_c$.

A computes: $z_a = x_a + y_a$
B computes: $z_b = x_b + y_b$
C computes: $z_c = x_c + y_c$

Note that:
\begin{align*}
P_x(1) &= x_a.\\
P_y(1) &= y_a.\\
\\
P_x(2) &= x_b.\\
P_y(2) &= y_b.\\
\\
P_x(3) &= x_c.\\
P_y(3) &= y_c.\\
\end{algin*}


We denote $P_{x + y} =(defined as)= P_x + P_y $.

\begin{align*}
x_a + y_a &= P_x(1) + P_y(1) = (P_x + P_y)(1) = P_{x + y}(1) \\
x_b + y_b &= P_x(2) + P_y(2) = (P_x + P_y)(2) = P_{x + y}(2) \\
x_c + y_c &= P_x(3) + P_y(3) = (P_x + P_y)(3) = P_{x + y}(3) \\
\end{align*}

Note that we have now computed a new secret sharing polynomial,
$P_{x + y}$. $z = P_x(0) + P_y(0) = x + y$ So, $z$ is encrypted with $P_{x + y}$.

\subsubsection{Constructing $*$ / Rabin Matching}
We denote $P_{x * y} =(defined as)= P_x * P_y $.
Note that we have now computed a new secret sharing polynomial,
$P_{x * y}$. $z = P_x(0) * P_y(0) = x * y$


$z <- x * y$
z_a <- x_a * y_a.
z_b <- x_b * y_b.
z_c <- x_c * y_c.

Polynomials are also homomorphic according to $*$
Hence, $z_a <- P_x * P_y = P_{x * y}$


However, this is a 2 degree polynomial! So, we now do not have P_{x * y}.

There is another problem: Shannon secret sharing gives us secrecy if the polynomial is *random*.
In this case, the polynomial is not random, since it is reducible (P_{x * y} = P_x * P_y). Reducible
polynomials are a subset of all polynomials (Sid question: what's the size of the subset?).


Now, can we somehow ``linearize'' P_{x * y} such that the  constant term remains the same,
and it is uniformly chosen across polynomials of degree t? (chosen over: $FF[x] / x^{t+1}$).


NOTE: I do not understand this final part properly!


We construct $z_a' = x_a * y_a$, $z_b' = x_b * y_B$, $z_c' = x_c * y_c$.
Then, we *share* z_a' as z_{aa}', z_{ab}', z_{ac}'. Repeat with z_{b}', z_{c}'.

They lie on a polynomial whose constant term is $z$.So,
\lambda_a z_a' + \lambda_b z_b' + \lambda_c z_c' = z.

We know that $z_a'$ is a linear combination of $z_{aa}'$, $z_{ab}'$, $z_{ac'}$
so, $z_a' = \lambda_a' z_{aa}' + \lambda_b' z_{ab}' + \lambda_c' z_{ac}'$.
Similarly for $z_b'$, $z_c'$.


So, $z$ will be a linear combination of all $z_{pq}'$ where $p, q \in {a, b, c}$

We can write this linear combination as:
z = $(\lambda_a \lambda_a' z_{aa}' +
     \lambda_b \lambda_a' z_{ba}' +
     \lambda_c \lambda_a' z_{ca}') +
     (same for b) +
     (same for c)$.

z = \lambda_a znew_a + \lambda_b znew_b + \lambda_c znew_c.

Note that znew_a, znew_b, znew_c are 1 degree.



--------
P_z(t) = r.t + z.

z_a = r + z
z_b = 2r + z
z_c = 3r + z

Find some linear combination of z_a, z_b, z_c such that we get z
z = t_a z_a + t_b z_b + t_c z_c
---------

OK, I don't know WTF happened. Read this tonight (21 feb 2018)



References for this:
1. BGW 1988 (in SoTC)
2. GRR 1998 (in ACM PoDC)

\subsubsection{Kannan philosphy}
Today is the first time we have brought in a different adversary. Usually, we computationally bound the machine.
Today, we are bounding the ``omnipresence factor'' of the machine.


\paragraph{Suppose $n = 3, t = 2$ (that is, any two should not get the secret, out of 3 machines)}
This is equivalent to 1 out of 2 (what we did last class), by clubbing two machines together.
So, for this, we will require the existence of trapdoor one-way functions (since this reduces to 1-of-2).



\section{Generalized Secret Sharing}

(a beginning towards a major unresolved
problem/issue in Crypto/Algo)

\section{Review of last class}
- Shamir's secret sharing, and using it for secure
multi-party communication.


\section{Kannan philosophy}

We will take generalized secret sharing and generalise it. This will crop
up to have ramifications on crypto and algo.

\section{Generalization}

We have a secret $S$ that we split into $n$ shares $S_1\cdots S_n$.

In Shamir's secret sharing, $\leq t$ learns nothing, $\geq t + 1$ learns
full information.

Implicit assumption: Network is homogeneous in trust. Adversary will attack
one part of the network just as likely as any other part of the network.


However, there can be situations where we believe that some subset of the
network is relatively more trustworthy than other sections of the network.

In this case, we want to get $n$ shares, such that $<= t_1$ in the first
$\frac{n}{2}$, $<= t_2$ in the next $\frac{n}{2}$ does not get the secret.
Other combinations can learn.

Example: let $n = 4$, $t = 2$, this means all subsets of size $>= 2 + 1$
can access the secret.

Example': we want to make a statement such as: This basis (aka "access structure")
and all supersets of the basis should be allowed to access the secret:
Eg, we can give a basis 
{{S1, S2}, {S1, S3, S4}}, and we want a secret sharing method that will let
us share secrets among these sets and their supersets.


Note that this *is a generalization* of the original. A $t$ access threshold is
this basis scheme, where the basis is all subsets of size $t + 1$.


\section{Alternate view of access structure as boolean functions}
We said that an access structure is *monotone* over over subsets of ${1..n}$ 
(MONOTONE: If a subset $S$ is in the access structure, all supersets of $S$
are in the access structure)


We can look at the access structure as $f: {0, 1}^n \rightarrow {0, 1}$. (There
is clearly a bijection between subsets and ${0, 1}^n$, so represent subset as
${0, 1}^n$. We define  $f(bitencode(subset)) = 1$ if subset is in access structure, $0$ otherwise.


Now, we need $f$ to be a monotone boolean function. that is,
$f(bitencode(S)) = 1 \implies forall S \subset S', f(bitencode(S')) = 1$ (if $S$ can access the secret,
all supersets of $S$ can access the secret).

So now, we have constructed a boolean function to represent our access structure.
We will now invoke complexity theory.

\subsection{Hardness}
We can construct an access structure which will be lower bounded in exponential
size of the secret?

Number of subsets of powerset of set $n$ is $2^(2^n)$.
 
The monotonicity does not reduce this by much (Sid: Proof?)

For every access structure, we will have a secret sharing scheme. So, we will
have $O(2^2^n)$ secret sharing scheme. So, the number of bits for some
secret sharing scheme will be $O(log(2^2^n)) = O(2^n)$.


So, we will  have a secret sharing scheme that is exponential. Note that
the length is the number of instructions in the scheme  (both the message itself
and the instructions for the secret sharing) will be exponential. However, the
*share* could be small.

So, we have a computation versus communication trade-off that we need to explore.
That is, we can trade-off the sizes of the computation (instructions length)
and the size of communication (that is, the share length).



\subsubsection{Unresolved Problems (Open problems)}
\begin{itemize}
    \item Does there exist an access structure on $n$ shares 
        such that *every* secret sharing scheme for it has super-polynomial
        length? (that is, is there an access structure that does not allow
        for efficient encoding in terms of share length).

    \item Suppose we convert an access structure to a monotone boolean function.
        Suppose we only care about those functions that are in $P$ 
        (that is, polynomial circuit depth).

        Do all efficiently computable (in $P$) access structures have
        efficient secret sharing schemes?

        Given an access structure that is *efficiently computable*
        (in $P$), can we construct a secret sharing scheme that is in P?


        Relationship to algorithms:

        We have a notion of "input size" in algorithms.
        If we have a distributed algorithm, assume it is parametrised by
        number of nodes $n$, and say $<= t$ nodes that are allowed to be faulty.


        Assume we had captured the fault tolerance in terms of our
        boolean function, $f: {0, 1}^n -> {0, 1}$. When
        $f(S) = 0$, it is faulty, and failure here need to be tolerated.
        When $f(S) = 1$, it is not faulty, and failure here need not be tolerate.

        Note that these are equivalent to the old definition. Sets with
        $f(S) = 1$ are the "critical nodes", which need to be present. We can
        have $f(S) = 1 forall |S| >= t + 1$, and this gives us back our
        old definition based on number of nodes that are faulty.

        So, distributed systems can also say, we wish to tolerate some
        monotone $f$ in general, so we have generalized distributed system
        tolerance to a general "critical nodes" notion.

        So now, the size of our description of our distributed systems
        algorithm depends on the **size of $f$** (Since $f$ is now a
        parameter to our distriuted system algorithm)

        Before, our distributed system algorithm was $dsalgo(n, t)$, Now it is
        $dsalgo'(n, f)$, so we need to give f as a parameter.


        Consider the old problem with tolerance described in terms of the
        number $t$. Encoded as a function, it is: $f(S) = 1 if |S| > t, 0 when |S| <= t$.
        If we have to *describe* f, then if we decide to describe it in terms
        of the basis, we will have $nC(t + 1)$ elements in the basis. 
        What used to be a **number**(t) is now a set of size $nC(t + 1)$, since
        we generalized $t$ to $f$.

        
        If I choose to view fault tolerance as an access structure problem
        always, then the basis size is in itself $nC(t + 1)$. So, my
        input size is $nC(t + 1)$. So, any ridiculous crap of $nC(t + 1)$ 
        will be "polynomial". 

        However, the old encoding will consider a polynomial in $nC(t + 1)$
        as exponential.

        So, our choice of encoding is skewing our notion of what is
        "small".

        So, we need some reasonable way to define "size of access structure",
        that does not allow blowup like this.


        Why can't we argue that the input size is size of the function $f$?
        Size of $f$ has two notions: one as the length of the description of $f$,
        and the other as the *time taken for $f$ to execute*. That is, 
        I can have a small description that takes exponential time, or I can
        have an exponential description that takes constant time.
        (list of tuples verus encoding the smallest program)

        This leads us to: what is the size of a program? If two programs
        have the same length, but one is faster, then we would like to 
        consider the one that is faster (maybe?)

        Why should runtime play a role? Kannan argument: of what use is a program
        that will not give us output for billions of years. This is as good as
        not giving us a program at all.

        So, we would like to optimise on both length and running time.


        This fucks us over when the program that we choose is used to
        describe something about our runtime environment, like the access
        structure as given above.


        This naturally leads us to the question, given that the access
        strucrure is efficiently computable, can we have an efficient
        secret sharing scheme for that?


    \item Assume that it is *impossible* to have efficient secret sharing
        schemes for efficient access structures, so we wish to crypto it
        and solve-the-impossibility.

        We do not even know if we can solve this assuming the whole crypto-model.
        That is, adversary is PPTM, negligible prob. of error, one way
        functions exist, do all efficient access structures have efficient
        secret sharing schemes? Unknown.

    \item Supposedly, similar problems crop in other areas. To quote kannan,
        "messy waters", "murky", etc.


        Coding theory:

        We have a noisy channel, and we have to model noise.
        Noise is modelled as: if I send $n$ bits/symbols, $<= t$ symbols
        are in error. Can we give an error correction code that can 
        tolerate such noise?

        Again, do the same thing, replace $<= t$ with an access structure, as
        we did in distributed systems.  We land in the same problem of
        describing input size.


        Why would we need access structures for ECC ever? For example, consider 
        two channels between $S --> R$. One channel has 1/5 chance of error,
        other channel has 1/4 chance of error. Say I send the first n/2 bits
        over the first channel and the next n/2 bits over the next channel.

        This can be modeled using an access structure over the full n bit
        string. So, knowing details about where this "toggling" will happen
        should let us design better ECC schemes.

        For example, if our channel sent first n/2 bits correctly and next
        n/2 bits fully wrong, we have an ECC: only send data in the first n/2
        bits. This is *not* the same as stating that "50% of data is corrupted"
        In this case, there is no ECC that can solve this.

        So, having data regarding where the toggles happen is *useful information*
        to have. So, the access function is a *useful abstraction* in ECC, it
        is not frivolous.

        Thus, we are not allowed to say "access structures are useless in ECC".
        So now, our noise is a program / monotone boolean function.  This now
        leads us to the thorny problem of defining size of monotone
        boolean function.


        For the past half-century, ECC has only worked on the threshold
        kind of $f$. We have not worked on ECC of the general access
        function kind.

        Rounding back, we need to define input size of $f$, which is the
        noise in the channel.
        
        If $f$ is very short in length but it takes exponential time to run, then we cannot
        "observe" the error since it takes exp time to run.

        Suppose we get a noisy channel in real life whose $f$ is not in $P$. Then the
        real world Channel is computing $f$ which is in NP.
        
        So, we can sample the channel (which is computing $f$) to solve NP
        problems. But such channels should not exist, since nature
        cannot solve NP problems in P time. So, we need
        $f$ which is in $P$.

        Q) does there exist efficient ECC for all realistic channel f \in P?

        Cute Scott Aaronson quote: "If P != NP were a question in physics, then
        it would have been a law by now (since we have *never* observed nature
        solve NP problems in P time".

        
\end{itemize}


Physics is ridiculous, since we expect reality == math prediction.

We can weaken it, by saying people should be able to distinguish reality 
and math prediction in polynomial time!

% read separately
% 
\chapter{Cryptography from channel noise, without 1-way functions}

\subsection{Introduction Kannan philosophy}
If the insecure channel connecting $A$ and $B$ is noisy, we can have secure communication.
This is possible because we have two adversaries -- the actual adversary and the noise
in the channel. Intuitively, these are adversaries of each other. \\

The ``noise injector'' is disrupting information that the eavesdropper wishes to understand. Can we design a channel that disrupts the eaversdropper more than it disrupts $B$? This was not initially considered to be  possible because of the seeming impossibility of harnessing the noise. \\

In previous lectures, we have shown how to simulate a secure machine on two insecure machines. The contents of this secure machine's memory were the bitwise \texttt{XOR} of the memories of the two insecure machines. This can be done by simulating secure \texttt{XOR} and secure \texttt{AND}. We have already shown how to simulate secure \texttt{XOR} without one-way functions. However, to simulate \texttt{AND} we used one-way functions. We now show a protocol to compute secure \texttt{AND} over a noisy channel without the use of one-way functions. 

\subsection{Secure \texttt{AND} without one-way functions}

Initially $A$'s memory consists of two bits, $x_A$ and $y_A$. Similarly $B$'s memory consists of two bits, $x_B$ and  $y_B$. The secure machine's memory consists of two bits $x$ and $y$ where $x = x_A + x_B$ and $y = y_A + y_B$. We want $A$ and $B$ to compute $z_A$ and $z_B$ respectively, such that $z = z_a + z_b = x \cdot y$. \\1

In general, we will introduce bits that are consistently named between $A$ and $B$. The naming scheme will be
$bitname_A, bitname_B$. This indicates that the secure virtual machine has a bit $bitname = bitname_A + bitname_B$.

\subsubsection{First simplification -- constraints on the noise}

As a pedagogical simplification, we assume that out of every four consecutive bits sent over the channel, \textit{exactly one} of the four bits will be toggled. This restriction will be lifted later. 

\subsubsection{Protocol}

We define a matrix $T$, representing the truth table of the bitwise \texttt{AND} operation. Note that $T$ is public and part of the protocol.

\[T = 
\begin{bmatrix}
0 & 0 & 0 \\
0 & 1 & 0 \\
1 & 0 & 0 \\
1 & 1 & 1
\end{bmatrix}
\]

\begin{algorithm}
% \caption{{\it }}
\label{a-enc-h}
\begin{algorithmic}[1]
\State \textbf{Initial state:} $A$ has two bits, $x_A$ and $y_A$. $B$ has two bits, $x_B$ and $y_B$.
\State $A$ generate four random bits $R = \langle r_0, r_1, r_2, r_3 \rangle$, and sends $R$ to $B$.
\State $B$ receives four bits from $A$, $S = \langle  s_0, s_1, s_2, s_3 \rangle$.
\State $A$ computes $R \times T = \langle a_A, b_A, c_A \rangle$, and $B$ computes $S \times T = \langle a_B, b_B, c_B \rangle$. Correspondingly, the secure machine has computed $(R + S) \times T = \langle a, b, c \rangle $.
\State $A$ noiselessly sends (using an error-correcting code) $x_A + a_A$ and $y_A + b_A$ to $B$. $B$ noiselessly sends $x_B + a_B$ and $y_B + b_B$ to $A$.
\State $A$ and $B$ both compute $X = (x_A + a_A) + (x_B + a_B)$ and $Y = (y_A + b_A) + (y_B + b_B)$.
\State $A$ defines $z_A = (X \cdot Y) + (x_A \cdot y) + (y_A \cdot x) + c_A$.
\State $B$ defines $z_B = (X \cdot Y) + (x_B \cdot y) + (y_B \cdot x) + c_b$. We have $z = z_A + z_B = x \cdot y$, as desired.
\State \textbf{Final state:} $A$ and $B$ have computed $z_a$ and $z_b$ respectively such that $z = z_a + z_b = (x_A + x_B) \cdot (y_A + y_B) = x \cdot y$
\end{algorithmic}
\end{algorithm}

\pagebreak

% Steps 1-3
Note that by the assumption on the channel noise, $R$ differs from $S$ at exactly one position. Therefore, $R + S$ is $1$ at exactly one position and $0$ at the other positions. However, the position where $R$ and $S$ differ is unknown, and it is this uncertainty that we harness to implement the secure computation. \\

% Step 4
In Line $4$, the secure machine computes $(R + S) \times T = \langle a, b, c \rangle $. Since $R + S$ has exactly one bit set to $1$ the remaining bits being set to $0$, $\langle a, b, c \rangle$ is in fact a row of $T$. Since $T$ is the truth table of the bitwise \texttt{AND} operation, it follows that $a \cdot b = c$. \\

We have computed the \texttt{AND} of some \textit{random bit} $c$, but \textit{not} that of the bit $z$ which we are interested in. We will use $\langle a, b, c \rangle$ to compute $z$. \\

% Step 5
In Line $5$, we simulate a noiseless channel over the noisy channel using error correction codes. We have used the noise of the channel to construct $c$. After this, we have no more use of the noise in the channel. \\

$A$ sends $(x_A + a_A)$ and $(y_A + b_A)$ to $B$. $B$ sends $(x_B + a_B), (y_B + b_B)$ to $A$. Here $a_A, a_B, b_A$, and $b_B$ function as one-time-pads since they are random bits. Hence, they prevent $A$ from getting information about $x_B$ and $y_B$, and prevent $B$ from getting information about $x_A$ and $y_A$.

% Step 6
Using the shared information, both $A$ and $B$ can now compute $X = (x_A + a_A) + (x_B + a_B)$ and $Y = (y_A + b_A) + (y_B + b_B)$. These will be used to construct $z_a$ and $z_b$ such that $z = z_a + z_b = x \cdot y$.

\begin{align*}
x &= x_a + x_b \\
X &= x_a + a_a + x_b + a_b \\
\text{Hence, } x &= X + (a_a + a_b) \\
\text{Similarly, } y &= Y + (b_a + b_b) 
\end{align*}
  
Let us compute $x \cdot y$.

\begin{align*}
x  \cdot y &= (X + (a_a + a_b)) \cdot (Y + (b_a + b_b)) \\
&= (XY) + X(b_a + b_b) + (a_a + a_b)Y +  (a_a + a_b) (b_a + b_b) \\
&= (XY) + X(b_a + b_b) + (a_a + a_b)Y +  \underline{a \cdot b} \\
&= (XY) + X(b_a + b_b) + (a_a + a_b)Y +  \underline{c} \\
&= (XY) + X(b_a + b_b) + (a_a + a_b)Y +  \underline{(c_a + c_b)}
\end{align*}

$A$ defines $z_a = XY + X b_a + a_a Y + c_a$. $B$ defines $z_b = X b_b + a_b Y + c_b$. Both $A$ and $B$ are using locally available bits, along with the globally known $X$ and $Y$. We see that $x \cdot y = z_a + z_b = z$.

\section{Relaxing the channel noise constraints}
Suppose that in each block of four, at most two of the bits can get flipped. This is reasonable because the overall number of bits flipped can be at most half for any channel over which communication is possible. We will relax this assumption later.\\

We now give a protocol for detecting if the number of bits flipped is odd or even. This is sufficient to detect whether the number of bits flipped is one under the assumption that at most two bits are flipped.\\

$A$ and $B$ compute and publish $\sum r_i$ and $\sum s_i$, respectively. $A$ and $B$ compute $p = (\sum r_i + \sum s_i)$. $p$ will be $1$ if an odd number of bits are toggled. Given our assumption that at most $2$ bits are toggled, if $p$ is $1$, then the number of bits toggled is $1$. Both $A$ and $B$ know $p$. If $p$ is not $1$, then we restart the protocol until our constraint on the channel is satisfied, i.e. until we have $p = 1$.

\subsection{Relaxing the constraint on local noise}
Let us suppose that there is no constraint on the number of bits that can be toggled. If $0, 2$ or $4$ bits are toggled, then $p$ will be $0$ and the protocol will be restarted. Hence, we need only worry about the case when $p = 3$. We repeat this protocol $n$ times with $p$ being $1$ or $3$ in each of the iterations. (we restart if $p$ is not $1$ or $3$.) The answers can be wrong only in the cases where $p = 3$. \\

Recall that in a distributed computation, if there are $n$ nodes and less than $n/3$ of them are actively disruptive, we can still perform the computation. In this case, we consider the iterations where the answer can be wrong to be analogous to disruptive nodes in the distributed computation. Therefore, if the number of iterations where $p = 3$ is less than $n/3$, then we can perform correct computation. \\

This is also a reasonable assumption because it is quite pathalogical for the noise to be clumped locally and flip exactly $3$ bits in many $4$-bit windows when the overall average is at most $2$ bits per $4$-bit window.

\subsection{Further relaxation using distillation}
We now further relax the channel noise. We will show that it is sufficient for the number of iterations where $p = 1$ to be greater than the number of iterations where $p = 3$.\\

TODO: Find out how to do this.

\subsection{Conjecture: Buggy software is more secure than bug free software}

In this situation, we exploited the unpredictable nature of the noise to achieve secure computation over the noisy channel. However, we conjecture that we can achieve the same result by considering other sources of unpredictability. For example, we could use race conditions as a source of unpredictability to perform this protocol over. \\

However, this is definitely not a "free lunch" theorem. We use the unpredictability in controlled ways. In particular, we need a way to turn it on or off -- recall that we disable the noise in the later parts of the algorithm by employing an error correction code.

\subsection{Future work}
Certain obvious directions of extension are to consider larger window sizes than just 4. With larger window sizes, we can pick larger $T$ matrices that allow us to perform multi-gate operations. However, this ability will be offset by a harder analysis on the error-detection side. The use of the parity bit $p$ will need to be generalised for higher window width variants. \\

We could also try to perform this over $GF_{p^n}$ rather than $GF_{2}$. In this case, each element of the field itself takes up $\log p^n$ space, and therefore local spikes in noise would need to be much stronger to disrupt the protocol. \\

Lastly, we would like to axiomatise precisely what sources of unpredictability can be used with this algorithm.

\documentclass[11pt]{article}
%\documentclass[10pt]{llncs}
%\usepackage{llncsdoc}
\usepackage{amsmath}
\usepackage{graphicx}
\usepackage{makeidx}
\usepackage{algpseudocode}
\usepackage{algorithm}
\evensidemargin=0.20in
\oddsidemargin=0.20in
\topmargin=0.2in
%\headheight=0.0in
%\headsep=0.0in
%\setlength{\parskip}{0mm}     
%\setlength{\parindent}{4mm}
\setlength{\textwidth}{6.4in}
\setlength{\textheight}{8.5in}
%\leftmargin -2in
%\setlength{\rightmargin}{-2in}
%\usepackage{epsf}
%\usepackage{url}
\usepackage{epsfig}
\usepackage{tabularx}
\usepackage{latexsym}
\newtheorem{lemma}{Lemma}
\newtheorem{observation}{Observation}
\newtheorem{proof}{Proof}
\newcommand\ddfrac[2]{\frac{\displaystyle #1}{\displaystyle #2}}

\def\qed{$\Box$}
\def\proof{\textit{Proof. }}
\newtheorem{corollary}{Corollary}
\newtheorem{theorem}{Theorem}

\begin{document}
\section{Agreement in a distributed system}

\section{Introduction: Philosophy of today's class}
This is the beginning of ``clash of philosophies''.
Agreement in a distributed system can be impossible! (which is why it is part of this course).
If one-way-functions exist, then much more is possible.

Different adversaries:
\begin{itemize}
\item Computational Adversary - All modern crypto
\item Practical Adversary - Noise based crypto (last class)
\item Natura Adversary - Quantum crypto (after mid-2)
\item Philosophical Adversary - ??? (So kannan, man)

  Our problem is a fundamental problem in distributed systems. Problems of agreement abound.
  There are instances where distrubuted systems textbooks give up, and give proofs of no-solutions.
  However, this is POIS class, so impossibility proofs are where we start our trade.

  
  Djikstra's paper in 1980 where consensus is not possible: three nodes connected in a synchronous network.
  We want to simulate a broadcast channel over P2P. That is, we wish to broadcast information over P2P.

  The problem is called ``byzantine agreement problem''.

  Each party has a single bit of input $x_i$. Each will give a single bit of output $b_i$ ($i \in \{ 1, 2, 3 \}$).
  Agreement  requirement - All non-faulty nodes have the same output (all $b_i$ are equal).

  If this was the \textit{only goal}, then this is trivial. We just have all of them output a constant.
  Validity requirement - The output of all non-faulty nodes is equal to the input of \textit{some} non-faulty node.


  We can simulate broadcast. Run the protocol that has agreement and valitiy. We want to simulate broadcast. We send $0$ to everybody,
  and then simualate the protocol that we have crafted. Since everyone has started with the same input, everyone will end with the
  same output.

  Is this actually such a tough problem? Why can't we just say ``send your input to everybody''? Why is this not equivalent to simulating
  a broadcast channel?


  Let's see what happens. Let $A$ be a source. Say $A$ sends message $m$ to $B$ and $C$. this looks like $A$ has broadcasted message $m$.
  If this were happening over a \textit{broadcast channel}, then $m$ would have been received by $B$ and $C$.

  However, on a unicast channel, it is possible that $A$ fails between $B$ receiving the message and $C$ receiving the message.
  Now, this is \textit{not broadcast}.

  We need broadcast to be atomic - either everyone gets the message, or no one gets the message!

  assume $A$ is adverserial: that is, $A$ is able to send $m_1$ to $B$, $m_2$ to $C$, $m_1 \neq m_2$. So, we need to make sure
  that people can't fuck up broadcast over unicast. However, if we had a physical broadcat channel, this fuck up can't happen.


  So, the problem \textit{is complicated}. We wish to simulate an atomic broadcast channel over a unicast channel, given
  adverserial nodes.
\end{itemize}

\section{Three nodes, one of them is byzantine faulty, no way to have both agreement and validity}
\subsection{Intutition}
Say $A$, which is byzantine faulty sends $0$ to $B$ and $1$ to $C$. $B$ and $C$ now exchange values,
they realise agreement has not actually happened.

If they \textit{know} that $A$ was faulty, then they could agree that A was faulty.

However, $B$ does not know whether $C$ is faulty or not. It could be that $A$ had sent
$0$ to $C$, but $C$ was faulty and thus chose to broadcast $1$ to $B$.

So, $B$ only knows that \textit{one of} $A$ and $C$ is faulty.
Similarly, $C$ only knows that \textit{one of} $A$ and $B$ is faulty.

\subsection{Proof}
Let $\Pi$ be a byzantine agreement protocol for this case.

Running with inputs $x_1, x_2, x_3$, we get outputs $b_1, b_2, b_3$, we have the guarantee that
1. all b_i are equal
2. b_i is equal to some x_i


We now show that such a $\Pi$ cannot exist.

The original proof is difficult. Then, at MIT, a new proof technique came out in distributed systems
which proved this. (Proof is called "hexagon proof"). 

Proof strategy:
\begin{itemize}
    \item Given that $\Pi$ is a byzantine agreement protocol.
    \item We create some other problem with some other network, and show that if $\Pi$ is a byzantine
        agreement protocol in this network, then that protocol should do *something* in the other network.
    \item We show that the protocol does not do anything (is not consistent)
\end{itemize}


Consider a network of six nodes in a ring topology (hexagon) (Call this network $Hex$)
Call the original network $Tri$.

The code delegated to three parties in $Tri$ is $\Pi = < \Pi_1, \Pi_2, \Pi_3>$.

$\Pi'$ is the program for $Hex$.  $\Pi' = <\Pi_1, \Pi_2, \Pi_3, \Pi_1, \Pi_2, \Pi_3>$
(That is, node $i$ gets program $\Pi'_i$).


In $Tri$, $\Pi_1$ got messages from $\Pi_2 , \Pi_3$. This is
the same structure in $Hex$. So, structurally, the networks are the same.
Syntactically, the code will work in $Hex$ - because the
"local" network topology is the same. So, $\Pi_$ running at node $1$ cannot 
distinguish if it is at $Tri$ or $Hex$.

Proof by induction on rounds. At round $0$, we cannot distinguish since
structurally, the locales are the same. Then, induction on rounds.

We now show that $Hex$ is an inconsistent protocol.

Input $0$ for nodes $1, 2, 3$, $1$ for nodes $4, 5, 6$.


Consider $(2, 3)$ in $Tri, Hex$. $(2, 3)$ in $Tri$ will be talking to $1$.

Say $1$ is faulty in $Tri$ (we can do this since Pi is a BA protocol).
$1_tri$  will send what goes from $4_hex$ to $3_hex$ to $3_tri$.
$1_tri$  will send what goes from $1hex$ to $2_hex$ to $2_tri$.

So, $2, 3$ do not know if there are working in $Tri$ or in $Hex$, because
the messages are the same!

We know that in $Tri$, there is a BA protocol, so it should terminate.
Moreover, it should terminate with $0$ since we assumed that the protocol
is a BA protocol.

Also, this means that output will be $0$ in $Hex$ for $2, 3$ as well, since
they are structurally isomorphic.


Now, let us look at $4, 5$ in $Hex$ (which has protocols $\Pi_1, \Pi_2$).
Let us say that $3'$sends $\alpha$ to $2'$ and \beta to 1'.

This is equivalent to a faulty 3 in Tri.

We know that the output must be 1 for 1', 2' from our starting conditions.


Now, looking at $(1, 3)$, once again, if 2 is faulty, make outputs the
same way such that they can't distinguish if they are running in Tri or in Hex.

In this case, 1' will have to output 0, since 3 outputs 0. However,
in the last case, we had agreed that 1' would have to output 1.

Conditions:
- 3 outputs 0
- 1' outputs 1
- output(3) = output(1').

Hence, such a $\Pi'$ cannot exist. Hence, $\Pi$ cannot exist.


\section{Clash of philosophies}

\subsection{Distributed systems spirit}
We use the term "non faulty" when defining Agreement, Validity. Originally,
the definition was, "if I gave a node \Pi, then it should execute \Pi,
not something else (\Pi')".

Kannan - anyone who contributes to the execution deserves the output

\subsection{Clash}

Assume there is a player who is participating in two programs, and
is non-faulty.

Suppose it is a byzantine agreement program (1 out of 3), which is impossible.

We force everyone to digitally sign their messages. Now, impossibility becomes
possible.

Now, if I was supposed to run Pi (which says sign your messages), so I run it.


In the background, my friend asks me to share my secret key with him, which I do.
This person is also part of the BA network, which causes my key to be revealed,
and thus the adversary can forge my signature, thus the protocol breaks.


Should the node in question be called faulty or non-faulty?

Assume we call this node faulty. This makes sense, because he is breaking our
crypto assumptions of key-privateness.

However, distributed systems is of the opinion that anyone who shares their
resources must reap the rewards. So, from a distributed systems viewpoint,
the node is non-faulty.

Now, the notion of faulty and non faulty depends on the implementation detail!


\subsection{Non modular attacks}
Assume a byzantine agreement protocol between 3 people. It solves the problem
using digital sinatures. Call this protocol $\Pi$.


They're running two byzantine agreement protocols in parallel.

Call the parties "1 = 1', 2 = 2', 3 = 3'", so we have $\Pi$ running
between $1, 2, 3$, and another instance of $\Pi$ running between
$1', 2', 3'$. (makes sense, two agreement protocols are running in parallel).


Let the input be $0$ for (1, 2), and $3$ is corrupt in $\Pi$.

Let the input be $1$ for (1', 3') and $2'$ is corrupt in $\Pi'$.


Adversary can fail \textit{both} protocols (Why?)

The adversary has access to the private key of 2' (which is the private key
of 2 as well). Since all processes receive the same private key, the adversary
in 2' has access to the private key of 2.

For example, 2' creates (1, signed 1). The adversary in 2' will drop the packet
that 2 tried to send to 1 (which was supposed to be (0, signed 0)),
and sends the (1, signed 1).

\subsection{Processes}

We assume that 2 processes do not have the same PID.


Consider three distributed processes $1, 2, 3$. We have no global PID
assuming one of these are faulty, since we showed that it is impossible to agree
on even a single bit.
So now, we have three PIDs, $PID_1, PID_2, PID_3$.


Let us say server, client are connected across two ports P, P'. In theory,
an adversary can send messages from P to P', and P' to P (flip P and P').


\subsection{Correctness versus Security}

Correct HW on top of a correct OS on top of a correct compiler on top of a 
correct program will give us a correct program.

In seurity,
Secure HW on top of a secure OS on top of a secure compiler on top a secure
program will give us a secure program (?)

This is indeed untrue.
Consider the distributed process with same PIDs case. the attack is neither
fully OS based nor network based. the router can arbitrarily swap PIDs 
acting as an adversary. We cannot construct consensus due to the byzantine
agreement theorem we have.

We can have cross-model attacks.

\section{Crypto, enter the stage}

The idea is that the adversary needs to simulate the hexagon protocol to 
construct a contradiction. However, if we can ensure that this process
is computationally intensive, then life is chill, because we cannot
have such an adversary.

So, we force people to sign their messages, so that the adversary
can't simulate.




\end{document}

\chapter{Quantum Secret Key Establishment}

\section{Three Polarizers Experiment}

Photon is a single Qubit system: Vector in $\mathbb{C}^2$

$|\psi> = a |0> + b|1>, a, b \in C$

Normalized, such that $|a|^2 + |b|^2 = 1$.


Postulates of QM:
\begin{itemize}
    \item For any system, there is a state vector in Hilbert space.
    \item Measurement postulate: If we measure a qubit, then with probablibity of $|a|^2$, we get $|0>$, with probablibity of $|b|^2$, we get $|1>$.
        After measurement, things collapse to one of the measured values
    \item  Evolution postulate: Will evolve according to schrodinger / unitary transformation
\end{itemize}


\section Quantum Key exchange

$A$, $B$. Have two channels, classical and quantum.
$E_{(ve)}$ is eavesdropping on both channels.


\begin{align*}
\ket{+} = \frac{\ket{0} + \ket{1}} {\sqrt(2)}
\ket{-} = \frac{\ket{0} - \ket{1}} {\sqrt(2)}
\end{align*}

Step 1: $A$ chooses to encode $0$ as either $\ket{0}$ or as $\ket{+}$
$A$ chooses to encode $1$ as either $\ket{0}$ or as $\ket{-}$


Consider a stream of random bits: $r_0, r_1, \dots, r_n$, $s_0, s_1, \dots s_n$

$r_i$ is the value we encode. $s_i$ dictates how we encode $r_i$.

$r_i = 0, s_i = 0$, then we sent $\ket{0}$.
$r_i = 0, s_i = 1$, then we sent $\ket{+}$.
Et cetra.


$A$ encodes $r_i$ with respect to $s_i$ and sends the qubits to bob.
So, $n$ qubits have been sent. All the qubits can be eavesdropped by $E$.



Step 2: $B$ receives these qubits and measures them in one of the two bases
\textit{at random}. (either 01 or plus-minus). $B$ records the answers.

$B$'s choice of basis will be governed by random bits $s_1', s_2', \dots s_n'$
Let the answers on measuring in the $s_i$ basis be $r_1', r_2', \dots r_n'$. 


Note that if $s_i = s_i'$, then $r_i = r_i'$. If $s_i \neq s_i'$, then
$r_i = r_i'$ with $0.5$ probability.



Step 3: $A$ and $B$ publish the $s_i$, $s_i'$. (publish measurement basis info)
Ignore all indeces where $s_i \neq s_i'$. So, from now on, we will consider
the index set $I_{eq} = { i \in [1\dots n] \vert  s_i = s_i' }$

Analyzing Eve: Eve has a $s_i''$ that is used to measure the channel.
When $s_i = s_i'$, but $s_i'' \neq s_i$ (that is, Eve is measuring using a 
different basis from $A$ and $B$).

Assume WLOG that $A$ and $B$ are using the 01 basis. Now, if Eve measures
using +- basis, then the original $\ket{0}$ or $\ket{1}$ will now become
$\ket{+}$ or $\ket{-}$.

So now, when Bob measures, he will only \textit{get the original with 0.5 probability!}

We can detect the presence of someone who is intercepting the channel
with this principle.


So, there will be ~25 percent chance that $s_i \neq s_i'$ when $r_i = r_i'$, if
Eve is eavesdropping. If not, then $s_i \neq s_i' \implies r_i \neq r_i'$.

Step 4: Randomly sample some subset of the $r_j$ for those indeces with $s_j = s_j'$.
If eavesdropper did not exist, then all the $r_j$  will match. If eavesdropper
exists, then some values will be mismatched.

This will let us \textit{detect} the presence of an eavesdropper.

So now, if the eavesdropper is not there, the rest of the
\textit{unpublished $r_k$'s} where $s_k = s_k'$ become
our secret key. This is shared, since we know that this will not
be corrupted, due to the lack of an eavesdropped. We also know that the
values will be equal since we use the measurement basis ($s_k = s_k'$)


\subsection{Why can't eve keep a copy of the qubits?}
In theory, Eve could have tried to keep a copy of the qubits, and then measured
once the $s_i$ have been published. However, no-cloning prevents this from happening.

\section{Dealing with noise}

On a noisy quantum channel, we will need to deal with the case that possibly
$s_i = s_i'$, but $r_i \neq r_i'$.


\section{Dealing with noise}

On a noisy quantum channel, we will need to deal with the case that possibly
$s_i = s_i'$, but $r_i \neq r_i'$. We can use classical error correction on
the $r_i$'s to deal with this stuff. We do not go this analysis in the lecture.


\section{Two qubit systems}

four classical possible outcomes: $\ket{00}$, $\ket{01}$, $\ket{10}$, $\ket{11}$.
Superposition of all four is allowed.

$\ket{\psi} = a_{00} \ket{00} + a_{01} \ket{01} + a_{10} \ket{10} + a_{11} \ket{11}$.

\section{$n$ qubit system}
$n$ qubit system will have $2^n$ classical outcomes - we will need $2^n$ complex amplitudes.
On the other hand, for am $n$ bit system, we will need, well, $n$ bits.

\section{Why quantum computer?}

Note that a quantum computer runs $2^n$ computations in parallel, samples one of
them (by nature's choice), and deletes all $2^n - 1$ values.

the argument is that, for a classical computer, even \textit{deleting}
$2^n - 1$ values will take exponential time. However, we can at least
exploit this "deleting" property.


\end{document}

\documentclass[11pt]{article}
%\documentclass[10pt]{llncs}
%\usepackage{llncsdoc}
\usepackage{amsmath}
\usepackage{graphicx}
\usepackage{makeidx}
\usepackage{algpseudocode}
\usepackage{algorithm}
\evensidemargin=0.20in
\oddsidemargin=0.20in
\topmargin=0.2in
%\headheight=0.0in
%\headsep=0.0in
%\setlength{\parskip}{0mm}     
%\setlength{\parindent}{4mm}
\setlength{\textwidth}{6.4in}
\setlength{\textheight}{8.5in}
%\leftmargin -2in
%\setlength{\rightmargin}{-2in}
%\usepackage{epsf}
%\usepackage{url}
\usepackage{epsfig}
\usepackage{tabularx}
\usepackage{latexsym}
\newtheorem{lemma}{Lemma}
\newtheorem{observation}{Observation}
\newtheorem{proof}{Proof}
\newcommand\ddfrac[2]{\frac{\displaystyle #1}{\displaystyle #2}}

\def\qed{$\Box$}
\def\proof{\textit{Proof. }}
\newtheorem{corollary}{Corollary}
\newtheorem{theorem}{Theorem}

\begin{document}
\section{Cryptocurrencies}

\section{Why are they impossible? (If they were possible, would not be part of the course!)}

Consider some cash. Once we spend it, we don't \textit{own it anymore}.
Copying physical currency is difficult, expensive, and innacurate.
Hence, double spending is difficult.

Copying digital media is possible. Hence, we can construct a central authority to keep
track of money. However, this causes us to lose out on decentralisation, and also lose out on
annonymity.


In a P2P system however, since agreement is ``eventual'', we can construct scenarios of
double spending. This is a famous problem, apparently. 


\section{Idea behind cryptocurrencies}

All transactions are public. However, how does this solve double spending?

\section{Solving the problem: Blockchains}

1. Transactions are in chains (time series). A change at $t=i$ needs to make a change
for all $i < t0 \leq \text{len(chain)}$. That is, a change in the middle must cascade
till the end of the chain.

2. Chain must grow with new transactions.

Note that this is already making double spending difficult. If someone wants to mess with an older
transaction, he has to be faster than the rest of the world.

3. Takes a 'lot of work' to make a change.

4. Longest chain is the honest chain.

This makes double spending somewhat impossible: Changing the chain is now very hard. Since the fastest growing
chain represents ``truth'', an adversary will need to change the *longest* chain, as the longest chain grows.


Notice that it makes sense for the longest chain to be the honest chain, since in general, the longest chain is the
one that most people in the distributed system agree on (or you have access to more computational power than everyone else *combined*).


The proof-of-work is critical. We need proof that someone spent a lot of time solving a problem. So, we use
crypto puzzles for this. (verification is easy, solving is difficult. So, some NP problem).

People who add to the ledger are termed ``miners''.
A miner can add an arbitrary transaction to the blockhain, by being allowed to add a transaction that ``creates money''.
That is, he can add a transaction that says ``add 5 BTC to my account''. The block will need to contain proof-of-work,
as well as a link to the previous block, and a chain of transactions.


The current proof-of-work time is ~10 minutes for the entire network.


At any stage, the network will probably fork, for, say, 1 block. However, the probability of the network agreeing on a longest chain
is very high.

\section{How do blockchains solve our problems?}

When $A$ sends some money to $B$, $B$ waits for the transaction to appear in some longest chain. Once the transaction appears,
$B$ will render its services to $A$.

$A$ cannot double spend, since everyone knows that $A$ has sent money to $B$.

To invalidate the transaction $A \rightarrow B$, then $A$ needs to fork the chain before the transaction and establish a new
longesst chain. This is impossible since $A$ will be then competing against everyone else.


\section{Details of blockchain}



Notice that there is still a single point of failure: the CA that maintains our digital signatures.

\section{Possible solution that would be environmentally friendly}

We have a way to build secure computers on top of insecure computers. IT will probably be a hybrid of proof-of-stake
and proof-of-work.


There will be many miners. All miners will not be wasting effort. They will be involved in simulating one machine.
This abstract machine will choose which the next miners will be. Since we pick the next miners that will be a
subset of the previous miners, we won't waste work. However, the choice will be done in a way we that no one can
influence who the next person is.

Envionmental friendliness is not a lasting problem.

\end{document}

\documentclass[11pt]{article}
%\documentclass[10pt]{llncs}
%\usepackage{llncsdoc}
\usepackage{amsmath}
\usepackage{graphicx}
\usepackage{makeidx}
\usepackage{algpseudocode}
\usepackage{algorithm}
\evensidemargin=0.20in
\oddsidemargin=0.20in
\topmargin=0.2in
%\headheight=0.0in
%\headsep=0.0in
%\setlength{\parskip}{0mm}     
%\setlength{\parindent}{4mm}
\setlength{\textwidth}{6.4in}
\setlength{\textheight}{8.5in}
%\leftmargin -2in
%\setlength{\rightmargin}{-2in}
%\usepackage{epsf}
%\usepackage{url}
\usepackage{epsfig}
\usepackage{tabularx}
\usepackage{latexsym}
\newtheorem{lemma}{Lemma}
\newtheorem{observation}{Observation}
\newtheorem{proof}{Proof}
\newcommand\ddfrac[2]{\frac{\displaystyle #1}{\displaystyle #2}}

\def\qed{$\Box$}
\def\proof{\textit{Proof. }}
\newtheorem{corollary}{Corollary}
\newtheorem{theorem}{Theorem}

\begin{document}
\section{Bit commitment and zero knowledge proofs}

Today's impossibility: blinding and yet binding: we need a binding decision to be made, but the
value that is bound is hidden.


\section{Bit commitment}

Commit phase: 
* Message is hidden.
* Any change to the message is detectable.

Reveal phase:
* message is publicly revealed.
* If we reveal something that is not what we committed, it is detectable.

\section{Zero Knowledge proof}
Assume we can solve bit-commitment.

``I want to prove I know something, but I don't want to reveal what I know''.

Example, Rand Co. We sell random numbers. Our client goes to court, saying that - how do I know that this
company is actually selling random numbers? The client's ulterior motive is to expose the program used by Rand Co.

Court would say: Can you prove that you perform random number generation, without revealing the source code?


\subsection{Zero knowledge with bit commitment}
We can prove that we differentiate taste from 1L bottle from that of 2L bottle, without revealing how.

Proof: I enter a locked room. I cannot see outside. I'll ask you to fill a glass of bottle with either 1L
or 2L. I tell what the water is by drinking it. If I get it correct $n$ times, the odds of that happening was $0.5^n$,
which is astronomically low to happen by luck.


\subsection{Zero knowledge proof for graph 3-coloring}

We now provide a zero knowledge proof system for graph 3-coloring.
since this is NP-complete, we have a ZKP protocol for all NP.

There is a graph $G$. Prover $P$ tells us that there is a 3-coloring of this graph. We, the
verifier $V$ wishes to be convinced of this, but $P$ does not wish to reveal the coloring.

We use the following tools:

1. Random permutation of vertices (Graph $H$ that is isomorphic to $G$).
2. $P$ \textit{commits} the colors of all vertices on the permutation graph, $H$.
3. $V$ asks to reveal the colour of end end points of any one random edge.
With probability of \frac{1}{|E|}, $P$ gets caught.
4. $V$ checks if the edge is properly colored (end vertices have different colors).
5. If this works, then repeat.

Lemma: IP = PSPACE =? ZKP.




\end{document}



\end{document}
