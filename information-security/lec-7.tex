\chapter{Probabilistic encryption}

Determinisim fucks over security. Since now-a-days, servers encrypt pretty much
everything you send them, you can try to mount a chosen plaintext attack.


\section{Truly random functions} 
Look at all functions from r to x. Pick one such function and use that. Number
of such functions: $2^{n^{2^n}}$ 

Number of bits to index this set: $$\log
(2^{n^{2^n}}) = 2^n log(2^n) = n \cdot 2^n$$

\section{Pseudorandom Function (PRF)}
We need distributions on functions. We define this by using keyed functions.

\begin{align*}
F: (k: \binary^n) \to (r : \binary^n) \to (x: \binary^n)
\end{align*}

firstst string is key, second string is what to encode, output is encoded.
In general, we fix a key, and then consider the function $F_k$. We assume
that $F_k$ is efficent. That is, there is a deterministic polynomial time
algorithm that can compute $(F_k(x)~\forall k, x)$.

Intuitively $F$ is called a pseudorandom function if the function $F_k$
for a randomly chosen $k$ is indistinguishable from a random function chosen
from the set of all functions having that domain and range. 

Note that the space of all functions is $(2^n)^{2^n}$, while the space of keys
is just $(2^k)$.


If we have key size as $n \cdot 2^n$, then $F_k$ (the kth function in the set
of all TRFS from r to x) will be truly random.

We formally define them as:
\begin{itemize}
\item Efficiency of computation:  given $x$, computing $f_k(x)$ is easy.
\item Pseudorandomness: for all PPTM A, 

\begin{align*}
|P [ A^{f_k(\cdot)} = 1 ] - P [ A^{f_n(\cdot)} = 1 ]| \leq \negl(n)
\end{align*} 
where $k \leftarrow \binary^k$, is a key that is chosen uniformly at
random from the key space, and $f_n \leftarrow (\binary^n \rightarrow \binary^n)$ 
is chosen uniformly at random from the space of functions.
\end{itemize}

\section{CPA security from pseudorandom functions}
Let $F$ be a pseudorandom function. Define an encryption scheme as follows:
\begin{itemize}
\item $\gen \equiv k \leftarrow \binary^n$. Choose a key at random
\item $\enc(m) \equiv ( r, F_k(r) \xor m)$
\item $\dec((r, c)) \equiv f_k(r) \xor c$.
\end{itemize}

This as seen before is CPA secure, but is problematically length doubling.

\section{Pseudorandom permutations}
A pseudorandom permutation is much like a pseudorandom function, except it
is bijective, and there is a polynomial time algorithm to compute both
$F_k(\cdot)$ and $F_k^{-1}(\cdot)$.

\begin{definition}
    Let $F: \binary^* \times \binary^* \rightarrow \binary^*$ be an efficent
    keyed permutation. We call $F$ a pseudorandom permutation if for all
    PPTM $D$, there exists a negligible function $\negl$ such that:
    \begin{align*}
        \left| \pr{D^{F_k(\cdot), F_k^{-1}(\cdot)} = 1} - 
        \pr{D^{f_n(\cdot), f_n^{-1}(\cdot)} = 1} \right| \leq \negl(n)
    \end{align*}
    Where $k \leftarrow \binary^n$ is chosen uniformly at random, and $f_n$
    is chosen uniformly at random from the set of all permutations of n
    bit strings.
\end{definition}

\section{Modes of operation}
A mode of operation is essentially a way to encrypt arbitrary length
messages using a block cipher. We will see methods that have better
\emph{ciphertext expansion}: The ratio between  the length of the message
and the length of the ciphertext.

\section{Cipher Block Chaining}

Like the name says, chain blocks for messages. we perform $c_k = F_k(m_k XOR
c_{k - 1})$. This creates a chain of dependences.

\section{Output feedback mode} $r_1 = public$ $r_k = f_k(r_{k - 1})$ $c_k = m_k
XOR r_k$


$G(x) = G_0(x) || G_1(x)$



