% // https://tex.stackexchange.com/questions/59702/suggest-a-nice-font-family-for-my-basic-latex-template-text-and-math
\documentclass[11pt]{book}
\usepackage[sc,osf]{mathpazo}   % With old-style figures and real smallcaps.
\linespread{1.025}              % Palatino leads a little more leading

% Euler for math and numbers
\usepackage[euler-digits,small]{eulervm}

%\documentclass[10pt]{llncs}
%\usepackage{llncsdoc}
\usepackage{amsmath,amssymb}
\usepackage{amsthm}
\usepackage{graphicx}
\usepackage{makeidx}
\usepackage{algpseudocode}
\usepackage{algorithm}
\usepackage{listing}
\usepackage{comment}
\evensidemargin=0.20in
\oddsidemargin=0.20in
\topmargin=0.2in
%\headheight=0.0in
%\headsep=0.0in
%\setlength{\parskip}{0mm}
%\setlength{\parindent}{4mm}
\setlength{\textwidth}{6.4in}
\setlength{\textheight}{8.5in}
%\leftmargin -2in
%\setlength{\rightmargin}{-2in}
%\usepackage{epsf}
%\usepackage{url}



\usepackage{booktabs}   %% For formal tables:
                        %% http://ctan.org/pkg/booktabs
\usepackage{subcaption} %% For complex figures with subfigures/subcaptions
                        %% http://ctan.org/pkg/subcaption
\usepackage{enumitem}
\usepackage{minted}
%\newminted{fortran}{fontsize=\footnotesize}

\usepackage{xargs}
\usepackage[colorinlistoftodos,prependcaption,textsize=tiny]{todonotes}

\usepackage{hyperref}
\hypersetup{
    colorlinks,
}

\usepackage{epsfig}
\usepackage{tabularx}
\usepackage{latexsym}
\newcommand\ddfrac[2]{\frac{\displaystyle #1}{\displaystyle #2}}

\def\qed{$\Box$}

%\newcommand{\NP}{\texttt{NP}}
%\newcommand{\PSPACE}{\texttt{PSPACE}}
%\newcommand{\NPSPACE}{\texttt{NPSPACE}}
%\newcommand{\TQBF}{\texttt{TQBF}}

\newcommand{\nptime}{\texttt{NP}}
\newcommand{\ptime}{\texttt{P}}
\newcommand{\coptime}{\texttt{co-P}}
\newcommand{\conptime}{\texttt{co-NP}}
\newcommand{\dspace}{\texttt{DPSACE}}
\newcommand{\pspace}{\texttt{PSPACE}}
\newcommand{\logspace}{\texttt{L}}
\newcommand{\nlogspace}{\texttt{NL}}
\newcommand{\conlogspace}{\texttt{co-NL}}
\newcommand{\ph}{\texttt{PH}}
\newcommand{\pathproblem}{\texttt{PATH}}
\newcommand{\copathproblem}{\overline{\texttt{PATH}}}
\newcommand{\clique}{\texttt{CLIQUE}}
\newcommand{\maxclique}{\texttt{MAXCLIQUE}}
\newcommand{\sat}{\texttt{SAT}}
\newcommand{\phtime}{\texttt{PH}}
\newcommand{\ppoly}{$\texttt{P}^{\texttt{poly}}$}
\newcommand{\pathbar}{$\overline{\texttt{PATH}}$}
\newcommand{\problempath}{\texttt{PATH}}
\newcommand{\ip}{\texttt{IP} }
\newcommand{\rp}{\texttt{RP} }
\newcommand{\corp}{\texttt{co-RP} }
\newcommand{\zp}{\texttt{ZP} }
\newcommand{\maxsat}{\texttt{MAXSAT} }

\newtheorem{theorem}{Theorem}
\newtheorem{corollary}[theorem]{Corollary}
\newtheorem{definition}[theorem]{Definition}
\newtheorem{lemma}[theorem]{Lemma}
\newtheorem{observation}[theorem]{Observation}
%\newtheorem{proof}[theorem]{Proof}

\renewcommand{\H}{\ensuremath{H}}
\newcommand{\E}{\ensuremath{\mathbb{E}}}
\newcommand{\R}{\ensuremath{\mathbb{R}}}
\newcommand{\N}{\ensuremath{\mathbb{N}}}
\newcommand{\Z}{\ensuremath{\mathbb{Z}}}
\newcommand{\KL}[2]{\ensuremath{D(#1 || #2)}}
\newcommand{\I}{\ensuremath{I}}

\title{Information theory}
\author{Siddharth Bhat}
\date{}

\begin{document}

\maketitle
\tableofcontents

\section{Preliminary definitions}

\begin{definition}
    Entropy(\H): The entropy of a random variable $X$ with probability
    distribution $p: X \rightarrow \R $ is defined as:
    \begin{align*}
        \H(X) \equiv - \sum_{x \in X} p(x) \log p(x) = \E[- \log \circ p]
    \end{align*}
\end{definition}

\begin{definition}
    Condtional entropy($\H(X|Y)$): The conditional entropy of a random 
    variable $X$ with respect to another variable $Y$ is defined as:
    \begin{align*}
        \H(X|Y) \equiv &- \sum_{y \in Y} p(y) \H(X|Y=y) \\
                       &= \sum_{y \in Y} p(y) \sum_{x \in X} - p(x|y) \log p(x | y)\\
                       &= \sum_{y \in Y} \sum_{x \in X} - p(y) p(x|y) \log p(x | y)\\
                       &= \sum_{y \in Y} \sum_{x \in X} - p(y \land x) \log p(x | y)\\
    \end{align*}
\end{definition}

\begin{definition}
    Kullback-Leibler divergence \KL{X}{Y}: The Kullback-Leibler divergence
    of $X \sim p $ with respect to $X' \sim q$ is:
    \begin{align*}
        \KL{X}{X'} \equiv \sum_{x \in X} p(x) \log \frac{p(x)}{q(x)}
    \end{align*}
\end{definition}

Note that \KL{X}{X'} is \emph{not symmetric}.

Intuition: extra cost of encoding $X$ if we thought the distribution were $X'$.

Useful extremal case to remember: Assume $X'$ has $q(x) = 0$ for some
letter $x \in X$. In this case, $\KL{X}{X'}$ would involve a term $\frac{p(x)}{0}$,
which is $\infty$. This is intuitively sensible, since $X'$ has no way to represent
$x$, and hence $X'$ is \textit{infinitely far away from encoding $X$}. However,
In this same case, one could have that $X$ is able to encode all of $X'$.

\begin{definition}
    Mutual information: \I(X;Y): This is the relative entropy between the
    joint and product distributions.

    \begin{align*}
        \I(X;Y) &\equiv \KL{p(x, y)}{p(x)p(y)} \equiv H(X) - H(X|Y) \\
                &= \sum_x \sum_y p(x, y) \log \frac{p(x, y)}{p(x)p(y)}
    \end{align*}
    
\end{definition}

\begin{theorem}
    Proof of equivalence of two definitions of mutual information:
\end{theorem}
\begin{proof}
    \begin{align*}
        &\I(X;Y)  = \sum_x \sum_y p(x, y) \log \frac{p(x, y)}{p(x)p(y)} \\
        &= \sum_x \sum_y p(x, y) \log \frac{p(x|y)}{p(x))} \\
        &= - \sum_x \sum_y p(x, y) \log p(x) - \big[ - \sum_x \sum_y p(x, y) \log p(x|y) \big] \\
        &= - \sum_x p(x) \log p(x) - \big[ - \sum_y \sum_x p(x, y) \log p(x|y) \big] \\
        &= - \sum_x p(x) \log p(x) - \big[ - \sum_y p(y) \sum_x \log p(x|y) \big] \\
        &= \H(X) - H(X|Y)
    \end{align*}
\end{proof}


Some notes about mutual information:
\begin{itemize}
    \item $I(X; Y) = I(Y;X)$. That is, $I$ is symmetric.

    \item Since $I(X;Y) = H(X) - H(X|Y)$, one can view it as the reduction
        in \textit{uncertainity} of $X$, after knowing $Y$. Another way of
        saying this is, what is the expected reduction in the number of
        yes/no questions to be answered to isolate the value of $X$ on knowing
        the value of $Y$.

    \item $I(X;Y) = 0$ iff $X, Y$ are independent. That is, knowing $X$ reduces
        no uncertainity about $Y$.

    \item $I(X;X) = H(X)$. So, knowing $X$ allows us to reduce our uncertainity
        of $X$ by $H(X)$. ie, we completely know $X$, since we have
        \textit{reduced our uncertainity of X} which was initially $H(X)$, by $H(X)$.

\end{itemize}

\begin{theorem}
    Chain rule for entropy: Let $X_1, X_2, \dots X_n \sim p(x_1, x_2, \dots, x_n)$
    Then:
    \begin{align*}
        \H(X_1, X_2, \dots X_n) = \sum_i \H(X_i | X_{i-1}, X_{i-2}, \dots X_1)
    \end{align*}
\end{theorem}
\begin{proof}
    \begin{align*}
    &\H(X_1, X_2) = \H(X_1) + \H(X_2 | X_1) \\
    &\H(X_1, X_2, X_3) = \H(X_1) + \H(X_2, X_3 | X_1) = \H(X_1) + H(X_2 | X_1) + H(X_3 | X_2, X_1) \\
    &\dots
    \end{align*}
\end{proof}
\end{document}
