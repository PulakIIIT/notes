\documentclass[11pt]{book}
%\documentclass[10pt]{llncs}
%\usepackage{llncsdoc}
\usepackage[sc,osf]{mathpazo}   % With old-style figures and real smallcaps.
\linespread{1.025}              % Palatino leads a little more leading
% Euler for math and numbers
%\usepackage[euler-digits,small]{eulervm}
\usepackage{physics}
\usepackage{amsmath,amssymb}
\usepackage{graphicx}
\usepackage{makeidx}
\usepackage{algpseudocode}
\usepackage{algorithm}
\usepackage{listing}
\usepackage{minted}
\usepackage{cancel}
\evensidemargin=0.20in
\oddsidemargin=0.20in
\topmargin=0.2in
%\headheight=0.0in
%\headsep=0.0in
%\setlength{\parskip}{0mm}
%\setlength{\parindent}{4mm}
\setlength{\textwidth}{6.4in}
\setlength{\textheight}{8.5in}
%\leftmargin -2in
%\setlength{\rightmargin}{-2in}
%\usepackage{epsf}
%\usepackage{url}

\usepackage{booktabs}   %% For formal tables:
                        %% http://ctan.org/pkg/booktabs
\usepackage{subcaption} %% For complex figures with subfigures/subcaptions
                        %% http://ctan.org/pkg/subcaption
\usepackage{enumitem}
%\usepackage{minted}
%\newminted{fortran}{fontsize=\footnotesize}

\usepackage{xargs}
\usepackage[colorinlistoftodos,prependcaption,textsize=tiny]{todonotes}

\usepackage{hyperref}
\hypersetup{
    colorlinks,
    citecolor=blue,
    filecolor=blue,
    linkcolor=blue,
    urlcolor=blue
}

\usepackage{epsfig}
\usepackage{tabularx}
\usepackage{latexsym}
\newcommand\ddfrac[2]{\frac{\displaystyle #1}{\displaystyle #2}}
\newcommand{\N}{\ensuremath{\mathbb{N}}}
\newcommand{\R}{\ensuremath{\mathbb R}}
\newcommand{\boldX}{\ensuremath{\mathbf{X}}}
\newcommand{\boldY}{\ensuremath{\mathbf{Y}}}
\newcommand{\B}{\ensuremath{\mathcal{B}}}
\newcommand{\Bomega}{\ensuremath{\mathcal{B}^\omega}}

% \newcommand{\braket}[2]{\ensuremath{\left\langle #1 \vert #2 \right\rangle}}


\def\qed{$\Box$}
\newtheorem{corollary}{Corollary}
\newtheorem{theorem}{Theorem}
\newtheorem{definition}{Definition}
\newtheorem{lemma}{Lemma}
\newtheorem{observation}{Observation}
\newtheorem{proof}{Proof}
\newtheorem{remark}{Remark}
\newtheorem{example}{Example}

\title{Software Foundations}
\author{Siddharth Bhat}
\date{Spring 2020}

\begin{document}
\maketitle
\tableofcontents

\chapter{Introduction}

Let $S = \langle X_s, X_s^0, U_s, \xrightarrow{s}, Y_s, h_s \rangle$ be
a transition system.

A \emph{finite run} originating from $x_0 \in X^0$  is a finite sequence
$(x_0, u_0, x_1, u_1, \dots u_{n-1}, x_n)$ such that $\forall i \in \{0, \dots, n-1\},~ x_i \xrightarrow{s}_{u_i} x_{i+1}$.
This is sometimes denoted as $x_0 \xrightarrow{s}_{u_0} x_1 \xrightarrow{s}_{u_1} \dots \xrightarrow{s}_{u_{n-1}} x_n$.
In the notation of Tabuada, this is called as the \emph{finite internal behaviour}.
A run has information about both states and transitions.

$\langle x_0, x_1, \dots, x_n \rangle$ is a \emph{finite trajectory} iff
$\exists u_0, \dots u_{n-1}$ such that $x_0 \xrightarrow{u_0} x_1 \xrightarrow{u_1} \dots \xrightarrow{u_{n-1}} x_n$
is a finite run. The trajectory has information only about states.

$\langle y_0, y_1, \dots, y_n \rangle$ is a \emph{finite trace} iff
there exists a finite trajectory $\langle x_0 \dots x_n \rangle$ and
$\forall i \in \{0, \dots, n \}, ~y_i = h_s(x_i)$. The finite trace has 
information only about projections of a state.

The \emph{finite behaviour} of a system $S$ is defined as the union of all
finite traces of $S$. This is notated as $\B(S)$.

The infinite behaviour of a system $S$ is the union of all infinite traces
of $S$, notated as $\Bomega(S)$.

The questions we are interested in answering are:
\begin{itemize}
    \item What is the algebra of systems? How do we compose systems?
    \item Can we look at the definition of a system and predict its behaviour?
        (Simulation and Bisimulation). This is different from extensionally
        looking at traces of the system.
\end{itemize}

\chapter{System composition}

\begin{itemize}
    \item Run: sequences of internal states 
    \item Trace: sequences of external states 
\end{itemize}

\begin{definition}
    Let $A$ and $B$ be two transition systems. The interconnect
    $\mathcal{I}$ between $A$ and $B$ is any relation between
    $x_a, u_a, x_b, u_b$
    ($u_a, u_b$ are the action spaces of $A$ and $B$):
    $$
    \mathcal I \subseteq X_A \times U_A \times X_B \times U_B
    $$
\end{definition}


If we want to feed both systems the same input, we can have 
$\mathcal I \equiv U_a = U_b$.

If we want to have both systems produce the same output, then we should have
$\mathcal I \equiv h_a(x_a) = h_b(x_b)$

Intuitively, the relation $\mathcal I$ provides constraints on the direct product
of the two systems that we want to enforce.


If we want to have feedback,
$\mathcal I \equiv u_b = h_a(x_a) \land u_a = h_b(x_b)$

\begin{definition}
    If $A, B$ are systems and $\mathcal{I}$ is the interconnect, then
    we have $A \times_{\mathcal I} B \equiv \langle X_c, X_c^0, U_c, \xrightarrow{c}, Y_c, h_c \rangle$:

    \begin{align*}
        X_c &\equiv \{ (x_a, x_b) : 
          x_a \in X_a, x_b \in X_b \land  (\exists u_a \in U_A, u_b \in U_B, 
          (x_a, u_a, x_b, u_b) \in \mathcal{I}) \} \\
        %
        X_c^0 &\equiv \{ (x_a, x_b) \in X_c  : 
          x_a \in X_a^0, x_b \in X_b^0 \} \\
        %
    U_c &\equiv U_a \times U_b \\
    (x_a, x_b) \xrightarrow{(u_a, u_b)} (x_a', x_b') &\equiv 
    x_a \xrightarrow{u_a} x_a', 
    x_b \xrightarrow{u_b} x_b', 
    (x_a, u_a, x_b, u_b) \in \mathcal{I} \\
    %
    h_c((x_a, x_b)) &\equiv (h_a(x_a), h_b(x_b))
    \end{align*}
\end{definition}
Question: why not define 
$U_c \equiv \{ (u_a, u_b) : 
\exists x_a \in X_a, x_b \in X_b, (x_a, u_a, x_b, u_b) \in \mathcal I$

\chapter{Control}

We have a classical plant-controller system, as in control theory. We can
model this as:
$$ x_c = f_c(x_p) \qquad x_p' = f_p(x_p, x_c) $$

That is, we have $x_p^0$. We find $x_c^0 = f_c(x_p^0)$. Then, 
$x_p^1 = f_p(x_p^0, x_c^0),x_c^1 = f_c(x_p^1)$ and the whole process continues.

\section{Modelling loops} 
Note that we can model a \texttt{while} loop as a pure controller (the predicate)
controlling the body of the function.  
\begin{verbatim}
while (C(x)) { x = F(x) }
\end{verbatim}
This becomes, as a feedback system:
$$
F'(c, x) = \begin{cases} F(x) & \text{if $c = 1$} \\ x & \text{otherwise} \end{cases}
\qquad
C(x) = \dots
$$

\section{Modelling loops with external feedback} 
\begin{verbatim}
while (C(x, e)) { x = F(x) }
\end{verbatim}
$$ x_c = f_c(x_p, e) \qquad x_p' = f_p(x_p, x_c) $$
\end{document}
