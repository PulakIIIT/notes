% Section 3.5 (Q1 and Q2)
% Section 4.5 (Q1)
% Section 6.8 (Q1)


\documentclass[11pt]{article}
%\documentclass[10pt]{llncs}
%\usepackage{llncsdoc}
\linespread{1.025}              % Palatino leads a little more leading
\usepackage{physics}
\usepackage{amsmath,amssymb}
\usepackage{graphicx}
\usepackage{makeidx}
\usepackage{algpseudocode}
\usepackage{algorithm}
\usepackage{listing}
\usepackage{minted}
\usepackage{tikz}
\usepackage{bbold}
\evensidemargin=0.20in
\oddsidemargin=0.20in
\topmargin=0.2in
%\headheight=0.0in
%\headsep=0.0in
%\setlength{\parskip}{0mm}
%\setlength{\parindent}{4mm}
\setlength{\textwidth}{6.4in}
\setlength{\textheight}{8.5in}
%\leftmargin -2in
%\setlength{\rightmargin}{-2in}
%\usepackage{epsf}
%\usepackage{url}

\usepackage{booktabs}   %% For formal tables:
                        %% http://ctan.org/pkg/booktabs
\usepackage{subcaption} %% For complex figures with subfigures/subcaptions
                        %% http://ctan.org/pkg/subcaption
\usepackage{enumitem}
%\usepackage{minted}
%\newminted{fortran}{fontsize=\footnotesize}

\usepackage{xargs}
\usepackage[colorinlistoftodos,prependcaption,textsize=tiny]{todonotes}

\usepackage{hyperref}
\hypersetup{
    colorlinks,
    citecolor=blue,
    filecolor=blue,
    linkcolor=blue,
    urlcolor=blue
}

\usepackage{epsfig}
\usepackage{tabularx}
\usepackage{latexsym}
\newcommand\ddfrac[2]{\frac{\displaystyle #1}{\displaystyle #2}}
\newcommand{\E}[1]{\ensuremath{\mathbb{E} \left[ #1 \right]}}
\renewcommand{\P}[1]{\ensuremath{\mathbb{P} \left[ #1 \right]}}
\newcommand{\N}{\ensuremath{\mathbb{N}}}
\newcommand{\R}{\ensuremath{\mathbb R}}

\newcommand{\Rgezero}{\ensuremath{\mathbb R_{\geq 0}}}
\newcommand{\powerset}{\ensuremath{\mathcal{P}}}

\newcommand{\boldX}{\ensuremath{\mathbf{X}}}
\newcommand{\boldY}{\ensuremath{\mathbf{Y}}}


\newcommand{\G}{\ensuremath{\mathcal{G}}}
\newcommand{\D}{\ensuremath{\mathcal{D}}}
\renewcommand{\H}{\ensuremath{\mathcal{H}}}
\newcommand{\X}{\ensuremath{\mathcal{X}}}
\DeclareMathOperator{\vcdim}{VCdim}
\newcommand{\Vcdim}{\ensuremath{\vcdim}}
\newcommand{\VCdim}{\ensuremath{\vcdim}}
\newcommand{\VCDim}{\ensuremath{\vcdim}}
\newcommand{\vc}{\ensuremath{\vcdim}}
\newcommand{\VC}{\ensuremath{\vcdim}}
\newcommand{\zoset}{\ensuremath{\{0, 1\}}} % set \{0, 1\}

\def\qed{$\Box$}
\newtheorem{question}{Question}
\newtheorem{theorem}{Theorem}
\newtheorem{definition}{Definition}
\newtheorem{lemma}{Lemma}
\newtheorem{observation}{Observation}
\newtheorem{proof}{Proof}
\newtheorem{remark}{Remark}
\newtheorem{example}{Example}

\title{Probabilistic graphical models, Assignment 3}
\author{Siddharth Bhat (20161105)}
\date{March 21st, 2020}

% Assignment for VC Dimension:
% Understanding ML Textbook
% 
% Section 6.8 (Exercises of chapter 6): Problems 1, 2, 3, 5, 9
% Submit solutions to any 4 of 5 problems.


\begin{document}
\maketitle
\section*{6.8, Q1:}
Monotonicity of VC dimension

Let $\H' \subseteq \H$.  Show that $\vc(\H') \leq \vc(\H)$.

\subsubsection*{Answer}
Recall that the definition of \vc is is that \vc(\H) is the maximal size of
a set $C \subseteq \X$ which can be \emph{shattered} by \H.

Expanding the definition of shattering, we get that the \vc(\H) is the maximal size
of \emph{any} set $C \subseteq X$ such that \H~restricted to $C$ is the set of all
functions from $C$ to \{0, 1\}.

Now, If $C \subseteq \X$ is shattered by $\H' \subseteq \H$, then this means
that:

\[
|\{ f|_C : f \in H' \}| = 2^{|C|}
\]

Since $\H' \subseteq \H$, we can
replace $\H'$ with $\H$ in the above formula to arrive at:

\[
|\{ f|_C : f \in H \}| = 2^{|C|}
\]

So, clearly, $\vc(\H') \leq \vc(\H)$.
However, there might be a set that is \emph{larger} than $C$ that can be shattered
by $\H$. This lets us get the strict equality $\vc(\H) < \vc(\H)$ in certain cases
--- that is, we \emph{cannot} assert that $\vc(H) \leq \vc(H')$.
For example, if we choose $\H' = \emptyset$ where $\H$ is a hypothesis class with
$\vcdim(\H) = 1$. Then $\vcdim(\emptyset) = 0 < 1 = \vcdim(\H)$. 


\section*{6.8, Q2:}
Given a finite domain $\X$, and a finite number $k \leq |\X|$s, find and prove
the VC dimension of:

\subsection*{A. Functions that assign 1 to exactly $k$ elements of \X}
$\H \equiv \bigg\{ h \in \zoset^{X} : ~|\{x : h(x) = 1\}| = k \bigg\}$.

\subsubsection*{Solution}
VC dimension is $k/2$. 

\subsection*{B. Functions that assign 1 to at most $k$ elements of \X}


\subsubsection*{Solution}
VC dimension is $k$.


\section*{6.8, Q3:}
Let $\X$ be the boolean hypercube $\{0, 1\}^n$. We define parity to be:
$$h_I: \X \rightarrow \{0, 1\}; h_I((x_1, x_2, \dots, x_n)) \equiv \sum_{i \in I} (x_i) ~ \mod 2.$$

What is the VC dimensions of the set of all parity functions? That is,
$$\H_p \equiv \{ h_I : I \subseteq \{1, 2, \dots n\} \}$$

\subsubsection*{Solution}

Once again, unwrapping the definition, our hypothesis class can compute the
sum modulo 2 of \emph{all of the subsets of $\vec x \in \X$}. We need to
use this to find the \emph{largest} set $C \subseteq \X \equiv \{0, 1\}^n$ such
that $|\H_C| = 2^{|C|}$. 

We can interpret elements $(h \in H)$ as a vector 
$h_I \in \{0, 1\}^n$, where $h_I$ is a vector with $1$'s at each index $i \in I$,
and $0$ at other indexes. That is:

$$
h_I \in \{0, 1\}^n \qquad
h_I[i] \equiv \mathbb{1}[i \in I] = 
\begin{cases} 1 & i \in I \\ 0 & \text{otherwise} \end{cases}
$$


We can reinterpret the function $h_I(x)$ as $h^T x$ where we have
a vector space over the galois field $GF_2$, where $\oplus$ denotes XOR (recall
that addition mod 2 is XOR).

\begin{align*}
h_I(x) = \bigoplus_{i \in I} x_i = \bigoplus_{i=1}^n \mathbb{1}[i \in I] x_i = \bigoplus_{i=1}^n h_I[i] x[i] =  h_I^T x
\end{align*}


Now, we can reinterpret the question of finding the VC dimension as finding
the largest collection of vectors $C \subseteq \X = \{0, 1\}^n$ such that
the function $C_{act}$ has full image, where the function $C_{act}$ is:

\begin{align*}
&C_{act}: \H \rightarrow \{0, 1\}^{|C|} \\
&C_{act}: \{0, 1\}^n \rightarrow \{0, 1\}^n \\
&C_{act}(h) \equiv (h(c_0), h(c_1), h(c_2), \dots h(c_n)) \\
&\quad = (h^T c_0, h^T c_1, \dots, h^T c_n) \\
& \quad = h^T (c_0, c_1, \dots c_n)
\end{align*}

If we regard $(c_0, c_1, c_2, \dots c_n) \subseteq \mathbb R^{n \times |C|}$
as a matrixm then we can see that  $C_{act}$ is a \emph{linear function}.


Now, if the set $C$ shatters $\H$, then:
\begin{itemize}
\item[1] The function $C_{act}$ will produce every element in $\{0, 1\}^{|C|}$
\item[2] the function $C_{act}$ will have full image. 
\item[3] This is  only possible when the dimension of the domain is less than or equal to the dimension of the range.
\item[4] the largest set that can be shattered is the largest matrix  $C \subseteq R^{n \times |C|}$  
         such that the function $C_{act}$ has full range.
\item[5] Thus, $|C| \leq n$ for $C_{act}$ to have full range.
\item[6] We can achieve $|C| = n$ by picking $C = I_{n \times n}$. In other words,
          the element $c_i$ will be the $i$th row of the identity matrix. That
          is $c_i[j] = \mathbb{1}[i = j] = \begin{cases} 1 & \text{i = j} \\ 0 & \text{otherwise} \end{cases}$.
         Clearly, this $C$ is shattered since the function $C_{act}$ is the identity function which
         will produce every single output in $\{0, 1\}^{|C|} = \{0, 1\}^n = \{0, 1\}^{|\H|}$.         
\item[7] $|C|=n$ is the largest possible, since $C_{act}$ is a function, and
         the size of its image is at most the size of the domain. Since the domain $\H$
         has $2^n$ elements, the image too can have at most $2^n$ elements, which
         it does when $|C| = n$, since $|{0, 1}^{|C|}| = 2^{|C|}$.
\item[7.5] $|C|=n$ is the largest possible, since $C_{act}$ is linear.
            For a linear function to be surjective, we need $Dim(domain)  Dim(range)$.
            Hence, $Dim(Domain) = Dim(\H) = n \geq Dim(range) = Dim(\{0, 1\}^{|C|}) = c$.
            That is, $n \geq |C|$.
\end{itemize}

Hence, we conclude that $\vc(\H) = n$.

\section*{6.8, Q5:}
Let $\H^d$ be the class of axis-aligned bounding boxes in $\R^d$. Show that
the VC dimensions of $\H^d$ is $2d$.


\subsubsection*{Solution}
Formally, we have 
$$
h_{\vec l, \vec r}(\vec p) \equiv
\begin{cases}
1 &  l[i] \leq p[i] \leq r[i]~\text{for all $i \in \{1, 2, \dots, n \}$ } \\
0 & \text{otherwise}
\end{cases} \qquad
\H^d \equiv \left\{ h_{\vec l, \vec r}: \R^d \rightarrow \{0, 1\} \mid \forall \vec l, \vec r \in \mathbb R^d \right\}
$$

We claim that the set of points:

\begin{align*}
S &\equiv S^+ \cup S^- \\
S^+ &\equiv \{ p[i] \equiv 1, p[j \neq i] \equiv 0 : i \in [d], p \in \R^d \} \\
S^- &\equiv \{ p[i] \equiv -1, p[j \neq i] \equiv 0 : i \in [d], p \in \R^d  \} \\
\end{align*}

shatters the hypothesis class $\H^d$.

We first show that $|H^d|$ restricted to $S$ expresses all functions $S \rightarrow \{0, 1\}$.

We will proceed by induction on the dimension $n$.
In the case of $(n = 2)$, we have already shown
this as part of the course (shattering of 4 points by rectangles). We assume that this
possible for $n=d-1$. We need to show that this possible for $n = d$.

Let us say that we are currently trying to shatter a set $T$. If the points
in $T$ lie in a $(d-1)$ subspace of 


To show that a set of size $2d+1$ cannot be shatt

\end{document}
