\documentclass[11pt]{book}
%\documentclass[10pt]{llncs}
%\usepackage{llncsdoc}
\usepackage{amsmath,amssymb}
\usepackage{graphicx}
\usepackage{makeidx}
\usepackage{algpseudocode}
\usepackage{algorithm}
\usepackage{listing}
\evensidemargin=0.20in
\oddsidemargin=0.20in
\topmargin=0.2in
%\headheight=0.0in
%\headsep=0.0in
%\setlength{\parskip}{0mm}
%\setlength{\parindent}{4mm}
\setlength{\textwidth}{6.4in}
\setlength{\textheight}{8.5in}
%\leftmargin -2in
%\setlength{\rightmargin}{-2in}
%\usepackage{epsf}
%\usepackage{url}

\usepackage{booktabs}   %% For formal tables:
                        %% http://ctan.org/pkg/booktabs
\usepackage{subcaption} %% For complex figures with subfigures/subcaptions
                        %% http://ctan.org/pkg/subcaption
\usepackage{enumitem}
%\usepackage[euler-digits,small]{eulervm}
%\usepackage{minted}
%\newminted{fortran}{fontsize=\footnotesize}

\usepackage{xargs}
\usepackage[colorinlistoftodos,prependcaption,textsize=tiny]{todonotes}

\usepackage{hyperref}
\hypersetup{
    colorlinks,
    citecolor=black,
    filecolor=black,
    linkcolor=black,
    urlcolor=black
}

\usepackage{epsfig}
\usepackage{tabularx}
\usepackage{latexsym}
\newcommand{\I}{\ensuremath{\mathcal{I}} }
\newcommand{\powerset}{\mathcal{P}}
\newcommand\ddfrac[2]{\frac{\displaystyle #1}{\displaystyle #2}}

\def\qed{$\Box$}
\newtheorem{corollary}{Corollary}
\newtheorem{theorem}{Theorem}
\newtheorem{definition}{Definition}
\newtheorem{lemma}{Lemma}
\newtheorem{observation}{Observation}
\newtheorem{proof}{Proof}
\newtheorem{exercise}{Exercise}
\newtheorem{example}{Example}

\title{IIIT Theory talks}
\author{Siddharth Bhat}
\date{Monsoon 2019}

\begin{document}

\maketitle
\tableofcontents

\chapter{Matroid theory}

We follow the combinatorialists' tradition and use  the notation 
${[m] = \{ 1, 2, \dots, m \}}$.

generalization of the notion of linear independence. Goal of the talk is to 
reach a decomposition theorem of Seymour, which simplifies a "regular matroid"
into simpler matroids.

\begin{exercise}
Let $A$ be an $m \times n$ matrix, columns $[a_1, a_2, \dots, a_n]$.
Let $I_1, I_2$ be a collection of linearly independent column indeces of $A$ such
that $|I_1| < |I_2|$ (\textit{strictly} less than). Show that there exists some column index ${e \in I_2 \setminus I_1}$
such that ${I_1 \cup \{e \}}$ is a set of linearly independent columns of $A$.
\end{exercise}
\begin{proof}
    Space spanned by $I_1$ has dimension $|I_1|$. Space spanned by $I_2$
    has dimension $|I_2|$. Hence, there must be at least one vector in $I_2$
    which is independent of $I_1$. If this is not the case, then the dimension of $I_2$
    will be equal to the dimension of $I_1$.

    So we pick the index of that vector in $I_2$ that is linearly independent of $I_1$.
\end{proof}

\begin{definition}
    Let $E$ be a finite set. Let $\I \subset \powerset(E)$ such that:
    \begin{itemize}
        \item $\emptyset \in \I$
        \item  $\forall I \in \I, ~ J \subset I \implies J \in \I$ (\I is closed under subsets)
        \item Let $I_1, I_2 \in \I$ such that $|I_1| < |I_2|$. Then there
            exists an element $e \in I_2 \setminus I_1$ such that $I_1 \cup \{ e \} \in \I$
    \end{itemize}
    Then, $(E, \I)$ is called as a matroid. \I is called the collection of independent sets. An
    element of \I is called as an independence set.
\end{definition}

\begin{example}
    For a matrix $A_{m \times n}$ over an arbitrary field. Let $E$ be the set
    of column indeces. Let $\I$ be all the subset of $E$ such that the subset 
    is linearly independent. Then $(E, \I)$ is a matroid.
\end{example}


\end{document}
