\chapter{Tree processing}
\section{Traversal via an Euler tour}
\begin{definition}
an \textbf{Euler tour} is a cycle of a graph that includes every edge of the
graph exactly once.
\end{definition}

\begin{lemma}
A directed graph $G$ \textbf{has an Euler tour} iff for every vertex,its in-degree
equals its out-degree.
\end{lemma}

For a tree $T = (V, E)$, to define an euler tour, we make it a directed graph.
$T_e = (V_e, E_e)$, where $V_e = V$, and $E_e = \cup_{(u,v) \in V} \{ (u, v), (v, u) \}$
That is, each $(u, v)$ in $E$ creates two edges $(u, v)$, and $(v, u)$ in $E_e$.
$T_e$ will have an Euler tour.


We have to define a successor function $s: E_e \to E_e$. Here, the successor for an edge.
For a node $u$ in $T_e$, order its \textbf{neighbours (both incoming and outgoing)}
$v_1, v_2, \dots v_d$. This can be done \textbf{independently at each node}. 
For $e = (v_i, u)$, set $s(e) = (u, v_{i + 1~\text{mod $d$}})$. This choice of $s$ is valid
since we always have both edges $(x, y)$ and $(y, x)$, and we are therefore
assured that $(v_i, u)$ will be an incoming edge, and $(u, v_{i + 1}$ will be
an outgoing edge. Also, compute $i + 1$ modulo $d$, so that we eventually
cycle.

\textbf{TODO: relabel vertices to $[0..(d - 1)]$ so that modulo works properly}
\textbf{TODO: add example}

\begin{theorem}
$s$ actually constructs a tour.
\end{theorem}
\begin{proof}
Induction on number of vertices. If $n = 1$, obviously true. 
If $n = 2$, at most one edge present. We will go along the edge and come back,
which is a valid tour.


\begin{itemize}
\item Let the tour be well defined for $n = k$. We will prove it for $n = k + 1$.
\item Every tree has at least one leaf, call it $l$. Create a tree $T' = T/\{l\}$.
\item Let $u$ be a neighbour of $l$ in $T$.
\item Let $N(u) = \{ v_0, v_1, \dots v_i = l, v_{i + 1}, \dots v_d \}$.
\item Set $s_{new}(u, v) \equiv (v, u)$. Set $s_{new}(v_{i - 1}, u) \equiv (u, v)$.
\item For all other vertices, $s_{new}(e) = s(e)$.
\end{itemize}
\end{proof}


\section{Using euler tours for traversal}
Operations on a tree such a rooting, preorder, and postorder traversal
can be converted to routines on an Euler tour.

\subsection{Rooting a tree}
Designate a node in a tree as the root. All edges in the tree are
directed towards (or away) from the root.

\begin{itemize}
\item let $\{v_1, v_2, \dots v_d\}$ be the neighbours of root node $r$. 
\item we mark the final edge of the tour as \texttt{NIL}, so we get an
Euler path, and not an Euler tour.
\item the edge $(r, v_i)$ appears before $(v_i, r)$.
\item so the edge $parent \to child$ appears before $child \to parent$
\item So, if $uv$ precedes $vu$, then set $u = parent(v)$. Orient the
edge $uv$ as $v \to u$ (that is, $child \to parent$), since we want all edges towards the root.
\end{itemize}

\subsection{Preorder traversal}
We have a rooted tree with $r$ as the root. In a preorder traversal, a node is
listed before any of the nodes in its subtrees.

In an Euler tour, nodes in a subtree are visited by entering subtrees,
and the exiting towards the parent.

If we can track the first occurence of a node in an euler path, this will
tell us the preorder traversal. Note that edges in the euler tour occur
first as $parent \to child$, and later as $child \to parent$. So, we can
look at the sequence of edges in the euler tour, and find the preorder
numbering.

\subsection{Expression tree evauation of binary trees}
Tree may not be balanced.

We use the \texttt{RAKE} technique to evaluate subexpressions. We rake the
leaves from the expression tree --- we remove the leaf node and its parent.

\begin{itemize}
\item $T = (V, E)$ is a tree rooted at root node $r$. $p: E \to E$ is the 
parent function.
\item One step of the rake operation at a leaf $l$ with $p(l) \neq r$ involves:
    \begin{itemize}
    \item Remove node $l$, $p(l)$ from the tree
    \item Make the sublings of $l$ as the child of $p(p(l))$. That is, graft
    the siblings of $l$ to the grandparent of $l$.
    \end{itemize}
\end{itemize}

Why is this a good technique? Can this be applied in parallel to several leaf
nodes? Yes, it can be applied to leaf nodes that don't share the same parent.
In general, there is a richness of leaf nodes in a tree, since there
are only $n - 1$ edges.

Each application of rake at all leaves reduces the  number of leaves by half.
Each application of \texttt{RAKE} is $O(1)$. So, total time is $O(\log n)$.


\begin{minted}{python}
def shrinkTree(R):
    compute labels for leaf nodes, store in array A (exclude leftmost
    and rightmost nodes in this A)

    for _ in range(k):
        apply rake operation to all odd numbered leaves that are
        the *left* children of their parent

        apply rake operation to all odd numbered leaves that are
        the *right* children of their parent

        update A to be the remaining even leaves
\end{minted}

Applying \texttt{Rake} means that we can process more than one leaf node
at the same time.

Fo expression evaluation, this may mean that an internal node with 
only one operand gets raked.

\begin{minted}{python}
          + g(u)
  
    + p(u) 

Y     X (u)


--After raking--

   + g(u)
Y
\end{minted}

\begin{itemize}
    \item Transfer the impact of applying the operaot at p(u) to the sibling of u
    \item $R_u = a_u X_u + b_u$
    \item $X_u$ is the result of the subexpression at node $u$ -- $X_u = f(left, right)$
    \item adjust $a_u$ and $b_u$ during any rake operation appropriately
    \item Initially, at each leaf node, $a_u = 1, b_u = 0$.
\end{itemize}



\begin{minted}{python}
          + g(u)
  
    + p(u) 
    X_w
    a_w
    b_w

v        (u) 5, 1, 0
X_v,
a_v,
b_v

--After raking--

     + g(u)
v
X_v'
a_v'
b_v'
\end{minted}

\begin{itemize}
    \item Before removing $p(u)$, the contribution of $p(u)$ to $g(u)$ will be $X_w a_w + b_w$.
    \item we want what $p(u)$ used to calculate to be what $v$ calculates after.
    \item $X_w = (X_u a_u + b_u) + (X_v a_v + b_v)= (X_v a_v) + (X_u a_u + b_u + b_v)$
    \item What $p(u)$ used to calculate is: $a_w X_w + b_w = a_w (a_v X_v + a_u x_u + b_u + b_v) + b_w = a_w a_v x_v + a_w (a_u X_u + b_u + b_v) + b_w$
    \item what $p(v)$ should be: $a_v' = a_w a_v$, $b_v' = a_w (\dots)$
\end{itemize}

For other operators, proceed in a similar fashion (\textbf{TODO: do this and send to kiko, he seems interested!})

