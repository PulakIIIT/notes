\chapter{\texttt{NLogSpace}-completeness}

\section{\texttt{Co-NLogSpace}}
$L \in \texttt{Co-NLogSpace} \equiv L^c \in \texttt{NLogSpace}$.
That is, complement the language $L$. if $L^c$ is in \texttt{NLogSpace},
then $L \in \texttt{Co-NLogSpace}$.

We intuitively believe that $\texttt{NP} \neq \texttt{Co-NP}$. However, we can
show that $\texttt{NLogSpace} = \texttt{Co-NLogSpace}$.

\begin{align*}
\texttt{PATH} &= \{ \langle G, u, v \rangle~\vert~\text{\textbf{exists} path between vertices $(u, v)$} \} \\
\overline{\texttt{PATH}} &=  \{ \langle G, u, v \rangle~\vert~\text{\textbf{no} path between vertices $(u, v)$} \}
\end{align*}

We assume that \pathbar is \conlogspace-Complete.

If we show that \pathbar is in \texttt{NLogSpace}, then every problem
in \conlogspace will be in \nlogspace

\subsection{Solving \pathbar in \nlogspace}
\begin{align*}
V_R &\equiv \{ \text{set of vertices reachable from $u$} \} \\
V_{NR} &\equiv \{ \text{set of vertices \textbf{not} reachable from $u$} \} \\
\end{align*}

\textbf{Sid confusion, why can't we use PATH as a subroutine:}
When we have an \texttt{NDTM}, we cannot \textit{observe that the NDTM returns a $0$}.
We can \textit{observe if an NDTM succeeds}, but there are weird paths and exponential
number of paths where the NDTM does not return a $0$? But if this is true,
then how is \texttt{PATH} \nlogspace-complete? I am very confused.

To represent $V_R$ and $V_{NR}$, we use $1$ bit per vertex 
(since $V_R$ and $V_{NR}$ are disjoint), so total space is $V$.

Assume we know $|V_R|$. In this case, we can check whether $v$ is unreachable
from $u$ --- Enumerate all vertices. If they are reachable from $u$, bump up
a counter.  If we don't hit $v$ till the counter gets to $|V_R|$, then what
we know that is $v$ is unreachable.

However, if $v$ were reachable from $u$, then as we enumerate, we would find
$v$ as we were going through all vertices (we would not hit $V_R$ unless we
visit $v$).

This is important, because in an NDTM, if \textit{any} of the paths
accept, then we accept.

\begin{align*}
V_R &= \cup_i V_R(i) \\
V_R(0) &= \{ u \} \\
\end{align*}

to compute $cur \in_? V_R(i + 1)$, first \textbf{recompute} that $pred \in V_R(i)$,
and then check that $(cur, pred) \in E(G)$. We cannot \textbf{store} $V_R(i)$, since
we don't have enough space.

eventually we will reach $V_R(|V|)$, where we stop.

We can compute $|V_R| = \sum_i |V_R(i)|$.
We compute $|V_R(i)|$ by checking over each vertex it's membership into $V_R(i)$.
And if it does, we bump up our counter. 

\textbf{Reference: Read \texttt{Sipser-Chapter 8}}

\begin{minted}{python}
def belongs(G, i, startv, endv, curv):
    """Check if curv belongs to V_R(i)"""
    if i == 1:
        return startv == curv
    else:
        # log(V)
        for pred in G.vertices:
            # This can use a modified version of PATH that stores lengths?
            if small_belongs(G, i - 1, startv, endv, pred):
                if isneighbour(pred, curv):
                    return True

        return False

def countcard(G, startv, endv):
    """ Count the cardinality of V_R"""
    card = 0
    # log(V)
    for i in len(G.vertices):
        # this is also log(V)
        for curv in G.vertices:
            if small_belongs(G, i, startv, endv, curv):
                card += 1
    return 1
\end{minted}

\section{Oracles}

For all inputs $w$ of length $|w| = n$, there exists a \textbf{single} advice
($a_n$ is allowed to be a single string that is polynomial in $n$).
So, $a: \mathbb{N} \to \Sigma^*$, and the advice of a given input $w$ is $a(|w|)$.


\subsection{\ppoly}

$L \in \text{\ppoly}$ if there is a polynomial time turing machine $M$ which
takes two inputs --- a string $x \in \Sigma^*$, and an advice $a_n \in \Sigma^*$,
such that for all inputs $w$ such that $|w| = n$, then there exists a polynomial
$p(n)$ with $|a_n| \leq p(|w|)$.

We force it to be polynomial in the word-length, because things like a lookup
table take exponential space in the word-length (number of strings of length $n$
is $2^n$).

We can see that the advice is somewhat "hardwired" into the machine given
the input length (since $a: \mathbb{N} \to \Sigma^*$). So, we have a 
sequence of machines $M_i: \mathbb{N} \to \{ \text{Turing machines} \}$, and we instantiate
the machine $M_{|w|}$ to check if $|w| \in L$.

\nptime is allowed to have a \textit{varying witness}, while \ppoly will have
the \textit{same} advice.

We don't even need to know if the advice string should be able to be found
in polynomial time.

\subsection{\ppoly contains non-recursive languages}
