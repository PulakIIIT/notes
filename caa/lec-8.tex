\chapter{Probabilistic proofs}
\section{Completeness and Soundness}
\begin{definition}
Completeness: For every true assertion, there is a valid proof.
\end{definition}

\begin{definition}
Soundness: For every false assertion, no valid proof exists.
\end{definition}

A good proof system must also be such that the verifier is efficient
(that is, polynomial time).

If we ask that a proof system must be sound and complete, there is no 
scope for error! Further, it is not clear if the verifier and the
prover can "talk" to each other. If we choose to allow interactions, what
are the implications?


We relax the assumptions this way --- Relaxed compleness states that
for every true assertion, there is a
proof strategy that will convince the verifier with probability 
at least $> \frac{1}{2}$.  
Similarly, relaxed soundness states that for every false assertion,
every proof strategy fails to convinve the verifier with probability
at least $> \frac{2}{3}$. 

The formalization is as follows:
\begin{definition}
Interactive proof systems
\begin{itemize}
\item An interactive proof system for a language $L$ consists of two
entities: a prover $P$ and a verifier $V$.
$P$ and $V$ share common input, and work for $R \in \mathbb{N}$ rounds.

\item In each round, the prover can send the verifier a message that 
is polynomial in the length of the input.

\item The verifier can send a polynomial length reply to the prover.

\item The verifier is a randomized polynomial time turing machine. Time
is measured as a function of the length of the input.

\item \textbf{Completeness}: $\forall x \in L$, there exists a prover strategy
so that the verifier accepts with probability $> \frac{2}{3}$.

\item \textbf{Soundness}: $\forall x \notin L$, any prover strategy will lad
the verifier to accept with probability  $< \frac{1}{3}$.
\end{itemize}
\end{definition}
