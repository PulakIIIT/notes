% // https://tex.stackexchange.com/questions/59702/suggest-a-nice-font-family-for-my-basic-latex-template-text-and-math
\documentclass[5pt]{book}
\usepackage[sc,osf]{mathpazo}   % With old-style figures and real smallcaps.
\linespread{1.025}              % Palatino leads a little more leading

% Euler for math and numbers
\usepackage[euler-digits,small]{eulervm}

%\documentclass[10pt]{llncs}
%\usepackage{llncsdoc}
\usepackage{amsmath,amssymb}
\usepackage{amsthm}
%\usepackage{graphicx}
\usepackage{makeidx}
\usepackage{listing}
\usepackage{comment}
\usepackage{dirtytalk}
\evensidemargin=0.20in
\oddsidemargin=0.20in
\topmargin=0.2in
%\headheight=0.0in
%\headsep=0.0in
%\setlength{\parskip}{0mm}
%\setlength{\parindent}{4mm}
\setlength{\textwidth}{6.4in}
\setlength{\textheight}{8.5in}
%\leftmargin -2in
%\setlength{\rightmargin}{-2in}
%\usepackage{epsf}
%\usepackage{url}



\usepackage{booktabs}   %% For formal tables:
                        %% http://ctan.org/pkg/booktabs
\usepackage{subcaption} %% For complex figures with subfigures/subcaptions
                        %% http://ctan.org/pkg/subcaption
\usepackage{enumitem}
\usepackage{minted}
%\newminted{fortran}{fontsize=\footnotesize}

\usepackage{xargs}
\usepackage[colorinlistoftodos,prependcaption,textsize=tiny]{todonotes}

\usepackage{hyperref}
\hypersetup{
    colorlinks,
}

\usepackage{epsfig}
\usepackage{tabularx}
\usepackage{latexsym}
\newcommand\ddfrac[2]{\frac{\displaystyle #1}{\displaystyle #2}}

\def\qed{$\Box$}

%\newcommand{\NP}{\texttt{NP}}
%\newcommand{\PSPACE}{\texttt{PSPACE}}
%\newcommand{\NPSPACE}{\texttt{NPSPACE}}
%\newcommand{\TQBF}{\texttt{TQBF}}

\newcommand{\nptime}{\texttt{NP}}
\newcommand{\ptime}{\texttt{P}}
\newcommand{\coptime}{\texttt{co-P}}
\newcommand{\conptime}{\texttt{co-NP}}
\newcommand{\dspace}{\texttt{DPSACE}}
\newcommand{\pspace}{\texttt{PSPACE}}
\newcommand{\logspace}{\texttt{L}}
\newcommand{\nlogspace}{\texttt{NL}}
\newcommand{\conlogspace}{\texttt{co-NL}}
\newcommand{\ph}{\texttt{PH}}
\newcommand{\pathproblem}{\texttt{PATH}}
\newcommand{\copathproblem}{\overline{\texttt{PATH}}}
\newcommand{\clique}{\texttt{CLIQUE}}
\newcommand{\maxclique}{\texttt{MAXCLIQUE}}
\newcommand{\sat}{\texttt{SAT}}
\newcommand{\phtime}{\texttt{PH}}
\newcommand{\ppoly}{$\texttt{P}^{\texttt{poly}}$}
\newcommand{\pathbar}{$\overline{\texttt{PATH}}$}
\newcommand{\problempath}{\texttt{PATH}}
\newcommand{\rp}{\texttt{RP} }
\newcommand{\corp}{\texttt{co-RP} }
\newcommand{\zp}{\texttt{ZP} }
\newcommand{\maxsat}{\texttt{MAXSAT} }
\newcommand{\ilp}{\texttt{ILP} }
\DeclareMathOperator{\shape}{dim}

\newtheorem{theorem}{Theorem}
\newtheorem{corollary}[theorem]{Corollary}
\newtheorem{definition}[theorem]{Definition}
\newtheorem{lemma}[theorem]{Lemma}
\newtheorem{observation}[theorem]{Observation}
%\newtheorem{proof}[theorem]{Proof}

\newcommand{\ip}{\texttt{IP} }
\newcommand{\lp}{\texttt{LP}}
\newcommand{\R}{\ensuremath{\mathbb{R}}}
\newcommand{\N}{\ensuremath{\mathbb{N}}}
\newcommand{\Z}{\ensuremath{\mathbb{Z}}}

\title{Critical thinking}
\author{Siddharth Bhat}
\date{}

\begin{document}

\maketitle
\tableofcontents

\begin{definition}
    Retrograde analysis: Reasoning that seeks to explain how things
    must have developed from things that have happened before.
\end{definition}

\section{Definitions}
\begin{definition}
    Definiendum: The word or symbol being defined
\end{definition}

\begin{definition}
    Definiens: A symbol or group of symbols that have the same definition
    as the Definiendum
\end{definition}

\begin{definition}
    Ambiguity: A term is ambiguous \textbf{in a given context} when it
    has more than one distinct meaning, and the context does not make clear
    what is intended.
\end{definition}

\begin{definition}
    Vagueness: A term is vague, when there are borderline cases where the
    term may or may not apply.
\end{definition}

\begin{itemize}
    \item \textbf{Stipulative definition}: A new definition where some meaning is
        arbitrarily defined. Such a definition cannot be correct or incorrect.
        Eg. Zetta has been stipulatively defined to mean $10^{12}$.
    \item \textbf{Lexical definition}: Reports the meaning that the Definiendum
        already has. The report maybe correct or incorrect, so such a definition
        maybe true or false.
    \item \textbf{Precising definition}: A definition used to eliminate 
        vagueness or ambiguity.

        It reports on a word that already exists, but it makes the known
        meaning more precise. It could increase the precision by stipulating,
        but the purpose of this stipulation is to \textit{improve a pre-existing}
        meaning.

    \item \textbf{Theoretical definition}: A definition that encapsulates
        an understanding of the theory in which the term is a key element.
    \item \textbf{Persuasive definition}: A definition used to resolve
        disputes  by influencing attitudes and stirring emotion. Often
        uses emotive language.
\end{itemize}

\section{Intention and extension}
\begin{definition} Extension: The set of all objects to which a term
    may be applied. $$extention(P) \equiv \{ x ~|~\forall x \in universe, ~P(x) \}$$
\end{definition}

\begin{definition} Intension: The attributes shared by all and only the objects
    in the class the term denotes. Or, the connotation of the term.

    $$intension(P) \equiv \bigcap_{\forall x \in extension(P)} attribs(x) $$
\end{definition}

\paragraph{The equilateral triangle example} 
Consider equi-\textbf{angular} triangle,
whose intention is different from equi-\textbf{lateral} triangle.

Both of these have the same \textit{extension} (since the sets are the
same), but have different \textit{intentions}.

\section{Extension and denotative definitions}
Means of defining extensive terms:
\begin{definition}Denotative definition: A definition that identifies
    the extention of a term, by (for eg.) listing the members
    of the definition. An extensive definition
\end{definition}


\begin{definition}Ostensive definition: A denotative definition, where
    the definition is made by pointing.

    For example, the word desk means \textit{this}.
\end{definition}

\begin{definition}Quasi Ostensive definition:
    Example: the desk is means \textit{this} article of furniture. This
    presupposes the meaning of \textit{article of furniture}.
\end{definition}

\section{Intention and Intensional Definitions}
\subsection{Types of intentional definitions}
\begin{definition}Subjective intension: Set of all attributes speaker
    believes are posessed by objects denoted by that word
\end{definition}

\begin{definition}Objective intension: Total set of characteristics
    shared by all objects in the word's extension
\end{definition}

\begin{definition}
    Conventional intension: Commonly accepted intension of a term; 
\end{definition}

\subsection{Techniques for creating intensional definitions}
\begin{definition}Synonymous definition: A word, phrase, etc. is defined
    in terms of another
\end{definition}

\begin{definition}Operational definition: Defined by means of operations.
    Example: length can be defined by the measuring procedure.
\end{definition}

\begin{definition}Definitions by genus and difference (Analytical definitions):
    Class whose membership is divided: \textbf{genus}. Subclasses: \textbf{species}.
\end{definition}

\section{Fallacies}
\subsection{Fallacies of relevance}
\begin{itemize}
    \item \textbf{Appeal to populace}: Appeal to people's emotions. Eg. speeches
        by fascist leaders.
    \item \textbf{Appeal to emotion}: Appeal to base emotions such as pity.
    \item \textbf{Red herring}: Distract from the actual argument with a
        \textit{deliberately misleading trail}. Etymology: People who used to
        try and save foxes from being hunted by leaving a smoked herring, which
        confuses dogs, and also turns red.
    \item \textbf{Strawman}: Misconstrue argument to make it seem weaker
        than it actually is, and then defeat the weakened argument.
    \item \textbf{Ad homenim --- abusive type (Argument against the person)}: Attack moral character
        of person. 
    \item \textbf{Ad homenim --- circumstantial type}: Attacking someone's argument
        based on their \textit{circumstance}. Eg. calling a non-vegeterian who
        argues for reduced meat consumption a hypocrite. This does not reduce the
        validity of the argument at all.
    \item \textbf{Appeal to force}: Appeal to threats to coerce the other
        person to accept your argument.
    \item \textbf{Missing the point}: One attacks a different thesis than
        the one the interlocutor was advancing.
\end{itemize}
\subsection{Fallacies of defective induction}
\begin{itemize}
    \item \textbf{Argument from ignorance}: Arguing that just because something
        is not \emph{proven} true, it \textit{must be} false, or the converse.
    \item \textbf{Appeal to inappropriate authority}: Appeal to the authority
        of someone who is not an authority on the subject at hand. 

        Example: Invoking Picasso on a discussion about economics.

    \item \textbf{False cause}: Arguing for a cause-and-effect relationship
        where none exists. Eg. you fell sick because of the bees this time
        of year.

    \item \textbf{Hasty Generalization}: Performing induction from a very
        small sample size.
\end{itemize}

\subsection{Fallacies of presumption}
These come from presuming unjustified assumptions.
\begin{itemize}
    \item \textbf{Accident}: Assuming a generalization applies to all concrete
        cases. 

        Example: It is wrong to steal. We can create corner cases such as 
        \say{what if the person was hungry}? This falls under accident.

    \item \textbf{Complex question}: Constructing a loaded question where
        refuting a part of the question implicitly provides truth to another
        part, which was unintended.

        Example: \say{With all of the hysteria, all of the fear, all of the
        phony science, could it be that man-made global warming is the greatest
    hoax ever perpetrated on the American people?}

    \item \textbf{Begging the question}: Assuming the conclusion in the 
        supposition.

        Example: \say{There is no such thing as knowledge which cannot be
        carried into practice, for such knowledge is really no knowledge at
        all.}
\end{itemize}


\end{document}

