% // https://tex.stackexchange.com/questions/59702/suggest-a-nice-font-family-for-my-basic-latex-template-text-and-math
\documentclass[11pt]{book}
\usepackage[sc,osf]{mathpazo}   % With old-style figures and real smallcaps.
\linespread{1.025}              % Palatino leads a little more leading

% Euler for math and numbers
\usepackage[euler-digits,small]{eulervm}

%\documentclass[10pt]{llncs}
%\usepackage{llncsdoc}
\usepackage{amsmath,amssymb}
\usepackage{amsthm}
\usepackage{graphicx}
\usepackage{makeidx}
\usepackage{algpseudocode}
\usepackage{algorithm}
\usepackage{listing}
\usepackage{comment}
\evensidemargin=0.20in
\oddsidemargin=0.20in
\topmargin=0.2in
%\headheight=0.0in
%\headsep=0.0in
%\setlength{\parskip}{0mm}
%\setlength{\parindent}{4mm}
\setlength{\textwidth}{6.4in}
\setlength{\textheight}{8.5in}
%\leftmargin -2in
%\setlength{\rightmargin}{-2in}
%\usepackage{epsf}
%\usepackage{url}



\usepackage{booktabs}   %% For formal tables:
                        %% http://ctan.org/pkg/booktabs
\usepackage{subcaption} %% For complex figures with subfigures/subcaptions
                        %% http://ctan.org/pkg/subcaption
\usepackage{enumitem}
\usepackage{minted}
%\newminted{fortran}{fontsize=\footnotesize}

\usepackage{xargs}
\usepackage[colorinlistoftodos,prependcaption,textsize=tiny]{todonotes}

\usepackage{hyperref}
\hypersetup{
    colorlinks,
}

\usepackage{epsfig}
\usepackage{tabularx}
\usepackage{latexsym}
\newcommand\ddfrac[2]{\frac{\displaystyle #1}{\displaystyle #2}}

\def\qed{$\Box$}

%\newcommand{\NP}{\texttt{NP}}
%\newcommand{\PSPACE}{\texttt{PSPACE}}
%\newcommand{\NPSPACE}{\texttt{NPSPACE}}
%\newcommand{\TQBF}{\texttt{TQBF}}

\newcommand{\nptime}{\texttt{NP}}
\newcommand{\ptime}{\texttt{P}}
\newcommand{\coptime}{\texttt{co-P}}
\newcommand{\conptime}{\texttt{co-NP}}
\newcommand{\dspace}{\texttt{DPSACE}}
\newcommand{\pspace}{\texttt{PSPACE}}
\newcommand{\logspace}{\texttt{L}}
\newcommand{\nlogspace}{\texttt{NL}}
\newcommand{\conlogspace}{\texttt{co-NL}}
\newcommand{\ph}{\texttt{PH}}
\newcommand{\pathproblem}{\texttt{PATH}}
\newcommand{\copathproblem}{\overline{\texttt{PATH}}}
\newcommand{\clique}{\texttt{CLIQUE}}
\newcommand{\maxclique}{\texttt{MAXCLIQUE}}
\newcommand{\sat}{\texttt{SAT}}
\newcommand{\phtime}{\texttt{PH}}
\newcommand{\ppoly}{$\texttt{P}^{\texttt{poly}}$}
\newcommand{\pathbar}{$\overline{\texttt{PATH}}$}
\newcommand{\problempath}{\texttt{PATH}}
\newcommand{\rp}{\texttt{RP} }
\newcommand{\corp}{\texttt{co-RP} }
\newcommand{\zp}{\texttt{ZP} }
\newcommand{\maxsat}{\texttt{MAXSAT} }
\newcommand{\ilp}{\texttt{ILP} }
\DeclareMathOperator{\shape}{dim}

\newtheorem{theorem}{Theorem}
\newtheorem{corollary}[theorem]{Corollary}
\newtheorem{definition}[theorem]{Definition}
\newtheorem{lemma}[theorem]{Lemma}
\newtheorem{observation}[theorem]{Observation}
%\newtheorem{proof}[theorem]{Proof}

\newcommand{\ip}{\texttt{IP} }
\newcommand{\lp}{\texttt{LP}}
\newcommand{\R}{\ensuremath{\mathbb{R}}}
\newcommand{\N}{\ensuremath{\mathbb{N}}}
\newcommand{\Z}{\ensuremath{\mathbb{Z}}}

\title{Optimisation methods: Linear and Integer optimisation}
\author{Siddharth Bhat}
\date{}

\begin{document}

\maketitle
\tableofcontents


\chapter{Introductory examples}


\section{Classification}
Consider a classification problem where we have labeled data at
$(x_i, y_i)$, with some labeled $+$ and other data points labeled $-$.

\chapter{\lp-relaxation}

We know that $z^*_\ip \leq z^*_\lp$. We can solve an \lp problem using
a solver.

\section{Bipartite Matching}
Let $G \equiv (V \equiv X \cup Y, E \subset V \times V, w: E \rightarrow \R)$. Graph is:
\begin{itemize}
    \item Undirected, so $(x, y) \in E \iff (y, x) \in E$, $w((v, v')) = w((v', v))$.
    \item Bipartite, so that $(v, v') \in E \implies (v \in X \land v'\in Y) \lor (v \in Y \land v' \in X)$
    \item It's a little annoying to write the condition, but basically, for
        every edge, there's a unique weight which we adjust, even though
        the graph is undirected.
\end{itemize}
        
        We wish to find $M \subseteq E$ such that:

\begin{align*}
\max_{e \in M} w_e \\
\end{align*}

Can be transformed to:
\begin{align*}
    &\max_{e \in E} x_e w_e \qquad x_e \in \{0, 1\} \\
    &\sum_{e \in E, e = (v, v')} x_e = 1 \qquad \forall v \in V
\end{align*}
Where the $x_e$ are variables to be discovered.  
We can now LP relax this, where $x_e \in [0, 1]$:
\begin{align*}
    &\max_{e \in E} x_e w_e \qquad x_e \in [0, 1] \\
    &\sum_{e \in E, e = (v, v')} x_e = 1 \qquad \forall v \in V
\end{align*}
How do we go from the optimal solution to this problem, to an
integer solution?
\begin{itemize}
    \item Assume the \lp~is infeasible. This means that we have a vertex $u$
        such that $\sum_{e \in E, e = (u, v)} x_e = 1$ fails. that is,
        there's a vertex in $u$ that is not connected to $v$. In this case,
        the \ip~is also infeasible.
    \item Now, we know that the \lp is feasible. $a_1 \rightarrow b_1$ is
        not saturated means that $b_1 \rightarrow a_2$ is not saturated
        which implies that $a_2 \rightarrow b_2$ is not saturated, hence
        $b_2 \rightarrow a_1$ is not saturated. (TODO: add tikz picture).
        We can get a full cycle of edges with:

        \begin{align*}
        &x_{e_i} < 1 \\
        &x_{e_i} \in {a_1 
            \xrightarrow{e_1} b_1 
            \xrightarrow{e_2} a_2 
            \xrightarrow{e_3} b_2
            \xrightarrow \dots 
            \xrightarrow{e_{i-1}} b_n 
        \xrightarrow{e_i} a_1}
    \end{align*}
        The number of
        edges here will be \textit{even}. We can now pick a value $\epsilon \in (0, 1)$
        such that:
        \begin{align*}
        y_e \equiv 
        \begin{cases}
            x_e^* + \epsilon & \text{$i$ is even, $x_e$ is in the cycle} \\
            x_e^* - \epsilon & \text{$i$ is odd, $x_e$ is in the cycle} \\
            x_e^* & \text{otherwise}
        \end{cases}
        \end{align*}
        Note that $y_e$ is a valid solution, since we can set $\epsilon$ to
        be smaller than the slack we had in the smallest value of $x_i$.
        We can show that the $cost(y) \equiv \sum_{e \in E} w_e y_e$ is equal to:
        \begin{align*}
            cost(y) = cost(x_e^*) + \epsilon \bigg(\Delta \equiv \sum_{i=1}^n (-1)^i w(e_i) \bigg)
        \end{align*}

        Remember that $x_e^*$ is the best solution, so we can have nothing
        better than $cost(x_e^*)$. Hence, $cost(y_e^*) \leq cost(x_e^*)$,
        and hence, we are forced to conclude that $\Delta = 0$ 
        (If $\Delta > 0$, pick $\epsilon > 0$, if $\Delta < 0$, pick $\epsilon
        < 0$). 

        Hence, we can keep moving $\epsilon$ till an even edge becomes $1$
        (alternatively, and odd edge becomes $0$). Hence, we can \textit{keep rounding}
        till all our edges become $\{0, 1\}$.
\end{itemize}

So, we managed to start from an \lp~solution, and then \textit{unrelax} it to
construct an \ip~solution from it!

\section{Min vertex cover}
$G\equiv (V, E)$. We want to pick the smallest $F \subseteq V$, such that one end
of all edges is in this cover.
\[ \forall (u, v) \in E, u \in F \lor v \in F \]
Intuitively, these vertices $f \in F$ are watching over the edges, and each
edge must be watched by at least one vertex.

TODO: add tikz picture


\begin{align*}
    &\text{Integer program for the problem:} \\
    &x_v \in \{0, 1\}~\forall v \in V \qquad
    \min \sum x_v \qquad
    \forall (u, v) \in E, x_u + x_v \geq 1
\end{align*}

\begin{align*}
    &\text{LP relaxed program for the problem:} \\
    &x_v \in [0, 1]~\forall v \in V \qquad
    \min \sum x_v \qquad
    \forall (u, v) \in E, x_u + x_v \geq 1
\end{align*}

From the LP, we construct:
\begin{align*}
    S_\lp \equiv \bigg\{ u ~\bigg|~ x_u^* \geq \frac{1}{2} \bigg\} \qquad
    \text{Claim: $S_\lp$ is a vertex cover}
\end{align*}

For each edge $(u, v) \in E$, since $u + v \geq 1$, we \textit{cannot have that}
$x_u < 0.5 \land x_v < 0.5$, since then $x_u + x_v < 1$. Hence, each
edge will have one of its vertices with $x_{vertex} \geq 0.5$, and thus
$S_\lp$ is \textbf{a} vertex cover.

We now show \textbf{optimality} of $S_\lp$.

\begin{align*}
    &\lp \leq \ip \qquad \text{since the problem is a minimization problem} \\
    &\sum_{u \in V} x_u \leq \sum_{u \in V} y_u \qquad \text{$x$ is \lp~solution, $y$ is \ip~solution} \\
    &|S_\lp| = \sum_{x \in S_\lp} 1 \text{(counting})\leq 
    \sum_{u\in S_\lp} 2 x_u  \text{(definition of $S_\lp$)} \leq
    \sum_{u \in V} 2 x_u \text{(enlarging $S_\lp$ to $V$)} \leq
        \sum_{u \in V} 2 y_u = 2|s_{opt}| \\
    &|S_{opt}| \leq S_\lp \leq 2 |S_{opt}|
\end{align*}

So, we are at worst twice the size of the best vertex cover.

\section{Maximum independent set}
A set $I \subseteq V$ is independent if no two vertices in $I$ are connected
by an edge. Once again, let $x_v \in \{0, 1\}$ be indicators for a vertex
being picked. We have the constraints:

\begin{align*}
    \text{maximise}~\sum_v x_v \qquad \text{subject to:}~\forall (u, v) \in E,~ x_u + x_v \leq 1
\end{align*}

in the LP case, There is always a solution by setting all $x_v = 1/2$. Hence,
the optimal LP is always $|V|/2$ or larger.

\chapter{Formulating common operations in terms of \ilp}

\section{TODO: MISSED CLASS! LOOKUP WHAT HAPPENED}

\chapter{Matrix decompositions}
\subsection{Cholesky}
Let $A$ be positive definite, $L$  be lower triangular. We decompose it as follows:

\paragraph{Computing the decomposition}
\begin{align*}
    &A = L L^T\\
    &\begin{bmatrix} a_{11} & A_{12} \\ A_{12} & A_{22} \end{bmatrix} = 
    \begin{bmatrix} l_{11} & 0 \\ L_{21} & L_{22} \end{bmatrix}
    \begin{bmatrix} l_{11} & L_{21}^T \\ 0 & L_{22}^T \end{bmatrix}  = 
%
    \begin{bmatrix}
        l_{11}^2 & l_{11} L_{21}^T \\ l_{11} L_{21}^T & L_{21} L_{21}^T + L_{22} L_{22}^T
    \end{bmatrix}
\end{align*}

\paragraph{Solving $Ax = b$}
\begin{itemize}
    \item We want to solve $Ax = b$, which is equivalent to $LL^Tx = b$
    \item let $L^Tx  = u$. Now, $LL^T x = b \equiv Lu = b$.
    \item Solve $Lu = b$ to find value of $u$.
    \item Solve $L^Tx = u$ to find value of $x$.
\end{itemize}

\paragraph{Solving $x = ((A^TA)^{-1}A^T)b$}
\begin{itemize}
    \item Let $B = A^TA$. Now, original equation is $Bx = A^T b$.
    \item Compute $B$.
    \item Compute $d = A^Tb$
    \item Solve $Bx = d$. This is possible since $B$ is positive definite.
\end{itemize}


\paragraph{Finding inverse}
\begin{itemize}
\item First decompose $A = LL^T$
\item Solve $A x_i = e_i$
\end{itemize}

\subsection{LU}
\begin{align*}
    &A = PLU
\end{align*}

\subsection{QR}
$A = QR$ where $\shape(A) = (m, n)$ $\shape(Q) = (m, n)$, $\shape(R) =  (n, n)$.
$Q$ is orthogonal, $R$ is triangular.

We care about this decomposition in certain cases. For example, consider
$x = (A^TA)^{-1}A^Tb$. Let $A = QR$. Now, the expression becomes
\begin{align*}
    &x = (A^TA)^{-1}A^Tb \\
    &x = ((R^TQ^T)(QR))^{-1} (R^TQ^T)b = (R^TR)^{-1}) (R^TQ^T) b = (R^{-1} (R^T)^{-1} R^T Q^T) b = R^{-1} Q^T b \\
    &Rx = Q^T b \qquad \text{Let $Q^Tb = d$} \quad Rx = d \\
    &\text{Solve $Rx = d$}
\end{align*}

\paragraph{Comparison of Cholesky and QR for Least squares}
For cholesky, we want to find $x = (A^TA)^{-1}A^Tb$. First, rewrite to
$A^TA x = A^T b$. $Bx = d$ where $B = A^TA$.
\begin{itemize}
    \item Compute $B$
    \item Compute $d$
    \item Cholesky of $B = L L^T$
    \item Solve $Lv = d$
    \item Solve $Lx = v$
\end{itemize}

For QR:
\begin{itemize}
    \item Factorize $A = QR$
    \item Compute $d = Q^Tb$
    \item Solve $Rx = d$
\end{itemize}


As we move from Cholesky to SVD, factorization cost increases, solution
time decreases.

\paragraph{Computing $QR$}
\begin{align*}
    &\begin{bmatrix}
        a_1 & A_2
    \end{bmatrix} = 
    \begin{bmatrix}
        q_1 & Q_2
    \end{bmatrix} 
    \begin{bmatrix}
        r_{11} & R_{12} \\ 0 & R_{22}
    \end{bmatrix} =  
    %
    \begin{bmatrix}
        q_1 r_{11}  & q_1 R_{12} + Q_2 R_{22}
    \end{bmatrix} 
    %
    \\
    %
    &a_1 = q_1 r_{11} \quad A_2 = q_1 R_{12} + Q_2 R_{22}
\end{align*}

Since $Q^TQ = I$, $\begin{bmatrix}q_1 Q_2\end{bmatrix}^T \begin{bmatrix} q_1 Q_2\end{bmatrix} = I$,
hence $q_1^T q_1 = 1$.

So, \[a_1^Ta_1 = (q_1 r_{11})^T (q_1 r_{11}) = (r_{11} q_1^T) (q_1 r_{11}) = r_{11}^2\].
Hence, \[r_{11} = \sqrt{a_1^T a_1} \qquad q_1 = a_1 / r_{11}\].

To find $R_{12}$, \textbf{TODO}

Next, Let $B = A_2 - q_1 R_{12} = Q_2 R_{22}$. Now perform QR on $B$.

\subsection{SVD}
$A = UDV^T$ where $\shape(A) = (m, n)$ $\shape(U) = (m, n)$, $\shape(D) =  (n, n)$.
$\dim(V) = (n, n)$.  $U, V$ are orthogonal, $D$ is a diagonal matrix.

\section{Bala's implicit enumeration algorithm}
Used to solve $0, 1$ binary ILP problems.

\href{https://www.sce.carleton.ca/faculty/chinneck/po/Chapter13.pdf}{good link for bala's algorithm}

\chapter{Practice for mid-1}
\begin{itemize}
    \item Proof for unrelaxing bipartite matching from LP to best solution
    \item Approximation ratio for vertex cover
    \item Phrasing maximum independent set in terms of ILP
    \item Different styles of constraints that can be encoded using ILP
        \begin{itemize}
            \item Functions of K discrete variables
            \item Either-or constraints
            \item K out of N constraints
            \item Compound alternatives
            \item Piecewise linear
        \end{itemize}
    \item Facility location problem
    \item Bala's method to solve BIP
    \item Scheduling and makespan
    \item Costs of different factorizations, and different factorizations:
        \begin{itemize}
            \item Cholesky: $1/3n^3$
            \item Solving system $L \vec x = b$ is $n^2$ by forward substitution.
            \item computing LU: $2/3n^3$
        \end{itemize}
\end{itemize}

\end{document}

