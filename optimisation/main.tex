% // https://tex.stackexchange.com/questions/59702/suggest-a-nice-font-family-for-my-basic-latex-template-text-and-math
\documentclass[11pt]{book}
\usepackage[sc,osf]{mathpazo}   % With old-style figures and real smallcaps.
\linespread{1.025}              % Palatino leads a little more leading

% Euler for math and numbers
\usepackage[euler-digits,small]{eulervm}

%\documentclass[10pt]{llncs}
%\usepackage{llncsdoc}
\usepackage{amsmath,amssymb}
\usepackage{amsthm}
\usepackage{graphicx}
\usepackage{makeidx}
\usepackage{algpseudocode}
\usepackage{algorithm}
\usepackage{listing}
\usepackage{comment}
\evensidemargin=0.20in
\oddsidemargin=0.20in
\topmargin=0.2in
%\headheight=0.0in
%\headsep=0.0in
%\setlength{\parskip}{0mm}
%\setlength{\parindent}{4mm}
\setlength{\textwidth}{6.4in}
\setlength{\textheight}{8.5in}
%\leftmargin -2in
%\setlength{\rightmargin}{-2in}
%\usepackage{epsf}
%\usepackage{url}



\usepackage{booktabs}   %% For formal tables:
                        %% http://ctan.org/pkg/booktabs
\usepackage{subcaption} %% For complex figures with subfigures/subcaptions
                        %% http://ctan.org/pkg/subcaption
\usepackage{enumitem}
\usepackage{minted}
%\newminted{fortran}{fontsize=\footnotesize}

\usepackage{xargs}
\usepackage[colorinlistoftodos,prependcaption,textsize=tiny]{todonotes}

\usepackage{hyperref}
\hypersetup{
    colorlinks,
}

\usepackage{epsfig}
\usepackage{tabularx}
\usepackage{latexsym}
\newcommand\ddfrac[2]{\frac{\displaystyle #1}{\displaystyle #2}}

\def\qed{$\Box$}

%\newcommand{\NP}{\texttt{NP}}
%\newcommand{\PSPACE}{\texttt{PSPACE}}
%\newcommand{\NPSPACE}{\texttt{NPSPACE}}
%\newcommand{\TQBF}{\texttt{TQBF}}

\newcommand{\nptime}{\texttt{NP}}
\newcommand{\ptime}{\texttt{P}}
\newcommand{\coptime}{\texttt{co-P}}
\newcommand{\conptime}{\texttt{co-NP}}
\newcommand{\dspace}{\texttt{DPSACE}}
\newcommand{\pspace}{\texttt{PSPACE}}
\newcommand{\logspace}{\texttt{L}}
\newcommand{\nlogspace}{\texttt{NL}}
\newcommand{\conlogspace}{\texttt{co-NL}}
\newcommand{\ph}{\texttt{PH}}
\newcommand{\pathproblem}{\texttt{PATH}}
\newcommand{\copathproblem}{\overline{\texttt{PATH}}}
\newcommand{\clique}{\texttt{CLIQUE}}
\newcommand{\maxclique}{\texttt{MAXCLIQUE}}
\newcommand{\sat}{\texttt{SAT}}
\newcommand{\phtime}{\texttt{PH}}
\newcommand{\ppoly}{$\texttt{P}^{\texttt{poly}}$}
\newcommand{\pathbar}{$\overline{\texttt{PATH}}$}
\newcommand{\problempath}{\texttt{PATH}}
\newcommand{\rp}{\texttt{RP} }
\newcommand{\corp}{\texttt{co-RP} }
\newcommand{\zp}{\texttt{ZP} }
\newcommand{\maxsat}{\texttt{MAXSAT} }

\newtheorem{theorem}{Theorem}
\newtheorem{corollary}[theorem]{Corollary}
\newtheorem{definition}[theorem]{Definition}
\newtheorem{lemma}[theorem]{Lemma}
\newtheorem{observation}[theorem]{Observation}
%\newtheorem{proof}[theorem]{Proof}

\newcommand{\ip}{\texttt{IP} }
\newcommand{\lp}{\texttt{LP}}
\newcommand{\R}{\ensuremath{\mathbb{R}}}
\newcommand{\N}{\ensuremath{\mathbb{N}}}
\newcommand{\Z}{\ensuremath{\mathbb{Z}}}

\title{Information theory}
\author{Siddharth Bhat}
\date{}

\begin{document}

\maketitle
\tableofcontents

\chapter{\lp-relaxation}

We know that $z^*_\ip \leq z^*_\lp$. We can solve an \lp problem using
a solver.

\section{Bipartite Matching}
Let $G \equiv (V \equiv X \cup Y, E \subseteq X \times Y, w: E \rightarrow \R)$. We wish 
to find $M \subseteq E$ such that:

\begin{align*}
\max_{e \in M} w_e \\
\end{align*}

Can be transformed to:
\begin{align*}
    &\max_{e \in E} x_e w_e \qquad x_e \in \{0, 1\} \\
    &\sum_{e \in E, e = (v, v')} x_e = 1 \qquad \forall v \in V
\end{align*}
Where the $x_e$ are variables to be discovered.  
We can now LP relax this, where $x_e \in [0, 1]$:
\begin{align*}
    &\max_{e \in E} x_e w_e \qquad x_e \in [0, 1] \\
    &\sum_{e \in E, e = (v, v')} x_e = 1 \qquad \forall v \in V
\end{align*}
How do we go from the optimal solution to this problem, to an
integer solution?
\begin{itemize}
    \item Assume the \lp~is infeasible. This means that we have a vertex $u$
        such that $\sum_{e \in E, e = (u, v)} x_e = 1$ fails. that is,
        there's a vertex in $u$ that is not connected to $v$. In this case,
        the \ip~is also infeasible.
    \item Now, we know that the \lp is feasible. $a_1 \rightarrow b_1$ is
        not saturated means that $b_1 \rightarrow a_2$ is not saturated
        which implies that $a_2 \rightarrow b_2$ is not saturated, hence
        $b_2 \rightarrow a_1$ is not saturated. (TODO: add tikz picture).
        We can get a full cycle of edges with 
        $x_e < 1, x_e \in {a_1 \rightarrow b_1 \rightarrow a_2 \rightarrow b_2 \rightarrow \dots \rightarrow b_n \rightarrow a_1}$. The number of
        edges here will be \textit{even}. We can now pick a value $\epsilon \in (0, 1)$
        such that:
        \begin{align*}
        y_e \equiv 
        \begin{cases}
            x_e^* + \epsilon & \text{$i$ is even, $x_e$ is in the cycle} \\
            x_e^* - \epsilon & \text{$i$ is odd, $x_e$ is in the cycle} \\
            x_e^* & \text{otherwise}
        \end{cases}
        \end{align*}
        Note that $y_e$ is a valid solution, since we can set $\epsilon$ to
        be smaller than the slack we had in the smallest value of $x_i$.
        We can show that the $cost(y) \equiv \sum_{e \in E} w_e y_e$ is equal to:
        \begin{align*}
            cost(y) = cost(x_e^*) + \epsilon \bigg(\Delta \equiv \sum_{i=1}^n (-1)^i w(e_i) \bigg)
        \end{align*}

        Remember that $x_e^*$ is the best solution, so we can have nothing
        better than $cost(x_e^*)$. Hence, $cost(y_e^*) \leq cost(x_e^*)$,
        and hence, we are forced that $\Delta = 0$ (If $\Delta > 0$, pick $\epsilon > 0$,
        if $\Delta < 0$, pick $\epsilon < 0$). 

\end{itemize}
\end{document}

