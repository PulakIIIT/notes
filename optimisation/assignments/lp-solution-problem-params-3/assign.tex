\documentclass[11pt]{article}
\usepackage[sc,osf]{mathpazo}   % With old-style figures and real smallcaps.
\linespread{1.025}              % Palatino leads a little more leading
% Euler for math and numbers
\usepackage[euler-digits,small]{eulervm}
\usepackage{amsmath}
\usepackage{amssymb}
\usepackage{physics}
\usepackage{tikz}
\usepackage{fancyhdr}
\author{Siddharth Bhat(20161105)}
\title{Optimization assignment --- LP solution problem parameter 3 }
\date{\today}

\pagestyle{fancy}
\fancyhf{}
\lhead{Siddharth Bhat (20161105)}
\rfoot{Page \thepage}

\begin{document}
\maketitle
\thispagestyle{fancy}
We are asked to mimise the dot product between the vector $\vec c$
and the vector $\vec x$, subject to the constraints $0 \leq x_i \leq 1$. Since
the objective is $c^t x = \sum_i c_i x_i$ under the constraints
$0 \leq x_i \leq 1$, we can consider each $x_i$
independently as there are no terms in the objective or in the constraints that
have cross-dependencies.

Note that if the $i$th component of the vector $c_i$ is positive, then the
minimum value of $x_i c_i$ is possible if $x_i = 0$(since to $0 \leq x_i \leq
1$). Similarly, if the $i$th component of the vector $c_i$ is negative, then
the minimum value of $x_i c_i$ is achieved when $x_i = 1$. If $c_i$ is $0$,
then the value of $x_i$ does not matter. Hence, the optimal value of the $x$
vector is:
\begin{align*}
    x_i = \begin{cases}
        1 & \text{if $c_i < 0$} \\
        0 & \text{if $c_i \geq 0$}
    \end{cases}
\end{align*}
\end{document}

