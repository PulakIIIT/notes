\documentclass[11pt]{article}
\usepackage[sc,osf]{mathpazo}   % With old-style figures and real smallcaps.
\linespread{1.025}              % Palatino leads a little more leading
% Euler for math and numbers
\usepackage[euler-digits,small]{eulervm}
\usepackage{amsmath}
\usepackage{amssymb}
\usepackage{physics}
\usepackage{tikz}
\usepackage{fancyhdr}
\author{Siddharth Bhat(20161105)}
\title{Optimization assignment --- TODO}
\date{\today}
\renewcommand{\l}{\left}
\renewcommand{\r}{\right}

\pagestyle{fancy}
\fancyhf{}
\lhead{Siddharth Bhat (20161105)}
\rfoot{Page \thepage}

\begin{document}
\maketitle
\thispagestyle{fancy}
Let us assume the linear program is of the form $\min A\vec{x} = \vec{b}$, 
subject to $\vec x \geq 0$.

Let the objective be $z = k_i x_i$ at some step, and the constraints be
$A'x = b$. At this point, we need to find the index $e$ for the entering
variable such that the \textit{total change} will be minimum. That is,
$k_e \delta_e$ is minimum, where $\delta_e$ is the maximum value $x_e$ can take.
We can read $k_e$ off from the objective function, and we can compute $\delta_e$
by computing the maximum value the value $x_i$ can take under the given system
of constraints.


In terms of the canonical form, consider the canonical form:
\begin{align*}
    \begin{bmatrix}
        -z & x_1 & x_2 & \dots & x_n & RHS \\
        1 & c_1 & c_2 & \dots & c_n & 0 \\
        0 & \alpha_{11} & \alpha_{12} & \dots & \alpha_{1n} & b_1 \\
        0 & \alpha_{21} & \alpha_{22} & \dots & \alpha_{2n} & b_2 \\
        0 & \vdots & \vdots & \ddots & \vdots & \vdots \\
        0 & \alpha_{m1} & \alpha_{m2} & \dots & \alpha_{mn} & b_m \\
    \end{bmatrix}
\end{align*}


For a fixed choice of column $l$, the rate of change is $-c_l$. The maximum
value that $x_l$ can take, if we choose to exist $x_r$,
is $b_r / c_{rl}$. Since we want the total negative change to be maximised, we need to
pick $(l, r)$ such that $-c_l \dot b_r / c_{rl}$ is minimum, since that is 
the quantity which is the rate of change in variable $x_l$ multiplied by the 
total delta that is possible when $x_r$ exists while $x_l$ enters.

\begin{align*}
    (l, r) = \text{argmin}_{l, r} \l( -c_l b_r / c_{rl} \r)
\end{align*}
\end{document}


