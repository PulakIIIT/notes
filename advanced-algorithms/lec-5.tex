\chapter{Advice \& Time Hierarchies}

\section{\ppoly}

This class could possibly be bigger than \ptime.

In \nptime, witnesses are different for each string. In \ppoly, witnesses
are fixed for strings of a given length.

The advice string need to even be found in polynomial time!

Recursive language: Halts on all inputs with yes/no
Recursively enumerable: Halts and returns yes on inputs which belong to the language.
On inputs that do not halt, undefined behavior.

\section{Unary language that is non-recursive}

$L~\text{is a unary language}~\equiv L \subseteq 1^*$

\begin{theorem}
Every unary language is decidable by \ppoly 
\end{theorem}
\begin{proof}
let $L$ be a unary language.

Since the only characteristic of a string in a unary language is its length,
for any given length, there is \textit{at most one string of that length}
in $L$. So, we can index the set $L$ by the string lengths! Hence, the
advice function allows us to build up a lookup table for \textit{any} unary
language.


 We construct the advice function $a_L: \mathbb{N} \to \{0, 1\}$ be such that
$a_L(n) = \text{does}~1^n~\text{belong to L}$?. Now, let $M$ decide $L$
as follows: $M(str) = a(|str|)$. Since we don't need to build $a$ (it's an
oracle we take for granted, the proof is done).
\end{proof}

\begin{theorem}
\ppoly contains non-recursive languages.
\end{theorem}
\begin{proof}
Let $L_{nr} \subset \{0, 1\}^*$ be a nonrecursive language. 
We define $L_w = \{ 1^{\#w}~\vert~w \in L_{nr} \}$,
which is a unary language. A string $1^k \in L_w$ acts as a witness for the
existence of some string $w \in L_{nr}$ as the lex-ordering-position of the string
$w$.

\begin{align*}
&\text{Example of \# evaluated on some strings}\\
&\#0 \to 0 \\
&\#1 \to 1 \\
&\#00 \to 3 \\
&\#01 \to 4 \\
&\#100 \to 5 \\
&\dots
\end{align*}

$L_{nr}$ has now been reduced to $L_u$, since the mapping with $\#$ is a 
\textit{bijection} . Also, $L_u$ can be decided by \ppoly. Hence,
$L_u$ can decide nonrecursive languages.

\textbf{Question: Is the set $\{0, 1\}^*$ countable? It doesn't feel like it is!}
\end{proof}


\subsection{Sparse language}
A \textbf{sparse language} is one where the number of strings of length $n$
is bounded by a polynomial. $|L \cap \{0, 1\}^n| \leq p(n)$.

\textbf{Idle thought: Is there a classification theorem for sparse languages? "sparse-complete"}

We study the relationship between \nptime and \ppoly, using sparse languages.

\subsection{Cook reduction}
A language $L_1$ cook reduces to a language $L_2$ if there is a 
polynomial-time turing machine $M_{L_1}$ that recognizes $L_1$ given oracle
access to $L_2$.

The machine $M_{L_1}$ Can query membership to $L_2$ multiple times (polynomial)
before deciding if a string $w \in_? L_1$.

\begin{lemma}
If $L_1$ Cook-reduces to $L_2$ and $L_2 \in P$, then $L_1 \in P$.
\end{lemma}
\begin{proof}
$L_1$ is decided by a  polynomial-time turing machine $M_{L_1}$,
so it can make at most polynomial queries to $L_2$. 
Since $L_2 \in P$, There exists a polynomial-time turing machine $M_{L_2}$
which solves the membership query.

The total running time for $M_{L_1}$ is in \ptime, so it can make at
most polynomial queries to $M_{L_2}$. Hence, $M_{L_1}$ can simulate
$M_{L_2}$ and solve the membership problem.
\end{proof}

\begin{theorem}
Every language $L \in \nptime$ is Cook-reducible to a sparse language
iff $\nptime \subseteq \text{\ppoly}$.
\end{theorem}

This theorem is significant because we strongly believe that no \nptime-complete
language is sparse! So, we believe that $\nptime \not\subset \text{\ppoly}$.

Since SAT is \nptime-complete, we simply need to show that SAT is cook-reducible
to a sparse language iff $\nptime \subseteq \text{\ppoly}$.

We will exhibit polynomial-time advice string for all inputs of a
given length, to use the power of \ppoly.

\begin{proof}
\textbf{(Forward)~SAT Cook-reducible to a sparse language $L \implies SAT \in \text{\ppoly}$}

There is a polynomial-time machine $M$ which can solve SAT given oracle
access to sparse language $L$.

We want to show that SAT is in $P^poly$.

Let $M$ run in time $p(n)$ on inputs of length $n$

The advice string $a(n)$ we want to give is the oracle behaviour on sparse language $L$.
Since the machine $M$ can ask for string of length at most $p(n)$.

Since the language is sparse, the set of all strings of a given length in $L$ is polynomial.
So, $a(n) = concat(\{ w \in L~\vert~|w| \leq p(n) \})$ where $concat$ concatenates
all the strings.
$a(n)$ will be polynomial in length since the length of each string $w$ is bounded by $p(n)$.
Let $sparse(n)$ be the polynomial that controls the sparsity of $L$ for any string $n$.
That is, for any length $i$, the language $L$ contains at most $sparse(i)$ strings.

The total number of strings in $a(n)$ will be $N = \sum_{i=0}^{p(n)} sparse(i)$, which
is a polynomial in $n$. Hence, $a(n)$ is a legal advice string.

We're done here, we converted oracle access to a sparse language into a 
polynomial advice string.

\textbf{(Backward)~$SAT \in \text{\ppoly} \implies$ SAT Cook-reducible to a sparse language $L$}

We are given a machine $M_{sat}$ which seeks advice $a(n): \mathbb{N} \to \{0, 1\}^*$.
The machine $M_{sat}$ runs for polynomial time $p_sat(n)$.

We need to construct a sparse language $L_{sparse}$, such that given oracle access
to $L_{sparse}$, we can solve SAT using a new machine $M'$.

Consider all strings that are queried by $M_{sat}$ to $M_{poly}$. 
For an input of length $n$, the machine $M_sat$ can query $a$ 
$p_{sat}(n)$ times at maximum. Hence, we the language consisting of the subset 
of $a$ that is sampled by $M_{sat}$ is a sparse language. 
Given access to this language, we can substitute the
function $a$ with the sparse language which contains all advice accessed from $a$.



\end{proof}


