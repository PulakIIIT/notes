\documentclass{article}
\usepackage{minted}
\usepackage{amsmath}
\usepackage{amssymb}
\newcommand*{\Perm}[2]{{}^{#1}\!P_{#2}}%
\newcommand*{\Comb}[2]{{}^{#1}C_{#2}}%
\usepackage{amsfonts}
\author{Siddharth Bhat (20161105)}
\title{Complexity and Advanced Algorithms -- Assignment 3}
\begin{document}
\newcommand{\threesat}{\texttt{3-SAT}~}
\newcommand{\pspace}{\texttt{PSPACE}~}
\newcommand{\np}{\texttt{NP}~}
\newcommand{\nat}{\mathbb{N}~}
\newcommand{\pred}{\mathbb{P}~}
\newcommand{\logspace}{\texttt{LOGSPACE}~}
\newcommand{\nlogspace}{\texttt{NLOGSPACE}~}
\newcommand{\ptime}{\texttt{P}~}
\newcommand{\nptime}{\texttt{NP}~}
\maketitle

\section{Give examples of sparse, non-sparse languages, context-free, non-context-free languages}

\subsection{Sparse language}

$L = \{ x~\vert~x \in \{0, 1\}^*, x~\text{has exactly $2$ bits equal to $1$ in its binary representation} \}$

This is a sparse language, because for a fixed $n$, there are $\Comb{n}{2} = n(n-1)/2 =
O(n^2)$ strings with  exactly $2$ bits as 1. Hence, the number of strings
for any $n$ is polynomial in $n$.


\subsection{Non-Sparse language}
$L = \{ x~\vert~ x \in {0, 1}^*, x~\text{is even} \}$

for a given $n$, there are $2^n$ binary strings, of which $2^n/2 = 2^{(n - 1)} = O(2^n)$
are even. Here, for a fixed $n$, the number of strings is exponential in $n$,
which makes this language non-sparse

\subsection{Context-free language}
$L = \{ a^nb^n~\vert~n \in \mathbb{N} \}$

We know that context free-languages are recognized by adeterministic 
push-down automata. So, we can construct an PDA which pushes 
every time it sees as $a$. When it sees the first $b$, it switches to a state
that pops $a$s. The PDA accepts if it empties the stack, when the string
has been exhausted.

\subsection{Non context-free language}
$L = \{ a^n b^n c^n~\vert~n \in \mathbb{N} \}$

This language is not context free (a rigorous proof can be arrived at by pumping).
Since context free languages can be recognized by push-down automata, the intuition
is that the push-down automata is forced to use the stack to match $a^n b^n$,
after which the stack is empty, and hence cannot match $c^n$.



\section{Is every regular, context-free language decidable in $\logspace$}
\subsection{Regular languages}
Every regular language can be encoded as a deterministic finite automata.
The size of a DFA does not change with problem size. Hence, a DFA can 
be simulated in $O(1)$ space, and is therefore definitely in $\logspace$


\subsection{Context-free languages}
Yes, this can be done in $\logspace$, by using the CYK algorithm.

We use the fact that any context-free grammar has an equivalent chomsky-normal
form. Hence, given a grammar in the chomsky-normal form, we can employ the
CYK algorithm to decide if the grammar is in $\logspace$.

The intuition of the algorithm is that, given a grammar $G$,
with nonterminals $R_1, R_2, \dots R_r$, and an input string
$I = i_1 i_2 \dots i_n$, we bottom-up construct an array
$P[n, n, r]$, where 

\begin{align*}
P[start, length, r] \equiv \text{Can non-terminal $R_r$ produce the substring
$a_{start} \dots a_{start + length -1}$?}
\end{align*}

We bottom-up construct this array $P$, which we will finally query with
$P[1, n, S]$ where $S$ is the start symbol of the grammar. This will tell
us if the "substring $a_1 \dots a_n \equiv a$ can be produced starting
with the non-terminal $S$ (which is the start symbol), thereby answering
the membership query.

Note that $|P| = n^2 r$, so we cannot directly "store" P. However,
since P is recursively defined, we can use recursion to find values
on-the-fly.


\subsubsection{deciding $CFG \in \nlogspace$}

\begin{minted}{python}
# a := input string is a global

def is_substring(start, len, r):
    """Returns whether the substring 
       "a_start a_{start+1} ... a_{start + len - 1}"
       can be produced from the nonterminal R_r

       r := O(1)
       start, len := O(log(n))
    """
    if (len == 1):
        # Return if R_r is of the correct shape to produce a_start
        return (R_r -> a[start])
    else:
        # Let there be a valid production := R_r -> R_p R_q
        # which can consume the string "a".
        p = <guess with non-determinism>
        plen = <guess with non-determinism>
        q = <guess with non-determinism>
        qlen = len - plen

        assert (plen >= 0 && qlen >= 0 && plen + qlen = len)

        # Note that the maximum value of (plen, qlen) = (n/2, n/2).
        # So, when we recurse, we will take at max log(n) steps
        # of the recursive call!
        if plen < qlen:
            if !is_substring(start, plen, p): return false
        else:
            if !is_substring(start + plen, qlen, q): return false


        # Now, this final recursive call is a tail-call, which
        # can be converted into a loop with no use of stack
        # Now that we have checked that the smaller string is legal,
        # we now check the larger substring can be legally produced
        # as a tail call, so we use no extra space
        if plen < qlen:
            return is_substring(start + plen, qlen, q)
        else:
            return is_substring(start, plen, p)

def membership():
    # The start symbol is the first non-terminal by convention
    return p(1, length(a), 0)
\end{minted}



\subsubsection{Deciding $CFG \in \logspace$}

\begin{minted}{python}
# a := input string is a global

def is_substring(start, len, r):
    """Returns whether the substring 
       "a_start a_{start+1} ... a_{start + len - 1}"
       can be produced from the nonterminal R_r

       r := O(1)
       start, len := O(log(n))
    """
    if (len == 1):
        # Return if R_r is of the correct shape to produce a_start
        return (R_r -> a[start])
    else:
        # Let there be a valid production := R_r -> R_p R_q
        # This is O(1), and needs O(1) space to index [1..r]
        for every production R_r -> R_p R_q:
            # Consider all partitions of (plen, qlen) for the strings (R_p, R_q)
            for plen in range(0, len + 1):
                qlen = len - plen

                # Note that the maximum value of (plen, qlen) = (n/2, n/2).
                # So, when we recurse, we will take at max log(n) steps
                # of the recursive call!
                if plen < qlen:
                    if is_substring(start, plen, p):
                        # this is a tail call and takes no extra space.
                        return is_substring(start + plen, qlen, q)

                    return false
                else:
                    if is_substring(start + plen, qlen, q):
                        return is_substring(start, plen, p)
                    return false

def membership():
    # The start symbol is the first non-terminal by convention
    return p(1, length(a), 0)
\end{minted}



\section{Give examples of functions that are not space constructible or time constructible}
\subsection{non time \& space constructible}
Trivially, any non-computable is not time-constructible (indeed, it is
not even constructible!). For example, pick the function $f(x) = \langle x \rangle (x) halts$
where $\langle x \rangle$ is $x$ interpreted as a program. This function
is not constructible by a diagonalization proof.

\section{$k$-ary search on an array}

It can be done in log-space. The intuition is that we require a logarithmic
number of recursive calls, and for each step of the recursion, we need 
a constant ($k$) number of pointers into the memory. Hence, the full
algorithm operates in log-space.

\begin{minted}{python}
# arr is global scope
# arr := list(int)

# needle is global scope
# needle := int

def kary_search():
    kary_search_recur(arr, 0, length(arr));

# [begin, end) indexing.
def kary_search_recur(begin, end):
    assert (end > begin)

    # base case
    # -------------
    if (end == begin + 1):
        return arr[begin] == ix

    # recursive case
    # -------------
    # O(log(|arr|))
    max_ix_smaller_than_x = begin - 1;

    # O(log(|arr|))
    min_ix_larger_than_x = end + 1;

    #k := O(1), since K is independent of the problem size
    for k in range(1, K + 1):
        # ix = O(log(|arr|))
        ix = (end - begin + 1) / K

        # Indexed arr[ix], arr[max_ix_smaller_than_x] = constant space
        if (ix > max_ix_smaller_than_x and  arr[ix] < needle)
            max_ix_smaller_than_x = ix

        if (min_ix_larger_than_x < ix && arr[ix] > needle):
            min_ix_larger_than_x = ix


    assert (max_ix_smaller_than_x < min_ix_larger_than_x)
    # The recursion won't take more stack space because it's a tail
    # recursive call. One could imagine the whole function surrounded
    # by a while(1) {..} loop, which simply re-assigns the
    # values of begin, end!
    # 
    # That is, one can transform a function call of then form:
    # def f(x, y):
    # ...
    # f(x', y')
    # into:
    # def f_no_stack(x, y):
    #     xcur = x, ycur = y
    #     while(1):
    #         <body>
    #         xcur = x', ycur = y'
    # note that f_no_stack consumes no extra stack!
    # https://stackoverflow.com/questions/33923/what-is-tail-recursion
    kary_search_recur(max_ix_smaller_than_x, min_ix_larger_than_x)
\end{minted}


\end{document}
