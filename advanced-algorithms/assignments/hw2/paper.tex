\documentclass{article}
\usepackage{minted}
\usepackage{amsmath}
\usepackage{amsfonts}
\author{Siddharth Bhat (20161105)}
\title{Complexity and Advanced Algorithms -- Assignment 2}
\begin{document}
\newcommand{\threesat}{\texttt{3-SAT}~}
\newcommand{\pspace}{\texttt{PSPACE}~}
\newcommand{\np}{\texttt{NP}~}
\newcommand{\nat}{\mathbb{N}~}
\newcommand{\pred}{\mathbb{P}~}
\newcommand{\logspace}{\texttt{LOGSPACE}~}
\newcommand{\ptime}{\texttt{P}~}
\newcommand{\nptime}{\texttt{NP}~}
\maketitle

\section{Write appropriate quantified formulae}
\subsection{infinitely many primes}
We create a predicate called $Prime : \nat \to \pred$ which is true
when a number is prime. We use this to define the infinitude of primes.

\begin{align*}
    &Prime: \nat \to \pred \\
    &Prime(n) \equiv \forall k \in \nat, 2 \leq k < n \implies n \% k \neq 0
    \\
    \\
    &Infitude \equiv \forall n \in \nat, Prime(n) \implies \exists m \in \nat, m > n \land Prime(m)
\end{align*}

Note that using the function $Prime$ is nothing special --- it is simply
shorthand for substituting the expression of $Prime$ into the expression
for $Infitude$. I will continue to do this in the next question as well.
This does not affect the correctness of the answer, since it can be converted
into one large logical formula. Writing it this way simply makes it
easier to reason about.

\subsection{Every pair of positive integers have a unique GCD}

We first define a predicate $CD: \nat \times \nat \times \nat \to \pred$, where
$CD(n, m, d)$ means that $d$ is a common divisor of $n, m$.

This is used to define $GCD: \nat \times \nat \times \nat \to \pred$, where
$GCD(n, m, d)$ means that $d$ is the $GCD$ of $n$ and $m$. 

Finally, the definition of $GCD$ is used to define the uniqueness of $GCD$.
\begin{align*}
    &CD: \nat \times \nat \times \nat \to \pred \\
    &CD(n, m, d) \equiv n \% d = 0 \land m \% d = 0
    \\
    \\
    &GCD: \nat \times \nat \times \nat \to \pred \\
    &GCD(n, m, d) \equiv CD(n, m, d) \land \forall d' \in \nat, d' > d \implies \lnot CD(n, m, d')
    \\
    \\
    &UniqueGCD \equiv \forall n, m, d,  d' \in \nat, GCD(n, m, d) \land GCD(n, m, d') \implies d = d'
\end{align*}

\section{Sorting  $\in \logspace$}
Assume we wish to sort in ascending order (The solution is symmetric in the
descending order case).


Let us index the output list as $\langle o_1, o_2, o_3, \dots, o_N$

In the output list, the numbers indexed by the set $[1, k_1]$ numbers are smaller
than all the others: that is, they are greater than $0$ of the other numbers in the list.

The numbers in the range of $[k_1 + 1, k_0]$ numbers are greater than $1$ of the
other numbers. 

If we generalize, the numbers $[k_i + 1, k_{i+1}]$ are greater than $i - 1$.

we can iterate over the count $i - 1$ with a variable
\texttt{nums\_greater\_required}, which represents the count of numbers 
the output must be greater than.

Next, for each \texttt{nums\_greater\_required}, we check each number in the list
(indexed by \texttt{cur\_num\_position}) the count of numbers it is greater than
(by walking the list using \texttt{compare\_num\_position}, and storing the count
of numbers the current number is greater than in \texttt{cur\_nums\_greater}).

If this count is equal to \texttt{nums\_greater\_required}, we have the correct
number.


\begin{minted}{python}
def logsort_unique(input):
    # nums_greater_required := space O(log(|input|))
    for nums_greater_required in range(len(input)):

        # cur_num_position := space(O(log(|input|)))
        for cur_num_position in range(len(input)):

            # 0 <= cur_nums_greater <= nums_greater_required => 
            # cur_nums_greater := space O(log(|input|))
            cur_nums_greater = 0


            # compare_num_position := space(O(log(|input|)))
            for compare_num_position in range(len(input)):


                # input[cur_num_position] := space O(1)
                # input[compare_num_position] := space O(1)
                if input[cur_num_position] > input[compare_num_position]:
                    cur_nums_greater += 1


            if cur_nums_greater == nums_greater_required:
                print (input[cur_num_position])
\end{minted}

The time complexity of this is $O(|input| * |input| * |input|) = O(|input|^3)$



\section{\pspace is closed under union and intersection}
Let $L_1$ and $L_2$ be languages in \pspace. Let $M_1$ and $M_2$ be the
turing machines which decide $L_1$ and $L_2$ respectively.

\subsection{$L_1 \cup L_2 \in \pspace$}
We create a new machine $M$ which accepts the language $L_1 \cup L_2$.

This machine first runs $M_1$, and stores the output of $M_1$ in bit $b_1$.
Since $M_1 \in \pspace$, $M$ will not run out of space. Since $M_1$
is a decision procedure, we are guaranteed it will halt, so $M$
waiting for $M_1$ to halti is legal.

Similarly, $M$ then runs $M_2$ and stores $b_2$. finally, $M$ outputs
$b_1 \lor b_2$. 

$M$ accepts only when either $M_1$ or $M_2$ accept, that is, $M$ accepts
if a string belongs to either language.


\subsection{$L_1 \cap L_2 \in \pspace$}
Repeat the exact same construction, except change the output bit from
$b_1 \lor b_2$ to $b_1 \land b_2$.

$M$ accepts only when both $M_1$ and $M_2$ accept, that is, $M$ accepts
only if a string belongs to both languages.



\section{Equivalence between \np as nondeterministic solver v/s deterministic verifier}
\subsection{nondeterministic solver $\implies$ deterministic verifier}
Consider a language $L$, which has a non-deterministic solver $S \in \nptime$.
We construct a verifier $V$, and a function $witness: L \to \{0, 1\}^*$ which
constructs the witness string $y = witness(w)$ for a given string $w \in L$.

Consider a successful run of the non-deterministic solver $S$ for a
given $w \in L$. We create the witness string $y$ by \textit{recording} all of the
values that the non-deterministic solver $S$ \textit{guessed for a successful run}.

So, $y = \text{collection of all guesses made by $S$}$

We create the verifier $V$ by running $S$ deterministically, and at every 
point $S$ tries to use non-determinism, we use the \textit{correct value}
that is stored in $y$. 

Thus, if a successful run exists for $S$ (that is, $w \in L$),
then the verifier $V$ will \textit{have access to the right guesses}.  On the other hand,
if $w \notin L$, the there \textit{are no right guesses}, so $V$ will reject.



\subsection{deterministic verifier $\implies$ nondeterministic solver}
Consider a language $L$, with a deterministic verifier $V \in \ptime$.

We design a nondeterministic solver $S \in \nptime$, which on input string
$w$, guesses the witnesss string $y$ using nondeterminism and
then querifies the verifier with $V(w, y)$. That is,

$S(w) = V(w, \overline{y})$ where $\overline{y}$ is guessed using
nondeterminism.  This is in \np, since the verifier $V$ runs in polynomial
time, and the non-determinism is used to guess $\overline{y}$.


\end{document}
