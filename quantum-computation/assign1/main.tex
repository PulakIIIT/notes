\documentclass[11pt]{book}
\usepackage[sc,osf]{mathpazo}   % With old-style figures and real smallcaps.
\linespread{1.025}              % Palatino leads a little more leading
% Euler for math and numbers
\usepackage[euler-digits,small]{eulervm}
\usepackage{amsmath,amssymb}
\usepackage{amsthm}
\usepackage{graphicx}
\usepackage{makeidx}
\usepackage{algpseudocode}
\usepackage{algorithm}
\usepackage{physics}
\usepackage{listing}
\usepackage{comment}
\usepackage{booktabs}
\usepackage{subcaption}
\usepackage{enumitem}
\usepackage{minted}

\usepackage{epsfig}
\usepackage{tabularx}
\usepackage{latexsym}

\title{Quantum information and computation: Assignment 1}
\author{Siddharth Bhat}
\date{}
\begin{document}
\section{Q1 -- matrix representation for $\ket{\phi_k}\bra{\phi_j}$, in the
orthonormal $\ket{v_i}$ basis}

Perform change of basis.

\section{Q2 -- positive operator is Hermitian}
We first show that a positive operator is normal, and this automatically
implies that it is Hermitian.

To show that a positive operator is normal, we consider $A^\dagger A$

Now that we know that it is normal, by spectral decomposition, it
posesses an eigenbasis. We now show that all of its eigenvalues are real.
This is now a matrix with real entries on the diagonal, which is hermitian.

To show that the eigenvalues are real, let $\ket \lambda$ be an eigenvector
with magnitude $1$ and eigenvalue $\lambda$.
\begin{align*}
    \bra \lambda A \ket \lambda \geq 0 \qquad
    \lambda \braket{\lambda} = \lambda \geq 0
\end{align*}
Hence, the eigenvalues are real and positive, and therefore it is Hermitian.

\begin{align*}
\end{align*}

\section{Q3 -- $A^\dagger A$ is positive}
\begin{align*}
    \forall v \in V, ~\bra v A^\dagger A \ket v = \bra {A v} \ket {A v} = \norm{A v}^2 \geq 0
\end{align*}
Hence, $A^\dagger A$ is positive.

\section{Q4. Eigenvalues of a projector $P$ are either $0$ or $1$}
Let $\ket \lambda$ be an eigenvector of $P$ with associated eigenvalue
$\lambda$.  

\begin{align*}
    P^2 (\ket \lambda) = \lambda (P \ket \lambda) = \lambda ^2 \ket \lambda \qquad
    P (\ket \lambda) = \lambda \ket \lambda
\end{align*}

However, since $P$ is a projector, $P^2 = P$, and therefore, $\lambda^2 = \lambda$.
The roots of this equation are $0, 1$. Hence, $\lambda \in \{0, 1\}$.

\section{Q5. Tensor product of two unitary operators is unitary}
Let $U, V$ be unitary operators.

\begin{align*}
    &\bra{U u \otimes V v} \ket{U u \otimes V v} = \\
    &\bra{u \otimes v} (U^\dagger \otimes V^\dagger) (U \otimes V) \ket{u \otimes v} =  \\
    &\bra{u \otimes v} (U^\dagger U \otimes V^\dagger V) \ket{u \otimes v} = \\
    &\bra{u \otimes v} I \otimes I \ket{u \otimes v} = \\
    &\bra{u \otimes v} \ket{u \otimes v} = \\
\end{align*}
Hence, $U \otimes V$ is unitary since it preserves inner products.

\section{Q6. Tensor product of projectors is a projector}
Let $P, Q$ be projectors. $P \equiv \sum_{i=1}^l \ket{i}\bra{i}$.
$Q \equiv \sum_{j=1}^k \ket{j}\bra{j}$.


\begin{align*}
    P \otimes Q &\equiv (\sum_{i=1}^l \ket{i}\bra{i}) \otimes (\sum_{j=1}^k \ket{j}\bra{j}) \\
                &\equiv \sum_{i=1}^l \sum_{j=1}^k \ket{ij}\bra{ij} \\
\end{align*}

Which is in the form of a projector, in that it leaves $\ket{ij}$ unchanged,
and sends every other vector to $0$. So, it projects vectors onto the
subspace spanned by $\ket{ij}$.


\section{Q7. Find $\log$ and square root of matrix}
\section{Q8. Trace properties}
\subsection{$Tr(AB) = Tr(BA)$}
\begin{align*}
    Tr(AB) = \sum_z (AB)_{zz} = \sum_z \sum_k A_{zk} B_{kz} = \sum_z \sum_k B_{kz} A_{kz} = \sum_z (BA)_{zz} = Tr(BA)
\end{align*}

\subsection{$Tr(A + B) = Tr(A) + Tr(B)$}
\begin{align*}
    Tr(A + B) = \sum_z (A + B)_{zz} = \sum_z A_{zz} + B_{zz} = Tr(A) + Tr(B)
\end{align*}


\subsection{$Tr(2A) = 2Tr(A)$}
\begin{align*}
    Tr(2A) = \sum_z (2A)_{zz} = \sum_z 2 A_{zz} 2 \sum_z A_{zz} = 2 \Tr(A)
\end{align*}

\begin{align*}
\end{align*}

\section{Commutator properties}
\subsection{$[A, B] = -[B, A]$}
\begin{align*}
    [A, B] = AB - BA = - (BA - AB) = - [B, A]
\end{align*}

\subsection{$\frac{[A, B] + \{A, B\}}{2} = AB$}
\begin{align*}
    \frac{[A, B] + \{A, B\}}{2} = \frac{(AB - BA) + (AB + BA)}{2} = AB
\end{align*}

\section{Express polar decomposition as outer product}
\section{Find left and right polar decomposition}
\end{document}
