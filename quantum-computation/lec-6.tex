\chapter{Quantum Computing: Simon's Problem}

We are given a function $f: \{0, 1\}^n \to \{0, 1\}^n$. There exists
an (unknown to us) $s \in \{0, 1\}^n$.


We are promised that:
\begin{itemize}
\item $\forall x \in \{0, 1\}^n,~ f(x \oplus s) = f(x)$.
\item $f(y) = f(x) \iff  [y = x \xor s] \lor [y = x]$.
\end{itemize}
Goal is to find $s$.

\begin{align*}
f(y \xor x) = f((y \xor s) \xor (x \xor s))
\end{align*}


Supppose we have checked $f$ for $y_1, y_2, \dots y_k$.
For each pair, have a trial $s_{i, j} = y_i \xor y_j$.  If $y_i = y_j$, then
$s = s_{i, j}$.

What's the runtime? We can find a lower bound. There are $\frac{k(k-1)}{2}$
pairs, but some $s_{i, j}$ may be the same. 
\begin{align*}
\frac{k^2}{2} > \frac{k(k-1)}{2} > 2^n > \frac{2^n}{2} \qquad k > \sqrt{2^n}
\end{align*}

Classically, we will need to perform at least $\sqrt{2^n}$ tests.

\section{Quantum solution} 
We will create a linear algorithm.

\begin{align*}
    &\ket{\psi_1} \equiv \ket0^{\tensor n} \ket0^{\tensor n} \\
    &\ket{\psi_2} \equiv \H^{\tensor n} \tensor I^{\tensor n} (\psi_1) = 
    2^{\frac{-n}{2}} \sum_{y \in {0, 1}^n} \ket y \ket 0^{\tensor n} \\
    &\ket{\psi_3} \equiv U_f(\psi_2) = 2^\frac{-n}{2} \sum_{y \in \{0, 1\}^n} \ket y \ket{f(y)} \\
\end{align*}

Notice that since $f$ is two-to-one, for every component $\ket{f(y)}$ of the second
qubit, we will have first qubits: $\ket{y}, \ket{y \xor s}$.

\begin{align*}
    &\ket{\psi_3}  \equiv 2^\frac{-n}{2} \sum_{y \in \{0, 1\}^n} \frac{\l( \ket y + \ket{y \xor s} \r) \ket{f(y)}}{2} \\
\end{align*}

We divide by two since we are double counting the pair $(y, y \xor s)$, since later
we will hit $(y' \equiv y \xor s, y' \xor s = y \xor s \xor s = y)$.

We now measure the second qubit:
\begin{align*}
    &\ket{\psi_4} \equiv M(\psi_3) =  \frac{\ket y + \ket{y \xor s}}{\sqrt 2} \qquad \text{(for some $y$ with prob. $2^{-(n - 1)}$)} \\
\end{align*}
We next apply hadamard to $\ket{\psi_4}$, which we will write using the
identity:
$$H^{\tensor n}\ket x \equiv 2^{\frac{-n}{2}} \sum_{z \in \{0, 1\}^n} (-1)^{x \cdot z} \ket z$$
Here, $x \cdot y \equiv \text{inner product over $\Z/2\Z$}$.

\begin{align*}
    &\ket{\psi_5} \equiv H^{\tensor n} (\psi_4) = 
    \frac{\sum_{z \in \{0, 1\}^n \l( (-1)^{y \cdot z} + (-1)^{(y \xor s) \cdot z} \r) \ket z }}
        {2^{- \frac{n-1}{2}}} \\
    = 
    &\frac{\sum_{z \in \{0, 1\}^n \l( (-1)^{y \cdot z} \l[ 1 + (-1)^{(s \cdot z)}\r] \r) \ket z }}
        {2^{- \frac{n-1}{2}}} \\
\end{align*}

If $(s . z = 1)$, then the coefficient $1 + (-1)^{s \cdot z} = 1 - 1 = 0$. So,
we will only get $\ket z$ which have $(s . z = 0)$.

Using this, we can get $z$'s such that $(s . z) = 0$.We can gather $(n - 1)$
of these and solve the system $(s . z_i) = 0$ for $s$.

\section{Probability computation}
Next, we perform some probability computations to check how many of the vectors
will be linearly independent.  The number of matrices is $2^{n \times (n - 1)}$.

Now, we need to check that this matrix is invertible for the system of linear
equations to be solvable. Note that for the first vector $z_1$, only the zero
vector is not allowed. So, the number of $z_1$'s is $(2^n - 1)$. Now,
the second vector $z_2$ must be linearly independent of the first vector $z_1$,
for which we have $(2^n - 2)$ choices, since we must remove the $2$ vectors
that are linearly dependent on $z_1$. For $z_3$, we need to eliminate the $4$ 
vectors that are combinations of $z_1, z_2$. So, number of choices for $z_3$
will be $(2^n - 2^2)$. For $z_i$, we will have $(2^n - 2^{i - 1})$.

Total: 

\begin{align*}
    \frac{\prod_{i=1}^n(2^n - 2^{i - 1})}{2^{n \times (n - 1)}} > \frac{1}{4}
\end{align*}
