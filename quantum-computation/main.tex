\documentclass[11pt]{book}
%\documentclass[10pt]{llncs}
%\usepackage{llncsdoc}
\usepackage{amsmath,amssymb}
\usepackage{graphicx}
\usepackage{makeidx}
\usepackage{algpseudocode}
\usepackage{algorithm}
\usepackage{listing}
\usepackage{comment}
\usepackage{physics}
% look for package for quantum computing!
\evensidemargin=0.20in
\oddsidemargin=0.20in
\topmargin=0.2in
%\headheight=0.0in
%\headsep=0.0in
%\setlength{\parskip}{0mm}
%\setlength{\parindent}{4mm}
\setlength{\textwidth}{6.4in}
\setlength{\textheight}{8.5in}
%\leftmargin -2in
%\setlength{\rightmargin}{-2in}
%\usepackage{epsf}
%\usepackage{url}



\usepackage{booktabs}   %% For formal tables:
                        %% http://ctan.org/pkg/booktabs
\usepackage{subcaption} %% For complex figures with subfigures/subcaptions
                        %% http://ctan.org/pkg/subcaption
\usepackage{enumitem}
\usepackage{minted}
%\newminted{fortran}{fontsize=\footnotesize}

\usepackage{xargs}
\usepackage[colorinlistoftodos,prependcaption,textsize=tiny]{todonotes}

\usepackage{hyperref}
\hypersetup{
    colorlinks,
}

\usepackage{epsfig}
\usepackage{tabularx}
\usepackage{latexsym}
\newtheorem{lemma}{Lemma}
\newtheorem{observation}{Observation}
\newtheorem{proof}{Proof}
\newcommand\ddfrac[2]{\frac{\displaystyle #1}{\displaystyle #2}}

\def\qed{$\Box$}
\def\proof{\textit{Proof. }}
\newtheorem{corollary}{Corollary}
\newtheorem{theorem}{Theorem}

\title{Quantum computation and information - Indranil Chakravarty}
\author{Siddharth Bhat}
\date{}

\begin{document}

\maketitle
\tableofcontents
% http://mirrors.ibiblio.org/CTAN/macros/latex/contrib/physics/physics.pdf
\newcommand{\qdot}{{\dot q}}

\chapter{Lagrangian, Hamiltonian mechanics}

Mechanics in terms of generalized coords.
\section{Lagrangian}
Define a functional. $L$ over the config. space of partibles $q^i$, $qdot^i$.
$L = L(q^i, qdot^i)$.  We have an explicit dependence on $t$.



$L = KE - PE$

Assuming a 1-particle system of unit mass,
$$L = \frac{1}{2} \qdot^2 - V(q)$$

Assuming an n-particle system of unit mass,
$$L = \sum_i \frac{1}{2} {qdot^i}^2 - V(q^i)$$ 

\section{Variational principle}

Take a minimum path from $A$ to $B$. Now notice that the path that is
slightly different from this path will have some delta from the minimum.

Action
$$S(t0, t1) = \int L \dd t = \int_{t0}^{t1} L(q^i, qdot^i) \dd t$$.
Least action: $\delta S = 0$

% \begin{align*}
%     \delta S &= \delta \int L(q^i, qdot^i) \dd{t} \\
%              &= \int \delta L(q^i, qdot^i) \dd{t} \\
%              &= \int \pdv{L}{q^i} \delta q^i + \pdv{L}{qdot^i} \delta qdot^i \dd{t} \\
% \begin{align*}





\end{document}
