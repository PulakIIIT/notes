\documentclass[11pt]{book}
%\documentclass[10pt]{llncs}
%\usepackage{llncsdoc}
\usepackage[sc,osf]{mathpazo}   % With old-style figures and real smallcaps.
\linespread{1.025}              % Palatino leads a little more leading
\usepackage{amsmath,amssymb}
\usepackage{graphicx}
\usepackage{makeidx}
\usepackage{algpseudocode}
\usepackage{algorithm}
\usepackage{listing}
\evensidemargin=0.20in
\oddsidemargin=0.20in
\topmargin=0.2in
%\headheight=0.0in
%\headsep=0.0in
%\setlength{\parskip}{0mm}
%\setlength{\parindent}{4mm}
\setlength{\textwidth}{6.4in}
\setlength{\textheight}{8.5in}
%\leftmargin -2in
%\setlength{\rightmargin}{-2in}
%\usepackage{epsf}
%\usepackage{url}

\usepackage{booktabs}   %% For formal tables:
                        %% http://ctan.org/pkg/booktabs
\usepackage{subcaption} %% For complex figures with subfigures/subcaptions
                        %% http://ctan.org/pkg/subcaption
\usepackage{enumitem}
%\usepackage{minted}
%\newminted{fortran}{fontsize=\footnotesize}

\usepackage{xargs}
\usepackage[colorinlistoftodos,prependcaption,textsize=tiny]{todonotes}

\usepackage{hyperref}
\hypersetup{
    colorlinks,
    citecolor=blue,
    filecolor=blue,
    linkcolor=blue,
    urlcolor=blue
}

\usepackage{epsfig}
\usepackage{tabularx}
\usepackage{latexsym}
\newcommand\ddfrac[2]{\frac{\displaystyle #1}{\displaystyle #2}}

\def\qed{$\Box$}
\newtheorem{corollary}{Corollary}
\newtheorem{theorem}{Theorem}
\newtheorem{definition}{Definition}
\newtheorem{lemma}{Lemma}
\newtheorem{observation}{Observation}
\newtheorem{proof}{Proof}

\title{Philosophy: God and knowledge}
\author{Siddharth Bhat}
\date{Monsoon 2019}

\begin{document}

\maketitle
\tableofcontents

\chapter{Introductory class}

The difference between mathematical truths and factual truths.

\section{The problem of psychologism}

Even though mathematics seems objective, we still filter
mathematics through our psychology, so it might not be
as objective as we want it to be.

\chapter{Descartes: Meditations, Doubt and Belief}

He begins with \textit{methodic doubt}, where we do not question each and
every particular belief. Rather, we question the source of knowledge of our
beliefs. If this source of knowledge is open to even the least doubt, then 
we reject all beliefs which arise from this knowledge.

We start with perception, the most obvious source of knowledge. We know that
our perception can easily deceive us --- for example, madmen are clearly
deceived by their perceptions. We do not need to presuppose such an
extreme position, however. Dreams are another example where our perceptions
deceive us. Yet another example is that of temperature: We can
change our perception of hot or cold, by dipping our hands into a
jug of hot or cold water.

As an aside, ``neurotics'' have some connection with reality, ``psychotics''
have a clean break with reality. So ``madmen'' can be subdivided into
these two classes, by Freud's classification. Freud says that normal
dreams as wish fulfillmentt. That is, you are able to fulfill your wishes
that have been repressed. Also, nightmares are your subconscious
trying to work through some form of trauma. It is the subconscious
trying to take trauma and rework it into a manageable form. They
had a patient who had a recurrent nightmare of being shot
in the head. Clearly, this is not ``wish fulfillment''.
He creates the explanation that we constantly relive a bad
experience to learn how to manage it.

Now, Descartes makes a decisive move --- Even if we are dreaming,
the \textit{content} of our dreams as mental images are
\textit{embedded in reality}.
For example, think of an imaginary/fictitious object, such
as a unicorn or a fantastic alien, or an orientable Mobius strip.
Even so, there would be elements which are recognizable.

At the level of mental images, the images cannot be doubted, though
what they are images \textit{of} may not exist. The content
of the images in terms of shape, figure, and color must exist in our head,
for them to \textit{be images}. Hence, we cannot reject the existence
of figure and color.

We are also unable to reject time. Dreams can \textit{distort} time, but
we can still feel the passage of time in a dream. We are trying to move
from something that is mental, to something that is more than our thought
process.

\textbf{Subjective idealism} says that it is just my flow of consciousness,
and everything is confined to \textit{my} consciousness.

\textbf{Objective idealism} says that to even possess consciousness, we need
abstract entities that transcend just our consciousness --- for example,
time.

Since time is required to be able to dream, time cannot be reduced
to dreaming. This is called as \textbf{Transcendental philosophy}.
``Transcendental'' since we find that at the very moment we
turn inwards, a different reality. We find a transcendence in
imminence (imminence = the inner). That is, me observing
myself internally causes me to come across certain structures that
I cannot reduce to myself. For example, that of temporality.

Hence, he believes that mathematics contains some measure of
certainty. The natural sciences have an element of empiricism
mixed into them. They presuppose things existing in the external
world. This presupposition is open to doubt. On the other hand,
arithmetic does not need to hypothesize the existence of
physical numbers. The ideal entity of numbers (in the sense of Plato's forms)
is something which is independent of our subjective experience.

TODO: I left this class to go back home. finish this

\chapter{Lecture 4}
Descartes is not happy with saying that this is how we are physically
constituted. So he's looking for this absolute, indubitable ground. Therefore,
he has to actually prove that there is some transcendent source which
guarantees your perception. "clear and distinct perception" is what we require,
but this is hard to get to.

"Doubt has taken the ground beneath my feet" --- At the same time, I can no
longer remain on the surface. So far, I had been on the surface and it was not
a problem for me since I had never doubted. Once I began this quest, I do not
find any absolutely certain ground. I cannot forgot about this shaky ground
and continue to live there.

He asks what he can be certain of, given hyperbolic doubt --- perception,
memory, body, figure, extension and movement are all fiction. He asks whether
his \textit{very existence}, his notion of I is in itself something real. The
act of thinking is the conformation of my existence as a thinking being. As long
as I am doubting, I exist as a doubting being. As long as I am being deceived,
I exist as a being who is deceived, and so on.

I think \textit{therefore} I am --- the \textit{therefore} is one of assigning
thinking and being equal / identity. It is \textbf{not} inference.

% need to have a \aside{...} to log interesting tangents.
% Maybe I should call it \kannan{...}
\subsection{The problem of personal identity}
Descartes walks into a bar. The bartender asks, "what will it be?" Descartes
responds, "I think not", and he promptly disappears.

Memory constitutes a large fragment of our personal identity. On Descartes'
theory, there is no reference to the past. We can only affirm out existence
in the now.

\subsection{Solipsilm: The sole I}

How does one get out of ones mind, to access the rest of the physical realm?
Currently, I only know my own mental appearances. Everything else is an
illusion.

Descartes wishes to reject this branch of thought. For example, he questions
whether he stops existing when we sleep. He wishes to come up with an 
explanation for our sense of continuity.


After introducing "I think therefore I am", he continues saying "I do not know
very clearly what I am, except for the fact that I am".

Imagination is imagistic, and images have already been rejected, insofar
as they refer to the external world. However, imagination which refers to the
internal world is indubitable.

The only proof certificate of existence we have found so far is
that of thinking.h

At some point, Descartes' defines the body as that which cannot be self-moved.
This is different from Aristotelian physics, where everything is self-moved
according to its telos. Descartes has broken tradition from the Aristotelian
model, to somewhat arrive at the notion of inertia. (Netwon, a century later
formalizes this).

Descartes rejects the reality of his perception, but marvels at the richness of
his perception. A mirage does not truly exist, but it exists \textit{for me}.
This brings out a gap between the external world and the internal world.
Yet, we believe that we know the external world better than the internal world.

\subsubsection{The wax example}
Descartes takes something we are absolutely sure of: For example, a piece of
wax.  He waxes poetic about this piece of wax, its physical characteristics
(sound, color, smell). When we begin to melt the wax, it heats up, loses color
and smell, becomes liquid. He then asks: "does the same wax remain after this
change?" We must confess that it does remain, no one would say otherwise.
However, by what criterion are we judging that it is the same? All of its
physical characteristics have been altered!

He wishes to retain physical objects. But by his standards for indubitable
knowledge, he is forced to recourse to God. 

\subsection{Why does modern science reject teleology?}
Teleology presupposes what is to be explained. If we say that something moves
because it is supposed to move, this is not an explanation.

\section{Lecture 5}
Empiricism is not the annihilation of reason. Rather, it is a viewpoint on where
reason lies: Does it lie in the mind, or does it lie within nature? That is,
are all our ideas and connections between ideas are "out there", and we are
passively listening to these. We are discovering the laws of nature, not
inventing them.

Descartes with "cogito ergo sum" is closing the gap between subject and object.
He is saying: "only that really exists for sure, which I can subjectively
grasp". The usual way of looking at things would be to say that a table exists,
and I am \textit{able to} perceive it. Descartes' view of this is that I can be
aware of the existence of the table \textit{since} I am perceiving it.

Principle of coherence: My beliefs are true if all my beliefs are coherent.

He decides to abstract from everything that is unnecessary from the piece of
wax to define the piece of wax. It's not the form, since the form changes. The
extension changes.


So, we need the notion of a god to protect us from a demon who might hinder our
clear and distinct perception. There is a circularity in his logic, where he
starts out by assuming that we need a clear and distinct perception, and
therefore we need a God. He then continues to use God to make whatever we see
through clear and distinct perception to be true.

\section{Lecture 6}

The intuition that descartes has is referred to as "clear and distinct perception",
"natural light of reason" --- This is used to describe "self-evident" objects
for Descartes. For Descartes, rationality is equivalent to self-evidence.
However, he still believes that is possible to be deceived by clear
and distinct perception due to his standards of truth.

\subsection{The circularity of the argument}

\begin{itemize}
    \item Whatever I clearly and distinctly perceive is true.
    \item I clearly and distinctly perceive that god is not a deceiver.
    \item Therefore, god has created me in such a way that everything I clearly and distinctly see is true
\end{itemize}

\begin{itemize}
    \item I can be certain that whatever I can see clearly and distinctly is if
        God exists
    \item I can know and be certain that God exists and is not a deceiver only
        if I know thatever I see clearly and distinctly is true
\end{itemize}

\subsection{Ideas involved in the arguments}

\begin{itemize}
    \item \emph{Formal reality}: Any idea as a mental object has formal reality. Even
        if we are hallucinating, the hallucination has a formal reality, as per
        the experience of the observer.
    \item \emph{Objective reality}: Idea insofar as it represents something outside
        itself. That is, the thought represents an object.
\end{itemize}

Formal reality cannot be false, due to "cogito ergo sum". There are all true as
figments of imagination, and therefore possess formal reality.

\subsection{Tangent}
Words like "this", "that", "now" and so on in modern philosophy are called
"indexicles". These words are somewhat special in language, since their meaning
is tied to the context of their utterance. When we say "this", we are stepping
out of language, into reality. Philosophers have argued that even "this" is a
concept which has a kind of universality to it. We can understand the word
"this" even if we do not know what specific "this" it refers to. Descartes was
the first to elucidate this with the wax example.

Wittgenstein argues against ostensive definitions (definitions which involve
pointing) --- ostensive definitions imply that meaning can be found in the
thing we are pointing to.  Alternatively, ostensive definitions imply that
language gets it meaning from reality. 

Let us say we say the sentence "this is a table", by pointing at a table.
However, the child who we are teaching about "table" needs to realize that
"table" does not refer to, say, the wood, or the color of the table.  

Wittgenstein calls this as "playing the language game".

\subsection{Formal versus Objective reality}

\begin{itemize}
    \item There is as much reality in the effect as there is in the cause.
    \item something cannot arise from nothing.
    \item something more perfect cannot arise from nothing.
    \item all ideas have formal reality.
    \item objective mode of being belongs to ideas, formal modes of being
          is available to all causes of ideas.
    \item god is infinite.
    \item Therefore I could not have generated the idea.
    \item hence god must exist to affirm god.
    \item now, god has a formal reality. However, we still need to be able to
          pull god out into an objective reality.
\end{itemize}

The cause must have as much reality as its effect. From this it follows,

\begin{itemize}
    \item something cannot proceed from nothing
    \item what has more reality within itself cannot proceed from something
        with less reality. If not, this too opens up a gap of reality
\end{itemize}

There is a formal reality to all ideas, but some have greater objective reality.

I have many ideas in my head, all of which have formal reality. We wish to
judge its objective reality. For example, let us pick the idea of substance".
The "substance" is a formal reality which has an "effect". For many of these
ideas like "motion", since I can think that I am moving, I can get the idea of
of "motion" myself, from the fact that I think I can move myself. So these
ideas of the physical world, I can access insofar as I can experience them.

However, there is only one idea that is so perfect, that I cannot derive from
myself since I am an imperfect being. Being finite, I cannot derive the idea of
the infinite. Being an imperfect being, I cannot derive the idea of the perfect
God.

The idea of God, insofar as it is a formal reality in my head, carries all of
these attributes. 

Now, every idea as an effect requires a cause, since nothing can come from
nothing. The only idea I can clearly and distinctly see, but cannot produce is
that of God. This depends on the principle that the cause (the ideas that lead
to the new idea) has to have at least as much reality as the effect (the new
idea that is generated). If this were not the case, then this too opens up a
gap, where the there is gap of reality between the cause and the effect.

Unfortunately for us, he does not believe that mathematics is perfect since he
believes in psychologism. 

However, the question is now whether God can also be attributed to
psychologism.  He believes that God is infinite, and therefore he cannot access
God. He also believes that he cannot get to the infinite by negating the
finite. In some sense, he's a intuioninst, in the mathematical sense of the
word.  Therefore, he claims that we need the capacity to negate a fact to be
able to negate it.

\section{Leture 7}

The correspondence theory of truth --- my ideas have to correspond to the
world outside.

If we have a physical object that we are viewing, then let us assume that
the object we are viewing has formal reality. Our thoughts naturally have
a formal reality (by definition). We now need to link the objective reality
of the object to the formal reality of the object.

In modern science, we do not have correspondence --- we do not need to see
atoms in order to posit atoms. What we see is indeed not what we get. The
modern scientific epistemology has broken with the idea that our ideas need to
correspond with reality in an immediate way.

The idea that what we see has to directly correspond with what we see is known
as naive realism --- naive as opposed to scientific realism, which explains what
we see in terms of complex structures.

There is another problem that occurs at the mathematical level. The correspondence
does not imply existence. For example, there is  mathematical correspondence
between reality and our mathematical ideas, but they (need not) correlate to
reality.

An example from aristotelian logic would be to say "all men are mortal" --- what
we are saying is that if there was such a thing as a man, then they would be mortal.
({\texttt{forall m in men, mortal(m)}}). Hence, there is a gap between truth
and validity.

An objection to descartes is that the ideas we have are in contradiction to the
\textit{formal} idea of God. Hence, we need some way to transport a formal god
into an objective god.

Now, he needs an argument called as the "ontological argument" which argues
that perfection \textit{must} exist.  That is, something that is perfect must
exist. This is extremely suspect.

\begin{itemize}
    \item As formal reality, all ideas that we have in our head have the same
    level of existence.  Ideas might have different contents, or objective
    referents.  
    
    \item This difference in content must have been generated from
    somewhere. If I am simply hallucinating, then I should not be able to
    create distinctions in my head.  There are specific differences in our
    ideas, which cannot simply be explained by formal existence. One can say
    that this come from other formal ideas. However, this would lead to some
    kind of infinite regress.
\end{itemize}


How are we guaranteed continuity of existence? Indeed, the great mystery of
time is that it is continuous (the existence of the passage of time or the
arrow of time).  At the same time, it is discrete,  in that it is broken up
into instants. As we speak, it becomes the past. We cannot point to the
present. If time were discrete, what connects the I across two instants?
If time were truly continuous, then it would be eternal, without change.


If we take time to be discrete, Descartes argues that at each instant of time,
we have a cause that produces the I in each moment. 

Buddhists don't believe in a soul. They believe that each preceding moment 
generates the succeeding moment out of itself. There is causality operating
at the level of our soul. The soul is nothing but causality occurring over
the string of moments. This is called the "theory of dependent origination".
A cool quote by heidigger: "In passing, time remains". Hereclitis said "you
cannot step into the same river twice".


\section{Escaping the vicious circle of God --- The trademark argument}

The cartesian god is a replacement for the problem of infinite regress when
it comes to describing the flow of time. If we have to postulate an I, then
this I needs to be "outside" / "transcendental" of time.


God as a creator left a trademark in my mind, through which I am capable of
conceiving god.  That is, there are innate ideas in my head. Since I am
imperfect, I could not have gotten the idea of perfection from myself.


Why should the pythagorean theorem hold? There is some objectivity to the truth
that is outside of my mental performance. However, I do need my mental
performance to make it explicit to me. Similarly, there is an innate notion of
God. But it is only when I exercise this capacity that I can fully form this
idea of God through the clear and distinct perception.

So, God has given us the clear and distinct perception. He then uses
the \emph{clear and distinct perception} to access the notion of God.


If we had separated clear and distinct perception as something that \emph{humans}
posess, then we get a vicious cycle. However, if we say that clear and
distinct perception was \emph{given my God}, but we need to \emph{exercise} this
clear and distinct perception.


For example, all of math is tautological. So in some sense, the entire
information content is zero. However, every time we prove a theorem, we have
gained some knowledge. So, Descartes' notion of knowledge is about making
these concepts explicit, like that of mathematics, or clear and distinct
perception. ${a = a}$ is a trivial truth. However, a derivation 
${a \rightarrow b \rightarrow \dots z}$
is still tautological, but is more useful, since it makes explicit that $a$
implies $z$.

\section{Kant's theory of morality}

God has the power to decieve us, but the desire to decieve us is a lack. 
Hence, God will choose not to decieve us.

If we are infinitely rational, we cannot but be moral, according to Kant.
Kant is a perfectly secular rationalist, and therefore provides more
accessible arguments to many of descartes' claims. 

If god cannot deceive us, how are we deceived? The example of 
temperature as felt by the hand, for example. Why did God not
make me perfect? Why is there evil in the world --- why did God
allow me to be evil? If I am capable of rational thought, how is it
that I can perform evil, which is irrational (Kant)?

Kant says that there are no actual moral values with content.
There is no God to guarantee morality.
Why should we be moral?

Take any subjective maxim. The test for whether it is moral is whether
it is universalizable. What would happen to such a society? If society
breaks down, then this is immoral.

For example, consider the notion of a promise. If I believe "I ought
to break a promise whenever it is convenient to me ". However, if I
universalize this, I will get "Everyone ought to break a promise
whenever it is convenient for anyone" --- this leads to a break down
of society.

In general, for us to break a social norm, we both need the social norm
or law to exist, so that we are able to break the law. So, in some sense,
we want both $A$ (for the law to exist) and ${\lnot A}$
(for us to be able to break the law).

Going back to the promise example, we would require the institution of promises
to exist for us to gain something by breaking it, so we have $Promise$. We also need
to be able to break a promise, so we are arguing for $\lnot Promise$.

For Kant, there is no cultural relativism. So, there is a best morality that is
accessible through rationality.

\section{Pulling back Kant to Descartes}

Descartes says that as long as we seen something clearly and distinctly and we
are absolutely sure of its truth, we cannot be mistaken. 

He creates a distinction between \emph{understanding} and \emph{judging}
something. In logic, this is the difference as introduced by Frege between the
content stroke $[- A]$(assume $A$ is true) and  the judgment stroke \texttt{[turnstile A]}
($A$ is the case).

We are temporal beings, so we can lose our clear and distinct perception --- for example,
we can forget the proof of a theorem which we had studied.

Our conception of the will: The will is infinite, and is what allows us to conceive of God.
This goes back to the negativity of thought: we are able to suspend a thought,
doubt it, and take a step back. The freedom of the will is the only mark of infinity
that we have. Therefore, we are able to conceive of notions of infinity, which we cannot
see clearly and distinctly, but we can still conceive it. Therefore, free will is
the mark of perfection that we possess. 

However, since this is freedom, we can use this freedom to make mistakes.
He makes an interesting observation that this kind of freedom is not merely indifference
between two equal possibilities (freedom is just another word for nothing left to lose).
However, for Descartes, freedom is the ability to gravitate towards truths that
are self-evident is a rationally determined will. Contrast this to the rebellious
will of a teenager.

\section{A review of Descartes}

Logical truth does not provide us with proof of existence. For example, the
nature of the triangle is such that even if there are no physical triangles.
However, Descartes is trying to derive existence from the conceivability of
something.

After Descartes, rationalism takes a different turn --- it grants that we
cannot actually say anything factual about the world. So, "phenomenology"
decides to talk about phenomena, \textit{as phenomena} --- we don't care
if phenomena refer to something beyond themselves.

Whatever I see clearly and distinctly, I cannot but help agreeing to it. For
example, the truths of mathematics. The reason for error is that my freedom of
will is much greater than my comprehension. If I actually see things cleary and
distinctly, I cannot be open to error anymore, because of the existence of a
non-deceiving god.

\chapter{Defending Hume}

We have pieces of music which are temporal, and do not have an identity
at an instant of time.

Indeed, it is just because our idea of a person allows for the momentary
existence of a person that we are able to ask questions about personal identity
of the first place. Only when we presuppose the idea that the self is not temporal
can we ask the question about the character of a person at a moment in time.
Hence, a person is \emph{not} like a piece of music, and therefore, the
person must be an object, such that the entirety of the object is present
at that piece of time. 

Numerical identity: identity of "one-ness". Specific identity: identity
of "exactly equal / specifically equal".


He wishes to show that numerical identity does not presuppose continuity.
We compare two types of identity: specific and numerical identity.
If we imagine a ship which has changed over time in the sense that
every piece of its plank, beam, etc. has been replaced. So we would say
They are no longer numerically identical, but they have the same specific
identity / resemblance, since the new ship is an exact resemblance of the
old ship. On the other hand, if the same ship continues over time, then
they are numerically identical.


On the other hand, if we imagine a human being growing up, I have a different
specific identity, since I am clearly different, but I intuitively
have the same numerical identity, since I feel like the same person.


We can have something have the same numerical identity but different
specific identity for us to be able to talk about change. Otherwise,we 
do not know that they belong to the same object.

Hume has said that we have no access to mind-independent reality. We only
have our ideas. There is no difference between ascribing identity to a thing
v/s ascribing an identity to oneself, since both are bare ideas.

This defense of Hume requires us to differentiate between a spatial and a temporal
object. So, the self is not a temporal object (since it is not like music), since
we can talk about it instantaneously, and therefore the notion of the self
is fiction.

If everything is an idea, space is also temporally experienced. We are mentally
synthesizing space as a phenomena. So, for example, when we see a chair,
we see some perspective projection of the chair. So, we are not given
all of the chair at once. However, we imagine the entire chair as a spatial
object, in a way that I do not with music.

John Berger in "ways of seeing", says that the artist has to unlearn what
he knows and paint what he sees.

\chapter{Kant (1760s)}
Hume was of the opinion that there is no causality, only association. However,
our associations are usually non-arbitrary. We presuppose constancy of association,
to be able to explain experience.

Kant is going to show how regular association must exist, and it therefore
cannot be at the empirical level. So, the unity of self transcends experience, 
but it's a logical posit, which we require to associate ideas, so we can form
the fiction of causality, the fiction of ideas, etc.

Kant has two deductions. He's trying to deduce the necessary conditions for
the possibility of experience. He formulates a deduction to deduce these categories.
There are two editions: The A deduction and the B deduction.

We no longer have access to the thing in itself, all we have is
experience (phenomenology). Both space and time are synthesized by consciousness.
In the case of a temporal object, is it easy to see why 
we need synthesis: for example, a melody needs a synthesis for it to be a melody.
For a spatial object, we still need synthesis to integrate the different
perspectives that we experience of the spatial object. 

From within experience, we can differentiate between space and time, since we can
reverse spatial synthesis, but not temporal synthesis. The arrow of time, in a
sense (or the hyperbolic metric of spacetime, if that's the cut of your jib).

Kant believes that time has a certain privilege over space: Everything is temporal.
Even though our sense of space is not reducible to time, the reversibility of space
still happens within time. To quote Kant:

\begin{quote}
	Every intuition contains in itself a manifold which can be represented
	as a manifold, only insofar as the mind distinguishes the time in the
	sequence of one impression upon another, for each representation, insofar
	as it is contained in a single moment, can never be anything other than
	absolute unity. in order that a unity may arise from the manifold, as is required
	for the representation of space, it must first be run through and
	held together. This act I name the \emph{synthesis of apprehension}.
\end{quote}

For Kant, "intuition" is some sense of "impression ". "representation "
is whatever it is that 
allows us to experience space (in its pure form), instead of things in space. This
is in the sense of Euclid: He is trying to understand how Euclidian Geometry
is the "science of space". It is concerned with the very \emph{form of space}.

Kant recognizes pure intuition, which is non-sensational,but is almost like 
Descarte's "clear and distinct perception ". Kant has a phrase --- 
"necessity is existence given through possibility itself".

For example, consider the universe of (mathematical) possibilities. What
we are seeing as a mathematician is possibility qua possibility. And pure 
intuition goes into being able to see mathematical necessity.

We have a pure intuition of space, which is
then synthesized in various ways to get Euclidian geometry.
The idea is a higher order constitution through synthesis. 

\chapter{Strassen - Freedom and Resentment}

The goal of philosophy, strictly speaking is to do meta-ethics. The idea is 
that we are trying to consider the source of ethics. What constitutes value,
for example. 

Strassen wishes to analyze our ordinary sensible understanding of ethics,
and how we actually behave. And then, we can see how this reflects these
deeper kind of convictions that we hold. These convictions might be implicit.

We can take natural science as the starting point, and claim that no one
deserves blame or praise, since all is deterministic.
Alternatively, we can start from the abstract notion of freedom, and argue that
people do deserve blame and praise.

In the deterministic model (there is no free will), can we have moral praise
or blame?

We do not have the same reactive attitudes between people who are, say, crazy.
We cannot be resentful of people like this. We treat them like an object to be
contained. This is in stark contrast to how we as a society treat violent
offenders.

There exists a "just deserve" theory --- if you kill someone, you open yourself up
for being killed. You deserve it because you are fully chose. Of course,
the "chose" is the real question here.

Strassen builds up an argumnt for showing how this entire exercise that
if determinism is true, there is no moral responsibility, and therefore
there is no guilt, etc. All of this stays at the intellectual level, and
is unable to account for the kinds of nuances that constitute the experience
of our humanity. In virtue of being humans, we understand implicitly this
very intricate structure.

What he does is to break the deadlock between the incompatibilist position 
(determinism and free will is not compatible), and the compatibilist position
(even if determinism is true, we can still have moral responsibility and punishment).
Both of these are carried out in the intellectual realm. He opens
by talking about two kinds of people. The moral optimist says the truth of
determinism does not prevent us from having reactive attitudes. He is
a compatiblist. The moral pessimist will say if determinism is true, then there
is no responsibility, and hence there can be no attribution of judgement. The
moral pessimist is the incompatibilist. The problem is that the way in which
the moral optimist reconciles these two postions is wrong. This is because the
moral optimist admits the possiblity of determinism. He goes on to say
"even \emph{if} determinism is true, \dots". But they do not appeal to the utility
of punishment.  Strassen goes on to prove that there is a social utility in
punishment even under determinism. This way, we do not have to deal with
intensionality of the agent. 

No matter how benevolent an action looks, people are doing it for their
own ends. This is unfalsifiable.  The position that all we are doing when we
are acting is out of self-intrest is called egoism. We can only show a
lack of self-interest in cases of conflict.

The deontology versus consequentialism is an offshoot of the rationalist versus
empericist divide. Since consequences can be measured, it leads itself to
empirical evidence. While deontological reasoning cannot be measured, since
motives and intentions are not tangible. Hence, they
require rationalist style argumentation.

We have so far not discussed Kant's view on morality, so we cannot be precise
when we discuss morality here, with respect to Kantian ethics.

\section{Strassen's argumentation}

compatibilist (Those who believe that morality exists even if free will does not)
is called as the moral optimist by Strassen. The incompatibilist is termed by Strassen as
a pessimist (That is, those who believe that moral attribution of any kind 
is not possible without free will).

The moral optimist could try and reconcile with the moral pessimist. Even
if determinism is true, holding someone responsible / morally judging someone
is possible due to its social efficacy. That is, these practices regulate
social behaviour in desirable ways. We are defending the compatiblist position
with a consequentialist argument. 

The moral pessimist rejects these as incorrect criteria. Morality is not
based on consequences, it's based on intentions.

The optimist might continue to pursue their line of argumentation, and
argue that there are many cases where freedom of will simply means that we 
are not coerced in any way. The optimist gives a \emph{negative} conception
of freedom. Under these circumstances, moral attribution is right / meaningful.
It is a minimalist notion of freedom. We don't know whether we are free in the
full fledged sense, of being pure agents with free will. But we can, in a 
concrete situation, figure out whether we were compelled or not. Negative
freedom is visible. Positive freedom is a metaphysical problem.

The pessimist is of the opinion that the full sense of morality depends on
\emph{positive freedom}, and this is what is lacking in the optimist's 
position.  The pessimist is of the opinion that
consequentialism is not the right criteria / notion of morality. Our
morality is based on a certain sense of freedom, which is not reducible to
consequentialism.

Now we are at an impasse. Straseen  will attempt to intervene here, and will
phrase the question in such a way that the optimist position can be saved,
though consequentialism drops out, in return for a concession by the
pessimist. What Strassen introduces is going to be a non metaphysical way
of argument. In the history of philosophy, this is a recognizable move. The only
way to undercut an impasse is to change the assumptions of the debate. What
Strasesen does is not by starting from the metaphysics, but by starting
from our ordinary everyday experience of morality. 

What is the sense of the moral that we have in our everyday, inter-personal
interactions? This is quite similar to the phenomenological move. When
seen from the history of moral philosophy, this is quite significant.


Let's take the non detached feeling we have when we have an interpersonal
relationship.  It's only here that I can feel love, hate. We normally make
moral judgements form a third person perspective. A judge is not personally
affected by the crime, but a judge sits in judgement.

Sartre has a great example of this. He gives an example of the peeping tom.
They are conscious of the people who they are seeing, since they 
are totally immernsed in the peeping.Hence, they are not self conscious.
Sartre says that self conscious is social. We have this only in the
presence of the other. It is the other's imagined gaze that leads us to
have a self conciousness. This is very different from the view point of
the meditator, who has a notion of self-reflection and self conciousness.
We also feel awkward at a party, ie, self-concious. So, self conciousness is
indeed social.

Revenge is first-order, Punishment is second-order. Punishment is mediated
by existing moral law.

Our reactive attitudes are mediated.

\end{document}
