\documentclass[11pt]{book}
%\documentclass[10pt]{llncs}
%\usepackage{llncsdoc}
\usepackage{amsmath,amssymb}
\usepackage{graphicx}
\usepackage{makeidx}
\usepackage{algpseudocode}
\usepackage{algorithm}
\usepackage{listing}
\evensidemargin=0.20in
\oddsidemargin=0.20in
\topmargin=0.2in
%\headheight=0.0in
%\headsep=0.0in
%\setlength{\parskip}{0mm}
%\setlength{\parindent}{4mm}
\setlength{\textwidth}{6.4in}
\setlength{\textheight}{8.5in}
%\leftmargin -2in
%\setlength{\rightmargin}{-2in}
%\usepackage{epsf}
%\usepackage{url}

\usepackage{booktabs}   %% For formal tables:
                        %% http://ctan.org/pkg/booktabs
\usepackage{subcaption} %% For complex figures with subfigures/subcaptions
                        %% http://ctan.org/pkg/subcaption
\usepackage{enumitem}
%\usepackage{minted}
%\newminted{fortran}{fontsize=\footnotesize}

\usepackage{xargs}
\usepackage[colorinlistoftodos,prependcaption,textsize=tiny]{todonotes}

\usepackage{hyperref}
\hypersetup{
    colorlinks,
    citecolor=black,
    filecolor=black,
    linkcolor=black,
    urlcolor=black
}

\usepackage{epsfig}
\usepackage{tabularx}
\usepackage{latexsym}
\newcommand\ddfrac[2]{\frac{\displaystyle #1}{\displaystyle #2}}

\def\qed{$\Box$}
\newtheorem{corollary}{Corollary}
\newtheorem{theorem}{Theorem}
\newtheorem{definition}{Definition}
\newtheorem{lemma}{Lemma}
\newtheorem{observation}{Observation}
\newtheorem{proof}{Proof}

\title{Philosophy: God and knowledge}
\author{Siddharth Bhat}
\date{Monsoon 2019}

\begin{document}

\maketitle
\tableofcontents

\chapter{Introductory class}

The difference between mathematical truths and factual truths.

\section{The problem of psychologism}

Even though mathematics seems objective, we still filter
mathematics through our psychology, so it might not be
as objective as we want it to be.

\chapter{Descartes: Meditations, Doubt and Belief}

He begins with \textit{methodic doubt}, where we do not question each and
every particular belief. Rather, we question the source of knowledge of our
beliefs. If this source of knowledge is open to even the least doubt, then 
we reject all beliefs which arise from this knowledge.

We start with perception, the most obvious source of knowledge. We know that
our perception can easily deceive us --- for example, madmen are clearly
deceived by their perceptions. We do not need to presuppose such an
extreme position, however. Dreams are another example where our perceptions
deceive us. Yet another example is that of temperature: We can
change our perception of hot or cold, by dipping our hands into a
jug of hot or cold water.

As an aside, ``neurotics'' have some connection with reality, ``psychotics''
have a clean break with reality. So ``madmen'' can be subdivided into
these two classes, by Freud's classification. Freud says that normal
dreams as wish fulfilment. That is, you are able to fulfil your wishes
that have been repressed. Also, nightmares are your subconcious
trying to work through some form of trauma. It is the subconcious
trying to take trauma and rework it into a managable form. They
had a patient who had a recurrent nightmare of being shot
in the head. Clearly, this is not ``wish fulfilment''.
He creates the explanation that we constantly relive a bad
experience to learn how to manage it.

Now, Descartes makes a decisive move --- Even if we are dreaming,
the \textit{content} of our dreams as mental images are
\textit{embedded in reality}.
For example, think of an imaginary/ficticious object, such
as a unicorn or a fantastic alien, or an orientable mobius strip.
Even so, there would be elements which are recognizable.

At the level of mental images, the images cannot be doubted, though
what they are images \textit{of} may not exist. The content
of the images in terms of shape, figure, and color must exist in our head,
for them to \textit{be images}. Hence, we cannot reject the existence
of figure and color.

We are also unable to reject time. Dreams can \textit{distort} time, but
we can still feel the passage of time in a dream. We are trying to move
from something that is mental, to something that is more than our thought
process.

\textbf{Subjective idealism} says that it is just my flow of conciousness,
and everything is confined to \textit{my} conciousness.

\textbf{Objective idealism} says that to even posess conciousness, we need
abstract entities that transcend just our consciousness --- for example,
time.

Since time is required to be able to dream, time cannot be reduced
to dreaming. This is called as \textbf{Transcendental philosophy}.
``Transcendental'' since we find that at the very moment we
turn inwards, a different reality. We find a transcendence in
imminence (imminence = the inner). That is, me observing
myself internally causes me to come across certain structures that
I cannot reduce to myself. For example, that of temporality.

Hence, he believes that mathematics contains some measure of
certainty. The natural sciences have an element of empericism
mixed into them. They presuppose things existing in the external
world. This presupposition is open to doubt. On the other hand,
arithmetic does not need to hypothesize the existence of
physical numbers. The ideal entity of numbers (in the sense of Plato's forms)
is something which is independent of our subjective experience.

TODO: I left this class to go back home. finish this

\chapter{Lecture 4}
Descartes is not happy with saying that this is how we are physically
constituted. So he's looking for this absolute, induibitable ground. Therefore,
he has to actually prove that there is some transcendent source which
guarantees your perception. "clear and distinct perception" is what we require,
but this is hard to get to.

"Doubt has taken the ground beneath my feet" --- At the same time, I can no
longer remain on the surface. So far, I had been on the surface and it was not
a problem for me since I had never doubted. Once I began this quest, I do not
find any absolutely certain ground. I cannot forgot about this shaky ground
and continue to live there.

He asks what he can be certain of, given hyperbolic doubt --- perception,
memory, body, figure, extension and movement are all fiction. He asks whether
his \textit{very existence}, his notion of I is in itself something real. The
act of thinking is the conformation of my existence as a thinking being. As long
as I am doubting, I exist as a doubting being. As long as I am being decieved,
I exist as a being who is decieved, and so on.

I think \textit{therefore} I am --- the \textit{therefore} is one of assinging
thinking and being equal / identity. It is \textbf{not} inference.

% need to have a \aside{...} to log interesting tangents.
% Maybe I should call it \kannan{...}
\subsection{The problem of personal identity}
Descartes walks into a bar. The bartender asks, "what will it be?" Descartes
responds, "I think not", and he promptly disappears.

Memory constitutes a large fragment of our personal identity. On Descartes'
theory, there is no reference to the past. We can only affirm out existence
in the now.

\subsection{Solipsilm: The sole I}

How does one get out of ones mind, to access the rest of the physical realm?
Currently, I only know my own mental appearances. Everything else is an
illusion.

Descartes wishes to reject this branch of thought. For example, he questions
whether he stops existing when we sleep. He wishes to come up with an 
explanation for our sense of continuity.


After introducting "I think therefore I am", he continues saying "I do not know
very clearly what I am, except for the fact that I am".

Imagination is imagistic, and images have already been rejected, insofar
as they refer to the external world. However, imagination which refers to the
internal world is indubitable.

The only proof certificate of existence we have found so far is
that of thinking.h

At some point, Descartes' defines the body as that which cannot be self-moved.
This is different from Aristotelian physics, where everything is self-moved
according to its telos. Descartes has broken tradition from the Aristotelian
model, to somewhat arrive at the notion of inertia. (Netwon, a century later
formalizes this).

Descartes rejects the reality of his perception, but marvels at the richness of
his perception. A mirage does not truly exist, but it exists \textit{for me}.
This brings out a gap between the external world and the internal world.
Yet, we believe that we know the external world better than the internal world.

\subsubsection{The wax example}
Descartes takes something we are absolutely sure of: For example, a piece of
wax.  He waxes poetic about this piece of wax, its physical characteristics
(sound, color, smell). When we begin to melt the wax, it heats up, loses color
and smell, becomes liquid. He then asks: "does the same wax remain after this
change?" We must confess that it does remain, no one would say otherwise.
However, by what criterion are we judging that it is the same? All of its
pyhysical characteristics have been altered!

He wishes to retain physical objects. But by his standards for indubitable
knowledge, he is forced to recourse to God. 

\subsection{Why does modern science reject teleology?}
Teleology presupposes what is to be explained. If we say that something moves
because it is supposed to move, this is not an explanation.

\section{Lecture 5}
Empericism is not the annhilation of reason. Rather, it is a viewpoint on where
reason lies: Does it lie in the mind, or does it lie within nature? That is,
are all our ideas and connections between ideas are "out there", and we are
passively listening to these. We are discovering the laws of nature, not
inventing them.

Descartes with "cogito ergo sum" is closing the gap between subject and object.
He is saying: "only that really exists for sure, which I can subjectively
grasp". The usual way of looking at things would be to say that a table exists,
and I am \textit{able to} perceive it. Descartes' view of this is that I can be
aware of the existence of the table \textit{since} I am perceiving it.

Principle of coherence: My beliefs are true if all my beliefs are coherent.

He decides to abstract from everything that is unnecessary from the piece of
wax to define the piece of wax. It's not the form, since the form changes. The
extension changes.


So, we need the notion of a god to protect us from a demon who might hinder our
clear and distinct perception. There is a circularity in his logic, where he
starts out by assuming that we need a clear and distinct perception, and
therefore we need a God. He then continues to use God to make whatever we see
through clear and distinct perception to be true.

\end{document}
