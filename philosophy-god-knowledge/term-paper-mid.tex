\documentclass{article} 
\usepackage[backend=biber,authordate]{biblatex-chicago}
\usepackage{amsmath}
\usepackage{amssymb}
\usepackage[version=0.96]{pgf}
\usepackage{tikz}

\addbibresource{term-paper-mid.bib} 
\title{Philosophy term paper: Critically evaluating Hume's skepticism of causality} 
\author{Siddharth Bhat}
\date{Monsoon 2019} 


\begin{document} 
\maketitle

\section{Introduction}
This short survey on  the skepticism of causality is divided into roughly three
parts: We begin by srveying Hume and his arguments on why we must be skeptical
of causality. We then move on to see certain arguments against this thesis.

First, we consider Karl Popper's notion of falsification, and how the
scientific method (as is broadly understood), truly allows us to gain access to
scientific truths.

Secondly, we consider the Bayesians and the statistical method, who take Hume
to heart and encode the notion of "habit" in terms of "probability". We examine
the ramifications of taking this perspective to the extreme, and how we lose
sight of the distinction between correlation and causation --- precisely as
anticipated by Hume. 

Lastly, we rebutt the Bayesians by considering the bleeding edge science
of causality, initiated by Judea Pearl, starting from the 1980's, which provides
an alternative definition of causality in terms of \emph{counerfactuals}. While
this definition appears useless, we provide a short introduction for how this
allows us to regain a reasonable notion of causality which can be mathematially
dissected, thereby triumphing over the anarchy of Hume's definition.

\section{Hume and Causal skepticism}

David Hume (1711-1776) was a contemporary of John Locke and George Berkely.
While all of them advocated using empiricism to arrive at knowledge, Hume was
the only consistent empiricist among the three, who applied his empirical
standards rigorously and unflichingly. This allowed him to arrive at oftentimes
counter-intuitive results, such as challenging the very existence of causation.

Since Aristotle, there was thought to be a difference between \emph{belief}
and \emph{scientific opinion}. Scientific knowledge consisted of providing
explanations in terms of necessary causes for an effect to manifest. As an
example, smoke is \emph{caused} by fire, since for smoke to exist, there must
\emph{necessarily} exist fire that causes this smoke. 


\subsubsection{Convention and Causality}
The crux of Hume's argument is that of convention. All that we suppose to be
"causal", is in fact simply convention, that we as humans have gotten used to
due our experience. He provides a famous example of billiard balls: When we see
that the motion of one billiard ball follows another, we are only observing
conjunction, not their connection. Indeed, his definition of "cause" is this:
\begin{quote} A \textbf{cause} is an object, followed by another, where all the
objects similar to the first are followed by objects similar to the second.
\end{quote}




\section{Karl Popper and falsification} 

These problems of causation and their
relation to science were eventually resolved (in a sense) by Karl
Popper(1902-1984), who gave an account of scientific knowledge in terms of a
strictly empirical account driven by falsification, instead of one driven by
induction. Roughly, his solution relates to the adage: \begin{quotation} All
models are wrong, some models are useful.  \end{quotation}

Where he views the entirety of science, and as an extension, our model of
causality, as precisely that: a model. No knowledge that we possess, not even
that of obvious causality, is certain. Knowledge is correct, insofar as it
provides a useful model of the world. A model, or a notion of causation can be
discarded as soon as it does not align with the reality. In this way, the
notion of \emph{falsifiability} takes center stage.

In short, Popper's theory can be summed up as:
\begin{itemize}
    \item There is no correct way to perform induction, given access to any amount
        of experimental data. This agrees with Hume's skepticism.
    \item Our theories (including our notion of causality, which is governed by our theory), is
        just a conjecture. We must attempt to falsily these conjectures. If we
        are unable to falsify a conjecture, then we have greater faith in the conjecture. However,
        we can never truly establish a fact (such as causality). All we can do
        is to attach a \emph{probability} to causality.
\end{itemize}


\subsection{Bayesians and Hume}
The bayesians were a school who took as fundamental Bayes' theorem, a general
theorem about the structure of probabilities which describes how to \emph{incorporate}
experimental knowledge (obervations) into our current knowledge of the world (the prior)
to arrive at an updated knowledge of the world (the posterior). The famous formula is:

\begin{align*}
    P(B) = \frac{P(B|A) \times P(A) }{P(A|B)}
\end{align*}

Far too much ink has been spilled analyzing the above formula, so I wil not
repeat that here. The basic point is that this theory provides us with the
\textit{best possible way to update our beliefs}, if our beliefs about the
nature of the universe are purely probabilistic.

Thus, bringing this back to Humian terms, bayesian theory allows us to decide
\textit{which conventions we should choose}, given raw experimental data.

However, this again loses a notion of causation, since we are again making
probabilistic statements about the world. In theory, a causation would be
repressented as a \textit{perfect correlation}. However, this has its own can
of worms. A simple problem with this model is that we lose the \textbf{directionality}
of cause and effect.

For example, we might have the readings on a thermometer, and the real temperature
of that day. Using this data, we would create a formula $r = \alpha t + \beta$,
where $r$ is the reading of the thermometer, $t$ is the real temperature, and
$\alpha, \beta$ are constants. This gives us a model for the relationship between
the thermometer reading and that of temperature, and indeed these two variables
are perfectly correlated. However, crucially, there are \textit{two} possible
interpretations of the formula: one saying that the temperature governs the
reading of the thermometer ($r = \alpha t + \beta$), and the other saying that
the reading of the thermometer governs the teperature ($t = (r - \beta) /
\alpha$). The structure of mathematical equations is such that both of these
are entirely valid views.

What we really need is a new form of mathematics, that allows us to depict
$(r \leftarrow \alpha t + \beta )$ where the $\leftarrow$ indicates that
the $r$ is determined by $t$ and not the other way round.


This was believed for roughly a century since the birth of statistics to be
impossible, by (fundamentally), an appeal to Popper and Hume: how, indeed,
can we establish such a directionality, if all we have is the data at hand?

\section{Judea Pearl, Causal models, and the do-calculus}

Here enters the fascinating study intiated by Judea Pearl into what are now
called \emph{Causal Models}. 

The insight is that one can provide a definition of causality, provided
\emph{counterfactuals} are available. One can \textit{define} causality 
as follows. For the sentence:
\begin{quote}
    The billiard ball A colliding with billiard ball B \textbf{caused} billiard ball B to move.
\end{quote}
We assign the meaning of the word \textbf{caused} as the meaning of this
equivalent sentence:
\begin{quote}
    \textbf{In all universes} where the billiard ball A \textbf{not} collided with
    billiard ball B, the billiard ball B \textbf{would not move}.
\end{quote}

This sentence does not seem to improve our state of affairs by much. It has managed
to provide a definition of causality. However, this definition relies on universal
quantification of an all universes, which is a tad unrealistic to test.

Note that this is a different notion of causality than that of hume, in that
it is verifiable, provided access to an infinite number of universes, and the
ability to perform experiments.

Luckily for us, due to the structure of mathematics, \emph{once we fix some causal model}, 
for example, (A collided with B~$\rightarrow$~B moved), we
can test how well the \emph{data} agrees with the model.

While as bayesians, we had:
(\begin{tikzpicture}
    \node (data) at (0, 0) {data};
    \node (correlations) at (2, 0) {correlations};
    \draw (data) edge[->] (correlations);
\end{tikzpicture})

Note that there is no human-in-the-loop. From the data, we compute, plug and chug
to arrive at correlations. To a statistician, this is all that we can say, and 
indeed, all that we need to say. Unfortunately, this is a disastrous position to
hold. After all, as we all know, "correlation does not imply causation". We have
yet been unable to escape the long shadow cast by Hume. "correlation" is simply
a fancy mathematical encoding of "convention", is it not?

Now as users of do-calculus, we have:
\begin{align*}
\begin{tikzpicture}
    \node (data) at (1, 0) {data};
    \node (model) at (5, 1) {causal model};
    \node (agreement) at (1, 1) {agreement (do-calculus)};
    \draw (data) edge[->] (agreement);
    \draw (model) edge[->] (agreement);
\end{tikzpicture}
\end{align*}

Notice that we \emph{explicitly} require a model to even begin applying do-calculus.
However, the upshot is that we can now discuss how well our proposed model of
reality agrees with the data that we have. Indeed, our process of "doing science"
can be now viewed as:


\begin{tikzpicture}
    \node (data) at (1, 0) {data};
    \node (agreement) at (1, 1) {agreement};
    \node (model1) at (5, 0) {causal model 1 $(P = 0.2)$};
    \node (model2) at (5, 1) {causal model 2 $(P = 0.5)$};
    \node (model3) at (5, 2) {causal model $n$ $(P = 0.1)$};
    \draw (data) edge[->] (agreement);
    \draw (model1) edge[->] (agreement);
    \draw (model2) edge[->] (agreement);
    \draw (model3) edge[->] (agreement);
\end{tikzpicture}

The takeaway is that \textit{relative to a model $i$}, we can judge how well
this model captures the causality of the data!

This is \textit{both subjective}, or bound by convention (in terms of the choice
of model), but also objective \textit{relative to a model}. Once we choose a
model, we can check its "causal agreement" with the data.

\section{Conclusion}
I argue that do-calculus and our 21st century understanding of 
causality has settled Hume's arguments to a far greater degree than Hume
believed was ever possible: under an intuitive definition of causality (counter-factuals),
we can build an edifice of mathematical machinery which allows us to test models
of reality against data. While this choice of model is non-canonical, and this
is where the human enters the loop, beyond this choice, we have objectivity.

\nocite{*} \printbibliography

\end{document}
