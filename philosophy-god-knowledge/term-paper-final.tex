\documentclass{article} 
\usepackage[backend=biber,authordate]{biblatex-chicago}
\usepackage{amsmath}
\usepackage{amssymb}
\usepackage[version=0.96]{pgf}
\usepackage{tikz}

% What further consequences follow from the ‘phenomenological’ or ‘bottom up’
% approach that Strawson adopts? Can it form the basis of overcoming the divide
% between the feminist ‘care’ perspective and the universal ‘justice’ perspective
% of traditional moral theory?          

% Does Strawson succeed in overcoming the impasse between the ‘moral optimist’
% and ‘pessimist’? Discuss

% 4. Critically evaluate Strawson’s conception of reactive attitudes with respect
% to Kantian moral theory.


\addbibresource{term-paper-final.bib} 
\title{Does Strawson's ‘phenomenological’ approach}
\author{Siddharth Bhat}
\date{Monsoon 2019} 


\begin{document} 
\maketitle

\section{Introduction}
Liberterian free will: humans are fully free to choose actions.
Hard determinism: all current events all caused by past events. Nothing other
that what would occur, could occur.
Principle of alternate possibilities: An action is free if the person doing
the action could have done otherwise.
We feel free: Liberterians argue that we should not discount our personal
experiences. 
Compatibilism: past determines the future, but there is something different
about human actions. We can still call an action free when the determinism
comes from within. The action couldn't not happened. But when an action
is self-determined, the action should be considered free.
Frankfurt cases: Have a brain implant that will force you to make a particular
choice if you try to take another choice. If you take the "right choice", were
you free? Principle of alternate possibilities says you are not free. 
Frankfurt argues that you chose what to do, even if you could not have done
otherwise.

How can we separate internal from external factors? Isn't personality shaped
by nurture? 

We can choose to talk about a gradation of free will, and some actions are
more or less free (Patricia Churchland). We can't help but hold people accountable,
and assign blame and praise.

Care ethics: We would be concerned with issues of needs rather than rights.
Restorative justice movement. Ethics of care promotes empathy. Meeting
needs is more important than seeking rights. 

Care ethics, premises: women and men are equal.


Freedom and resentment: we don't really know whether determinism is true,
or that god exists. We can't be sure that we are morally responsible agents.
Strawson allows us to sidestep this debate. It's a mistake to think that
moral responsibility requires a condition to be satisfied - we are presupposing
that we need to be free to do otherwise. Moral responsibility does not need
any condition of freedom at all. This presupposition is problematic. Us holding
people responsible and using reactive attitudes (punishment and reward) is
upholding moral responsibility. We should not seek to escape our reactive 
attitudes and take some "external", "objective" viewpoint. This is a general
mistake of thinking we can step out of a conceptual framework and justify it.
We can call into question some parts of the framework by using other parts
of the framework. So, justification is some kind of "internal" object.

Punishment and reward are ways of displaying our reactive attitudes. Strawson
is giving an account of moral responsibility that does not require a freedom
condition. Being morally responsible is being a fitting target of the
reactive attitude. According to Strawson, to be  a fitting target, one needs
to display a quality of will: eg, if you display me hostitility or indifference,
then I can punish you. If you show me graditude or happiness, I might reward
you. 



\nocite{*} \printbibliography

\end{document}
