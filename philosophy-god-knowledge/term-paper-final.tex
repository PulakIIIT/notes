\documentclass{article} 
\usepackage[backend=biber,authordate]{biblatex-chicago}
\usepackage{amsmath}
\usepackage{amssymb}
\usepackage[version=0.96]{pgf}
\usepackage{tikz}

% What further consequences follow from the ‘phenomenological’ or ‘bottom up’
% approach that Strawson adopts? Can it form the basis of overcoming the divide
% between the feminist ‘care’ perspective and the universal ‘justice’ perspective
% of traditional moral theory?          

% Does Strawson succeed in overcoming the impasse between the ‘moral optimist’
% and ‘pessimist’? Discuss

% 4. Critically evaluate Strawson’s conception of reactive attitudes with respect
% to Kantian moral theory.


\addbibresource{term-paper-final.bib} 
\title{Kant and Strawson}
\author{Siddharth Bhat}
\date{Monsoon 2019} 


\begin{document} 
\maketitle

\section{Introduction}
We first begin by providing an account of both Kant's utilitarian ethics and
how this related to moral culpability, as well as P F Strawson's account of
moral responsibility. Next, we examine both of these positions, and conclude
that these are not compatible. We discuss the points of incompatibility, and
possible modifications that can be made to each theory to agree with the other.
I lean towards rejecting P F Strawson's account of moral responsibility, and
therefore choose to sacrifice more of Strawson's account to reach a possible
middle ground.

\section{Free will and responsibility, the big picture}
We might begin with our naive intuition that humans are fully free to choose
their own actions, at all points in time. This position is known as
\textbf{Libertarian free will}. This is an intuitively appealing position, since
we all intuitively believe that we possess agency and free will.

There is another intuitively appealing proposition, that all current events all
caused by past events. This is known as \textbf{determinism}. 
If one believes in the laws of physics, then it appears
that (outside of freaky quantum phenomena), it is possible that the world
is entirely deterministic, and that having full knowledge of the state of
the world at any point in time will allow us to predict with
perfect accuracy at all future times. Indeed, one might argue this is the fundamental
goal of the basic sciences. A somewhat more unsettling rephrasing of the triumph of the scientific method is:
\begin{quote}
"Nothing other that what would occur, could occur."
\end{quote}

The problem is striking --- This view of the world implies that the current
state of the world determines the next state of the world, which includes the
actions we take and the decisions we make. This disagrees
disagree with our intuition of choosing our own actions.

We can sharpen our position on what choosing our own action means, by
considering the \textbf{Principle of alternate possibilities}. This states that
an action is truly free if the person doing the action could have done
otherwise.  The principle of alternate possibilities, which is a sharpening of
the idea of Libertarian free will, \emph{does not jive} with determinism. So,
we need to give up on one or the other. Libertarians argue that we should not
discount our personal experiences, which very clearly make us feel that we do
have free will. However, personal intuitions does not a theory make, so we move
forward, to investigate how to modify our notion of free will to line with
determinism. We could, of course, have tried to modify determinism to suit
libertarian free will, but that seems harder, since the author feels that there
is more scientific evidence for determinism than there is for Libertarian free
will.

The position of \textbf{compatibilism} argues that while determinism is true,
and the past does determine the future, there is something \emph{different}
about human actions. We can still call an action free when the determinism of
the action comes from \emph{within}. We know from determinism that action
couldn't not happened. But when an action is \emph{self-determined}, the action
should be considered free.


This can come under attack from multiple fronts.We first consider a famous thought
experiment known as a \textbf{Frankfurt case}. Let us assume a woman has a brain
implant that will force her to make a particular
choice if she tries to take any other choice. For example, let's say that Alice's
brain has been secretly hijacked such that if she chooses to eat anything other
than a pizza for lunch, the device will activate and will send an electrical impulse
that will force her to utter the words "I'd like a pizza, please" to the cashier.
Let's now say that Alice does choose pizza for lunch every day of the week,
thereby leaving the device dormant. Now, did Alice truly make this choice our
of her own free will? The Principle of alternate possibilities says she is not
free, since she could not have chosen otherwise (to not eat pizza).
Frankfurt on the contrary argues that since Alice chose to eat pizza, even if she
could not have done otherwise, she is indeed free.

The second (graver, in my opinion) attack one can mount is by questioning
the separation between internal and external factors that is required for 
the notion of "the action from \emph{within}". What is the criteria to separate
internal and external factors? Isn't our rich inner life shaped by our interactions
with the external world? Does it really make sense to separate the two, and
does such a separation even exist in the first place? Intuitively, I feel
that this does not exist, and hence this particular notion of compatibilism is
indefensible.

We now move on to see the positions of Strawson and Kant, and how they choose
to defend the compatibilist position with more refined arguments. However,
their axioms contradict each other, and thus we will have to consider
which system appeals to us more, or try to resolve their contradictions.

\section{Kantian ethics}

Kantian ethics argues that we have to use Reason to find out what is right.
\textbf{Categorical imperatives} are what he govern his system of ethics, and
these are derived from pure reason. We can determine what is right and wrong.
To judge if an action is moral, we must universalize the action, and see what
effect this has on the outcome of our action. 

Let's consider a case of theft: Is it legal for Bob to steal something? Let's
say for the sake of argument it is. If we universalize this, we get an
imperative that reads "It is legal for \emph{anyone} to steal anything". This
is clearly a contradiction, for if everyone is stealing, then no one really is. 

Hence, we are led to the opposite conclusion, that no one can lie!
This can lead to counter-intuitive results. For example, Claire might have to lie
about Alice's whereabouts to her furious husband Bob, as he just found out
that Alice has cheated on him for the last year, and is prone to stabbing her
in a fit of rage. However, under Kant's system, Claire cannot lie, under
even this extreme circumstance.

In Kant’s view, morality requires us to be free --- we must have had the
ability to do otherwise. Consider an example of Duncan, who steals a loaf
of bread in soviet russia to keep his family fed for the night.
Kant asserts for this action to be morally wrong, it must have been in his control.
That is, he defines "control" as "had the ability to do otherwise" --- that is,
it was within his power to \emph{not} have committed the theft. If it was
not under his control, then it incorrect to call this action morally wrong.
Punishing his act might be useful, insofar as it can mold the behavior of others,
but his action per se continues to be morally right.  Moral rightness and
wrongness apply only to free agents who control their actions.

% Justice: The ancient greeks understood Justice as harmony.  Violating ones place
% in the social order is unjust. There is a utilitarian account of justice, where
% a just society is one that results in the most good for the most people. For a
% political liberterian, a just society is one that allows its citizens to be
% maximally free. The view of justice as pushed forward by Rawles is that justice
% is fairness. Rawles reasoned that the world is full of natural inequalities (one
% can speak about "moral luck"). 

\section{P F Strawson: Freedom and resentment}

P F Strawson jokingly remarked that he would work on moral philosophy when his
insight waned \cite{TODO}. With that disclaimer out of the way, we show how
Strawson manages to sidestep the entire question of determinism with a slick
argument.


Strawson begins by asserting that it is a mistake to think that moral responsibility
requires free will. He rejects Kant's style of constructing a system of ethics,
whereby one begins from moral principles and then constructs an ethical system.
Rather, he wishes to build a system from the ground up. 

He examines our everyday intuitions of moral responsibility. We notice that
we do assign moral blame and praise to other people who we believe are blame
or praise worthy. He calls these styles of reaction where we hold people
morally culpable as our \textbf{reactive attitudes}. There are 
individuals such as children and madmen whom we do not hold morally
responsible. However, our interactions with children and madmen are objective,
where we hold our reactive attitudes in check, to interact with them with
what Strawson calls the \textbf{objective attitude}

This is an account of moral responsibility that does not require a freedom
condition. Being morally responsible is being a fitting target of the
reactive attitude. According to Strawson, to be a fitting target, one needs
to display a quality of will --- if you display me hostility or indifference,
then I can punish you. If you show me gratitude or happiness, I might reward
you. 

The truth of determinism cannot force us to give up our subjective viewpoint,
since the reactive attitudes are too deeply embedded in our
humanity. Hence, all we can do is to \emph{embrace} the reactive attitudes.
This forces us to conclude that even without free will, one can hold people
morally responsible. Thus, Strawson is a compatibilist.

\section{Synthesis}

My problem with Kant's account of ethics is that it is far too easy to break
it with thought experiments that lead to absurd results. What I do enjoy
is the fact that Kant begins with a reasonable framework for how ethics
\emph{ought} to be constructed.

On the other hand, while Strawson's account agrees with our everyday experience,
this is the problem --- Strawson chooses to defend the state of the world with
\emph{is} statements, and essentially dodges the question of
"what should responsibility be".

It is hard to bring these two views together, but an attempt can be made. It
is reasonable to argue that Strawson's position is the reasonable one, since
it does reflect our moral intuitions, and does not lead to paradoxical situations
the way Kant's does. So, let us try to construct a set of axioms that would
allow us to derive Strawson's position. These might be:

\begin{itemize}
    \item Humans interact with one another using our emotions, and these are 
        the most important form of social interaction.
    \item Our reactive attitudes define the shape of our social interactions.
\end{itemize}

\nocite{*} \printbibliography

\end{document}
