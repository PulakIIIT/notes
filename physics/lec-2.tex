\chapter{Functional calculus}

this chapter develops a completely handwavy physics version of functional
analysis.


\begin{definition}
    A \textbf{functional} $F$ is a function: $F : (\reals \to \reals) \to \reals$
\end{definition}

\begin{notation}
Evaluation of a functional $F$ with respect to $f$ is denoted by $F[f]$.
\end{notation}

\section{Functional Derivative - take 1}

Consider a functional $F: (\reals \to \reals) \to \reals$, 
a function $f: \reals \to \reals$, 
and a "test function" $\phi: \reals \to \reals$.

Consider a functional $F$.  We only define the derivative of a functional $F$
with respect to a function $f$ by what happens under an integral sign as follows:

\begin{align*}
\int \fdv{F}{f} (x) \phi(x) \dd x = 
\lim_{\epsilon \to 0}
\frac{F[f + \epsilon \phi] - F[f]}{\epsilon}
\end{align*}

Now, we can define a small variation in $F$ as:

\begin{align*}
    \delta F &: (\reals \to \reals) \times (\reals \to \reals) \to \reals \\
    \delta F(f, \phi) &\equiv \int \fdv{F}{f} (x) \phi(x) \dd x
\end{align*}

Intuitively, $\delta F$ tells us the variation of the function $f$ along
a test function $\phi$. So, it encapsulates some kind of "directional derivative".


So, we can look at $\fdv{F}{f}$ as a functional as follows:
\begin{align*}
    \fdv{F}{f} &: (\reals \to \reals) \to \reals \\
    \fdv{F}{f} (\phi) &=  \delta F (f, \phi)
\end{align*}

Wehre $\fdv{F}{f}$ allows us to "test" the change of $F$ with respect to $f$
along a given "direction" $\phi$.

\section{Functional Derivative as taught in class}

Substitute $\phi = \delta (x - p)$. Now, the quantity:

\begin{align*}
    &\fdv{F}{f}(x) \phi(x) =
    \delta F (f, \delta (x - p))  \\
    % 
    &\text{Rewriting $\delta F$ by sticking it under an integral:} \\
    %
    &\int \fdv{F}{f} (x) \delta (x - p) \dd x =  
    \lim_{\epsilon \to 0}
    \frac{F[f + \epsilon \delta (x - p)] - F[f]}{\epsilon}  \\
    %
    & \fdv{F}{f}{\bigg \rvert_p} = 
    \lim_{\epsilon \to 0}
    \frac{F[f + \epsilon \delta (x - p)] - F[f]}{\epsilon}
\end{align*}

That is, we can start talking about "derivative of the functional $F$ with
respect to a function $f$ at a point $p$" as long as we only test the functional $F$ against
$\delta$-functions.

So, we can alternatively define this quantity as:

\begin{align*}
\fdv{F}{f}{\bigg \rvert_p} \equiv 
\lim_{\epsilon \to 0} \frac{F[f + \epsilon \delta (x - p)] - F[f]}{\epsilon}
\end{align*}

While this does not "look like a functional", it actually is, if we
mentally replace:

\begin{align*}
    p \to  \int ~\rule{0.3cm}{0.1mm}~\delta (x - p) \dd x
\end{align*}

This is how mukku got that expression.

\section{Common functional derivatives}

\subsection{$F[f] = \int_0^\infty f dx$}


\begin{align*}
    \frac{\delta F[f]}{\delta f (x_0)} &=
        \lim_{\epsilon \to 0} 
         \frac{\int_0^\infty (f + \epsilon \delta (x - x_0)) \dd x- \int_0^\infty f \dd x }{\epsilon} \\
         %%%%
         &= \int_0^\infty \delta(x - x_0) \dd x = 1
\end{align*}

\subsection{Derivative of $F[\phi] \equiv \int (\pdv{\phi}{y})^2$}

$\fdv{F}{f}{\bigg \rvert_p} = \int (\pdv{\phi}{y})^2$


\section{Deriving E-L from functional magic}

\section{Weird things in Functional Analysis as taught in class}

Consider the functional

$J[f] = \int g[f'] dy$:

since $g$ is a functional, it has a type $g: (\reals \to \reals) \to \reals$.
So, our integrand must be some function $df$, and not some
\textbf{space component} $dy$. \textbf{I don't understand what the definition of J means}.

