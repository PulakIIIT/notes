\chapter{Functional calculus}

this chapter develops a completely handwavy physics version of functional
analysis.


\begin{definition}
    A \textbf{functional} $F$ is a function: $F : (\reals \to \reals) \to \reals$
\end{definition}

\begin{notation}
Evaluation of a functional $F$ with respect to $f$ is denoted by $F[f]$.
\end{notation}

\section{Functional Derivative - take 1}

We can only use a functional under an integral sign. Consider a functional $F$.
We define the derivative of this functional as:

\begin{align*}
\int \fdv{F}{f} (x) \phi(x) \dd x = 
\lim_{\epsilon \to 0}
\frac{F[f + \epsilon \phi] - F[f]}{\epsilon}
\end{align*}

So,
\begin{align*}
    \fdv{F}{f} &: (\reals \to \reals) \to \reals \\
    \fdv{F}{f} (\phi) &=  \int \fdv{F}{f} (x) \phi(x) \dd x
\end{align*}

\section{Functional Derivative as taught in class}

Substitute $\phi = \delta (x - p)$. Now, the quantity:

\begin{align*}
    \int \fdv{F}{f}(x) \phi(x) \dd x = 
    \int \fdv{F}{f}(x) \delta (x - p) = 
    \fdv{F}{f} (p)
\end{align*}

That is, we can start talking about "derivative of the functional $F$ with
respect to a function $f$ at a point $p$" as long as we only test the functional $F$ against
$\delta$-functions.

So, we can alternatively define this quantity as:

\begin{align*}
\fdv{F}{f}{\bigg \rvert_p} \equiv 
\lim_{\epsilon \to 0} \frac{F[f + \epsilon \delta (x - p)] - F[f]}{\epsilon}
\end{align*}

While this does not "look like a functional", it actually is, if we
mentally replace:

\begin{align*}
    p \to  \int ~\rule{0.3cm}{0.1mm}~\delta (x - p) \dd x
\end{align*}

