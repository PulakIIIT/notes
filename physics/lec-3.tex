\chapter{Maxwell's equations in Minkowski space}
% http://www.physics.ucc.ie/apeer/PY4112/Tensors.pdf

Let us first review Maxwell's equations:

\begin{align*}
&\div E = \frac{\rho}{\epsilon_0}~\text{(Electric charges produce fields)}\\
&\div B = 0~\text{(Only magnetic dipoles exist)}\\
&\curl E = - \pdv{B}{t}~\text{(Lenz Law - time varying magnetic field induces current that opposes it)} \\
&\curl B =  \mu_0 \bigg(J + \epsilon_0 \pdv{E}{t} \bigg)~\text{(Ampere's law + fudge factor)}
\end{align*}

Begin with the equation that $\div B = 0$. This tells that $B$ can be written
as the curl of some other field --- $B = \curl A$. Hence

\begin{equation}
    \boxed{ B^i = \levicevita^{ijk}  \partial_j A^k }
\end{equation}

Next, take $\curl E = - \pdv{B}{t}$.


\begin{align*}
&\curl E = - \pdv{B}{t} = \pdv{(\curl A)}{t} = \curl{\pdv{A}{t}} \\
&\curl (E + \pdv{A}{t}) = 0 \\
&\text{writing this as the divergence of some field $\phi$ scaled by $\alpha : \reals$} \\
&E + \pdv{A}{t} = \alpha \big(\div \phi\big) \\
&E = \alpha \div \phi - \pdv{A}{t}
\end{align*}

Since electrostatics is time-independent, we choose to think of $\alpha = -1$, 
so we can interpret $\phi$ as the potential.

\begin{equation}
     E^i = - \pdv{\phi}{x^k}  g^{ik} - \pdv{A}{t}^i
\end{equation}

A slight reformulation (since we know that in Minkowski space, $\partial_t = \partial_0$)
we get the equation:


\begin{equation}
    \boxed{ E^i = - g^{ik} \partial_k \phi - \partial_0 A^i}
\end{equation}

We get the metric $g^ik$ involved to raise the covariant $\pdv{\phi}{x^k}$
into the contravariant $E^i$.

(\textbf{Sid question:} how does one justify switching $\curl$ and $\partial$? It feels like some algebra)

\textbf{Here be magic!} We define A new rank-$2$ tensor in Minkowski space-time,
called $F$ (for Faraday),

\begin{equation}
    \boxed{F_{\mu \nu} = \partial_\mu A_\nu - \partial_\nu A_\mu}
\end{equation}

(\textbf{Sid question:} why is this object $F_{\mu \nu}$ covariant? What does this \textit{mean}?)

\begin{lemma}
$F_{\mu \nu}$ is antisymmetric.
\end{lemma}

\begin{lemma}
$F_{\mu \nu}$ has 6 degrees of freedom
\end{lemma}
\begin{proof}
Number of degrees of freedom of $F$: 
\begin{align*}
\frac{4^2~\text{(total)} - 4~\text{(diagonal)}}{2~\text{(anti-symmetry)}} = 6
\end{align*}
\end{proof}

Notice that $F$ is a 1-form!

We now wish to re-expresss $B^{ij}$ and $E^{ij}$ in terms of $F$, so that
this $F$ captures all of maxwell's equations.

\begin{align*}
    B^i &= \levicevita^{ijk}  \partial_j A^k = \levicevita^{ikj} \partial_k A^j \tag*{by $k$, $j$ being free variables} \\
    B^i &= \frac{1}{2} \bigg( \levicevita^{ijk} \partial_j A^k + \levicevita^{ikj} \partial_k A^j \bigg) \\
        &\text{Substituting $\partial_j A_k - \partial_k A_j = F_{jk}$, } \\
    B^i &= \frac{1}{2} \levicevita^{ijk} F_{jk}
\end{align*}


So, $B$ in terms of $F$ is:
\begin{equation}
    \boxed{B^i = \frac{1}{2} \levicevita^{ijk} F_{jk}}
\end{equation}

Similarly, we wish to write $E$ in terms of $F$. The algebra is as follows:
\begin{align*}
    E^i = -g^{ik} \partial_k \phi - \partial_0 A^i
\end{align*}

