% http://mirrors.ibiblio.org/CTAN/macros/latex/contrib/physics/physics.pdf
\newcommand{\qdot}{{\dot q}}

\chapter{Lagrangian, Hamiltonian mechanics}

Mechanics in terms of generalized coords.
\section{Lagrangian}
Define a functional. $L$ over the config. space of partibles $q^i$, $qdot^i$.
$L = L(q^i, qdot^i)$.  We have an explicit dependence on $t$.



$L = KE - PE$

Assuming a 1-particle system of unit mass,
$$L = \frac{1}{2} \qdot^2 - V(q)$$

Assuming an n-particle system of unit mass,
$$L = \sum_i \frac{1}{2} {qdot^i}^2 - V(q^i)$$ 

\section{Variational principle}

Take a minimum path from $A$ to $B$. Now notice that the path that is
slightly different from this path will have some delta from the minimum.

Action
$$S(t0, t1) = \int L \dd t = \int_{t0}^{t1} L(q^i, qdot^i) \dd t$$.
Least action: $\delta S = 0$

In physics, we try to minimise the action $L = T - V$ where $T$ is the
Kinetic energy (Travail), and $V$ (Voltage) is the Potential energy.

So, the question is, why does minimising the lagrangian work,
and how do we get the euler-lagrange equations from this?
