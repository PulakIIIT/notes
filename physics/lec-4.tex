\chapter{Gauge theories}
We construct a 1-dimensional guage theory and study its symmetries.

\section{Euler-Lagrange equations for a field}
\section{Klein-gordon equations}
\section{Lagrangian for a massive scalar field}
\section{Symmetries of a scalar field Lagrangian}
\section{Derving the force of the EM-field from the Lanrangian}

Recall that $B = \curl A$, $E = \grad \phi - \pdv{A}{t}$, and the force
on a particle is $\va F = q (\va E + \va v \times \va B)$.

\begin{align*}
ma &= e (\grad \phi - \pdv{A}{t}) + e (v \times (\curl A)) \\
ma &= e (\grad \phi - \pdv{A}{t}) + e (\div (v \vdot A) - (v \vdot \grad) A)
\end{align*}

Note that $(v \cdot \grad) A$ is:
\begin{align*}
v \vdot \grad = \dv{x}{t} \pdv{x} + \dv{y}{t} \pdv{y} + \dv{z}{t} \pdv{z} \\
(v \vdot \grad) A = \dv{x}{t} \pdv{A}{x} + \dv{y}{t} \pdv{A}{y} + \dv{z}{t} \pdv{A}{z} \\
\end{align*}

However, now let us compare $\dv{A}{t}$ and $(v \cdot \grad) A$:

\begin{align*}
\dv{A}{t} = \pdv{A}{t} + \dv{x}{t}\pdv{A}{x} + \dv{y}{t}\pdv{A}{y} \dv{z}{t}\pdv{A}{z} \\
\dv{A}{t} = \pdv{A}{t} + (v \cdot \grad) A
\end{align*}

Now, rewriting $ma$,
\begin{align*}
ma &= e (\grad \phi - \pdv{A}{t}) + e (\div (v \vdot A) - (v \vdot \grad) A) \\
ma &= e (\grad \phi - \pdv{A}{t}) + e (\div (v \vdot A) - (v \vdot \grad) A)
\end{align*}

\section{Local and global symmetries}

\chapter{Charged particle interaction in fields, or, how maxwell's equations have $U(1)$ symmetry}
(written in red ink pen)

Consider a scalar field $\phi$, and the lagrangian:

\begin{align*}
    &\Lag(\phi, \partial_\mu \phi) = \frac{1}{2} (\partial_\mu \phi) (\partial_\mu \phi) - \frac{1}{2} m^2 \phi^2
\end{align*}

Next, we want to consider charged particle interactions, which comes from
$H = \frac{(p - eA)^2}{2m}$, which creates the lorentz force $e \va v \times \va B = e \va v \vdot \va A$.

We know that $\Lag$ is invariant under global rotation $e^{i \theta}$. Now we
study local gauge invariance by making $\theta$ a function of space. That is,
$\theta \to \theta(x)$.

We use the complex form of the lagrangian:

\begin{align*}
    \Lag = (\partial_\mu \phi) (\partial^\mu \phi)^* - m^2|\phi|^2
\end{align*}

We now consider the transform:

\begin{align*}
    phi \to e^{i e \theta(x)} \phi
\end{align*}

This implies the transforms:
\begin{align*}
    &\partial_\mu \phi = 
    \partial_\mu (e^{i e \theta(x)} \phi) = 
    e^{i e \theta(x)} \partial_\mu \phi + i e e^{i e \theta(x)} \phi \partial_\mu \theta = \\
    &e^{i e \theta(x)}(\partial_\mu \phi + i e \phi \partial_\mu \theta)
\end{align*}

Hence, for invariance, we get:

\begin{align*}
    &(\partial_\mu \phi ) (\partial^\mu \phi )^*  \\
    =~&(e^{i e \theta(x)}(\partial_\mu \phi + i e \phi \partial_\mu \theta))
    (e^{i e \theta(x)}(\partial^\mu \phi + i e \phi \partial^\mu \theta))^* \\
     =~&(\partial_\mu \phi + i e \phi \partial_\mu \theta)(\partial^\mu \phi^* - i e \phi^* \partial^\mu \theta)
\end{align*}

We introduce:
\begin{align*}
    D_\mu \phi \equiv (\partial_\mu - i A_\mu) \phi (\partial^\mu + i A^\mu) \phi^*
\end{align*}

We shall check that this transforms in a lorentzian way. In the new coordinate
system:
\begin{align*}
    \bar D_\mu \bar \phi \equiv &([\partial_\mu - i \bar A_\mu) \bar \phi] 
        [(\partial^\mu + i \bar A^\mu) \bar \phi^*] \\
    % 
    =~&[(\partial_\mu - i \bar A_\mu) (e^{i e \theta(x)}) \phi] 
    [(\partial^\mu + i \bar A^\mu) (e^{i e \theta(x)}\phi)^*]  \\
    %
    =~&[\partial_\mu (e^{i e \theta(x)} \phi) - i \bar A_\mu e^{i e \theta(x)} \phi]
    [\partial_\mu (e^{-i e \theta(x)} \phi^*) + i \bar A_\mu e^{-i e \theta(x)} \phi^*] \\
    % 
    &\text{evaluate $\partial_\mu(UV)$} \\
    =~&[e^{i e \theta(x)} (\partial_\mu \phi) + i e  e^{i e \theta(x)} (\partial_\mu \theta)\phi - i \bar A_\mu e^{i e \theta(x)} \phi]
    [e^{- i e \theta(x)} (\partial_\mu \phi^*) - i e  e^{- i e \theta(x)} (\partial_\mu \theta) \phi^* + i \bar A_\mu e^{- i e \theta(x)} \phi^*] \\
    % 
    =~&[e^{i e \theta(x)} (\partial_\mu \phi + i e (\partial_\mu \theta) \phi - i \bar A_\mu  \phi)]
    [e^{- i e \theta(x)} (\partial_\mu \phi^* - i e (\partial_\mu \theta) \phi^* + i \bar A_\mu  \phi^*)] \\
    %
    &\text{cancelling $e^{i e \theta(x)}$ with $e^{- i e \theta(x)}$} \\
    =~&[\partial_\mu \phi + i e (\partial_\mu \theta) \phi - i \bar A_\mu  \phi]
    [\partial_\mu \phi^* - i e (\partial_\mu \theta) \phi^* + i \bar A_\mu  \phi^*] \\
    %
    =~&[\partial_\mu + i e (\partial_\mu \theta) - i \bar A_\mu  ]\phi
    [\partial_\mu  - i e(\partial_\mu \theta) + i \bar A_\mu ] \phi^* \\
\end{align*}

Comparing:
\begin{align*}
    D_\mu \phi &\equiv (\partial_\mu - i A_\mu) \phi (\partial^\mu + i A^\mu) \phi^* \\
    %
    \bar D_\mu \bar \phi &\equiv [\partial_\mu + i e (\partial_\mu \theta) - i \bar A_\mu  ]\phi
    [\partial_\mu  - i e (\partial_\mu \theta)  + i \bar A_\mu ] \phi^* \\
    %
    -i A_\mu &= + i e (\partial_\mu \theta) - i \bar A_\mu \\
    -A_\mu &= e (\partial_\mu \theta) - \bar A_\mu \\
    \bar A_\mu &= e (\partial_\mu \theta) + A_\mu \\
   &\textbf{TODO: This is not what mukku got! mukku got $\bar A_\mu = A_\mu - e(\partial_\mu \theta)$. HOW?}
\end{align*}


So, now, our modified lagrangian is:

\begin{align*}
    \Lag (\phi, \phi^*, \partial_\mu \phi, \partial_\mu \phi^*, A_\mu) = 
    [(\partial_\mu - i A_\mu) \phi][(\partial^\mu - i A^\mu) \phi]^* - \frac{1}{2} m |\phi|^2
\end{align*}
