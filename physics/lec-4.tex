\chapter{Gauge theories}
We construct a 1-dimensional guage theory and study its symmetries.

\section{Euler-Lagrange equations for a field}
Consider a Lagrangian:
$$\Lag(\phi, \phi', \dot \phi) = {\dot \phi}^2  - \phi'^2$$

When written in terms of $M^4$ (minkowski space), we know that
$$
\partial_\mu \equiv (\partial_{t}, - \grad)
$$

Hence, in minkowski space, the lagrangian becomes a function of \textit{only}:
$$
\Lag(\phi, \partial_\mu \phi) \equiv \dots
$$

So we managed to unify the space derivative and the time derivative (yay).

Now, we can consider the action of this Lagrangian:

\begin{align*}
    S[\phi] = \int L(\phi, \partial_\mu \phi)~\dd^4x
\end{align*}

Minimising the functional $S$,
\begin{align*}
    &\delta S[\phi] = 0 \\
    & \delta \int L(\phi, \partial_\mu \phi)~\dd^4x = 0 \\
    %
    &\text{For an analogy, consider $dL = \pdv{L}{\phi} \dd \phi + \pdv{L}{\psi} \dd \psi + \dots$} \\
    %
    & \int \bigg[\pdv{L}{\phi} \var \phi + 
    \pdv{L}{\partial_\mu \phi} \var (\partial_\mu \phi) \bigg] \dd^4 x  = 0\\
    %
    &\text{Using linearity of integration, and commuting of $\delta, \partial_\mu$}\\
    %
    & \int \pdv{L}{\phi} (\var \phi) \dd^4 x + 
      \int \pdv{L}{\partial_\mu \phi}  (\partial_\mu \var \phi) \dd^4 x = 0 \\
    %
    &\text{Using $\int U \dd V = UV - \int V \dd U$, 
        $V = \delta \phi$, 
    $U = \pdv{L}{\partial_\mu \phi}$ } \\
    \\
    %
    & \int \pdv{L}{\phi} (\var \phi) \dd^4 x + 
       \eval{\pdv{L}{\partial_\mu \phi}  (\var \phi)}_{endpoints} 
      -  \int \delta \phi \bigg(\partial_\mu \fdv{L}{\partial_\mu \phi} \bigg) \dd^4 x  = 0\\
  %
  &\text{Since fields decay at endpoints, forget the integral} \\
  %
  & \int \pdv{L}{\phi} (\var \phi) \dd^4 x  
  -  \int \delta \phi \bigg(\partial_\mu \pdv{L}{\partial_\mu \phi} \bigg) \dd^4 x = 0\\
  %
  &\text{Refactoring to pull the common $\delta \phi$,}\\
  %
  & \int \bigg(\pdv{L}{\phi} -  \partial_\mu \pdv{L}{\partial_\mu \phi} \bigg) (\var \phi) \dd^4 x  = 0 \\
  %
  & \text{Since this is true for all perturbations $\delta \phi$ implies that:} \\
  %
  &\pdv{L}{\phi} -  \partial_\mu \pdv{L}{\partial_\mu \phi} = 0
\end{align*}


Hence, we have the Euler-Lagrange equation for \textbf{scalar fields}:
\begin{equation}
    \boxed{\pdv{L}{\phi} -  \partial_\mu \pdv{L}{(\partial_\mu \phi)} = 0} 
\end{equation}

So next, let's consider a Lagrangian (which is supposedly the free particle
analogue):
(\textbf{TODO: find out why this is free particle KE})

\begin{align*}
    &(\partial_\mu \phi)^2 \equiv \partial_\mu \phi \partial^\mu \phi~\text{(This is notation)} \\
    &\text{Note that:} \\
    &\partial_\mu \phi \partial^\mu \phi = \partial_\mu \phi \eta^{\mu \nu} \partial_\nu \phi 
    ~\text{(where $\eta^{\mu \nu}$ is the metric; lowering indeces)} \\
    %
    \\
    &\text{We define the Lagrangian as:}\\
    &\Lag(\phi, \partial_\mu \phi) = \frac{1}{2}(\partial_\mu \phi)^2
\end{align*}

Calculating the terms in the EL equation:
\begin{align*}
    &\pdv{L}{\phi} = 0 \\
    \\
    \\
    &\pdv{L}{(\partial_\sigma \phi)} 
    = \pdv{(\partial_\sigma \phi)} \bigg( \frac{1}{2}(\partial_\mu \phi)^2  \bigg) \\
    %
    &=\pdv{(\partial_\sigma \phi)}  \bigg( \frac{1}{2}\eta^{\mu \nu} (\partial_\mu \phi) (\partial_\nu \phi) \bigg) \\
    %
    &=\frac{1}{2}\eta^{\mu \nu} \pdv{(\partial_\mu \phi)}{(\partial_\sigma \phi)} (\partial_\nu \phi) + 
    \frac{1}{2}\eta^{\mu \nu} (\partial_\mu \phi) \pdv{(\partial_\nu \phi)}{(\partial_\sigma \phi)} \\
    %
    &\textbf{(TODO: understand why $\pdv{(\partial_\mu \phi)}{(\partial_\sigma \phi)}  = \delta_{\mu}^\sigma$)}
    \\
    &=\frac{1}{2}\eta^{\mu \nu} \delta_{\mu}^{\sigma} (\partial_\nu \phi) + 
    \frac{1}{2}\eta^{\mu \nu} (\partial_\mu \phi) \delta_{\nu}^{\sigma} \\
    %
    &\text{(Contracting on $\mu$)}\\
    %
    &=\frac{1}{2}\eta^{\sigma \nu} (\partial_\nu \phi) + 
    \frac{1}{2}\eta^{\mu \sigma} (\partial_\mu \phi) \\
    %
    &\text{(Replacing dummy index $\mu \equiv \nu$)} \\
    %
    &=\eta^{\nu \sigma} \partial_{mu} \phi \\
    %
    &\textbf{(TODO: understand how this happens, something about $\eta$'s signature)} \\
    %
    &=\partial^\sigma \phi
    \\
\end{align*}
    Hence, the inner part of the second term in the EL equation is:
\begin{equation}
    \pdv{L}{(\partial_\sigma \phi)}  = \partial^\sigma \phi
\end{equation}
Now, the second term of the Lagrangian is:

\begin{align*}
    &\partial_\mu \pdv{L}{(\partial_\mu \phi)} = \partial_\mu (\partial^\mu \phi) 
    =~\partial_0 \partial^0  \phi - \laplacian \phi = \Box \phi
\end{align*}

Hence, finally, our Euler-Lagrange equation is:
\begin{align*}
    &\pdv{L}{\phi} -  \partial_\mu \pdv{L}{(\partial_\mu \phi)} = 0 \\
    &0 + \Box \phi  = 0
\end{align*}

So, finally, The Euler-Lagrange equations for a scalar field governed by
$L = \frac{1}{2} (\partial_\mu \phi)^2$ is:
\begin{equation}
    \boxed{\Box \phi  = 0}
 \end{equation}

\section{Klein-gordon equations}
\section{Lagrangian for a massive scalar field}
\section{Symmetries of a scalar field Lagrangian}
\section{Derving the force of the EM-field from the Lanrangian}

Recall that $B = \curl A$, $E = \grad \phi - \pdv{A}{t}$, and the force
on a particle is $\va F = q (\va E + \va v \times \va B)$.

\begin{align*}
ma &= e (\grad \phi - \pdv{A}{t}) + e (v \times (\curl A)) \\
ma &= e (\grad \phi - \pdv{A}{t}) + e (\div (v \vdot A) - (v \vdot \grad) A)
\end{align*}

Note that $(v \cdot \grad) A$ is:
\begin{align*}
v \vdot \grad = \dv{x}{t} \pdv{x} + \dv{y}{t} \pdv{y} + \dv{z}{t} \pdv{z} \\
(v \vdot \grad) A = \dv{x}{t} \pdv{A}{x} + \dv{y}{t} \pdv{A}{y} + \dv{z}{t} \pdv{A}{z} \\
\end{align*}

However, now let us compare $\dv{A}{t}$ and $(v \cdot \grad) A$:

\begin{align*}
\dv{A}{t} = \pdv{A}{t} + \dv{x}{t}\pdv{A}{x} + \dv{y}{t}\pdv{A}{y} \dv{z}{t}\pdv{A}{z} \\
\dv{A}{t} = \pdv{A}{t} + (v \cdot \grad) A
\end{align*}

Now, rewriting $ma$,
\begin{align*}
ma &= e (\grad \phi - \pdv{A}{t}) + e (\div (v \vdot A) - (v \vdot \grad) A) \\
ma &= e (\grad \phi - \pdv{A}{t}) + e (\div (v \vdot A) - (v \vdot \grad) A)
\end{align*}

\section{Local and global symmetries}

\chapter{Charged particle interaction in fields, or, how maxwell's equations have $U(1)$ symmetry}
(written in red ink pen)

Consider a scalar field $\phi$, and the lagrangian:

\begin{align*}
    &\Lag(\phi, \partial_\mu \phi) = \frac{1}{2} (\partial_\mu \phi) (\partial_\mu \phi) - \frac{1}{2} m^2 \phi^2
\end{align*}

Next, we want to consider charged particle interactions, which comes from
$H = \frac{(p - eA)^2}{2m}$, which creates the lorentz force $e \va v \times \va B = e \va v \vdot \va A$.

We know that $\Lag$ is invariant under global rotation $e^{i \theta}$. Now we
study local gauge invariance by making $\theta$ a function of space. That is,
$\theta \to \theta(x)$.

We use the complex form of the lagrangian:

\begin{align*}
    \Lag = (\partial_\mu \phi) (\partial^\mu \phi)^* - m^2|\phi|^2
\end{align*}

We now consider the transform:

\begin{align*}
    phi \to e^{i e \theta(x)} \phi
\end{align*}

This implies the transforms:
\begin{align*}
    &\partial_\mu \phi = 
    \partial_\mu (e^{i e \theta(x)} \phi) = 
    e^{i e \theta(x)} \partial_\mu \phi + i e e^{i e \theta(x)} \phi \partial_\mu \theta = \\
    &e^{i e \theta(x)}(\partial_\mu \phi + i e \phi \partial_\mu \theta)
\end{align*}

Hence, for invariance, we get:

\begin{align*}
    &(\partial_\mu \phi ) (\partial^\mu \phi )^*  \\
    =~&(e^{i e \theta(x)}(\partial_\mu \phi + i e \phi \partial_\mu \theta))
    (e^{i e \theta(x)}(\partial^\mu \phi + i e \phi \partial^\mu \theta))^* \\
     =~&(\partial_\mu \phi + i e \phi \partial_\mu \theta)(\partial^\mu \phi^* - i e \phi^* \partial^\mu \theta)
\end{align*}

We introduce:
\begin{align*}
    D_\mu \phi \equiv (\partial_\mu - i A_\mu) \phi (\partial^\mu + i A^\mu) \phi^*
\end{align*}

We shall check that this transforms in a lorentzian way. In the new coordinate
system:
\begin{align*}
    \bar D_\mu \bar \phi \equiv &([\partial_\mu - i \bar A_\mu) \bar \phi] 
        [(\partial^\mu + i \bar A^\mu) \bar \phi^*] \\
    % 
    =~&[(\partial_\mu - i \bar A_\mu) (e^{i e \theta(x)}) \phi] 
    [(\partial^\mu + i \bar A^\mu) (e^{i e \theta(x)}\phi)^*]  \\
    %
    =~&[\partial_\mu (e^{i e \theta(x)} \phi) - i \bar A_\mu e^{i e \theta(x)} \phi]
    [\partial_\mu (e^{-i e \theta(x)} \phi^*) + i \bar A_\mu e^{-i e \theta(x)} \phi^*] \\
    % 
    &\text{evaluate $\partial_\mu(UV)$} \\
    =~&[e^{i e \theta(x)} (\partial_\mu \phi) + i e  e^{i e \theta(x)} (\partial_\mu \theta)\phi - i \bar A_\mu e^{i e \theta(x)} \phi]
    [e^{- i e \theta(x)} (\partial_\mu \phi^*) - i e  e^{- i e \theta(x)} (\partial_\mu \theta) \phi^* + i \bar A_\mu e^{- i e \theta(x)} \phi^*] \\
    % 
    =~&[e^{i e \theta(x)} (\partial_\mu \phi + i e (\partial_\mu \theta) \phi - i \bar A_\mu  \phi)]
    [e^{- i e \theta(x)} (\partial_\mu \phi^* - i e (\partial_\mu \theta) \phi^* + i \bar A_\mu  \phi^*)] \\
    %
    &\text{cancelling $e^{i e \theta(x)}$ with $e^{- i e \theta(x)}$} \\
    =~&[\partial_\mu \phi + i e (\partial_\mu \theta) \phi - i \bar A_\mu  \phi]
    [\partial_\mu \phi^* - i e (\partial_\mu \theta) \phi^* + i \bar A_\mu  \phi^*] \\
    %
    =~&[\partial_\mu + i e (\partial_\mu \theta) - i \bar A_\mu  ]\phi
    [\partial_\mu  - i e(\partial_\mu \theta) + i \bar A_\mu ] \phi^* \\
\end{align*}

Comparing:
\begin{align*}
    D_\mu \phi &\equiv (\partial_\mu - i A_\mu) \phi (\partial^\mu + i A^\mu) \phi^* \\
    %
    \bar D_\mu \bar \phi &\equiv [\partial_\mu + i e (\partial_\mu \theta) - i \bar A_\mu  ]\phi
    [\partial_\mu  - i e (\partial_\mu \theta)  + i \bar A_\mu ] \phi^* \\
    %
    -i A_\mu &= + i e (\partial_\mu \theta) - i \bar A_\mu \\
    -A_\mu &= e (\partial_\mu \theta) - \bar A_\mu \\
    \bar A_\mu &= e (\partial_\mu \theta) + A_\mu \\
   &\textbf{TODO: This is not what mukku got! mukku got $\bar A_\mu = A_\mu - e(\partial_\mu \theta)$. HOW?}
\end{align*}

So now, we know how $A_\mu$ transforms:
\begin{equation}
    \boxed{\bar A_\mu = A_\mu - e(\partial_\mu \theta)}
\end{equation}

So, now, our modified lagrangian is:

\begin{align*}
    \Lag (\phi, \phi^*, \partial_\mu \phi, \partial_\mu \phi^*, A_\mu) = 
    [(\partial_\mu - i A_\mu) \phi][(\partial^\mu - i A^\mu) \phi]^* - \frac{1}{2} m |\phi|^2
\end{align*}



We want to understand if we can add interaction terms for $A_\mu$. If not,
what the obstacles are. We would like the interaction terms to be a Lorentz
tensor, and we would like it to respect local symmetrics.

\begin{align*}
    &\mu^2 A_\mu A^\mu \to \mu^2 \bar A_\mu \bar A^\mu = \mu^2 (A_\mu + e \partial_\mu \theta) (A^\mu + e \partial^\mu \theta)
\end{align*}
So clearly, we cannot have mass terms of the form $\mu^2 A_\mu A^\mu$. We must
try other things and check if they are interesting.

We now try to inspect some invariants of $A_\mu$.  For example, we can
construct:
$$F_{\mu\nu} = \partial_\mu A_\nu - \partial_\nu A_\mu$$

We already know that $F$ is a Lorentz tensor. Let's check if it's gauge
invariant.

\begin{align*}
    \bar F_{\mu\nu} =~&\partial_\nu \bar A_\mu - \partial_\mu A_\nu \\
    =~&\partial_\nu(A_\mu - e (\partial_\mu \theta)) - \partial_\mu (A_\nu - e \partial_\nu \theta)) \\
    %
      &\text{Since partial derivatives commute:} \\
    %
    =~&\partial_\nu A_\mu - \cancel{e \partial_\nu \partial_\mu \theta} - \partial_\mu A_\nu - \cancel{e \partial_\mu \partial_\nu \theta})) \\
    =~&\partial_\nu A_\mu - \partial_\mu A_\nu = F_{\mu \nu}
\end{align*}

Hence, $F_{\mu \nu}$ is also gauge invariant.

Here, there is some sentence of how "there are only two invariants possible
for $F_{\mu \nu}$:

\begin{itemize}
    \item $F^{\mu \nu} F_{\mu \nu}$
    \item $e^{\mu \nu \alpha \beta} F_{\mu \nu} F_{\alpha \beta}$ (what is $e$ in this context?)
        Supposedly, we can ignore this since it relates to divergence (??)
        \textbf{TODO: figure this out}
\end{itemize}

So, extending the lagrangian with the $F_{\mu \nu}$ term gives us:

\begin{align*}
    &\Lag (\phi, \phi^*, \partial_\mu \phi, \partial_\mu \phi^*, A_\mu) = 
    \frac{1}{4} F^{\mu \nu} F_{\mu \nu} + [D_\mu \phi][D^\mu \phi]^* - \frac{1}{2} m |\phi|^2 \\
    %
    &\text{where:}\\
    %
    &D_\mu \equiv (\partial^\mu - i A^\mu) \\
    &F_{\mu\nu} \equiv \partial_\mu A_\nu - partial_\nu A_\mu
\end{align*}

Now, performing variational calculus on this lagrangian, we should theoretically
regain maxwell's equations:

\textbf{TODO: follow the last part of the stuff written in red ink pen to
understand what in the actual fuck happened}

