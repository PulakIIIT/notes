\newcommand{\cobpp}{\texttt{co-BPP}}
\newcommand{\ip}{\texttt{IP}}
\newcommand{\zkp}{\texttt{ZKP}}
\chapter{Exploring probabilistic complexity classes}

\begin{itemize}
    \item $\bpp \subset \Sigma^p_2 \cap \Pi^p_2$ (Sipser-Gaes theorem)
    \item $\bpp \subset \ppoly$ (Adleman's theorem)
    \item Hierarchy theorem for circuit complexity --- $SIZE(f(n))$
\end{itemize}


\begin{theorem}
    $\bpp \subseteq \Sigma^p_2 \cap \Pi^p_2$ (Sipser-Gaes theorem)
\end{theorem}
\begin{proof}
    First, notice that $\bpp = \cobpp$ (flip the answer, similar to $\ptime = \coptime$).

    Showing $\bpp \subseteq \Sigma_2^p$, it would automatically imply that $\bpp \subseteq \Pi_2^p$,
    since $\bpp = \cobpp$.
    So now, all we need to show is that $\bpp \subseteq \Sigma_2^p$.

    \begin{align*}
    &\text{Definition of \bpp~for a language $L$:} \\
    &\exists \text{poly time}~M, \forall w \in L, P[M(w, r) = accept] \geq \frac{2}{3} \\
    &\exists \text{poly time}~M, \forall w \notin L, P[M(w, r) = accept] \leq \frac{1}{3}
    \end{align*}

    By repeating, we can make the probabilities:
    \begin{align*}
    &\text{Definition of \bpp~for a language $L$:} \\
    &\exists \text{poly time}~M, \forall w \in L, P[M(w, r) = accept] \geq \ 1 - \frac{1}{2^n} \\
    &\exists \text{poly time}~M, \forall w \notin L, P[M(w, r) = accept] \leq \frac{1}{2^n}
    \end{align*}
\end{proof}

\begin{align*}
    S_x = \{ r~\vert~ r~\text{is good for input x} \}
\end{align*}

That is, given $r$ as the randomness for input $x$ on machine $M$, $M$ will
accept with high probability (ie. $M(x, r)$ accepts with high probability).


There are only two cases: Either $|S_x|$ is large, or $|S_x|$ is very small,
since $x \in L \lor x \notin L$.  Let $|r| = m$. Now, we count the size
of $|S_x|$ depending on the cases of $x \in_? L$.


%% TODO: fix this
\begin{align*}
    x \in L &\implies |S_x| \geq (1 - \frac{1}{2^n}) \cdot 2^m \\
    x \notin L &\implies |S_x| \leq \frac{1}{2^n} \cdot 2^m
\end{align*}

We can exploit this idea to show the theorem: "every string either has a large
$|S_x|$ or a small $|S_x|$ can be written as a "good" $\Sigma_2^p$ statement.


Let $S$ be any set, $|S|< 2^{m - n}$ ($|S|$ is small). Let $k = ceil(m / n) + 1$.
Let us define:

$$
S + u_i \equiv \{ x + u_i~\vert~x \in S \}
$$

where $+$ is bitwise-XOR.

First of all, note that $|S + u_i| = |S|$ ($\texttt{blank} + u_i$ = XOR = bijective, hence we don't change cardinality).


\begin{align*}
    \bigcup_{i=1}^k |(S + u_i)| \leq \sum_{i=1}^k |S + u_i| = \sum_{i=1}^k |S| = k |S| = k \cdot 2^{m - n}
\end{align*}


\begin{align*}
    \forall u_1, u_2, \dots u_k \in \{0, 1\}^*, |u_i| = m, \bigcup_{i=1}^k (S + u_i) \neq \{0, 1\}^m
\end{align*}
Immediate, because $|\bigcup_{i=1}^k (S + u_i)| \leq k \cdot 2^{m - n}$,
$|\{0, 1\}^m| = 2^m$.  ($m = poly(n)$, hence $k = poly(n)$, now argue inequalities).


Next, we argue something similar for \textit{large} sets. 
If $|S| \geq (1 - \frac{1}{2^n}) \cdot 2^m$,

\begin{align*}
    Stmt \equiv \exists u_1, u_2, \dots u_k \in \{0, 1\}^*, |u_i| = m, \bigcup_{i=1}^k (S + u_i) = \{0, 1\}^m
\end{align*}

We prove this using the \textbf{probabilistic method}.  $P[Stmt] > 0$ is what we want to show.
Let us show this by considering the \textit{converse}: $P[\lnot Stmt] < 1$. 


We first make some definitions to state formally: 
$$B_r \equiv r \notin \bigcup (S + u_i)$$
 $$B_r^i \equiv r \notin S + u_i$$

The converse is now the probability that $B_r$ does not cover $\{0, 1\}^m$.  
We know that $$|S + u_i| \geq (1 - \frac{1}{2^n}) \cdot 2^m$$.
Note that in this probability, we range over all $u_i$.

\begin{align*}
    P[B_r^i] \leq \texttt{TODO: complete this} = 2^{-n} \\
    P[B_r] \leq  k \cdot 2^{-n} \leq 1
\end{align*}

Hence, there must exist $\{ u_i \}$.


So now, we can pose the membership query $x \in_? L$ as:

\begin{align*}
    \exists u_1, u_2, \dots u_k, \forall r \in \{0, 1\}^m, r \in \bigcup_{i = 1}^k (S_x + u_i)
\end{align*}

If the statement had been $r \in S_x$, then this means that $M(x, r) = \texttt{ACCEPT}$.
However, here we know that $r \in \bigcup_{i = 1}^k (S_x + u_i)$. From this, 
we infer that $\exists i, r \in S_x + u_i$.  This means
that $r = r_0 + u_i$ where $M(x, r_0) = \texttt{ACCEPT}$. Hence, this means
that $r + u_i$ will accept $M(x, r + u_i) = M(x, r_0 + u_i + u_i) = M(x, r_0) = \texttt{ACCEPT}$.

So, we can create a new machine $M'(x, r) = \text{run}~M(x, r + u_i),  \forall i \in [1, k]$.
Note that $M'$ is in poly, since $M$ is poly, and $k = ceil(m, n) = poly(n)$.

\begin{align*}
    \exists u_1, u_2, \dots u_k, \forall r \in \{0, 1\}^m, M'(x, r) = \texttt{ACCEPT}
\end{align*}

This statement is now in $\Sigma_2^p$, since $M' \in \ptime$, and we have the correct
$\exists \forall$ structure.

