chapter{Review of the last 3 lectures, after add-drop}

\begin{itemize}
    \item Is it easier to \textit{pose} problems than to \textit{solve} them?
    \item Can every "solvable" problem have a solution that uses finite
        resources?
    \item What problems are \textit{interesting}?
    \item Are all interesting problems solved in an \textit{interesting} way?
        (P v/s NP)
    \item Can things get more interesting? (Quantum Mechanics, Approximation,
        Randomness, Interactivitiy, \dots)
\end{itemize}

Are there problems with infinite length input / output but can still be
posed in finite time?  Eg. output $\pi$ in decimal. However, we decided that
both input/output should be finite. We decided this does not belong to problems
we wish to solve it, since we cannot solve it in finite time. If we believe
that nature is inherently noisy, or nature is quantized, or nature has finite
precision, then we cannot consider problems that require infinite time as
problems in this universe (since Nature / the universe itself cannot pose
such a problem).

Quantum mechanics (which is a theory of quantization) is developed over
infinite precision mathematics ($\mathbb{C}$). Does this really make sense?  
There is a way in which a quantized universe can be infinite precision: This
is by using 'external help': There are infinite such quantized universes which
intersect at some points, and at those points, precision will increase. 
(If we both have a resolution of 1 pixel but are at a gap of 1/2, my least
count is now 1/2). If there are an infinite number of universes overlapping
at a single point, then we can construct "infinite precision". 
(\textit{I feel this is crazy. Is this really crazy?})


Posing a question is creating a language $L \subset \Sigma^*$. (Sid: a solution
is a classifier for $L$). 

Kannan's view: 
\begin{itemize}
    \item Finite space $\equiv$ finite information can be stored.
        (Turing: finite tape alphabet. Since a cell demarcates a finite volume,
        we want to have a finite amount of info in this cell)
    \item Information travels at finite speed. If we have cells, we should not
        be able to store and retreive information "equally" (based on how far
        we are from it). Hence, all infinite memory must be sequential memory
        since information travels at finite speed.
    \item Finite program $\equiv$ finite control.
\end{itemize}

Solution to these choices is a TM.


Do all languages have a TM recognizing it? No (\texttt{RE} = solvable by TM).

The class \texttt{R} = decidable by a TM (TM halts on all inputs). Diagonalization
led us to Halting problem.


We have the class $P$, and we claimed that $P$ is interesting. Given that $P$
is considered interesting because of feasibility, it is possible that there are
questions that are interesting even though \textbf{solving them} is not
feasible. For example, if we can actually \textbf{understand} the solution, or
the proof of non-existence of solutions, then we will care. \texttt{IP =
PSPACE} is one such magical case where if someone can solve with a lot more
power than you have access to, you can learn things from them interactively in
reasonable time.
