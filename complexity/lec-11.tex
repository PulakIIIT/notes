\chapter{\ip}

\section{The class \ip}
TODO: copy notes from the algorithms class, they're identical. Unfortunately,
I missed this because I was sick.

We know that \dip~(deterministic IP) ~= \nptime. We also conjecture that
\ptime~= \bpp. So, clearly, randomization and interactivity alone don't
give much.

It is surprising that $\nptime~\subseteq \ip$ (since GNI is in \ip), since we
are combining two powers that are useless on their own. So, it's shocking
that \ip~= \pspace!


\section{Arithmetization}

We define a new problem, $$\hashsat \equiv \{ (\phi, k)~\vert~ \phi~\text{is a 3-CNF boolean formula with exactly $k$ satisfying clauses} \}$$.

Note that $\overline{\texttt{3-SAT}} = \hashsat(0)$. Hence, \hashsat~ is a
generalization of a problem that is not known to be in NP ($\overline{\texttt{3-SAT}}$).  We will use \textit{arithmetization} to show that $\hashsat \in \ip$.

We have a prover $P$, and a formula $\phi$, with $k$ satisfying clauses.
$P$ knows $(\phi, k)$, and wants to convince the verifier $V$ of this fact.

We have a boolean formula $\phi$ with boolean variables. If we choose to
work in a larger field $F_p$, $\{0, 1\} \in F_p$. A boolean function
can be viewed as a polynomial in a larger field, which agrees with the
boolean formula on $\{0, 1\}$. For instance, if $\phi = x_1 \land x_2$,
we can create a polynomial $q(x_1, x_2) = x_1 x_2$. Clearly, $q$ and $\phi$ agree
on $x_1, x_2 \in \{0, 1\}$.

\begin{itemize}
\item $\phi(x) = \lnot x \leftrightarrow q_\phi(x) = (1 - x)$
\item $\phi(x_1, x_2) = x_1 \land x_2 \leftrightarrow q_\phi(x_1, x_2) = x_1 x_2$
\item $\phi(x_1, x_2) = x_1 \lor x_2 \leftrightarrow q_\phi(x_1, x_2) = 1 - (1 - x_1)(1 - x_2)$
\end{itemize}
Now, we can express every boolean formula as a polynomial.


\subsection{Rewrite \hashsat in terms of arithmetization}
$$
(\phi, k) \in \hashsat \leftrightarrow k = 
    \sum_{x_1 \in \{0, 1\}} \sum_{x_2 \in \{0, 1\}} \dots \sum_{x_n \in \{0, 1\}}
    Q_\phi(x_1, x_2, \dots, x_n)
$$



\subsection{Interactive proof for $k$ in terms of $Q_\phi$}
We provide a recursive solution to verify
\begin{align*}
k = 
    \sum_{x_1 \in \{0, 1\}} \sum_{x_2 \in \{0, 1\}} \dots \sum_{x_n \in \{0, 1\}}
    Q_\phi(x_1, x_2, \dots, x_n)
\end{align*}

Define $h(x_1) \equiv Q_\phi(x_1, b_2, b_3, \dots b_n)$, where $b_i \in \{0, 1\}$ are constant.
Now, $h(x_1)$ is a univariate polynomial of degree $d$.

If we consider all possibilities of $b_2 \dots b_n$, we get $2^{n - 1}$ variants
of the $h(x_1)$ polynomial.  We consider all of them by summing over all possiblities,
and collecting all of them in $H$.

\begin{align*}
H(x_1) = 
     \sum_{b_2 \in \{0, 1\}} \dots \sum_{b_n \in \{0, 1\}} Q_\phi(x_1, b_2, b_3, \dots b_n)
\end{align*}

Note that $H(x_1)$ still has degree $d$, since it is the sum of many $h(x_1)$
polynomials.  However, see that $H(x_1)$ is an exponential sum (it has exponential number of terms), and therefore
the verifier can't simply construct $H(x_1)$ it \textbf{needs} the prover
to proxy for $H(x_1)$ in some sense. This is where we need the unbounded prover.


We can now see that the original statement is the same as saying
\begin{align*}
k = H(0) + H(1)
\end{align*}

\subsection{Using this property to verify}
\subsubsection{$n = 1$}
If $n = 1$, then $V$ checks that $k = Q_{phi}(x_1)$.

\subsubsection{$n > 1$}
\begin{itemize}
\item If $n > 1$, $V$ asks $P$ to give $H(x_1)$. $P$ gives $S(x)$.

\item $V$ checks if $k = S(0) + S(1)$. If not, reject. If success, then we need
to verify that $S =_? H$.

\item $V$ chooses $a \in_{random} F_p$, and computes $S(a)$.

\item Recursively ask for a proof that $S(a) = \sum_{b_2} \sum_{b_3} \dots \sum_{b_n} Q_\phi(a, b_1, b_2, \dots b_n)$.
This is a recursive step since I can write this as:

\begin{align*}
S(a) &= 
     \sum_{x_2 \in \{0, 1\}} \dots \sum_{x_n \in \{0, 1\}}
    Q_\phi(a, x_2, \dots, x_n) \\
    &\text{Compare to} \\
k &= 
    \sum_{x_1 \in \{0, 1\}} \sum_{x_2 \in \{0, 1\}} \dots \sum_{x_n \in \{0, 1\}}
    Q_\phi(x_1, x_2, \dots, x_n)
\end{align*}

and now $Q'_\phi(x_2, \dots x_n) \equiv Q_\phi(a, x_2, \dots, x_n)$ is a polynomial
of degree $n - 1$.


\subsection{Completeness}
The prover $P$ that can actually construct the correct $H$ will cause the verifier
to accept with probability $1$.

\subsection{Soundness}
We are currently checking of $(H - S)(a) = 0$. We had chosen $a \in_{random} F_p$.
Only if $a$ is a root of $(H - S)$, will the prover escape undetected.

There are only $d$ roots for the polynomial $(H - S)$, and $F_p$ has $p - 1$
elements. So, the prover will be undetected with probability $P < \frac{d}{p - 1}$.

Probability he is caught in the first round is $d/p$. 

So, the probability a cheating prover will be detected will be $\big(1 - d/p\big)^{n}$.

\textbf{TODO: Fix this, my bounds here are broken as fuck}
\end{itemize}

\section{\ip~= \pspace}

We show that $\tqbf \in \ip$, hence $\pspace \subseteq \ip$, since \tqbf~is
a \pspace-complete problem.


\begin{align*}
Q = \forall x_1 \exists x_2 \forall x_3 \exists x_4 \dots \phi(x_1, x_2, \dots, x_n) =_? 1
\end{align*}

We will now show how to convert $\exists$ and $\forall$.

\begin{itemize}
\item $\forall x_1 Q(x_1, x_2, \dots x_n) \leftrightarrow Q(0, x_2, \dots, x_n) \times Q(1, x_2, \dots x_n)$
\item $\exists x_1 Q(x_1, x_2, \dots x_n) \leftrightarrow Q(0, x_2, \dots, x_n) + Q(1, x_2, \dots x_n)$. This
is slightly annoying, since it goes to 2 if it passes for both $x_1 = \{0, 1\}$. Rather, we can write
$\exists x, p(x)$ in terms of $\lnot (\forall x, \lnot p(x))$.
\end{itemize}


Now, how do we check if this polynomial $Q_{\tqbf}(x_1, x_2, \dots x_n) = 1$?
We can try to adapt our $\hashsat$ proof. However, here, the degree of the
polynomial will not be polynomial in the input size, it will be exponential!

\subsection{linearisation}
To beat the curse of expoential polynomial size, we use a technique called
\textbf{linearisation}. If $x \in \{0, 1\} \implies \forall k \in \nats, x^k = x$.

\begin{align*}
&L_1(Q(x_1, x_2, \dots, x_n)) = x_1 Q(1, x_2, \dots x_n) + (1 - x_1) L_1Q(0, x_2, \dots x_n)
\end{align*}

So now, the new $L_1 (L_2 (\dots (L_n (Q(x_1, x_2, \dots, x_n)))\dots))$ has degree at most $n$ (\textbf{TODO: really? check this}). \textbf{TODO: this is not what the linearized polynomial looks like, it should have
$O(n^2) = 1 + 2 + \dots n$ $L$ terms}.

We can now repeat the proof of \hashsat~here.

\begin{itemize}
\item If we are checking $\exists x_i, Q$, ask for $H(x_i)$, obtain $S$, check of $S(0) + S(1) =_? k$.
If not, reject. Else, recursively check that $S(a) = (\dots)$


\item If we are checking $\forall x_i, Q$, ask for $H(x_i)$, obtain $S$, check of $S(0) \times S(1) =_? k$.
If not, reject. Else, recursively check that $S(a) = (\dots)$

\item If we are checkin the linearization operator, $L_i(Q(~))$, check that $x_i S(1) + (1 - x_i) S(0) =_? k$.
\end{itemize}


To show that $\ip \subseteq \pspace$, notice that 
