\documentclass[11pt]{book}
%\documentclass[10pt]{llncs}
%\usepackage{llncsdoc}
\usepackage{amsmath,amssymb}
\usepackage{graphicx}
\usepackage{makeidx}
\usepackage{algpseudocode}
\usepackage{algorithm}
\usepackage{listing}
\evensidemargin=0.20in
\oddsidemargin=0.20in
\topmargin=0.2in
%\headheight=0.0in
%\headsep=0.0in
%\setlength{\parskip}{0mm}
%\setlength{\parindent}{4mm}
\setlength{\textwidth}{6.4in}
\setlength{\textheight}{8.5in}
%\leftmargin -2in
%\setlength{\rightmargin}{-2in}
%\usepackage{epsf}
%\usepackage{url}

\usepackage{booktabs}   %% For formal tables:
                        %% http://ctan.org/pkg/booktabs
\usepackage{subcaption} %% For complex figures with subfigures/subcaptions
                        %% http://ctan.org/pkg/subcaption
\usepackage{enumitem}
%\usepackage{minted}
%\newminted{fortran}{fontsize=\footnotesize}

\usepackage{xargs}
\usepackage[colorinlistoftodos,prependcaption,textsize=tiny]{todonotes}

\usepackage{hyperref}
\hypersetup{
    colorlinks,
    citecolor=black,
    filecolor=black,
    linkcolor=black,
    urlcolor=black
}

\usepackage{epsfig}
\usepackage{tabularx}
\usepackage{latexsym}
\newcommand\ddfrac[2]{\frac{\displaystyle #1}{\displaystyle #2}}

\def\qed{$\Box$}
\newtheorem{corollary}{Corollary}
\newtheorem{theorem}{Theorem}
\newtheorem{definition}{Definition}
\newtheorem{lemma}{Lemma}
\newtheorem{observation}{Observation}
\newtheorem{proof}{Proof}

%\newcommand{\P}{\texttt{P}}
%\newcommand{\NP}{\texttt{NP}}
%\newcommand{\PSPACE}{\texttt{PSPACE}}
%\newcommand{\NPSPACE}{\texttt{NPSPACE}}
%\newcommand{\TQBF}{\texttt{TQBF}}

\newcommand{\cobpp}{\texttt{co-BPP}}
\newcommand{\ip}{\texttt{IP}}
\newcommand{\dip}{\texttt{DIP}}
\newcommand{\zkp}{\texttt{ZKP}}

\newcommand{\textbb}[1]{$\mathbb{#1}$}
\newcommand{\nats}{\mathbb{N}}
\newcommand{\reals}{\mathbb{R}}


\newcommand{\hashsat}{\texttt{\#SAT}}
\newcommand{\tqbf}{\texttt{TQBF}}


\newcommand{\pptm}{\texttt{PPTM}}
\newcommand{\dtm}{\texttt{dtm}}


\title{Computational Complexity Theory}
\author{Siddharth Bhat}
\date{}

\begin{document}

\maketitle
\tableofcontents

% http://mirrors.ibiblio.org/CTAN/macros/latex/contrib/physics/physics.pdf
\newcommand{\qdot}{{\dot q}}

\chapter{Lagrangian, Hamiltonian mechanics}

Mechanics in terms of generalized coords.
\section{Lagrangian}
Define a functional. $L$ over the config. space of partibles $q^i$, $qdot^i$.
$L = L(q^i, qdot^i)$.  We have an explicit dependence on $t$.



$L = KE - PE$

Assuming a 1-particle system of unit mass,
$$L = \frac{1}{2} \qdot^2 - V(q)$$

Assuming an n-particle system of unit mass,
$$L = \sum_i \frac{1}{2} {qdot^i}^2 - V(q^i)$$ 

\section{Variational principle}

Take a minimum path from $A$ to $B$. Now notice that the path that is
slightly different from this path will have some delta from the minimum.

Action
$$S(t0, t1) = \int L \dd t = \int_{t0}^{t1} L(q^i, qdot^i) \dd t$$.
Least action: $\delta S = 0$

% \begin{align*}
%     \delta S &= \delta \int L(q^i, qdot^i) \dd{t} \\
%              &= \int \delta L(q^i, qdot^i) \dd{t} \\
%              &= \int \pdv{L}{q^i} \delta q^i + \pdv{L}{qdot^i} \delta qdot^i \dd{t} \\
% \begin{align*}




\section{Operators}

\subsection{Projectors --- $P$}

Suppose $W$ is a $k$-dimensional vector subspace of the $d$-dimensional 
vector space $V$. 

Using Gram-Schmidt, it is possible to construct an orthonormal basis
$\ket{1}, \ket{2}, \dots \ket{d}$ for $V$ such that $\ket{1} \dots \ket{k}$
is an orthonormal basis for $W$. Then the projector $P$ is defined as:
\begin{align*}
    P_W \equiv \sum_{i=1}^k \ketbra{i}
\end{align*}

\begin{itemize}
\item $P^\dagger = P$ (Immediate from writing in $\ket{i}$ basis)
\item $P^2 = P$ (Immediate from writing in $\ket{i}$ basis)
\end{itemize}

$Q = I - P$ is the projector onto orthogonal complement of the subspace that $P$.
projects into. This projects onto the $\ket{k+1} \dots \ket{d}$ basis.

\subsection{Normal operator}
\begin{align*} A A^\dagger = A^\dagger A \end{align*}


\begin{theorem}
Spectral theorem for normal operators:
Any normal operator $M$ on a vector space $V$ is diagonal with respect to some
orthonormal basis for $V$.
\end{theorem}
\begin{proof}
Let $\lambda$ be an eigenvalue of $M$. $P_\lambda$ is the projector onto
$\lambda$'s eigenvector. $Q_\lambda = P_\lambda^\bot$ is the orthogonal complement projector
of $P$.

We first establish a fact about $P M Q$:
\begin{align*}
&M M^\dagger \ket \lambda = M^\dagger (M \ket \lambda) = \lambda M^\dagger \lambda \\
&\text{Hence, $M^\dagger v \in P$.} \\
&Q (M^\dagger P) = 0 \implies (P M Q)^\dagger = 0 \implies P M Q = 0
\end{align*}

Next, we prove some properties of $QM$ and $QM^\dagger$
\begin{align*}
QM = QM(P + Q) = QMP + QMQ = QMQ \\
QM^\dagger = QM^\dagger(P + Q) = QM^\dagger P + QM^\dagger Q = (PMQ)^\dagger + QM^\dagger Q
\end{align*}

\begin{align*}
&\text{QMQ is normal:} \\
&(QMQ)^\dagger(QMQ) = Q^\dagger M^\dagger Q^\dagger Q M Q = Q M^\dagger Q M Q = Q M^\dagger M Q \\
&(QMQ)(QMQ)^\dagger = (Q M Q) (Q^\dagger M^\dagger Q^\dagger) = Q M Q M^\dagger Q = 
Q M M^\dagger Q = Q M^\dagger M Q = (QMQ)^\dagger QMQ
\end{align*}

\begin{align*}
&M = (P + Q) M (P + Q) \\
&M = P M P + P M Q + Q M P + Q M Q \\
&M = P M P + Q M Q \\
&M = \lambda_i \ketbra{i} + Q M Q \\
&\text{Since $Q M Q$ is normal, and we are performing induction on dimension, and $P \bot Q$,} \\
&M = \lambda_i \ketbra{i} + \sum_k \lambda_k \ketbra{k} \\
&\text{Hence M is normal}
\end{align*}
\end{proof}

\begin{theorem}
Any diagonalizable operator is normal
\end{theorem}
\begin{proof}
Let $M$ be diagonal with respect to basis $\ket{i}$.
Then, $M \equiv \sum_i \lambda_i \ketbra{i}$.
Now, $M^\dagger= \sum_i \lambda_i^* \ketbra{i}$. 
\begin{align*}
&M M^\dagger = \bigg(\sum_i \lambda_i \ketbra{i}\bigg)
    \bigg(\sum_j \lambda_j^* \ketbra{j}\bigg) \\
&M M^\dagger = \sum_i \lambda_i^* \lambda_i \ketbra{i} \\
&\text{Similarly,}  \quad M^\dagger M = (\sum_i \lambda_i^* \lambda_i \ketbra{i}) 
\end{align*}
\end{proof}

\subsection{Unitary operator}
\[ U U^\dagger = U^\dagger U = I \]
\begin{itemize}
\item unitary operator is normal.
\item unitary operator preserves inner products.
\begin{align*}
\bra{b'} \ket{a'} = \bra{b} U^\dagger U \ket{a} = \bra{b} I \ket{a}
\end{align*}
\end{itemize}

\subsection{Positive operator}
Special class of Hermitian operator.

\begin{align*}
 \forall v \in V, \bra v A \ket v \geq 0
\end{align*}

If the inner product is strictly greater than zero, then such an operator
is called as \emph{positive definite}. If it is greater than or equal
to zero, it is called \emph{positive semidefinite}.

\begin{theorem}
A positive operator is Hermitian
\end{theorem}
\begin{proof}
\textbf{TODO}. Proof most likely follows real case, where we use
cholesky to write it as $A^T A$ and then show that it is normal. We then
use the fact that its eigenvalues are greater than or equal to zero
to establish that it is Hermitian.
\end{proof}


\chapter{Maxwell's equations in Minkowski space}
% http://www.physics.ucc.ie/apeer/PY4112/Tensors.pdf

Let us first review Maxwell's equations:

\begin{align*}
&\div E = \frac{\rho}{\epsilon_0}~\text{(Electric charges produce fields)}\\
&\div B = 0~\text{(Only magnetic dipoles exist)}\\
&\curl E = - \pdv{B}{t}~\text{(Lenz Law - time varying magnetic field induces current that opposes it)} \\
&\curl B =  \mu_0 \bigg(J + \epsilon_0 \pdv{E}{t} \bigg)~\text{(Ampere's law + fudge factor)}
\end{align*}

\section{Constructing $F$, or Tensorifying Maxwell's equations}

Begin with the equation that $\div B = 0$. This tells that $B$ can be written
as the curl of some other field:

\begin{equation}
    \boxed{B \equiv \curl A}
\end{equation}

Expanding this equation of $B$ in tensorial form:
\begin{equation}
    \boxed{ B^i = \levicevita^{ijk}  \partial_j A^k }
\end{equation}

Next, take $\curl E = - \pdv{B}{t}$.


\begin{align*}
&\curl E = - \pdv{B}{t} = \pdv{(\curl A)}{t} = \curl{\pdv{A}{t}} \\
&\curl (E + \pdv{A}{t}) = 0 \\
&\text{writing this as the divergence of some field $\phi$ scaled by $\alpha : \reals$} \\
&E + \pdv{A}{t} = \alpha \big(\div \phi\big) \\
&E = \alpha \div \phi - \pdv{A}{t}
\end{align*}

Since electrostatics is time-independent, we choose to think of $\alpha = -1$, 
so we can interpret $\phi$ as the potential.

\begin{equation}
     E^i = - \pdv{\phi}{x^k}  g^{ik} - \pdv{A}{t}^i
\end{equation}

A slight reformulation (since we know that in Minkowski space, $\partial_t = \partial_0$)
we get the equation:


\begin{equation}
    \boxed{ E^i = - g^{ik} \partial_k \phi - \partial_0 A^i}
\end{equation}

We get the metric $g^ik$ involved to raise the covariant $\pdv{\phi}{x^k}$
into the contravariant $E^i$.

(\textbf{Sid question:} how does one justify switching $\curl$ and $\partial$? It feels like some algebra)

\textbf{Here be magic!} We define A new rank-$2$ tensor in Minkowski space-time,
called $F$ (for Faraday),

\begin{equation}
    \boxed{F_{\mu \nu} \equiv \partial_\mu A_\nu - \partial_\nu A_\mu}
\end{equation}

(\textbf{Sid question:} why is this object $F_{\mu \nu}$ covariant? What does this \textit{mean}?)

\begin{lemma}
$F_{\mu \nu}$ is antisymmetric.
\end{lemma}

\begin{lemma}
$F_{\mu \nu}$ has 6 degrees of freedom
\end{lemma}
\begin{proof}
Number of degrees of freedom of $F$: 
\begin{align*}
\frac{4^2~\text{(total)} - 4~\text{(diagonal)}}{2~\text{(anti-symmetry)}} = 6
\end{align*}
\end{proof}

Notice that $F$ is a 1-form!

\section{Expressing $B$, $E$ in terms of $F$}
We now wish to re-expresss $B^{ij}$ and $E^{ij}$ in terms of $F$, so that
this $F$ captures all of maxwell's equations.

\begin{align*}
    B^i &= \levicevita^{ijk}  \partial_j A^k = \levicevita^{ikj} \partial_k A^j \tag*{by $k$, $j$ being free variables} \\
    B^i &= \frac{1}{2} \bigg( \levicevita^{ijk} \partial_j A^k + \levicevita^{ikj} \partial_k A^j \bigg) \\
        &\text{Substituting $\partial_j A_k - \partial_k A_j = F_{jk}$, } \\
    B^i &= \frac{1}{2} \levicevita^{ijk} F_{jk}
\end{align*}


So, $B$ in terms of $F$ is:
\begin{equation}
    \boxed{B^i = \frac{1}{2} \levicevita^{ijk} F_{jk}}
\end{equation}

Similarly, we wish to write $E$ in terms of $F$. The algebra is as follows:
\begin{align*}
    E^i &= -g^{ik} \partial_k \phi - \partial_0 A^i \\
    E^i &= -g^{ik} \partial_k \phi - \partial_0 g^{ik} A_k  \tag*{Is this allowed? Am I always allowed to insert the $g_{ik}$?} \\
    E^i &= -g^{ik} (\partial_k \phi + \partial_0 A_k) \\
\end{align*}

Since $k = \{1, 2, 3\}$ ($k$ is spacelike coordinates), and we would like to
relate $\phi$ with $A$ (to unify $E$), we \textbf{set}:

\begin{equation}
    \boxed{A_0 \equiv - \phi}
\end{equation}

Continuing the derivation,



\begin{align*}
    E^i &= -g^{ik} (\partial_k (- A_0) + \partial_0 A_k) \\
    E^i &= -g^{ik} (\partial_0 A_k - \partial_k A_0 ) \\
    E^i &= -g^{ik} F_{0k}
\end{align*}


So, finally, the relation is:

\begin{equation}
    \boxed{E^i = -g^{ik} F_{0k}}
\end{equation}

\textbf{TODO: Find out how $E^i = c F^{i0}$}


\begin{equation}
    \boxed{E^i = c F^{i0}}
\end{equation}

\section{Other ramifications of Maxwell's equations on $F$}

\subsection{Ramification 1}
\subsubsection{First part, using 4th equation}

We next consider the 4th Maxwell equation:

\begin{align*}
    \curl B &= \mu_0 J + \epsilon_0 \mu_0 \pdv{E}{t} \\
    \curl B &= \mu_0 J + \frac{1}{c^2} \pdv{E}{t} \\
            &\text{Converting to indices,}\\
    (\curl B)^i &= \mu_0 J^i + \frac{1}{c} \pdv{E^i}{ct} \tag{From $\partial_{ct} = \frac{1}{c} \partial_t$} \\
                &= \mu_0 J^i + \frac{1}{c} \pdv{E^i}{X^0} \\
                &= \mu_0 J^i + \pdv{F^{i0}}{X^0} \tag{From $E^i = c F^{i0}$} \\
                &= \mu_0 J^i + \partial_0 F^{i0}
\end{align*}

Now, we start to simplify the LHS, $\curl B$:

\begin{align*}
    &(\curl B)^i = \levicevita^{ijk} \partial_j B_k \\
    %
    &\text{Since $B^k = \frac{1}{2} \levicevita^{klm} F_{lm}$,} \\
    %
    &\text{$B_k = \frac{1}{2} \levicevita_{klm} F^{lm}$,} \tag{\textbf{TODO:} this is scam} \\
    %
    &(\curl B)^i = \levicevita^{ijk} \partial_j (\frac{1}{2} \levicevita_{klm} F^{lm}) =
    \frac{1}{2} \levicevita^{ijk} \levicevita_{klm} \partial_j F^{lm}\\
\end{align*}

\textbf{Aside: We need to know how to evaluate $\levicevita^{ijk} \levicevita_{klm}:$}
\begin{align*}
    \levicevita_{i_1, i_2, \dots, i_n} \levicevita_{j_1, j_2, \dots j_n} =  
    \det{
    \begin{vmatrix}
        \delta_{i_1 j_1} & \delta_{i_1 j_2} &\dots &\delta_{i_1 j_n} \\
        \delta_{i_2 j_1} &\delta_{i_2 j_2} &\dots &\delta_{i_2 j_n} \\
        \vdots           &\vdots  & \ddots & \vdots \\
        \delta_{i_n j_1} & \delta_{i_n j_2} & \dots & \delta_{i_n j_n}
\end{vmatrix}}
\end{align*}

Hence, \textbf{TODO: HOW?}
$\levicevita^{ijk} \levicevita^{ilm} = \frac{-1}{2} (\delta^j_i \delta^k_m - \delta^j_m \delta^k_l)$


Plugging both equations together,

\begin{align*}
    \frac{1}{2} \levicevita^{ijk} \levicevita_{klm} \partial_j F^{lm} &=  \mu_0 J^i + \partial_0 F^{i0}  \\
    %
    \frac{1}{2} \big[ 
   \frac{-1}{2} \big(\delta^i_l \delta^j_m - \delta^i_m \delta^j_l\big) \big]
   \partial_j F^{lm} &=  \mu_0 J^i + \partial_0 F^{i0} \\
    %
    \textbf{Something is fucked here with respect to $\partial_m F^{mi}$} \\
    %
    \frac{1}{2} \big[ \frac{-1}{2} \big(
    \partial_m F^{im} - \partial_m F^{mi} \big) \big] &= \mu_0 J^i + \partial_0 F^{i0}   \\
       %
        \textbf{$F$ is anti-symmetric, so rewriting $-\partial_m F^{mi} = \partial_m F^{im}$} \\
       %
    -\frac{1}{2} \big[ \partial_m F^{im} \big] &= \mu_0 J^i + \partial_0 F^{i0}   \\
       %
       \textbf{In the notes, the $\frac{1}{2}$ does not exist} \\
       %
    - \big[ \partial_m F^{im} \big] &= \mu_0 J^i + \partial_0 F^{i0}   \\
       % 
    \mu_0 J^i + \partial_0 F^{i0}  + \partial_m F^{im}  &= 0 \\
        % 
    \mu_0 J^i + \partial_\mu F^{i\mu} &= 0 \tag{$\mu = \{0, 1, 2, 3 \}$}
\end{align*}

This gives us a continuity-style equation, linking the current density $J$ to
the rate of change of $F$.
\begin{equation}
    \boxed{ \mu_0 J^i + \partial_\mu F^{i\mu} = 0 \tag{$\mu = \{0, 1, 2, 3 \}$} }
\end{equation}


\subsubsection{Second part, using 1st equation}

\begin{align*}
    &\grad E = \frac{\rho}{\epsilon_0} \\
    &\partial_i E^i = \frac{\phi}{\epsilon_0} \\
    &\text{Substituting $E^i = c F^{i0}$, } \\
    &c \partial_i F^{i0} = \frac{\rho}{\epsilon_0}  = \frac{\rho \mu_0}{\mu_0 \epsilon_0} = \rho c^2 \\
    &\partial_i F^{i0} = \mu_0 c \rho \\
    \text{Since $F$ is anti-symmetric, $F^{00} = 0$, Hence:}
    &\partial_0 F^{00} + \partial_i F^{i0} = \mu_0 c \rho \\
    &\partial_\mu F^{\mu 0} = \mu_0 c \rho
\end{align*}

\begin{equation}
    \boxed{ \partial_\mu F^{\mu0} = \mu_0 c \rho}
\end{equation}

\subsubsection{Combining part 1 and part 2:}


\begin{align*}
    \mu_0 J^i + \partial_\mu F^{i\mu} = 0 \tag{From $B$}  \\
    \partial_\mu F^{i\mu} = -\mu_0 J^i 
    \partial_\mu F^{\mu 0} = \mu_0 c \rho \\
    \partial_\mu F^{0 \mu} = - \mu_0 c \rho \\
\end{align*}

To combine these equations, \textbf{we set:}
\begin{equation}
    \boxed{J^0 \equiv c \rho}
\end{equation}
We arrive at the unified equation:

\begin{align*}
    \partial_\mu F^{\nu \mu} = - \mu_0 J^{\nu}
\end{align*}

Choose units such that $c = \frac{h}{2 \pi} = G_n = 1$, which gives us:


\begin{align*}
    &\partial_\mu F^{\nu \mu} = -  J^{\nu} \\
    &\textbf{$F$ is antisymmetric, so flipping indices} \\
    &\partial_\mu F^{\mu \nu} =  J^{\nu} \\
\end{align*}

\begin{equation}
    \boxed{ \partial_\mu F^{\mu \nu} =  J^{\nu} }
\end{equation}

Note that this is \textbf{Ampere's law!}

\chapter{Hierarchy Theorems}

$\exists L$, such that $\forall f: \mathbb{N} \to \mathbb{N}$, where $f$ is
space/time constructible,
\begin{align*}
    Space(f) &\supsetneq Space(o(f)) \\
    Time(f) &\supsetneq Time \bigg (\frac{o(f)}{log f} \bigg )
\end{align*}

So, there is a Hierarchy of complexity classes in time and space.

\subsection{Proof sketch}
We exhibit a languag $A$, such that $A \in Space(f(n))$, and $A \notin Space(o(f(n))$.

Let $D$ decide $A$. D's definition:
\begin{itemize}
    \item compute $f(n)$ and mark the end of $f(n)$ cells. If the read-write head
        ever crosses it, \texttt{REJECT}, \texttt{HALT}. We first need
        $f(n)$ to use $f(n)$ cells or less to compute. This is called as 
        \textbf{space-constructribility}. ($f: \mathbb{N} \to \mathbb{N}$ is
        space-constructible iff given $n$, $\exists$ TM which computes $f(n)$ 
        using at most $f(n)$ cells). Also, we want $f(n)$ to be at least
        $log(n)$. Clearly, this process is in space $f(n)$.

    \item We now need to "seaparate" A from the smaller classes. If A can be
        solved in a smaller space (ie, we cannot separate $A$), then
        there must be a TM (say, $D'$) which decides A in space less than $f(n)$.
        So now, we need to choose some input such that $D'$ is different from $D$.
        We can use diagonalization to construct such a function.

    \item let the input be $x$. Let $x = M10*$ for some TM $M$. if not,
        \texttt{REJECT}, \texttt{HALT}.

    \item Let $D$ simulate $M$ on input $M$. If $M$ takes less that $f(n)$
        time to run on $M$, then $M$ can decide A in time less than $f(n)$.
        So now, $D$ knows how much space $M(\langle M \rangle)$ requires. if $M(\langle M \rangle)$
        accepts, we reject.  If $M(\langle M \rangle)$ rejects, we accept (diagonalization).

    \item To find out whether $M(\langle M \rangle)$ rejects, note that it is space-bounded, so
        we can just check how many states of the configuration space it visits.
        If it has not halted after visiting all states in the configuration
        space, we can conclude that $M(\langle M \rangle)$ does not halt. The configuration space
        is $O(2^{f(n)})$. So we need to run $D$ for time $O(2^f(n))$, and then
        \texttt{REJECT} if it continues running.
\end{itemize}

\textbf{Arjun Q:} Are there examples of non-space constructible functions, which are non-trivial?
Other than ones that are too-small?


Proofs of time are similar to the space separation theorem.
\begin{itemize}
    \item compute $t(n)$ ($t(n)$ should be time-constructible).
        decrement a counter initialized to $t(n)$. if this hits $0$, 
        \texttt{REJECT, HALT}. We get a $log$ factor due to the slowdown
        of keeping time. (People are trying to speed this up).

    \item once again, repeat the same construction used for $SPACE$.
\end{itemize}

\section{Savitch's Theorem:  $\texttt{NSPACE($f(n)$)} \subseteq \texttt{PSPACE($f(n)^2$)}$}
$NSPACE(f(n))$ -- one branch of a NTM $N$ decides $L$ in space $O(f(n)$.

Configuration space is $O(\texttt{alphabet}^{f(n)}) = O(2^f(n))$ -- otherwise,
configurations are repeated.

Our branch depth is exponential in $f(n)$. So, we need to keep track of
$O(2^{f(n)})$ data.

Given $\langle C_1 \in Config(N), C_2 \in Config(N), t \in \mathbb{N} \rangle$
if we can find whether $C_1$ goes to $C_2$ in $t$ space, then we can solve
our original problem.

This can be solved by recursion by asking if there exists a $C_{mid}$, such that
$C_1 \to C_{mid}$ in $t/2$ steps, similarly $C_{mid} to C_2$ in $t/2$ steps.

% F(C1, C2, t)
% for all config C{mid}:
% a1 = F(C1, Cmid, t/2)
% a2 = F(Cmid, C2, t/2)
% accept if both accept

\section{Cook Levin theorem}
$L$ is \texttt{NP-complete}, if 

\begin{itemize}
    \item $L \in NP$
    \item $\forall L' \in NP$, there exists a Karp reduction from $L'$ to $L$: $L' \leq_p L$ (\texttt{NP-hard})
\end{itemize}

$A \leq_p B$ if there exists a poly time computable function $f: \Sigma^* \to
\Sigma^*$ such that 

$$w \in A \Leftrightarrow f(a) \in B$$
Karp reduction = poly time mapping reduction.

Define \texttt{SAT}, and show that \texttt{SAT} is \texttt{NP-complete}.


We have boolean formulas $\phi$, which is given in CNF.

$$\texttt{SAT} = \{ \phi~\vert~\phi~\text{is in CNF (product of sums),}~\phi~\text{is satisfiable} \}$$.

This is clearly decidable since we can try all possible assignments.

It is in NP since a NDTM can try to guess assignments.

To show that this is NP-complete, take any language $L'$ in \texttt{NP}. We provide
a karp reduction to \texttt{SAT}. We take the poly-time checker for $L'$ into a
SAT problem $\psi$, such that \textbf{iff} a solution for $\psi$ exists, then the poly
time checker will accept the string, and vice versa (for reject).



\section{\texttt{EXPSPACE} completeness - $EQ_{REG\uparrow}$}
$r \uparrow \equiv \exists k \in \mathbb{N}. r \uparrow$ is regular if $r$ is
regular.  We need this operator to control input size.

Question: Check if two regular expressions are the same -- We show that
this $\notin \texttt{PSPACE}$, and hence $\notin \texttt{PTIME}$. We show that
this problem is $\texttt{EXPSPACE}$ complete.

\section{Information Theory, CPA security - Lecture 5}


Most theorems will read as: if X is true, then the protocol $\Pi$ is secure.

\section{Our first example of circumventing an impossibility}

\section{PRNGs - Pseudo random number generators}
This allows us to break $|K| \geq |M|$.
This is still a one-time pad, but it allows us to create $|K| << |M|$.


Deterministic program $G$.  Takes as input $n$-bit string, returns $l(n)$ bit string. We have two
assumptions.
\begin{itemize}
\item 1. $l(n) > n$. Expansion.
\item 2. Pseudorandomess.
\end{itemize}

\subsubsection{Pseudorandomness}
for all PPTM(probabilistic polynomial turing machine) $D$,
$$ |P[D(r) = 1] - P[D(G(s)) = 1]| \leq \negl(|s|) \quad  r \leftarrow \binary^{l(n)} \quad  s \leftarrow \binary^n$$
$r, s$ are chosen uniformly at random. The probability distribution is over the
random coins used by $D$, along with the uniform distributions of $r$ and $s$.


\begin{itemize}
\item Strings of length $l(n)$. pick one at random. probability of picking one of them is
\item Strings of length $n$, and then we inject into $l(n)$ with $G$. Clearly, $|Im(G)| < 2^{l(n)}|$.
So, we can sample all $|Im(G)|$. If we are in a pseudo-random world, it will repeat for sure (with $P = 1$).
If we are in the non-PRNG world (true randomness), the chance that something repeats will be negligibly small.
\item We cannot distinguish with polynomial samples, however. So, PPTM is a good choice for a distinguisher.
\end{itemize}


Given that we have to assume PRNGs exist, there are different ways to proceed:
\begin{itemize}
\item Heuristics - Assume that the PRNG we write is a true PRNG, and then get to work.
\item Specific mathematical assumptions - Assume that certain problems are hard. Build PRNGs from this mathematical assumption.
\item Provable Security - If there exists even one hard problem $P$, then we can use that to build a PRNG.
\item Proven security - prove PRNGs exist.
\end{itemize}


\subsubsection{Assume PRNGs exist. We will build a secure encryption scheme}

% \begin{align*}
%     &M = \binary^{l(m)} \quad K = \binary^m \\
%     &Gen: k = {0, 1}^n$ \\
%     &Enc_k(m) = m \xor G(k) \quad Dec_k(c) = c \xor G(k)
% \end{align*}


Note that this is just one time. If they attacker can see two ciphertexts, they can XOR the ciphertexts to get the XOR of the cleartexts.

\paragraph{Proof that this is sane}.
If the adversary can differentiate between $M_0$ $M_1$, we will use it to break the PRNG (as in, distinguish between PRNG and RNG).
Call the adversary A. It can generate 2 messages $M_0$ and $M_1$. When given $encryption(M_b) = G(k) xor M_b$, he can guess $b = 0 \/ b = 1$ with non-negligible probability.
Call the distinguisher $D$. $D$ has to distinguish between truly random and pseudo random world for our proof.
Given a string $w$ and ask if $w$ can be distinguished by $A$. We can pick $w$ from the PRNG world or the RNG world.

If $A(w xor M0, w xor M1)$ gives us the  correct value(can distinguish), then we are using the PRNG. Otherwise, it is the RNG.


\subsubsection{CPA secure}

Adversary gets to pick $M_0$ and $M_1$, we choose a bit $b$ at random and give encryption of $M_b$. He has an oracle that has oracle access to the encryption algorithm. Even with this, he should not be able to guess $b$.

\paragraph{No determistic algorithm can be CPA secure}

The adversary will ask for encryption of $M_0$ and encryption of $M_1$. He gets back $C_0$ and $C_1$. Then, we can compare that to our result, and find the random bit $b$.


\paragraph{How to create CPA secure}
$C = <R, c xor enc(R)>$ where R is a random string.
Decryption will never fail. if we know R, we can xor twice.
Encryption will not fail because encryption of random data is still random.

We have a problem of length doubling: For one length of data, we need R as well.

\subsubsection{Indexable PRNGS}
A PRNG that we can index at a point, and it will start generating from that index. They are called ``pseudorandom functions''.

Consider $Z/pZ^x$. All numbers except $1$ in $Z/pZ$ are generators.

Discrete log: Given $g^x mod p$, given $g$, given $p$, find $x$. (log in a group). We know that Discrete log is hard. Let us try and build a PRNG.


Step 1. Given a PRNG that expands 1 bit, we can use it to create a PRNG that expands any number of bits $n$
$s = seed$. 
$G(s), G(G(s)), G(G(G(s))), G^n(s)$, take the extra bits from each $G^i(s)$. This is a PRNG.

This is a PRNG.

Assume we can break this PRNG. $s_1 s_2 ... s_n$ = stuff from PRNG is distinguishable from $r_1 r_2 r_3 ... r_n$ = Random info.

Construct $s_0 s_1 s_2 .. s_n$, $r_0 s_1 s_2.. s_n$, $r_0 r_1 s_2 s_3 ... s_n$. $r_0 r_1 r_2 r_3 .. r_n$. We know that we can
distinguish first from last. Hence, there must be an adjacent set of strings that can be distinguished, since ``distinguishable'' is transitive (why?)
so, if $r_i  dist r_{i+2}$, we need to have either $r_i dist r_{i + 1}$ or $r_{i + 1} dist r_{i + 2}$. However, between these strings, we have only edited $s_i$.
So, we are able to distinguish one bit extra. This means we can actually distinguish the output of $G$.


Step 2. if we can find $MSB(x)$, we can find x in polynomial time. So, all we need to do is to break $MSB(x)$.


Step 3. Create PRNG that produces one bit output using discrete log.

Take seed s. output $MSB(s_1 = g^s mod p)$. So we now have a PRNG that can create one bit. Second output: $MSB(s_2 = g^{s_1} mod p )$
Third output: $MSB(s_3 = g^{s_3} mod p)$.

Hence, if discrete log is hard, we can get a PRNG. 

\subsection{Multi message Indistinguishability experiment}

This is defined for an encryption scheme $(\gen, \enc, \dec)$.
\begin{itemize}
\item Adversary outputs a pair of vector of messages $(\vec m_0, \vec m_1)$. Each
    vector contains the same number of messages, and the $i$th messages have the
    same length. That is, $|m_0[i]| = |m_1[i]|$.
\item A random key is created: $ k \leftarrow \gen$, and a random bit
    $b \leftarrow \binary$ is chosen. For all $i$, $c[i] \leftarrow \enc_k(m[i])$
    is computed. $\vec c$ is given to the adversary $A$.
\item The adversary $A$ outputs a bit $b'$. The output of the experiment is $1$
    if $b = b'$, and $0$ otherwise.
\end{itemize}

Security definition of the cryptosystem remains unchanged.


\paragraph{Weakness of one time pads under this threat model}
Note that one time pads will fall to this threat model, since repeatedly
ciphering data with a one-time pad will allow us to extract data from the
one-time pad. Indeed, any deterministic scheme can be attacked under
this threat model. So, we now need probabilistic encryption schemes.

\paragraph{Attacking all deterministic cryptosystems under multi message threat model}
Let $m_0 \equiv (0^n, 0^n), m_1 \equiv (0^n, 1^n)$. Run this through the experiment. We will
be given $c \equiv (c_0, c_1)$. If $c_0 = c_1$, then  we know that the message
was $(0^n, 0^n) = m_0$, and is $m_1$ otherwise. We know this since the encryption
function is deterministic, and  hence $\enc_k(m_0) = \enc_k(m_1) \implies m_0 = m_1$.


\chapter{Gaps in space and time}

We wish to study what is not computable given some resource.
If there resource is time, we want to understand what can be solved
in $t(n)$ but not in smaller than $t(n)$ --- in the sense of $o(t(n))$.

We can try to construct a hierarchy of problems that can be solved
given increasing time. 

\begin{align*}
f(n) \in o(g(n)) &\equiv \lim_{n \to \infty} \frac{f(n)}{g(n)} = 0 \\
f(n) \in O(g(n)) &\equiv \lim_{n \to \infty} \frac{f(n)}{g(n)} \in  O(1)
\end{align*}

\section{Space Hierarchy}

A function $f: \mathbb{N} \to \mathbb{N}$ is said to be \textbf{space constructible}
if there exists a turing machine that on input $1^n$, it computes $f(n)$
using space $O(f(n))$. So the output can be $1^{f(n)}$ say, since that uses
space $O(f(n))$.

Most common functions such as polynomials, exponentials, and logarithms
are all space constructible.

\begin{theorem}
Let $f$ be a space-constructible function. There exists a language $L$ which
can be decided in $O(f(n))$ space, but not in $o(f(n))$ space.
\end{theorem}
\begin{proof}
The proof is to \textbf{construct} a language which can be decided on $O(f(n))$
space, but not in $o(f(n))$ space. Such a language tends to be artificial due
to the construction having to work \textit{for all $f$}.

We need two properties for this language $L$ we create:

\begin{itemize}
\item It is \textbf{not decidable} in $o(f(n))$ space.
\item It \textbf{is} decidable in $O(f(n))$ space.
\end{itemize}

We will use diagonalization to show an construct an $L$ that 
\textbf{cannot be decided} in $o(f(n))$ space. List each TM that runs in 
$o(f(n))$ space. This collection of all TMs (viewed as strings) is written as:

$$ALLTM = \cup_{i=0}^\infty \{0, 1\}^i$$


We will define a language $L$ which cannot be decided by \textbf{any} TM
on the above list.

We will create a matrix of the form $DECIDE(i, j) = M_i(\langle M_j \rangle)$.
That is, we feed $M_i$ the string of $M_j$.($\langle M_j \rangle$ interprets
the machine $M_j$ as a string).

Now, create a language $L$:

\begin{align*}
L \equiv \{ M~\vert~M ( \langle M \rangle ) = 0 \}
\end{align*}

Note that $L$ is \textbf{not decidable} in $o(f(n))$ space. Proof by contradtiction:
Assume such a machine $M_c$ ($c$ for contradiction) exists. We now ask if $\langle M_c \rangle \in L$?

\begin{itemize}
\item If $\langle M_c \rangle \in L$, then $M_c (\langle M_c \rangle) = 0$ (by the definiton of $L$).
But since $M_c$ \textbf{decides} $L$,
$M_c (\langle M_c \rangle) = 0 \implies  \langle M_c \rangle \notin L$. \textbf{Contradiction}.

\item On the other hand, say that $\langle M_c \rangle \notin L$, then $M_c (\langle M_c \rangle) = 1$
(by the definition of $L$).
But since $M_c$ \textbf{decides} $L$, $M_c (\langle M_c \rangle) = 1 \implies \langle M_c \rangle \in L$. 
\textbf{Contradiction}.
\end{itemize}


We now move to show that $L$ \textbf{can be decided} in $O(f(n))$ space.
Consider a machine \texttt{INTERPRET} that does this:

\begin{minted}{python}
def INTERPRET(w):
    Mw = convert_to_TM(w)

    # Naive solution: Try to run Mw, see what happens.
    # flag = Mw.run(w)

    # Problem 1: How do we know it runs in o(f(n)) space?
    # flag = Mw.run_with_bounded_space(w, space_bound=f(n))

    # Problem 2: How do we know that Mw halts?
    # Count the size of the config. space, and reject if Mw
    # takes more steps than the configuration space size.
    flag = Mw.run_wth_bounded_space_and_steps(w, space_bound=f(n), 
                                                 steps_bound=Mw.config_space_size())


    return !flag
\end{minted}
\textbf{For more details, read Sipser chapter 9}
\end{proof}



\begin{corollary}
For two functions $f1, f2 : \mathbb{N} \to \mathbb{N}$, if
$f1 \in o(f2)$, then $\dspace~(f1) \subsetneq \dspace~(f2)$. 
(\textbf{Sid note: we do not need the condition that $f1 \neq f2$ thanks to the fact
that in $o(n)$, the limit tends to $0$})
\end{corollary}

\section{Time Hierarchy}

\begin{theorem}
Let $f$ be a time-constructible function. There exists a language $L$ which
can be decided in $O(f(n))$ time, but not in $o \bigg(\frac{f(n)}{\log(f(n)} \bigg)$ time.
\end{theorem}
\begin{proof}
Proof is the same as that of space hierarchy (roughly).

We get the $\log$ factor for us to simulate a $f(n)$ time turing machine.
We do not know how to perform the simulation with constant overhead.
\end{proof}

\begin{corollary} $\texttt{P} \subsetneq \texttt{EXPTIME}$ \end{corollary}


\chapter{Probabilistic encryption}

Determinisim fucks over security. Since now-a-days, servers encrypt pretty much
everything you send them, you can try to mount a chosen plaintext attack.


\section{Truly random functions} 
Look at all functions from r to x. Pick one such function and use that. Number
of such functions: $2^{n^{2^n}}$ 

Number of bits to index this set: $$\log
(2^{n^{2^n}}) = 2^n log(2^n) = n \cdot 2^n$$

\section{Pseudorandom Function (PRF)}
We need distributions on functions. We define this by using keyed functions.

\begin{align*}
F: (k: \binary^n) \to (r : \binary^n) \to (x: \binary^n)
\end{align*}

firstst string is key, second string is what to encode, output is encoded.
In general, we fix a key, and then consider the function $F_k$. We assume
that $F_k$ is efficent. That is, there is a deterministic polynomial time
algorithm that can compute $(F_k(x)~\forall k, x)$.

Intuitively $F$ is called a pseudorandom function if the function $F_k$
for a randomly chosen $k$ is indistinguishable from a random function chosen
from the set of all functions having that domain and range. 

Note that the space of all functions is $(2^n)^{2^n}$, while the space of keys
is just $(2^k)$.


If we have key size as $n \cdot 2^n$, then $F_k$ (the kth function in the set
of all TRFS from r to x) will be truly random.

We formally define them as:
\begin{itemize}
\item Efficiency of computation:  given $x$, computing $f_k(x)$ is easy.
\item Pseudorandomness: for all PPTM A, 

\begin{align*}
|P [ A^{f_k(\cdot)} = 1 ] - P [ A^{f_n(\cdot)} = 1 ]| \leq \negl(n)
\end{align*} 
where $k \leftarrow \binary^k$, is a key that is chosen uniformly at
random from the key space, and $f_n \leftarrow (\binary^n \rightarrow \binary^n)$ 
is chosen uniformly at random from the space of functions.
\end{itemize}

\section{CPA security from pseudorandom functions}
Let $F$ be a pseudorandom function. Define an encryption scheme as follows:
\begin{itemize}
\item $\gen \equiv k \leftarrow \binary^n$. Choose a key at random
\item $\enc(m) \equiv ( r, F_k(r) \xor m)$
\item $\dec((r, c)) \equiv f_k(r) \xor c$.
\end{itemize}

This as seen before is CPA secure, but is problematically length doubling.

\section{Pseudorandom permutations}
A pseudorandom permutation is much like a pseudorandom function, except it
is bijective, and there is a polynomial time algorithm to compute both
$F_k(\cdot)$ and $F_k^{-1}(\cdot)$.

\begin{definition}
    Let $F: \binary^* \times \binary^* \rightarrow \binary^*$ be an efficent
    keyed permutation. We call $F$ a pseudorandom permutation if for all
    PPTM $D$, there exists a negligible function $\negl$ such that:
    \begin{align*}
        \left| \pr{D^{F_k(\cdot), F_k^{-1}(\cdot)} = 1} - 
        \pr{D^{f_n(\cdot), f_n^{-1}(\cdot)} = 1} \right| \leq \negl(n)
    \end{align*}
    Where $k \leftarrow \binary^n$ is chosen uniformly at random, and $f_n$
    is chosen uniformly at random from the set of all permutations of n
    bit strings.
\end{definition}

\section{Modes of operation}
A mode of operation is essentially a way to encrypt arbitrary length
messages using a block cipher. We will see methods that have better
\emph{ciphertext expansion}: The ratio between  the length of the message
and the length of the ciphertext.

\subsection{ECB --- Electronic Code Book} 
Given plaintext $m = \langle m_1 m_2 \dots m_l \rangle$, cyphertext
encrypts each block randomly, usin the PRF: $(c = \langle F_k(m_1), F_k(m_2) \dots F_k(m_l) \rangle)$

This is deterministic, and so cannot be CPA-secure.  It doesn't even have
indistinguishable encryption in the presence of an eavesdropper. This is
because if the same block repeats in the plaintext, it repeats in the
ciphertext.


\subsection{CBC --- Cipher Block Chaining}

Like the name says, chain blocks for messages. we perform $c_k = F_k(m_k \xor c_{k - 1})$. This creates a chain of dependences.

\subsection{Output feedback mode} 
Here, we set $r_n = f_k^n(r_0)$, where $r_0 \equiv IV$. This allows us to
compute $r_n$ without having to have the entire history.

\begin{align*}
    &r_0 = \text{public} \\
    &r_n = f_k(r_{n - 1}) \\
    &c_k = m_k \xor r_k
\end{align*}




\chapter{Probabilistic proofs}
\section{Completeness and Soundness}
\begin{definition}
Completeness: For every true assertion, there is a valid proof.
\end{definition}

\begin{definition}
Soundness: For every false assertion, no valid proof exists.
\end{definition}

A good proof system must also be such that the verifier is efficient
(that is, polynomial time).

If we ask that a proof system must be sound and complete, there is no 
scope for error! Further, it is not clear if the verifier and the
prover can "talk" to each other. If we choose to allow interactions, what
are the implications?


We relax the assumptions this way --- Relaxed compleness states that
for every true assertion, there is a
proof strategy that will convince the verifier with probability 
at least $> \frac{1}{2}$.  
Similarly, relaxed soundness states that for every false assertion,
every proof strategy fails to convinve the verifier with probability
at least $> \frac{2}{3}$. 

The formalization is as follows:
\begin{definition}
Interactive proof systems
\begin{itemize}
\item An interactive proof system for a language $L$ consists of two
entities: a prover $P$ and a verifier $V$.
$P$ and $V$ share common input, and work for $R \in \mathbb{N}$ rounds.

\item In each round, the prover can send the verifier a message that 
is polynomial in the length of the input.

\item The verifier can send a polynomial length reply to the prover.

\item The verifier is a randomized polynomial time turing machine. Time
is measured as a function of the length of the input.

\item \textbf{Completeness}: $\forall x \in L$, there exists a prover strategy
so that the verifier accepts with probability $> \frac{2}{3}$.

\item \textbf{Soundness}: $\forall x \notin L$, any prover strategy will lad
the verifier to accept with probability  $< \frac{1}{3}$.
\end{itemize}
\end{definition}

\newcommand{\cobpp}{\texttt{co-BPP}}
\newcommand{\ip}{\texttt{IP}}
\newcommand{\zkp}{\texttt{ZKP}}
\chapter{Exploring probabilistic complexity classes}

\begin{itemize}
    \item $\bpp \subset \Sigma^p_2 \cap \Pi^p_2$ (Sipser-Gaes theorem)
    \item $\bpp \subset \ppoly$ (Adleman's theorem)
    \item Hierarchy theorem for circuit complexity --- $SIZE(f(n))$
\end{itemize}


\begin{theorem}
    $\bpp \subseteq \Sigma^p_2 \cap \Pi^p_2$ (Sipser-Gaes theorem)
\end{theorem}
\begin{proof}
    First, notice that $\bpp = \cobpp$ (flip the answer, similar to $\ptime = \coptime$).

    Showing $\bpp \subseteq \Sigma_2^p$, it would automatically imply that $\bpp \subseteq \Pi_2^p$,
    since $\bpp = \cobpp$.
    So now, all we need to show is that $\bpp \subseteq \Sigma_2^p$.

    \begin{align*}
    &\text{Definition of \bpp~for a language $L$:} \\
    &\exists \text{poly time}~M, \forall w \in L, P[M(w, r) = accept] \geq \frac{2}{3} \\
    &\exists \text{poly time}~M, \forall w \notin L, P[M(w, r) = accept] \leq \frac{1}{3}
    \end{align*}

    By repeating, we can make the probabilities:
    \begin{align*}
    &\text{Definition of \bpp~for a language $L$:} \\
    &\exists \text{poly time}~M, \forall w \in L, P[M(w, r) = accept] \geq \ 1 - \frac{1}{2^n} \\
    &\exists \text{poly time}~M, \forall w \notin L, P[M(w, r) = accept] \leq \frac{1}{2^n}
    \end{align*}
\end{proof}

\begin{align*}
    S_x = \{ r~\vert~ r~\text{is good for input x} \}
\end{align*}

That is, given $r$ as the randomness for input $x$ on machine $M$, $M$ will
accept with high probability (ie. $M(x, r)$ accepts with high probability).


There are only two cases: Either $|S_x|$ is large, or $|S_x|$ is very small,
since $x \in L \lor x \notin L$.  Let $|r| = m$. Now, we count the size
of $|S_x|$ depending on the cases of $x \in_? L$.


%% TODO: fix this
\begin{align*}
    x \in L &\implies |S_x| \geq (1 - \frac{1}{2^n}) \cdot 2^m \\
    x \notin L &\implies |S_x| \leq \frac{1}{2^n} \cdot 2^m
\end{align*}

We can exploit this idea to show the theorem: "every string either has a large
$|S_x|$ or a small $|S_x|$ can be written as a "good" $\Sigma_2^p$ statement.


Let $S$ be any set, $|S|< 2^{m - n}$ ($|S|$ is small). Let $k = ceil(m / n) + 1$.
Let us define:

$$
S + u_i \equiv \{ x + u_i~\vert~x \in S \}
$$

where $+$ is bitwise-XOR.

First of all, note that $|S + u_i| = |S|$ ($\texttt{blank} + u_i$ = XOR = bijective, hence we don't change cardinality).


\begin{align*}
    \bigcup_{i=1}^k |(S + u_i)| \leq \sum_{i=1}^k |S + u_i| = \sum_{i=1}^k |S| = k |S| = k \cdot 2^{m - n}
\end{align*}


\begin{align*}
    \forall u_1, u_2, \dots u_k \in \{0, 1\}^*, |u_i| = m, \bigcup_{i=1}^k (S + u_i) \neq \{0, 1\}^m
\end{align*}
Immediate, because $|\bigcup_{i=1}^k (S + u_i)| \leq k \cdot 2^{m - n}$,
$|\{0, 1\}^m| = 2^m$.  ($m = poly(n)$, hence $k = poly(n)$, now argue inequalities).


Next, we argue something similar for \textit{large} sets. 
If $|S| \geq (1 - \frac{1}{2^n}) \cdot 2^m$,

\begin{align*}
    Stmt \equiv \exists u_1, u_2, \dots u_k \in \{0, 1\}^*, |u_i| = m, \bigcup_{i=1}^k (S + u_i) = \{0, 1\}^m
\end{align*}

We prove this using the \textbf{probabilistic method}.  $P[Stmt] > 0$ is what we want to show.
Let us show this by considering the \textit{converse}: $P[\lnot Stmt] < 1$. 


We first make some definitions to state formally: 
$$B_r \equiv r \notin \bigcup (S + u_i)$$
 $$B_r^i \equiv r \notin S + u_i$$

The converse is now the probability that $B_r$ does not cover $\{0, 1\}^m$.  
We know that $$|S + u_i| \geq (1 - \frac{1}{2^n}) \cdot 2^m$$.
Note that in this probability, we range over all $u_i$.

\begin{align*}
    P[B_r^i] \leq \texttt{TODO: complete this} = 2^{-n} \\
    P[B_r] \leq  k \cdot 2^{-n} \leq 1
\end{align*}

Hence, there must exist $\{ u_i \}$.


So now, we can pose the membership query $x \in_? L$ as:

\begin{align*}
    \exists u_1, u_2, \dots u_k, \forall r \in \{0, 1\}^m, r \in \bigcup_{i = 1}^k (S_x + u_i)
\end{align*}

If the statement had been $r \in S_x$, then this means that $M(x, r) = \texttt{ACCEPT}$.
However, here we know that $r \in \bigcup_{i = 1}^k (S_x + u_i)$. From this, 
we infer that $\exists i, r \in S_x + u_i$.  This means
that $r = r_0 + u_i$ where $M(x, r_0) = \texttt{ACCEPT}$. Hence, this means
that $r + u_i$ will accept $M(x, r + u_i) = M(x, r_0 + u_i + u_i) = M(x, r_0) = \texttt{ACCEPT}$.

So, we can create a new machine $M'(x, r) = \text{run}~M(x, r + u_i),  \forall i \in [1, k]$.
Note that $M'$ is in poly, since $M$ is poly, and $k = ceil(m, n) = poly(n)$, since
$m$ is the amount of randomness the machine $M$ consumes.

\textbf{Figure out where we need the $k = \dots + 1$ precisely (why the $+1$)}

\begin{align*}
    \exists u_1, u_2, \dots u_k, \forall r \in \{0, 1\}^m, M'(x, r) = \texttt{ACCEPT}
\end{align*}

This statement is now in $\Sigma_2^p$, since $M' \in \ptime$, and we have the correct
$\exists \forall$ structure.


\textbf{Intuitive picture: if $x \in L$, then the set $S_x$ is large enough that we
only need a polynomial number of $\{u_i\}$ to "spread" $S_x$ around to cover all
possible $r$, and the converse}.

\section{$\bpp \subset \ppoly$ (Adleman's theorem)}

\begin{align*}
    P(M[x, r] = L(x)]) \geq  1 - \frac{1}{2^{n + 1}}
\end{align*}

Let the length of $r$ be $m$, and hence there are $2^m$  number of $r$.
The number of bad $r$'s is $\leq \frac{2^m}{2^{n + 1}}$.

There are only $2^n$ possible $x$ inputs to the machine $M$.

Let us try to count the number of bad things for \textit{any} $x$. This will 
be $\leq 2^n \cdot \frac{2^m}{2^{n + 1}} = \frac{2^m}{2} < 2^m$.

So, there must exist a random string $r$ that is good for \textbf{all} $x$.
Give this as the advice string to derandomize. So now, we got the advice
sring for \ppoly~for every $n$.

This does not allow us to reduce to \ptime, because we still need to know
this special $r$ that allows us to derandomize every string of length $n$.

Note that we do need polynomial time here, so we can get exponentially close
to 1.

What we are using is that for a string that belongs in the language, there
are \textit{many} certificates, and similarly for strings that don't belong,
there are \textit{very few} certificates (the "barren middle" exists for \bpp).

\section{Computation is messy (Philosophy)}
Computation is in itself a "messy" object. Small changes in input leads
to huge changes in output. 

In most of known mathematics, it is not as sensitive. 

\chapter{Design models of parallel algorithms}

\section{Partitioning}
This is similar to divide-and-conquer, but we don't need to \textit{combine}
solutions! We can treat problems independently and solve it in parallel.
Examples are parallel merging and searching.

We generate subproblems that are independent of each other.
Example is quicksort. Once we partition the array into two subarrays,
we sort the subarrays recursively.

\subsection{Merging in parallel by partitioning}
Two sorted arrays $A$ and $B$ are to be merged into an array $C$.


We define a function $Rank(x_0, X) = |\{ x < x_0~\vert~x \in X \}|$. Note
that the position of $x_0$ in $sorted(X)$ is equal to $Rank(x_0, X)$.
\textbf{Claim:} $Rank(x, C) = Rank(x, A) + Rank(x, B)$.


For $x \in A$, $Rank(x, A)$ is immediately available (since $A$ is sorted).
We need to find $Rank(x, B)$, but we can find this using binary search through $B$.


Time for each binary search is $O(\log n)$. Total time for merging is
$O(\log n)$, since we are doing each binary search in parallel --- we just need
to read the array $B$, no need to update. The total work is $O(n \log n)$, since
we are performing $O(\log n)$ work for $n$ elements.


Note that this is \textbf{non optimal}. The sequential algorithm has
a time complexity of $O(n)$.


We are going to try and reduce the work to $O(n)$. 

\subsubsection{Merging, take 2: optimal work}
General technique is to solve a smaller problem in parallel, and then
extend the solution to the entire problem!

\begin{itemize}
    \item The problem size to be solved is guided by the factor of non-optimality
        in the current algorithm. We need to reduce the total work to $O(n)$.

        For input size $n$, we do $O(n \log n)$ work. So, for input size $n / \log n$,
        we do $O(n / \log n \times \log (n / \log n) \sim O(n) + O(\log(\log(n)) \sim O(n)$.

    \item We pick every $\log n$th element of $A$. We merge the selected elements
        of $A$ and $B$. However, we still perform binary search on the entireity of $B$.

    \item Pick elements $A[\log n], A[2 \log n], \dots, A[ n - \log n], A[n]$, and
        rank the, in $B$ (ie, find their corresponding positions in $B$.)

    \item Define $[B_{r(i)}, \dots, B_{r(i + 1)}] \equiv \text{portion of $B$ between $A[r \log i]$, $A[(r + 1) \log i]$ in $B$}$.

        \begin{minted}{py}
        A = (5) 6 9 12 (15) 18 19 (21) 23 26
        B = 1 4 (..5..) 7 8 10 11 12 (..15..) 16 17 20 (..21..) 22

        In the output array, we can merge
        the array of B between the (..) elements of A
        \end{minted}

        The problem is that the size of $\log n$ per chunk in $A$ does not mean
        that the size is $\log n$ in $B$.


        \begin{minted}{py}
        A = (5) 6 9 12 (15) ... (...) ...
        B = 6 6 6 6 6 6 6 ... 6

        In this case, the entireity of B is between [5, 15]
        \end{minted}

        So, if we can somehow control the size of $B$, so, we can perform
        binary search in $O(\log n)$, with $n / \log n$ processors.

        We then need to perform the merge with $O(\log n)$, 
        \textbf{under certain conditions}.  There are again $n / \log n$
        such merges.


        The work is $O(n)$.


        So now, the only thing we need to control is the size of partitions
        of $B$.


    \item If $[B_{r(i)}, \dots, B_{r(i + 1)}]$ is too large, then we can
        pick $\log n$ items of this section, and we can rank them in $A$!
        Each piece in $A$ will be smaller than $\log n$, since the partition
        of $A$ was already $\log n$.

    \item we can merge two sorted arrays of size $n$ in time $O(\log n)$
        with work $O(n)$.  This algorithm works in \texttt{CREW}.
        We can improve this  further, we will see this later.
\end{itemize}

\subsection{Searching faster}

Each binary search takes $O(\log n)$ time, and we have $O(n / \log n)$ subproblems,
each of size $O(\log n)$. 

Can we make search faster?

\begin{itemize}
    \item Consider a sorted array $A$ with $n$ elements. We want to search
        for an element $x$.
        Given $p$ processors, we can search at the indeces $1, n / p, 2n/p, \dots, n$.

    \item Record the result of each comparison as $1$ or $0$.
        $cmp[i] = 1 \equiv A[i] < x$, $cmp[i] = 0 \equiv A[i] \geq x$.
        More succinctly, \verb|cmp = map (\a -> a < x) A|.

    \item $cmp$ will either have all 0s, all 1s, or a shift from 1s to 0s.

    \item If we have a shift from 1s to 0s, we know that $x$ is likely
        in the $n/p$ segment corresponding to the shift from 1 to 0.

    \item So now, we can recursively search that small segment.

    \item $T(n) = T(n / p) + O(p)$. ($O(p)$ since $cmp$ has length $p$).
        Hence, $T(n) = T(n / p) + O(1)$. This gives us $O(\log n)$ when $p = 1$
        (make sure this is correct, there is some \textbf{off by one here}.

    \item When $p = O(\sqrt n)$, the time taken will be $O(\log n / \log (\sqrt n)) = O(1)$
        This looks useless from a work point of view, but we want to see what this is
        good for!
\end{itemize}

\subsection{From parallel search to merge}
\begin{itemize}
    \item We have two sorted arrays $A$ and $B$, which we want to merge.
    \item We want to rank some elements of $A$ to create paritions of $B$.
    \item Let us take $\sqrt n$ elements of $A$ in $B$.
    \item We have $n$ processors, so each search can use $ n / \sqrt n = \sqrt n$ processors.
    \item each search now finishes in $O(1)$  time.
    \item the problem is that the partitions of $A$ are much larger now (they are $\sqrt n$ large).
    \item we have a $\sqrt n$ sized piece of $A$, and we have a size of B that is of size $(?)$.
        Note that for each piece of $A$, we now choose to allocate $\sqrt n$ processors.
    \item So, we pick $n^\frac{1}{4}$ elements of $A$ in $B$, each of which
        uses $n^\frac{1}{4}$ processors. Size of each piece is now $n^\frac{1}{4}$.
    \item So, we pick $n^\frac{1}{8}$ elements of $A$ in $B$, each of which
        uses $n^\frac{1}{8}$ processors. Size of each piece is now $n^\frac{1}{8}$.
    \item We reduce the sequence $n \to \sqrt n \to n^{\frac{1}{4}} \to n^\frac{1}{8} \dots \to O(1)$.
        This can be done in $\log \log n$ steps!
\end{itemize}
\end{document}

\section{Hashing - collision resistant hashing schemes}

Key superscript: index
Key subscript: secret.

Probability that we can find collision for $H^s$ by a PPTM must be negligible.


\section{Merkle Damgard Transform}

Given a collision resistant, hash function of the form $(h: \{0, 1\}^{2n} \to {0, 1}^n)$.
(Fixed length)

Then we can construct a collision resistant
hash function of the form $(H: \{0, 1\})^{*} \to \{0, 1\}^n)$.

$m = m1 || m2 || m3 || .. m_n$

$m_i$ is $n$ bits.


$z_1 = h(m_1, IV)$
$z_2 = h(m_2, z_1)$
$z_t = h(m_t, z_{t - 1})$

Finally, $H(m) = h(z_t, |m|) (?)$

If $h$ is collision resistant, then $H$ is collision resistant

\chapter{Tree processing}
\section{Traversal via an Euler tour}
\begin{definition}
an \textbf{Euler tour} is a cycle of a graph that includes every edge of the
graph exactly once.
\end{definition}

\begin{lemma}
A directed graph $G$ \textbf{has an Euler tour} iff for every vertex,its in-degree
equals its out-degree.
\end{lemma}

For a tree $T = (V, E)$, to define an euler tour, we make it a directed graph.
$T_e = (V_e, E_e)$, where $V_e = V$, and $E_e = \cup_{(u,v) \in V} \{ (u, v), (v, u) \}$
That is, each $(u, v)$ in $E$ creates two edges $(u, v)$, and $(v, u)$ in $E_e$.
$T_e$ will have an Euler tour.


We have to define a successor function $s: E_e \to E_e$. Here, the successor for an edge.
For a node $u$ in $T_e$, order its \textbf{neighbours (both incoming and outgoing)}
$v_1, v_2, \dots v_d$. This can be done \textbf{independently at each node}. 
For $e = (v_i, u)$, set $s(e) = (u, v_{i + 1~\text{mod $d$}})$. This choice of $s$ is valid
since we always have both edges $(x, y)$ and $(y, x)$, and we are therefore
assured that $(v_i, u)$ will be an incoming edge, and $(u, v_{i + 1}$ will be
an outgoing edge. Also, compute $i + 1$ modulo $d$, so that we eventually
cycle.

\textbf{TODO: relabel vertices to $[0..(d - 1)]$ so that modulo works properly}
\textbf{TODO: add example}

\begin{theorem}
$s$ actually constructs a tour.
\end{theorem}
\begin{proof}
Induction on number of vertices. If $n = 1$, obviously true. 
If $n = 2$, at most one edge present. We will go along the edge and come back,
which is a valid tour.


\begin{itemize}
\item Let the tour be well defined for $n = k$. We will prove it for $n = k + 1$.
\item Every tree has at least one leaf, call it $l$. Create a tree $T' = T/\{l\}$.
\item Let $u$ be a neighbour of $l$ in $T$.
\item Let $N(u) = \{ v_0, v_1, \dots v_i = l, v_{i + 1}, \dots v_d \}$.
\item Set $s_{new}(u, v) \equiv (v, u)$. Set $s_{new}(v_{i - 1}, u) \equiv (u, v)$.
\item For all other vertices, $s_{new}(e) = s(e)$.
\end{itemize}
\end{proof}


\section{Using euler tours for traversal}
Operations on a tree such a rooting, preorder, and postorder traversal
can be converted to routines on an Euler tour.

\subsection{Rooting a tree}
Designate a node in a tree as the root. All edges in the tree are
directed towards (or away) from the root.

\begin{itemize}
\item let $\{v_1, v_2, \dots v_d\}$ be the neighbours of root node $r$. 
\item we mark the final edge of the tour as \texttt{NIL}, so we get an
Euler path, and not an Euler tour.
\item the edge $(r, v_i)$ appears before $(v_i, r)$.
\item so the edge $parent \to child$ appears before $child \to parent$
\item So, if $uv$ precedes $vu$, then set $u = parent(v)$. Orient the
edge $uv$ as $v \to u$ (that is, $child \to parent$), since we want all edges towards the root.
\end{itemize}

\subsection{Preorder traversal}
We have a rooted tree with $r$ as the root. In a preorder traversal, a node is
listed before any of the nodes in its subtrees.

In an Euler tour, nodes in a subtree are visited by entering subtrees,
and the exiting towards the parent.

If we can track the first occurence of a node in an euler path, this will
tell us the preorder traversal. Note that edges in the euler tour occur
first as $parent \to child$, and later as $child \to parent$. So, we can
look at the sequence of edges in the euler tour, and find the preorder
numbering.

\subsection{Expression tree evauation of binary trees}
Tree may not be balanced.

We use the \texttt{RAKE} technique to evaluate subexpressions. We rake the
leaves from the expression tree --- we remove the leaf node and its parent.

\begin{itemize}
\item $T = (V, E)$ is a tree rooted at root node $r$. $p: E \to E$ is the 
parent function.
\item One step of the rake operation at a leaf $l$ with $p(l) \neq r$ involves:
    \begin{itemize}
    \item Remove node $l$, $p(l)$ from the tree
    \item Make the sublings of $l$ as the child of $p(p(l))$. That is, graft
    the siblings of $l$ to the grandparent of $l$.
    \end{itemize}
\end{itemize}

Why is this a good technique? Can this be applied in parallel to several leaf
nodes? Yes, it can be applied to leaf nodes that don't share the same parent.
In general, there is a richness of leaf nodes in a tree, since there
are only $n - 1$ edges.

Each application of rake at all leaves reduces the  number of leaves by half.
Each application of \texttt{RAKE} is $O(1)$. So, total time is $O(\log n)$.


\begin{minted}{python}
def shrinkTree(R):
    compute labels for leaf nodes, store in array A (exclude leftmost
    and rightmost nodes in this A)

    for _ in range(k):
        apply rake operation to all odd numbered leaves that are
        the *left* children of their parent

        apply rake operation to all odd numbered leaves that are
        the *right* children of their parent

        update A to be the remaining even leaves
\end{minted}

Applying \texttt{Rake} means that we can process more than one leaf node
at the same time.

Fo expression evaluation, this may mean that an internal node with 
only one operand gets raked.

\begin{minted}{python}
          + g(u)
  
    + p(u) 

Y     X (u)


--After raking--

   + g(u)
Y
\end{minted}

\begin{itemize}
    \item Transfer the impact of applying the operaot at p(u) to the sibling of u
    \item $R_u = a_u X_u + b_u$
    \item $X_u$ is the result of the subexpression at node $u$ -- $X_u = f(left, right)$
    \item adjust $a_u$ and $b_u$ during any rake operation appropriately
    \item Initially, at each leaf node, $a_u = 1, b_u = 0$.
\end{itemize}



\begin{minted}{python}
          + g(u)
  
    + p(u) 
    X_w
    a_w
    b_w

v        (u) 5, 1, 0
X_v,
a_v,
b_v

--After raking--

     + g(u)
v
X_v'
a_v'
b_v'
\end{minted}

\begin{itemize}
    \item Before removing $p(u)$, the contribution of $p(u)$ to $g(u)$ will be $X_w a_w + b_w$.
    \item we want what $p(u)$ used to calculate to be what $v$ calculates after.
    \item $X_w = (X_u a_u + b_u) + (X_v a_v + b_v)= (X_v a_v) + (X_u a_u + b_u + b_v)$
    \item What $p(u)$ used to calculate is: $a_w X_w + b_w = a_w (a_v X_v + a_u x_u + b_u + b_v) + b_w = a_w a_v x_v + a_w (a_u X_u + b_u + b_v) + b_w$
    \item what $p(v)$ should be: $a_v' = a_w a_v$, $b_v' = a_w (\dots)$
\end{itemize}

For other operators, proceed in a similar fashion (\textbf{TODO: do this and send to kiko, he seems interested!})



\chapter{Questions}
$f (n) = f (n − 1) + 10$
$f (n) = f (n − 1) + n$
$f (n) = 2f (n − 1)$
$f (n) = f (n/2) + 10$
$f (n) = f (n/2) + n$
$f (n) = 2f (n/2) + n$
$f (n) = 3f (n/2)$
$f (n) = 2f (n/2) + O(n^2)$


\chapter{Review}
\section{PH}
\subsection{$\nptime = \conptime$ implies that PH collapses}
\subsection{$\text{\ppoly} \subset \nptime$ implies PH collapses to level 2}
intuitiion: ppoly's advice string lets us do something like quantifier
exchange, since it gives us an outer there exists for the advice string,
which can be guessed by and $NP^{NP}$ oracle. Also relies on the trick that
$\exists = \lnot \forall$, and you can not in this case due to being oracle call.
\section{BPP}
\subsection{BPP subset P/poly}
Intuition: can guess random seeds that are good for \emph{all} inputs.
\subsection{$BPP \subset \Sigma_p^2 \cap \Pi_p^2$}
Intutiion: can show we can find some $u_i$ to spread randomness around.
If we are on the accept side, the good seeds are large enough that we
can use randomness to cover the full set. If we are on the bad side, the
spreading with $u_i$ will not let us cover enough ground. So, we can 
convert it to $\exists u_i, \forall x_i, ...$ which is a PH style problem.

\end{document}
